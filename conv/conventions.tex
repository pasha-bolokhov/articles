\documentclass[14pt]{article}

\newcommand{\slashed}[1]{\hbox{{$#1$}\llap{$/$}}}
\begin{document}
\renewcommand{\thefootnote}{\fnsymbol{footnote}}

\begin{center}
	{\Large\bf Conventions on signs and constants}
\end{center}

\bigskip

	The Fourier transform is such that
$ p_\mu \to i \partial_\mu $. 
	This way, a derivative in an
	interaction term, acting on a propagator will yield
$ \pm i k_\mu $\footnote[2]{The upper sign corresponds to an outgoing
particle, the lower ($ - i k^\mu $) --- to an incoming one},
	where
$ k_\mu $
	is the momentum running through the propagator line.

	Electron charge is 
$ e = - |e| $ \footnote[3]{This agrees with Peskin \& Schr\"oder and also
with Lifshitz \& Pitaevsky.}.

	The QED action written for the \underline{electron} is
$ \mathcal{L}_{\rm int} = -e\bar{\psi}\slashed{A}\psi $.

	The covariant derivative is 
$ \mathcal{D}_\mu = \partial_\mu + i e A_\mu $.

	The vertex term for 
$ i M $\footnote[8]{ 
$ M $ is the amplitude, so that 
$ S = 1 + i M (2\pi)^4 \delta \left(p_1 + p_2 - p_3 - p_4\right) $.}
	is 
$ - i e \gamma^\mu $.

	Each new vertex we introduce acquires a factor of 
``$ i $''
	when converted to Feynman rules. That is, if
$ \mathcal{L}_{\rm int} = \eta_1 \bar\psi \slashed{n} 
	\left(n\cdot\partial\right)^2 \psi $,
	its vertex will correspond to
$ i\eta_1 \slashed{n} \left ( n \cdot (-i k)\right)^2 $.

\bigskip
	Feynman rules:
\renewcommand{\arraystretch}{2.3}
\begin{tabular}{lc}
	fermion propagator &
$ \frac{i\left(\slashed{p} + m \right)}{p^2 - m^2 + i\epsilon} $ \\

	photon propagator &
$ \frac{-i g_{\mu\nu}}{p^2 + i\epsilon} $ \\

	photon-electron vertex & 
$ -i e \gamma^\mu $ \\
\end{tabular}

\bigskip
	where we take the lagrangian
$ \mathcal{L}_{\rm QED} = - \frac{1}{4} F_{\mu\nu}F^{\mu\nu} +
		\bar{\psi}\left( i \slashed{\partial} - m \right)\psi -
		e\bar{\psi}\slashed{A}\psi $.

	The scalar field lagrangian is
$ \mathcal{L}_{\phi} = \partial_\mu \bar{\phi} \partial^\mu \phi -
			m^2 \bar{\phi} \phi $.

\renewcommand{\arraystretch}{1.0}
	We assume $\gamma_5$ to be 
$\left[\begin{array}{cc}
      0  & 1 \\
      1  & 0
\end{array}\right]$, and the {\it left} projector
$\frac{1 - \gamma^5}{2}$.

	We assume the generators of SU(N) to be normalized as
$ {\rm Tr}\left(T^a T^b\right) = \frac{1}{2} \delta^{ab} $.

\end{document}
