% This is version 3. A shorter version has appeared in the 
% World Scientific book.
% Remember: There Is No Cabal.
\documentstyle[11pt,sprocl]{article}
%\documentstyle[sprocl]{article}
% The following are tree-friendly modifications for the xxx.lanl.gov
% bulletin board^H^H^H^H^H^H^H^H^H^H^H^H^H^H archive.
%% toggle next 6 lines
\oddsidemargin 0.125in         
\evensidemargin 0.125in
%\topmargin=-0.31in 
%\topmargin=0.15in 
%\topmargin=0.16in 
\topmargin=-0.465in 
%\footheight=0.4in
\footheight=0.37in
\textheight = 9.04in
%\textheight = 9.1in
%\textwidth 6.12in
\textwidth 6.02in
\input{psfig.sty}
\bibliographystyle{unsrt}
\renewcommand{\theequation}{\arabic{section}.\arabic{equation}}
\def\thefootnote{\fnsymbol{footnote}}

%% toggle following line
%\pagestyle{empty}
\hyphenation{Y-u-k-a-w-a}
\hyphenation{neu-t-r-a-l-ino}
\hyphenation{non-per-tur-b-a-t-iv-e-ly}
\hyphenation{non-re-n-orm-a-l-i-z-a-ble}
\hyphenation{per-tur-b-a-t-iv-e-ly}
\hyphenation{char-g-ino}
\hyphenation{neu-t-r-a-l-i-n-os}
\hyphenation{char-g-i-n-os}
\hyphenation{Ma-j-e-r-o-t-t-o}
\hyphenation{Ta-ta}
\hyphenation{Ka-r-a-t-as}
% A useful Journal macro
\def\Journal#1#2#3#4{{#1} {\bf #2}, #3 (#4)}
% Some useful journal names
\def\perspectives{in {\it Perspectives on Supersymmetry}, edited by
G.L.~Kane
(World Scientific, Singapore, 1998)}
\def\beq{\begin{eqnarray}}
\def\eeq{\end{eqnarray}}
\def\bea{\begin{eqnarray*}}
\def\eea{\end{eqnarray*}}
\def\NCA{\em Nuovo~Cimento}
\def\IJMP{\em Intl.~J.~Mod.~Phys.}
%\def\NPB{{\em Nucl.~Phys.}~B}
\def\NP{\em Nucl.~Phys.}
\def\PLB{{\em Phys.~Lett.}~B}
\def\JETPLett{{\em JETP Lett.}}
%\def\PL{\em Phys.~Lett.}
\def\PRL{\em Phys.~Rev.~Lett.}
\def\MPL{\em Mod.~Phys.~Lett.}
\def\PRD{{\em Phys.~Rev.}~D}
\def\PR{\em Phys.~Rev.}
\def\PRP{\em Phys.~Rep.}
\def\ZPC{{\em Z.~Phys.}~C}
\def\PTP{{\em Prog.~Theor.~Phys.}}
% Some other macros used in the sample text
\def\Baryon{{\rm B}}
\def\Lepton{{\rm L}}
\def\sbar{\overline}
\def\stilde{\widetilde}
\def\st{\scriptstyle}
\def\sst{\scriptscriptstyle}
\def\vac{|0\rangle}
\def\argh{{{\rm arg}}}
\def\G{\stilde G}
\def\Wmess{W_{\rm mess}}
\def\NI{\stilde N_1}
\def\antivac{\langle 0|}
\def\infinity{\infty}
\def\mco{\multicolumn}
\def\epp{\epsilon^{\prime}}
\def\psibar{\overline\psi}
\def\nmess{N_5}
\def\chibar{\overline\chi}
\def\lagr{{\cal L}}
\def\drbar{\overline{\rm DR}}
\def\msbar{\overline{\rm MS}}
\def\conj{{{\rm c.c.}}}
\def\Et{{\slashchar{E}_T}}
\def\Etot{{\slashchar{E}}}
\def\mZ{m_Z}
\def\MPlanck{M_{\rm P}}
\def\mW{m_W}
\def\cbeta{c_{\beta}}
\def\sbeta{s_{\beta}}
\def\cW{c_{W}}
\def\sW{s_{W}}
\def\deltaeps{\delta}
\def\sigmabar{\overline\sigma}
\def\epsilonbar{\overline\epsilon}
\def\vep{\varepsilon}
\def\ra{\rightarrow}
\def\half{{1\over 2}}
\def\ko{K^0}
\def\be{\beq}
\def\ee{\eeq}
\def\bea{\begin{eqnarray}}
\def\eea{\end{eqnarray}}

%  \gsim and \lsim provide >= and <= signs.
\def\centeron#1#2{{\setbox0=\hbox{#1}\setbox1=\hbox{#2}\ifdim
\wd1>\wd0\kern.5\wd1\kern-.5\wd0\fi
\copy0\kern-.5\wd0\kern-.5\wd1\copy1\ifdim\wd0>\wd1
\kern.5\wd0\kern-.5\wd1\fi}}
\def\ltap{\;\centeron{\raise.35ex\hbox{$<$}}{\lower.65ex\hbox{$\sim$}}\;}
\def\gtap{\;\centeron{\raise.35ex\hbox{$>$}}{\lower.65ex\hbox{$\sim$}}\;}
\def\gsim{\mathrel{\gtap}}
\def\lsim{\mathrel{\ltap}}

%%%%%%%%%%%%%%%%%%%%%%%%%%%%%%%%%%%%%%%
%  Slash character...
\def\slashchar#1{\setbox0=\hbox{$#1$}           % set a box for #1
   \dimen0=\wd0                                 % and get its size
   \setbox1=\hbox{/} \dimen1=\wd1               % get size of /
   \ifdim\dimen0>\dimen1                        % #1 is bigger
      \rlap{\hbox to \dimen0{\hfil/\hfil}}      % so center / in box
      #1                                        % and print #1
   \else                                        % / is bigger
      \rlap{\hbox to \dimen1{\hfil$#1$\hfil}}   % so center #1
      /                                         % and print /
   \fi}                                        %

%%EXAMPLE:  $\slashchar{E}$ or $\slashchar{E}_{t}$
\setcounter{tocdepth}{2}

%%%%%%%%%%%%%%%%%%%%%%%%%%%%%%%%%%%%%%%%%%%%%%%%%%
%                                                %
%    BEGINNING OF TEXT                           %
%                                                %
%%%%%%%%%%%%%%%%%%%%%%%%%%%%%%%%%%%%%%%%%%%%%%%%%%
\begin{document}
\setcounter{footnote}{1}
\begin{flushright}
{\large
hep-ph/9709356 \\
v3 April 7, 1999\\
}
\end{flushright}
\vspace{0.25in}

\title{A SUPERSYMMETRY PRIMER}

\author{ STEPHEN P. MARTIN 
%% toggle next four lines
\footnote{Since October 1, 1998: 
Department of Physics, Northern Illinois University, DeKalb IL 60115,
{\it and}\\
Theoretical Physics, Fermi National Accelerator Laboratory, Batavia IL
60510. email: {\tt spmartin@fnal.gov}}}
\address{Randall Physics Laboratory, University of Michigan\\
Ann Arbor MI 48109-1120 USA}

\maketitle\abstracts{
I provide a pedagogical introduction to supersymmetry. The level of 
discussion is aimed at readers who are familiar with the Standard Model 
and quantum field theory, but who have little or no prior exposure to 
supersymmetry. Topics covered include: motivations for supersymmetry; the 
construction of supersymmetric Lagrangians; supersymmetry-breaking
interactions; the Minimal Supersymmetric Standard Model (MSSM); $R$-parity 
and its consequences; the origins of supersymmetry breaking; the mass 
spectrum of the MSSM; decays of supersymmetric particles; experimental 
signals for supersymmetry; and some extensions of the minimal framework. 
%% toggle next two lines
This is an extended version of a contribution to the book {\em
Perspectives on Supersymmetry}, edited by G.~L.~Kane (World Scientific,
Singapore 1998).}

\tableofcontents
%\vfill\eject
\section{Introduction}\label{sec:intro}
\setcounter{equation}{0}
\setcounter{footnote}{1}

The Standard Model of high energy physics provides a remarkably successful
description of presently known phenomena. The experimental
high-energy frontier has advanced into the hundreds of GeV range
with no confirmed deviations from Standard Model predictions
and few unambiguous hints of additional structure. Still, it seems quite
clear that the Standard Model
is a work in progress and will have to be extended to
describe physics at arbitrarily high energies.
Certainly a new framework will be required at
the reduced Planck scale
$\MPlanck = (8 \pi G_{\rm Newton})^{-1/2}
=
2.4 \times 10^{18}$ GeV, where quantum gravitational effects become
important. Based only on a proper respect
for the power of Nature to surprise us, it seems nearly as obvious that
new physics exists in the 16 orders of magnitude in energy between
the presently explored territory and the Planck scale.

The mere fact that the ratio $\MPlanck/M_W$ is so huge is
already a
powerful clue
to the
character of physics beyond the Standard Model, because of the infamous
``hierarchy problem".\cite{hierarchyproblem}
This is not really a difficulty
with the Standard Model itself, but rather a disturbing
sensitivity of the Higgs potential to new physics
in almost any imaginable extension of the Standard Model.
The electrically neutral part of the Standard Model Higgs field
is a complex scalar $H$ with a classical potential given by
\beq
V = m_H^2 |H|^2 + {\lambda} |H|^4\> .
\label{higgspotential}
\eeq
The Standard Model requires a non-vanishing vacuum expectation value
(VEV) for $H$
at the minimum of the potential.
This will occur if $m_H^2 < 0$, resulting in $\langle H \rangle
= \sqrt{-m_H^2/2\lambda}$. Since we know experimentally that
$\langle H \rangle 
= 174$ GeV from measurements of the properties of the weak interactions,
it must be that $m_H^2$ is very roughly of order $-$(100 GeV)$^2$.
However,
$m_H^2$ receives enormous quantum corrections from the virtual effects
of every particle which couples, directly or indirectly, to the Higgs field.

\begin{figure}
\centerline{\psfig{figure=susyhiggscorr1.ps,height=1in}}
\caption{Quantum corrections to the Higgs (mass)$^2$.
\label{fig:higgscorr1}}
\end{figure}
For example, in Fig.~\ref{fig:higgscorr1}a
we have a correction to $m_H^2$ from
a loop containing a Dirac fermion $f$ with mass $m_f$. If
the Higgs field couples to $f$
with a term in the lagrangian $-\lambda_f H \sbar f f$,
then the Feynman diagram in Fig.~\ref{fig:higgscorr1}a  yields a
correction
\beq
\Delta m_H^2 =  {|\lambda_f|^2\over 16 \pi^2}
\left [-2 {\Lambda_{\rm UV}^2 + 6 m_f^2 \> {\rm ln}(\Lambda_{\rm UV}/m_f) +
\ldots } \right ].
\label{quaddiv1}
\eeq
Here $\Lambda_{\rm UV}$ is an ultraviolet
momentum cutoff used to regulate the
loop integral; it should be interpreted
as the energy scale at which new physics
enters to alter the high-energy behavior of the theory.
The ellipses represent terms which depend on the precise manner in which
the momentum cutoff is applied, and which do
not get large as $\Lambda_{\rm UV}$ does.
Each of the leptons and quarks of the
Standard Model can play the role of $f$; for quarks, eq.~(\ref{quaddiv1})
should be multiplied by 3 to account for color.
The largest correction comes when $f$ is the top quark
with $\lambda_f\approx 1$. The problem is that if $\Lambda_{\rm UV}$
is of order $\MPlanck$, say, then this quantum correction to $m_H^2$
is some 30 orders of magnitude larger than the aimed-for value of
$m_H^2 \sim -(100$ GeV$)^2$.
This is only directly a problem for corrections to the
Higgs scalar boson (mass)$^2$, because quantum
corrections to fermion and gauge boson masses do not have the quadratic
sensitivity to $\Lambda_{\rm UV}$ found in eq.~(\ref{quaddiv1}).
However, the quarks and leptons and the electroweak
gauge bosons $Z^0$, $W^\pm$
of the Standard Model all owe their masses to
$\langle H \rangle$, so that the entire mass spectrum of the
Standard Model is directly or indirectly sensitive to the cutoff
$\Lambda_{\rm UV}$.

One could imagine that the solution is to simply pick an ultraviolet
cutoff $\Lambda_{\rm UV}$ which is not too large. However,
one still has to concoct some new physics at the scale $\Lambda_{\rm UV}$
which not only alters the
propagators in the loop, but actually
cuts off the loop integral.
This is not easy to do in a theory whose lagrangian does
not contain more than two
derivatives, and higher derivative theories
generally suffer from a loss of unitarity.
In string theories, loop integrals are cut off at
high Euclidean momentum $p$ by factors $e^{-p^2/\Lambda^2_{\rm UV}}$, but
then $\Lambda_{\rm UV}$ is a string scale which
is usually thought to be not very far below $\MPlanck$.
Furthermore, there is a contribution similar to eq.~(\ref{quaddiv1}) from
the virtual effects of any
arbitrarily heavy particles which might exist.
For example, suppose there exists a
heavy complex scalar particle $S$ with mass $m_S$
which couples to the Higgs with a lagrangian term
$
-\lambda_S |H|^2 |S|^2$. Then the Feynman diagram
in Fig.~\ref{fig:higgscorr1}b  gives a correction
\beq
\Delta m_H^2 = {\lambda_S\over 16 \pi^2}
\left [\Lambda_{\rm UV}^2 - 2 m_S^2
\> {\rm ln}(\Lambda_{\rm UV}/m_S) + \ldots
\right ].
\label{quaddiv2}
\eeq
If one rejects a physical interpretation of $\Lambda_{\rm UV}$
and uses dimensional regularization
on the loop integral instead of
a momentum cutoff, then there will be no $\Lambda_{\rm UV}^2$ piece.
However, even then the term proportional to $m_S^2$ cannot be eliminated
without the physically unjustifiable tuning of a counter-term specifically
for that purpose. So $m_H^2$ is sensitive to the
masses of the {\it heaviest} particles that $H$ couples to;
if $m_S$ is very large, its effects on the Standard Model
do not decouple, but instead make it very difficult to understand
why $m_H^2$ is so small.

This problem arises even if there is no direct coupling
between the Standard Model Higgs boson and the unknown heavy particles.
For example,
suppose that there exists a heavy fermion $F$ which, unlike the
quarks and leptons of the Standard Model, has vector-like
quantum numbers and therefore gets a large mass $m_F$ without
coupling to the Higgs field. [In other words, an arbitrarily large mass
term of the form
$m_F \overline F F$ is not forbidden by any symmetry, including
$SU(2)_L$.] In
that case, no diagram like
Fig.~\ref{fig:higgscorr1}a exists for $F$. Nevertheless there
will be a correction to
$m_H^2$ as long as $F$ shares
some gauge interactions with the Standard Model Higgs field; these may
be the familiar electroweak interactions, or some unknown gauge
forces which are broken at a very high energy scale inaccessible
to experiment. In any case, the two-loop
Feynman diagrams in Fig.~\ref{fig:higgscorr2}  yield a correction
\begin{figure}
\centerline{\psfig{figure=susyhiggscorr2.ps,height=1.02in}}
\caption{Two-loop corrections to the Higgs (mass)$^2$ due to a heavy
fermion.
\label{fig:higgscorr2}}
\end{figure}
\beq
\Delta m_H^2 = x\left ( {g^2 \over 16 \pi^2} \right )^2
\left [ a \Lambda_{\rm UV}^2 + 48 m_F^2 \>{\rm ln} (\Lambda_{\rm UV}/m_F)
+ \ldots \right ],
\label{quaddiv3}
\eeq
where $g$ is the gauge coupling in question, and $x$ is a group theory
factor of order 1. (Specifically, $x$ is the product of the
quadratic Casimir invariant of $H$ and the Dynkin index of $F$
for the gauge group in question.)
The coefficient $a$ depends on the precise method of cutting
off the momentum integrals. It does not arise at all if
one rejects the possibility of
a physical interpretation for $\Lambda_{\rm UV}$ and uses
dimensional regularization, but the $m_F^2$
contribution is always present.
The numerical factor $(g^2/16 \pi^2)^2$ may be quite small (of order
$10^{-5}$ for electroweak interactions), but the important
point is that these contributions
to $\Delta m_H^2$ are sensitive to
the largest masses and/or ultraviolet cutoff in the theory, presumably
of order $\MPlanck$. The ``natural" (mass)$^2$ of a fundamental Higgs
scalar, including quantum corrections, seems to be more like $\MPlanck^2$
than the experimentally favored value!
Even very indirect contributions from
Feynman diagrams with three or more
loops can give unacceptably
large contributions
to $\Delta m_H^2$.
If the Higgs boson is a fundamental particle,
we have two options: either we must make
the rather bizarre assumption that there do not exist {\it any}
heavy particles which couple (even indirectly or extremely weakly)
to the Higgs scalar field,
or some rather striking cancellation is needed between the various
contributions to $\Delta m_H^2$.

The systematic cancellation of the dangerous
contributions to $\Delta m_H^2$ can
only be brought about by the type of conspiracy which is better known
to physicists as a symmetry. It is apparent from comparing
eqs.~(\ref{quaddiv1}), (\ref{quaddiv2})
that the new symmetry ought to relate fermions and bosons, because
of the relative minus sign between fermion loop and boson loop
contributions to $\Delta m_H^2$. (Note that $\lambda_S$ must be positive
if the scalar potential is to be bounded from below.) If
each of the quarks and leptons of the Standard Model is accompanied
by two complex scalars with $\lambda_S = |\lambda_f|^2$,
then the $\Lambda_{\rm UV}^2$ contributions
of Figs.~\ref{fig:higgscorr1}a and \ref{fig:higgscorr1}b will neatly
cancel.\cite{quadscancel}
Clearly, more restrictions on the theory will be necessary
to ensure that this success persists to higher orders, so that,
for example, the contributions in Fig.~\ref{fig:higgscorr2}  and
eq.~(\ref{quaddiv3}) from a
very heavy
fermion are cancelled by the two-loop effects of some very heavy bosons.
Fortunately, conditions for cancelling all such
contributions to scalar masses are not only possible,
but are actually unavoidable once we merely assume that a symmetry relating
fermions and bosons, called a {\it supersymmetry}, should exist.

A supersymmetry transformation turns a bosonic state into a fermionic
state, and vice versa. The operator $Q$ which generates
such transformations must be an anticommuting spinor, with
\beq
Q |{\rm Boson}\rangle = |{\rm Fermion }\rangle; \qquad\qquad
Q |{\rm Fermion}\rangle = |{\rm Boson }\rangle .
\eeq
Spinors are intrinsically complex objects, so $Q^\dagger$ (the hermitian
conjugate of $Q$) is also
a symmetry generator. Because $Q$ and $Q^\dagger$ are fermionic operators,
they carry spin angular momentum 1/2, so
it is clear that supersymmetry must be a spacetime symmetry.
The possible forms for such symmetries in
an interacting quantum field theory are highly
restricted by the Haag-Lopuszanski-Sohnius extension of the
Coleman-Mandula theorem.\cite{HLS}
For realistic theories which, like the
Standard Model, have chiral fermions (i.e., fermions whose left-
and right-handed pieces transform differently under the gauge group)
and thus the possibility of parity-violating interactions,
this theorem implies that the
generators $Q$ and $Q^\dagger$ must satisfy an
algebra of anticommutation and commutation relations with
the schematic form
\beq
&&\{ Q, Q^\dagger \} = P^\mu \label{susyalgone}
\\
&&\{ Q,Q \} = \{ Q^\dagger , Q^\dagger \} = 0 \label{susyalgtwo}
\\
&&[ P^\mu , Q  ] = [P^\mu, Q^\dagger ] = 0 \label{susyalgthree}
\eeq
where $P^\mu$ is the momentum generator of spacetime translations.
Here we have ruthlessly suppressed the spinor indices on $Q$ and
$Q^\dagger$; after developing some notation we will,
in section \ref{subsec:susylagr.freeWZ}, derive the precise version of
eqs.~(\ref{susyalgone})-(\ref{susyalgthree}) with indices restored.
In the meantime, we simply note that the appearance of $P^\mu$
on the right-hand side of eq.~(\ref{susyalgone})
is unsurprising, since it transforms
under Lorentz boosts and rotations as a spin-1 object while $Q$
and $Q^\dagger$ on the left-hand side each transform as spin-1/2 objects.

The single-particle
states of a supersymmetric theory
fall naturally into irreducible representations of the supersymmetry
algebra which are called {\it supermultiplets}.
Each supermultiplet contains both fermion and boson states, which
are commonly known as {\it superpartners} of each other.
By definition, if $|\Omega\rangle$ and
$|\Omega^\prime \rangle$ are members of the same supermultiplet,
then $|\Omega^\prime\rangle$
is proportional to some combination of $Q$ and $Q^\dagger$ operators
acting on $|\Omega\rangle $, up to a
spacetime translation or rotation. The
(mass)$^2$ operator $-P^2$ commutes with
the operators $Q$, $Q^\dagger$,  and with all spacetime rotation and
translation operators,
so it follows immediately that particles which inhabit the same
irreducible supermultiplet must have equal eigenvalues of
$-P^2$, and therefore equal masses.

The supersymmetry generators $Q,Q^\dagger$ also commute with the
generators of gauge transformations.
Therefore particles in the same supermultiplet must also be in the same
representation of the gauge group, and so must have the same
electric charges, weak isospin, and color degrees of freedom.

Each supermultiplet contains an equal number of fermion and boson degrees
of freedom. To prove this, consider the operator
$(-1)^{2s}$
where $s$ is the spin angular momentum.
By the spin-statistics theorem, this operator
has eigenvalue $+1$ acting on a bosonic state and eigenvalue $-1$
acting on a fermionic state.
Any fermionic operator will turn a bosonic state into a
fermionic state and vice versa.
Therefore $(-1)^{2s}$ must anticommute
with every fermionic operator in the theory, and in particular
with $Q$ and $Q^\dagger$. Now consider the subspace of states
$| i \rangle$
in a
supermultiplet
which have the same eigenvalue $p^\mu$ of the four-momentum operator
$P^\mu$.
In view of eq.~(\ref{susyalgthree}), any combination of $Q$ or
$Q^\dagger$ acting on $|i\rangle$ will give another state $|i^\prime
\rangle$ which has the same four-momentum eigenvalue. Therefore
one has a completeness relation $\sum_i |i\rangle\langle i | = 1$
within this subspace of states.
Now one can take a trace over all such states
of the operator
$(-1)^{2s} P^\mu$
(including each spin helicity state separately):
\beq
\sum_i \langle i | (-1)^{2s} P^\mu | i \rangle
&=&
\sum_i \langle i | (-1)^{2s} Q Q^\dagger|i\rangle
+\sum_i\langle i | (-1)^{2s} Q^\dagger Q | i \rangle
\nonumber\\
&=&
\sum_i \langle i | (-1)^{2s} Q Q^\dagger | i \rangle
+ \sum_i \sum_j \langle i | (-1)^{2s} Q^\dagger |j \rangle \langle j | Q
| i \rangle\qquad{}
\nonumber\\
&=&
\sum_i \langle i | (-1)^{2s} Q Q^\dagger | i \rangle +
\sum_j \langle j | Q (-1)^{2s}  Q^\dagger | j \rangle
\nonumber\\
&=&\sum_i \langle i | (-1)^{2s} Q Q^\dagger | i \rangle -
\sum_j \langle j |  (-1)^{2s} Q Q^\dagger | j \rangle
\nonumber \\
&=& 0.
\eeq
The first equality follows from the supersymmetry algebra relation
eq.~(\ref{susyalgone}); the second and third from use of the completeness
relation;
and the fourth from the fact that $(-1)^{2s}$ must
anticommute with $Q$. Now $\sum_i \langle i | (-1)^{2s} P^\mu | i
\rangle = \, p^\mu$ Tr[$(-1)^{2s}$] is just proportional to
the number of bosonic degrees of freedom $n_B$
minus the number of fermionic degrees of freedom $n_F$
in the trace,
so that
\beq
n_B= n_F
\label{nbnf}
\eeq
must hold for a given $p^\mu\not= 0$ in each supermultiplet.

The simplest possibility for a supermultiplet
%% toggle ~ in the following line
which is consistent with eq.~(\ref{nbnf}) has a single
Weyl fermion (with two helicity states, so $n_F=2$) and two
real scalars (each with $n_B=1$). It is natural to assemble the two real
scalar degrees of freedom into a complex scalar field; as we will
see below this provides for convenient formulation of the supersymmetry
algebra, Feynman rules, supersymmetry violating effects, etc.
This combination of a two-component Weyl fermion and a complex
scalar field is called a {\it chiral} or {\it matter} or {\it scalar}
supermultiplet.

The next simplest possibility for a supermultiplet
contains a spin-1 vector boson.
If the theory is to be renormalizable this
must be a gauge boson which is massless, at least before the
gauge symmetry is spontaneously broken. A massless spin-1
boson has two helicity states, so 
the number of bosonic degrees of freedom is $n_B=2$.
Its superpartner is therefore a massless
spin-1/2 Weyl fermion, again with two
helicity states, so $n_F=2$.
(If one tried instead to use a massless spin-3/2 fermion, the theory would
not be renormalizable.)
Gauge bosons must transform as the adjoint
representation of the gauge group, so their fermionic partners, called
{\it gauginos},
must also.
Since the adjoint representation of a gauge group is always its
own conjugate, this means in particular that these fermions must have
the same gauge transformation properties for left-handed and
for right-handed components.
Such a combination of spin-1/2 gauginos and spin-1 gauge bosons
is called a {\it gauge} or {\it vector} supermultiplet.

There are other possible combinations of particles with spins
which can satisfy eq.~(\ref{nbnf}). However, these are always
reducible to combinations of chiral and gauge supermultiplets if they have
renormalizable interactions, except in
certain theories
with ``extended" supersymmetry. Theories with extended supersymmetry 
have more than one distinct copy of the supersymmetry generators
$Q,Q^\dagger$.
Such theories are mathematically amusing, but evidently do not have
any phenomenological prospects. The reason is that extended
supersymmetry in four-dimensional field theories cannot allow for chiral
fermions or
parity violation as observed in the Standard Model. So we will not discuss
such possibilities further, although extended supersymmetry in higher
dimensional field theories might describe the real world if the
extra dimensions are compactified, and extended supersymmetry in four
dimensions provides interesting toy models. The ordinary, non-extended,
phenomenologically-viable
type of supersymmetric model is sometimes called $N=1$ supersymmetry, with
$N$ referring to the number of supersymmetries (the number of distinct
copies of $Q, Q^\dagger$).

In a supersymmetric extension of the Standard
Model,\cite{FayetHsnu,FayetMSSM,Rparity}
each of the
known fundamental particles must therefore be in either
a chiral or gauge
supermultiplet and have a superpartner with spin
differing by 1/2 unit.
The first step in understanding the exciting
phenomenological consequences of this prediction is to decide how the
known particles fit into supermultiplets, and to give them appropriate
names. A crucial observation here is that only chiral
supermultiplets can contain fermions whose left-handed parts transform
differently under the gauge group than their right-handed parts.
All of the Standard Model fermions
(the known quarks and leptons) have this property, so they must be
members of chiral supermultiplets.\footnote{In particular, one cannot
attempt to make a spin-1/2 neutrino be the superpartner of the spin-1
photon; the neutrino is in a doublet, and the photon
neutral, under weak isospin.}
The names for the spin-0 partners of the quarks and leptons are
constructed by prepending an ``s", which is short
for scalar. Thus generically they are called
{\it squarks} and {\it sleptons}
(short for ``scalar quark" and ``scalar lepton").
The left-handed and right-handed pieces of the quarks and leptons
are separate two-component Weyl fermions with different gauge
transformation properties in the
Standard Model, so each must have its own complex scalar partner.
The symbols for the
squarks and sleptons are the same as for the corresponding fermion,
but with a tilde used to denote the superpartner of a Standard Model
particle. For example, the superpartners of the left-handed and
right-handed parts of the electron Dirac field
are called left- and right-handed selectrons,
and are denoted $\stilde e_L$ and $\stilde e_R$.
It is important to keep in mind that the ``handedness" here
does not refer to the helicity of the selectrons (they are
spin-0 particles) but to that of their superpartners.
A similar nomenclature
applies for smuons and staus: $\stilde \mu_L$, $\stilde\mu_R$,
$\stilde\tau_L$, $\stilde \tau_R$. In the Standard Model
the neutrinos are always left-handed, so the sneutrinos are denoted
generically by $\stilde\nu$, with a possible subscript indicating which
lepton flavor they carry: $\stilde\nu_e$, $\stilde\nu_\mu$,
$\stilde\nu_\tau$.
Finally, a complete list of the squarks is
$\stilde q_L$, $\stilde q_R$ with $q=u,d,s,c,b,t$.
The gauge interactions of each of these squark and slepton field
are the same as for the corresponding Standard Model fermion;
for instance, a left-handed squark like $\stilde u_L$ will couple to    
the $W$ boson while $\stilde u_R$ will not.

It seems clear that the Higgs scalar boson must reside in a chiral
supermultiplet, since it has spin 0.
Actually, it turns out that one chiral supermultiplet
is not enough. One way to see this is to note that
if there were only one Higgs chiral supermultiplet,
the electroweak gauge symmetry would suffer a triangle gauge anomaly,
and would be inconsistent as a quantum theory.
This is because the conditions for cancellation of gauge
anomalies include
$
{\rm Tr}[Y^3] = {\rm Tr}[T_3^2 Y] = 0,
$
where $T_3$ and $Y$ are the third component of weak isospin and the
weak hypercharge, respectively, in a normalization where the
ordinary electric charge is $Q_{\rm EM} = T_3 + Y$. The traces
run over all of the left-handed
Weyl fermionic degrees of freedom in the theory. In the Standard Model,
these conditions are already satisfied, somewhat miraculously,
by the known quarks and leptons. Now, a fermionic partner of
a Higgs chiral supermultiplet must be a weak isodoublet
with weak hypercharge $Y=1/2$ or $Y=-1/2$. In either case alone,
such a fermion will make a non-zero contribution to the traces and spoil
the anomaly cancellation.
This can be avoided if there are two Higgs supermultiplets,
one with each of $Y=\pm 1/2$. In that case
the total contribution to the anomaly traces from the two
fermionic members of the Higgs chiral supermultiplets will vanish.
As we will see in section \ref{subsec:mssm.superpotential}, both of these
are also necessary for another completely different reason:
because of the structure of supersymmetric theories,
only a $Y=+1/2$ Higgs chiral
supermultiplet can have the Yukawa couplings necessary to give
masses to charge $+2/3$ up-type quarks (up, charm, top),
and only a $Y=-1/2$ Higgs can
have the Yukawa couplings necessary to give masses to
charge $-1/3$ down-type quarks (down, strange, bottom) and to
charged leptons.
We will call the $SU(2)_L$-doublet complex scalar fields corresponding to
these two cases $H_u$ and $H_d$
respectively.\footnote{Other notations which are popular in the
literature have $H_d,H_u \rightarrow H_1, H_2$ or $H,\sbar H$.
The one used here has the virtue of making it easy to remember which Higgs
is responsible for giving masses to which quarks.}
The weak isospin components of $H_u$ with $T_3=(+1/2$, $-1/2$) have
electric charges $1$, $0$ respectively, and are denoted ($H_u^+$,
$H_u^0$). Similarly, the $SU(2)_L$-doublet complex scalar $H_d$
has $T_3=(+1/2$, $-1/2$) components ($H_d^0$, $H_d^-$). The neutral
scalar that corresponds to the physical Standard Model
Higgs boson is in a linear combination of $H_u^0$ and $H_d^0$;
we will discuss this further in section \ref{subsec:MSSMspectrum.Higgs}.
The generic nomenclature
for a spin-1/2 superpartner is to append ``-ino" to the name of
the Standard Model particle, so the fermionic partners of the
Higgs scalars are called higgsinos. They are denoted by
$\stilde H_u$, $\stilde H_d$ for the $SU(2)_L$-doublet left-handed
Weyl spinor fields, with
weak isospin components $\stilde H_u^+$, $\stilde H_u^0$ and
$\stilde H_d^0$, $\stilde H_d^-$.

\renewcommand{\arraystretch}{1.4}
\begin{table}[tb]
\caption{
Chiral supermultiplets in
the Minimal Supersymmetric Standard Model.\label{tab:chiral}}
\vspace{0.4cm}
\begin{center}
\begin{tabular}{|c|c|c|c|c|}
\hline
\mco{2}{|c|}{Names} & spin 0 & spin 1/2 & $SU(3)_C
,\,
SU(2)_L
,\,
U(1)_Y$\\
\hline\hline
squarks, quarks & $Q$ & $({\stilde u}_L\>\>\>{\stilde d}_L )$&
 $(u_L\>\>\>d_L)$ & $(\>{\bf 3},\>{\bf 2}\>,\>{1\over 6})$
\\
($\times 3$ families) & $\sbar u$
&${\stilde u}^*_R$ & $u^\dagger_R$ & $(\>{\bf \overline 3},\>
{\bf 1},
\>-{2\over 3})$
\\ & $\sbar d$
&${\stilde d}^*_R$ & $d^\dagger_R$ & $(\>{\bf \overline 3},\> {\bf 1},\>
{1\over 3})$
\\
\hline
sleptons, leptons & $L$ &$({\stilde \nu}\>\>{\stilde e}_L )$&
 $(\nu\>\>\>e_L)$ & $(\>{\bf 1},\>{\bf 2}\>,\>-{1\over 2})$
\\
($\times 3$ families) & $\sbar e$
&${\stilde e}^*_R$ & $e^\dagger_R$ & $(\>{\bf 1},\> {\bf 1},\>1)$\\
\hline
Higgs, higgsinos &$H_u$ &$(H_u^+\>\>\>H_u^0 )$&
$(\stilde H_u^+ \>\>\>
\stilde H_u^0)$& $(\>{\bf 1},\>{\bf 2}\>,\>+{1\over
2})$
\\ &$H_d$
& $(H_d^0 \>\>\> H_d^-)$ & $(\stilde H_d^0 \>\>\> \stilde H_d^-)$& $(\>{\bf
1},\>{\bf
2}\>,\>-{1\over 2})$
\\
\hline
\end{tabular}
\end{center}
\end{table}
We have now found all of the chiral supermultiplets of a minimal
phenomenologically viable extension of the Standard Model.
They are summarized in Table 1,
classified according to their transformation properties under
the Standard Model gauge group
$SU(3)_C\times SU(2)_L \times U(1)_Y$, which combines $u_L,d_L$ and
$\nu,e_L$ degrees of freedom into $SU(2)_L$ doublets.
Here we have followed the standard convention that all chiral
supermultiplets are defined in terms of left-handed Weyl spinors,
so that the {\it conjugates} of the
right-handed quarks and leptons (and their
superpartners) appear in Table 1. This protocol for defining chiral
supermultiplets turns out to be very useful for constructing
supersymmetric lagrangians, as we will see in section \ref{sec:susylagr}.
It is useful also to have a symbol for each of the chiral supermultiplets
as a whole; these are indicated in the second column of Table 1. Thus
for example $Q$ stands for the $SU(2)_L$-doublet chiral supermultiplet
containing $\stilde u_L,u_L$ (with weak isospin component $T_3=+1/2$),
and
$\stilde d_L, d_L$ (with $T_3=-1/2$), while $\sbar u$ stands
for the $SU(2)_L$-singlet
supermultiplet containing $\stilde u_R^*,  u_R^\dagger$. There are
three families for each of the quark and lepton supermultiplets, but
we have used first-family representatives in Table 1. Below,
a family index $i=1,2,3$ will be affixed to the chiral supermultiplet names
($Q_i$, $\sbar u_i, \ldots$) when needed, e.g.~$(\sbar e_1, \sbar e_2,
\sbar e_3)= (\sbar e, \sbar \mu, \sbar \tau)$. The bar
on $\sbar u$, $\sbar d$, $\sbar e$ fields is  part of the name, and does
not denote
any kind of conjugation. 

It is interesting to
note that the Higgs chiral supermultiplet $H_d$
(containing $H_d^0$, $H_d^-$, $\stilde H_d^0$, $\stilde H_d^-$)
has exactly the
same Standard Model
gauge quantum numbers as the left-handed sleptons and
leptons $L_i$, e.g.~($\stilde \nu$, $\stilde e_L$, $\nu$, $e_L$).
Naively
one might
therefore suppose that we could have been more economical in our
assignment by taking a neutrino and a Higgs scalar to be superpartners,
instead of putting them in separate supermultiplets.
This would amount to the proposal that the Higgs boson and a sneutrino
should be the same particle. This is a nice try which played a key role
in some of the first attempts to connect supersymmetry to
phenomenology,\cite{FayetHsnu}
but it is now known not to work.
Even ignoring the anomaly cancellation problem mentioned above,
many insoluble phenomenological
problems would result,
including lepton number violation
and a mass for at least one of the neutrinos in gross violation of
experimental bounds. Therefore, all of the superpartners of Standard
Model particles are really new particles, and cannot be identified
with some other Standard Model state.

\renewcommand{\arraystretch}{1.55}
\begin{table}[t]
\caption{
Gauge supermultiplets in
the Minimal Supersymmetric Standard Model.\label{tab:gauge}}
\vspace{0.4cm}
\begin{center}
\begin{tabular}{|c|c|c|c|}
\hline
Names & spin 1/2 & spin 1 & $SU(3)_C, \> SU(2)_L,\> U(1)_Y$\\
\hline\hline
gluino, gluon &$ \stilde g$& $g$ & $(\>{\bf 8},\>{\bf 1}\>,\> 0)$
\\
\hline
winos, W bosons & $ \stilde W^\pm\>\>\> \stilde W^0 $&
 $W^\pm\>\>\> W^0$ & $(\>{\bf 1},\>{\bf 3}\>,\> 0)$
\\
\hline
bino, B boson &$\stilde B^0$&
 $B^0$ & $(\>{\bf 1},\>{\bf 1}\>,\> 0)$
\\
\hline
\end{tabular}
\end{center}
\end{table}
The vector bosons of the Standard Model clearly must reside
in gauge supermultiplets. Their fermionic superpartners are
generically referred to as gauginos.
The $SU(3)_C$ color gauge interactions of
QCD are mediated by the gluon, whose spin-1/2
color-octet supersymmetric partner is the gluino. As usual,
a tilde is used to denote the supersymmetric partner of
a Standard Model state, so the symbols for the gluon and
gluino are $g$ and $\stilde g$ respectively. The electroweak
gauge symmetry $SU(2)_L\times U(1)_Y$
has associated with it spin-1
gauge bosons $W^+, W^0, W^-$ and $B^0$, with spin-1/2 superpartners
$\stilde W^+, \stilde W^0, \stilde W^-$ and $\stilde B^0$,
called {\it winos} and {\it bino}.
After electroweak symmetry breaking, the $W^0$, $B^0$ gauge
eigenstates mix to give mass eigenstates $Z^0$ and $\gamma$.
The corresponding gaugino mixtures of $\stilde W^0$ and $\stilde B^0$
are called zino ($\stilde Z^0$) and photino ($\stilde \gamma$); if
supersymmetry were unbroken, they would be
mass eigenstates with masses $m_Z$ and 0.
Table 2 summarizes the gauge supermultiplets of a minimal supersymmetric
extension of the Standard Model.

The chiral and gauge supermultiplets in Tables 1 and 2 make up the
particle content of the Minimal Supersymmetric Standard Model
(MSSM).
The most obvious and interesting feature of this theory is that
none of the superpartners of the Standard Model particles has been
discovered as of this writing.
If supersymmetry were unbroken, then there would have to be
selectrons $\stilde e_L$ and $\stilde e_R$ with masses exactly equal to
$m_e = 0.511...$ MeV. A similar statement applies to each of the other
sleptons and squarks, and there would also have to be a massless gluino and
photino. These particles would have been extraordinarily easy to detect
long ago.
Clearly, therefore, {\it supersymmetry is a broken symmetry} in the
vacuum state chosen by nature.

A very important clue as to the nature of supersymmetry breaking
can be obtained by returning to the motivation provided by
the hierarchy problem.
Supersymmetry forced us to introduce two complex scalar fields for
each Standard Model Dirac fermion, which is just what is needed to enable
a cancellation of the quadratically divergent $(\Lambda_{\rm UV}^2)$
pieces of eqs.~(\ref{quaddiv1}) and (\ref{quaddiv2}). This sort of
cancellation also requires that the associated dimensionless couplings
should be related (e.g.~$\lambda_S = |\lambda_f|^2$). The
necessary
relationships between couplings indeed occur
in unbroken supersymmetry, as we will see in section \ref{sec:susylagr}.
In fact, unbroken supersymmetry guarantees that the quadratic divergences
in
scalar squared masses must vanish to all orders in perturbation
theory.\footnote{A simple way to understand this is to note
that unbroken supersymmetry
requires
the degeneracy of scalar and fermion masses. Radiative corrections to
fermion masses are known to diverge at most logarithmically, so the same
must be true for scalar masses in unbroken supersymmetry.}
Now, if broken supersymmetry
is still to provide a solution to the hierarchy problem,
then the relationships between dimensionless couplings which
hold in an unbroken supersymmetric theory must
be maintained.
Otherwise, there would be quadratically divergent radiative corrections
to the Higgs scalar masses of the form
\beq
\Delta m_H^2 = {1\over 8\pi^2} (\lambda_S - |\lambda_f|^2)
\Lambda_{\rm UV}^2 + \ldots .
\eeq
We are therefore led to consider ``soft" supersymmetry breaking.
This means that the effective
lagrangian of the MSSM can be written
in the form
\beq
\lagr = \lagr_{\rm SUSY} + \lagr_{\rm soft},
\eeq
where $\lagr_{\rm SUSY}$ preserves supersymmetry invariance,
and $\lagr_{\rm soft}$ violates supersymmetry but contains only mass terms
and couplings with {\it positive} mass dimension.
Without further justification, soft supersymmetry breaking might
seem like a rather arbitrary requirement.
Fortunately, we will see in section \ref{sec:origins} that theoretical
models for
supersymmetry breaking
can indeed yield
effective lagrangians with
just such terms for $\lagr_{\rm soft}$.
If the largest mass scale
associated with the soft terms is
denoted $m_{\rm soft}$, then the additional
non-supersymmetric corrections to the Higgs scalar (mass)$^2$
must vanish in the $m_{\rm soft}\rightarrow 0$ limit, so by dimensional
analysis they cannot be proportional to $\Lambda_{\rm UV}^2$.
More generally, these models
maintain the cancellation of quadratically divergent terms
in the radiative corrections of all scalar masses, to all orders
in perturbation theory.
The corrections also cannot go like $\Delta m_H^2 \sim m_{\rm soft}
\Lambda_{\rm UV}$, because in general the loop momentum integrals
always diverge either quadratically or logarithmically, not linearly,
as $\Lambda_{\rm UV} \rightarrow \infty$. So
they must be of the form
\beq
\Delta m_{H}^2 =
m_{\rm soft}^2
\left [{\lambda\over 16 \pi^2}\> {\rm ln}(\Lambda_{\rm UV}/m_{\rm soft})
+ \ldots \right ].
\label{softy}
\eeq
Here $\lambda$ is schematic for various 
dimensionless couplings,
and the ellipses stand both for
terms which are independent of $\Lambda_{\rm UV}$ and for
higher loop corrections (which
depend on $\Lambda_{\rm UV}$ through powers of logarithms).

Since the mass splittings between the known Standard Model particles
and their
superpartners
are just determined by the parameters $m_{\rm soft}$ appearing
in $\lagr_{\rm soft}$, eq.~(\ref{softy})
tells us that the superpartner masses cannot be too huge.
Otherwise, we would lose our successful cure for the hierarchy
problem since the $m_{\rm soft}^2$ corrections to the Higgs
scalar (mass)$^2$ would be
unnaturally large compared to the electroweak breaking scale of 174 GeV.
The top and bottom squarks and the winos and bino give especially
large contributions to $\Delta m_{H_u}^2$ and $\Delta m_{H_d}^2$,
but the gluino mass and all the other squark and slepton masses also feed
in indirectly, through radiative corrections
to the top and bottom squark masses.
Furthermore, in most viable models of supersymmetry breaking that
are not unduly contrived, the superpartner masses do not
differ from each other by more than about an order of magnitude.
Using $\Lambda_{\rm UV} \sim \MPlanck$ and $\lambda \sim 1$ in
eq.~(\ref{softy}),
one finds that roughly speaking $m_{\rm soft}$, and therefore
the masses of at least the lightest few
superpartners, should be at the most about 1 TeV or so,
in order for the MSSM scalar potential to provide a Higgs
VEV resulting in $m_W,m_Z$ = 80.4, 91.2 GeV
without miraculous cancellations.
This is the best reason for the optimism
among many theorists that supersymmetry will be discovered at
LEP2, the Tevatron, the
LHC, or a next generation lepton linear
collider.

However, it is useful to keep in mind that
the hierarchy problem was {\it not} the historical motivation for
the development of supersymmetry in the early 1970's. The supersymmetry
algebra and supersymmetric field theories were originally concocted
independently in various disguises
\cite{RNS,Golfand,WessZumino,Volkov}
which bear little resemblance to the MSSM. It is quite impressive that
a theory which was developed for quite different reasons, including
purely aesthetic ones, can later be found to
provide a solution for the hierarchy problem.

One might also wonder if there is any good reason why all of
the superpartners of the Standard Model particles should be
heavy enough to have avoided discovery so far. There is.
All of the particles in the MSSM which have been discovered so far have
something in common; they would necessarily be massless in the absence of
electroweak symmetry breaking. In particular, the masses of the
$W^\pm, Z^0$ bosons and all quarks and leptons are equal to
dimensionless coupling constants times the Higgs VEV $\sim 174 $ GeV,
while the photon and gluon are required to be massless by electromagnetic
and QCD gauge invariance. Conversely, all of the undiscovered particles
in the MSSM have exactly the opposite property, since each of them can
have a lagrangian mass term in the absence of
electroweak symmetry breaking.
For the squarks, sleptons, and Higgs scalars this follows from a general
property of complex scalar fields that a mass term $m^2 |\phi|^2$
is always allowed by all gauge symmetries. For the higgsinos and gauginos,
it follows from the fact that they are fermions in a real representation
of the
gauge group. So, from the point of view of the MSSM, the discovery
of the top quark in 1995 marked a quite natural milestone; the
already-discovered particles are precisely those which had to
be light, based on the principle of electroweak gauge symmetry.
There is a single exception: one neutral Higgs scalar boson
should be lighter than about 150 GeV if supersymmetry
is correct, for reasons to be discussed in section
\ref{subsec:MSSMspectrum.Higgs}.

A very important feature of the MSSM is that the
superpartners listed in Tables 1 and 2 are not necessarily the
mass eigenstates of the theory. This is because after electroweak
symmetry breaking and supersymmetry breaking effects are included,
there can be mixing between the electroweak gauginos and the higgsinos,
and within the various sets of squarks and sleptons and Higgs
scalars which have the same electric
charge. The lone exception is the gluino, which is a color octet fermion
and therefore does
not have the appropriate quantum numbers to mix with any other particle.
The masses and mixings of the superpartners are obviously of paramount
importance to experimentalists. It is perhaps slightly less obvious that
these phenomenological issues are all quite directly related to one
central question which is also
the focus of much of the theoretical work in supersymmetry:
``How is supersymmetry broken?" The reason for this is that most
of what we do not already know about the MSSM has to do with
$\lagr_{\rm soft}$. The structure of supersymmetric lagrangians
allows very little arbitrariness, as we will see in section
\ref{sec:susylagr}. In fact, all of the dimensionless couplings
and all but one mass term in the supersymmetric part of the
MSSM lagrangian correspond directly to some parameter in the ordinary
Standard Model which has already been measured by
experiment. For example, we will find out that
the supersymmetric coupling of a gluino to
a squark and a quark is determined by
the QCD coupling constant $\alpha_S$. In contrast, the
supersymmetry-breaking part of the lagrangian apparently contains
many unknown parameters and a considerable amount of arbitrariness.
Each of the mass splittings
between Standard Model particles and their superpartners correspond
to terms in the MSSM lagrangian which are purely supersymmetry-breaking
in their origin and effect.
These soft supersymmetry-breaking terms can also introduce a
large number of mixing angles and
CP-violating phases not found in the Standard Model.
Fortunately, as we will see in section \ref{subsec:mssm.hints}, there is
already
rather strong evidence that the supersymmetry-breaking terms
in the MSSM are actually not arbitrary at all.
Furthermore, the additional parameters will be measured
and constrained as the superpartners are detected. From a theoretical
perspective, the challenge is to explain all of these parameters with
a model for supersymmetry breaking.

The rest of our discussion is organized as follows.
Section \ref{sec:notations} provides a list of important notations.
In section \ref{sec:susylagr}, we will learn how to
construct lagrangians for
supersymmetric field theories. Soft supersymmetry-breaking couplings
are described in section \ref{sec:soft}. In section \ref{sec:mssm},
we will apply the preceding general results to the special case
of the MSSM, introduce the concept of $R$-parity, and emphasize the
importance of the structure of the soft terms. Section
\ref{sec:origins} outlines some considerations for understanding
the origin of supersymmetry breaking, and the consequences of various
proposals. In section \ref{sec:MSSMspectrum}, we will study the mass and
mixing angle patterns of the new particles predicted by the MSSM.
Their decay modes are considered in section \ref{sec:decays},
and some of the qualitative features of experimental signals
for supersymmetry are reviewed in section \ref{sec:signals}. Section
\ref{sec:variations} describes some sample variations on the standard MSSM
picture. The discussion
will be lacking in historical accuracy or
perspective, for which the author apologizes in advance. The
reader is encouraged to consult the many outstanding
textbooks,\cite{WessBaggerbook}$^{\!-\,}$\cite{Ramondbook}
review articles,\cite{HaberKanereview}$^{\!-\,}$\cite{Tatareview}
and the reprint volume,\cite{reprints}
which contain a much more consistent guide to the original literature.

\section{Interlude: Notations and Conventions}\label{sec:notations}
\setcounter{equation}{0}
\setcounter{footnote}{1}

Before proceeding to discuss the construction of supersymmetric
lagrangians, we need to specify our notations. It is
overwhelmingly convenient to employ two-component Weyl notation
for fermions, rather than four component Dirac or Majorana spinors.
The lagrangian of the Standard Model (and supersymmetric extensions
of it) violates parity; each Dirac fermion has
left-handed and right-handed parts with
completely different electroweak gauge interactions.
If one used four-component notation, one would therefore
have to include clumsy left- and right-handed projection operators
\beq
P_{L,R} = (1\pm \gamma_5)/2
\eeq
all over the place. The two-component Weyl fermion notation has the
advantage of treating fermionic degrees of freedom with
different gauge quantum numbers separately from the start (as Nature
intended for us to do).
But an even better reason for using two-component notation
here is that in supersymmetric models the minimal building blocks of
matter are chiral supermultiplets, each of which
contains a single two-component Weyl fermion.

Since two-component fermion notation may be unfamiliar to some readers, we
will specify our conventions \footnote{The conventions used here are
the same as in Ref.\cite{WessBaggerbook},
except that we use a dagger rather than a
bar to indicate hermitian conjugation for Weyl spinors.}
by showing how they correspond to the four-component fermion language.
A four-component
Dirac fermion $\Psi_{\sst D}$ with mass $M$ is described by the lagrangian
\beq
\lagr_{\rm Dirac}
= -i \overline\Psi_{\sst D} \gamma^\mu \partial_\mu \Psi_{\sst D}
-  M \overline \Psi_{\sst D}
\Psi_{\sst D}\> .
\label{diraclag} \eeq
We use a spacetime metric $\eta_{\mu\nu} = $ diag($-1,1,1,1$). For our
purposes it is convenient to use the specific representation of
the 4$\times$4 gamma matrices given in $2\times$2 blocks by
\beq
\gamma_\mu = \pmatrix{ 0 & \sigma_\mu \cr
                       \sigmabar_\mu & 0\cr};\qquad
\qquad\gamma_5 = \pmatrix{1 & 0\cr 0 & -1\cr},
\eeq
where
\bea
&&\sigma_0 = \sigmabar_0 = \pmatrix{1&0\cr 0&1\cr};\qquad
\>\>\>\>\>\>\sigma_1 = -\sigmabar_1 = \pmatrix{0&1\cr 1&0\cr}; \nonumber\\
&&\sigma_2 = -\sigmabar_2 = \pmatrix{ 0&-i\cr i&0\cr};\qquad
\>\>\>\sigma_3 = -\sigmabar_3 = \pmatrix{1&0\cr 0&-1\cr}
\> .
\label{pauli}
\eea
In this basis, a four component Dirac spinor is written
in terms of
2 two-component, complex, anticommuting objects $(\xi)_\alpha$
with $\alpha=1,2$
and $(\chi^\dagger)^{\dot{\alpha}}$ with $\dot{\alpha}=1,2$:
\beq
\Psi_{\sst D} =
\pmatrix{\xi_\alpha\cr {\chi^{\dagger\dot{\alpha}}}\cr};
\qquad\qquad
\overline\Psi_{\sst D}  =
\pmatrix{\chi^\alpha &
                           \xi^\dagger_{\dot{\alpha}}\cr }
\> .
\label{psid}
\eeq
The undotted (dotted) indices are used for the first (last) two
components of a Dirac spinor.
The heights of these indices are important;
for example, comparing eqs.~(\ref{diraclag})-(\ref{psid}), we observe
that the matrices
$(\sigma^\mu)_{\alpha\dot{\alpha}}$ and $(\sigmabar^\mu)^{\dot{\alpha}\alpha}$
defined by eq.~(\ref{pauli})
carry indices with the heights as indicated.
The spinor indices are raised and lowered
using the antisymmetric symbol $\epsilon^{12} = -\epsilon^{21} =
\epsilon_{21} = -\epsilon_{12} = 1$; $\epsilon_{11} = \epsilon_{22} =
\epsilon^{11} = \epsilon^{22} = 0$, according to
\beq
\xi_\alpha = \epsilon_{\alpha\beta}
\xi^\beta;\qquad\qquad\!\!\!\!\!\!\!\!\!
\xi^\alpha = \epsilon^{\alpha\beta}
\xi_\beta;\qquad\qquad\!\!\!\!\!\!\!\!\!
\chi^\dagger_{\dot{\alpha}} = \epsilon_{\dot{\alpha}\dot{\beta}}
\chi^{\dagger\dot{\beta}};\qquad\qquad\!\!\!\!\!\!\!\!\!
\chi^{\dagger\dot{\alpha}} = \epsilon^{\dot{\alpha}\dot{\beta}}
\chi^\dagger_{\dot{\beta}}\>.\qquad{}
\eeq
This is consistent since
$\epsilon_{\alpha\beta} \epsilon^{\beta\gamma} =
\epsilon^{\gamma\beta}\epsilon_{\beta\alpha} = \delta_\alpha^\gamma$
and
$\epsilon_{\dot{\alpha}\dot{\beta}} \epsilon^{\dot{\beta}\dot{\gamma}} =
\epsilon^{\dot{\gamma}\dot{\beta}}\epsilon_{\dot{\beta}\dot{\alpha}} =
\delta_{\dot{\alpha}}^{\dot{\gamma}}$.
The field $\xi$ is called a ``left-handed Weyl spinor" and
$\chi^\dagger$ is a ``right-handed Weyl spinor". The names fit, because
\beq
P_L \Psi_{\sst D} = \pmatrix{\xi_\alpha \cr 0\cr};\qquad\qquad
P_R \Psi_{\sst D} = \pmatrix{0\cr \chi^{\dagger\dot{\alpha}}\cr}
\> .
\eeq
The hermitian conjugate of a left-handed Weyl spinor is a right-handed
Weyl spinor $(\psi_\alpha)^\dagger = (\psi^\dagger)_{\dot{\alpha}}$
and vice versa $( \psi^{\dagger\dot{\alpha}} )^\dagger =
\psi^\alpha$. Therefore any particular fermionic degrees of freedom
can be described equally well using a Weyl spinor which is left-handed
(with an undotted index)
or by one which is right-handed (with a dotted index).
By convention, all names
of fermion fields are chosen so that left-handed Weyl spinors do not
carry daggers and right-handed Weyl spinors do carry daggers, as in
eq.~(\ref{psid}).

It is useful to abbreviate expressions with two spinor fields by suppressing
undotted indices contracted like ${}^\alpha{}_\alpha$ and dotted indices
contracted like ${}_{\dot{\alpha}}{}^{\dot{\alpha}}$. In particular,
\beq
\xi\chi \equiv \xi^\alpha\chi_\alpha = \xi^\alpha \epsilon_{\alpha\beta}
\chi^\beta = -\chi^\beta \epsilon_{\alpha\beta} \xi^\alpha =
\chi^\beta \epsilon_{\beta\alpha} \xi^\alpha = \chi^\beta \xi_\beta \equiv
\chi\xi
\label{xichi}
\eeq
with, conveniently, no minus sign in the end.
[A minus sign appeared in eq.~(\ref{xichi}) from exchanging the order of
anticommuting
spinors, but it disappeared due to the antisymmetry of the $\epsilon$ symbol.]
Likewise,
$\xi^\dagger \chi^\dagger$ and $\chi^\dagger \xi^\dagger $
are equivalent abbreviations
for $\chi^\dagger_{\dot{\alpha}} \xi^{\dagger \dot{\alpha}} =
(\xi \chi)^*$, the complex conjugate of $\xi\chi$. In a similar way,
\beq
\xi^\dagger \sigmabar^\mu \chi = -\chi \sigma^\mu \xi^\dagger
=
(\chi^\dagger \sigmabar^\mu \xi)^* = -(\xi\sigma^\mu\chi^\dagger)^*
\label{yetanotheridentity}
\eeq
stands for $\xi^\dagger_{\dot{\alpha}}(\sigmabar^\mu)^{\dot{\alpha}\alpha}
\chi_\alpha$, etc.
With these conventions, the Dirac lagrangian
eq.~(\ref{diraclag}) can now be rewritten:
\beq
\lagr_{\rm Dirac}
&\! = \! & -i \overline\Psi_{\sst D} \gamma^\mu \partial_\mu
\Psi_{\sst D}
-  M \overline \Psi_{\sst D} \Psi_{\sst D}
\\
&\! = \! & -i \xi^\dagger \sigmabar^\mu \partial_\mu \xi -
i \chi^\dagger \sigmabar^\mu \partial_\mu \chi -
M(\xi\chi + \xi^\dagger \chi^\dagger)
\eeq
where we have dropped a total derivative piece
$i\partial_\mu(\chi^\dagger \sigmabar^\mu \chi)$
which does not affect the action.

A four-component Majorana spinor can be obtained from the
Dirac spinor of eq.~(\ref{psid}) by imposing the constraint
$\chi = \xi$, so that
\beq
\Psi_{\rm M} = \pmatrix{\xi_\alpha \cr \xi^{\dagger\dot{\alpha}}\cr};
\qquad\qquad
\overline\Psi_{\rm M}
= \pmatrix{
\xi^\alpha
& 
\xi^\dagger_{\dot{\alpha}}
\cr
}.
\eeq
The lagrangian for a Majorana fermion with mass $M$
\beq
\lagr_{\rm Majorana} =
-{i\over 2}\overline\Psi_{\rm M} \gamma^\mu \partial_\mu \Psi_{\rm M}
- {1\over 2} M \overline\Psi_{\rm M} \Psi_{\rm M}
\eeq
in the four-component Majorana spinor form can therefore be rewritten
\beq
\lagr_{\rm Majorana} = -i\xi^\dagger \sigmabar^\mu\partial_\mu \xi -
{1\over 2} M(\xi\xi +
\xi^\dagger\xi^\dagger)
\eeq
in the more economical two-component Weyl spinor representation.
[Note that even though $\xi_\alpha$ is anticommuting, $\xi\xi$ and its
complex conjugate $\xi^\dagger\xi^\dagger$ do not vanish,
because of the suppressed $\epsilon$ symbol, see eq.~(\ref{xichi}).]

More generally, any theory involving
spin-1/2 fermions can {\it always} be written
down in terms of a collection of left-handed Weyl spinors $\psi_i$
with
\beq
\lagr = - i \psi^{\dagger i} \sigmabar^\mu \partial_\mu\psi_i
+ \ldots
\eeq
where the ellipses represent possible
mass terms, gauge interactions, and Yukawa interactions
with scalar fields. Here the index $i$ runs over the appropriate gauge
and flavor indices of the fermions; it is raised or lowered by
hermitian
conjugation. There is a different $\psi_i$
for the left-handed piece and for the hermitian conjugate of the
right-handed piece of a Dirac fermion.
If one has any expression involving bilinears in four-component
spinors
\beq
\Psi_1 = \pmatrix{ \xi_1\cr\chi_1^\dagger\cr}\qquad{\rm and}\qquad
\Psi_2 = \pmatrix{ \xi_2\cr\chi_2^\dagger\cr},
\eeq
then one can translate into two-component Weyl spinor language
(or vice versa) using the dictionary:
\beq
&&\overline\Psi_1 P_L \Psi_2 = \chi_1\xi_2;\qquad\qquad\qquad
\overline\Psi_1 P_R \Psi_2 = \xi_1^\dagger \chi_2^\dagger;\qquad\>{}\\
&&\overline\Psi_1 \gamma^\mu P_L \Psi_2 = \xi_1^\dagger \sigmabar^\mu \xi_2
;\qquad\qquad
\overline\Psi_1 \gamma^\mu P_R \Psi_2 = \chi_1 \sigma^\mu \chi^\dagger_2
\qquad\>\>\>{}
\eeq
etc. We will introduce a few other Weyl spinor identities in the following
as they are needed.

Let us now see how the Standard Model quarks and leptons are described in
this notation.
The complete list of left-handed Weyl spinors can be given names
corresponding to the chiral supermultiplets in Table 1:
\beq
Q_i & = &
(u\> d),\>\, (c\> s),\>\, (t\> b)
\\
\sbar u_i & = &
\>\>\sbar u ,\>\sbar c,\> \sbar t
\qquad\qquad\qquad\>\>
\sbar d_i \> = \>
\>\>\sbar d ,\>\sbar s,\> \sbar b
\\
L_i & = &
(\nu_e\> e),\>\, (\nu_\mu\> \mu),\>\, (\nu_\tau\> \tau)\\
\sbar e_i & = &
\>\>\sbar e ,\>\sbar \mu,\> \sbar \tau    .
\eeq
Here $i=1,2,3$ is a family index. The bars on these fields are part of the
names of the fields, and do {\it not}
denote any kind of conjugation. Rather, the unbarred fields are
the left-handed pieces of a Dirac spinor, while the barred fields
are the names given to the conjugates of the right-handed piece
of a Dirac spinor. For example,
$e$ is the same thing as $e_L$ in Table 1, and $\sbar e$ is the same as
$e_R^\dagger$.
Together they form a Dirac spinor:
\beq
\pmatrix{e\cr {\sbar e}^\dagger} \equiv \pmatrix{e_L \cr e_R}
\label{espinor}
\eeq
with similar equations for all of the other quark and charged lepton
Dirac spinors. (The neutrinos of the Standard Model are not part
of a Dirac spinor.) The
fields $Q_i$ and $L_i$ are weak isodoublets which always go
together when one is constructing interactions invariant under
the full Standard Model gauge group $SU(3)_C\times SU(2)_L \times
U(1)_Y$. Suppressing all color and weak isospin indices, the
purely kinetic part of the Standard Model fermion lagrangian density
is then
\beq
\lagr &=&
-iQ^{\dagger i}\sigmabar^\mu\partial_\mu Q_i
-i\sbar u^{\dagger i}\sigmabar^\mu\partial_\mu \sbar u_i
-i\sbar d^{\dagger i}\sigmabar^\mu\partial_\mu \sbar d_i
%\nonumber
%\\
%&&
-iL^{\dagger i}\sigmabar^\mu\partial_\mu L_i
-i\sbar e^{\dagger i}\sigmabar^\mu\partial_\mu \sbar e_i
\qquad{}
\eeq
with the family index $i=1,2,3$ summed over.

\section{Supersymmetric lagrangians}\label{sec:susylagr}
\setcounter{equation}{0}
\setcounter{footnote}{1}

In this section we will describe the construction of supersymmetric
lagrangians. Our aim is to arrive at a sort of recipe which will
allow us to write down the allowed interactions and mass terms of
a general supersymmetric theory, so that later we can apply
the results to the special case of the MSSM.
We will not use the superfield language,\cite{superfields}
which is often more elegant and
efficient for those who know it, but which might seem rather
cabalistic to some readers.
Our approach is therefore intended to be rather complementary to the
superfield derivations given in
Refs.\cite{WessBaggerbook}$^{\!-\,}$\cite{Ramondbook}
We begin by considering the
simplest example of a supersymmetric theory in four dimensions.

\subsection{The simplest supersymmetric model: a free chiral
supermultiplet}\label{subsec:susylagr.freeWZ}

The minimum fermion content of any theory in four dimensions consists of
a single left-handed two-component Weyl fermion $\psi$.
Since this is an intrinsically complex object, it seems sensible to
choose as its superpartner a complex scalar field $\phi$.
The simplest action we can write down for these fields
just consists of kinetic energy terms for each:
\beq
&&S = \int d^4x\>
\left (\lagr_{\rm scalar} + \lagr_{\rm fermion}\right )
\label{Lwz} \\
&&\lagr_{\rm scalar} = - \partial^\mu \phi^* \partial_\mu \phi;
\qquad\qquad
\lagr_{\rm fermion} = - i \psi^\dagger \sigmabar^\mu \partial_\mu \psi .
\eeq
This is called the massless, non-interacting {\it Wess-Zumino
model},\cite{WessZumino}
and it corresponds to a single chiral supermultiplet as discussed in
the Introduction.

A supersymmetry transformation should turn the scalar boson
$\phi$ into something
involving the fermion $\psi_\alpha$. The simplest possibility for the
transformation of the scalar field is
\beq
\deltaeps \phi = \epsilon \psi;\qquad\qquad
\deltaeps \phi^* = \epsilon^\dagger \psi^\dagger
\label{phitrans}
\eeq
where $\epsilon^\alpha$ is an infinitesimal, anticommuting, two-component
Weyl fermion object which
parameterizes the supersymmetry transformation. Until section
\ref{subsec:origins.gravitino}, we will
be
discussing global supersymmetry, which means that $\epsilon^\alpha$
is a constant, satisfying $\partial_\mu \epsilon^\alpha=0$.
Since
$\psi$ has dimensions of
(mass)$^{3/2}$ and $\phi$ has dimensions of (mass), it must be that
$\epsilon$ has dimensions of (mass)$^{-1/2}$. Using
eq.~(\ref{phitrans}),
we find that the
scalar part of the lagrangian transforms as
\beq
\deltaeps
\lagr_{\rm scalar} = -\epsilon \partial^\mu \psi \> \partial_\mu \phi^*
-\epsilon^\dagger \partial^\mu \psi^\dagger \> \partial_\mu \phi .
\label{Lphitrans}
\eeq
We would like for this to be cancelled by
$\deltaeps\lagr_{\rm fermion}$, at least up to a total derivative,
so that the action will be invariant under the supersymmetry
transformation.
Comparing eq.~(\ref{Lphitrans}) with $\lagr_{\rm fermion}$,
we see that for this to have any chance of
happening, $\deltaeps \psi$ should be linear in $\epsilon^\dagger$
and in $\phi$ and contain one spacetime derivative.
Up to a multiplicative constant,
there is only one possibility to try:
\beq
\deltaeps\psi_\alpha
=
i (\sigma^\mu \epsilon^\dagger)_\alpha\> \partial_\mu \phi;
\qquad\qquad
\deltaeps\psi^\dagger_{\dot{\alpha}}
=
-i (\epsilon\sigma^\mu)_{\dot{\alpha}}\>   \partial_\mu \phi^* .
\label{psitrans}
\eeq
With this guess, one immediately obtains
\beq
\deltaeps \lagr_{\rm fermion} =
-\epsilon \sigma^\mu \sigmabar^\nu \partial_\nu \psi\> \partial_\mu \phi^*
+\psi^\dagger \sigmabar^\nu \sigma^\mu \epsilon^\dagger \>
\partial_\mu \partial_\nu \phi
\> .
\label{preLpsitrans}
\eeq
This can be put in a slightly more useful form by employing the
Pauli matrix identities
\beq
&&
\bigl[ \sigma^\mu \sigmabar^\nu + \sigma^\nu \sigmabar^\mu
\bigr ]_\alpha^\beta  =
-2 \eta^{\mu\nu} \delta_\alpha^\beta ;\qquad\>\>
\bigl[ \sigmabar^\mu \sigma^\nu + \sigmabar^\nu \sigma^\mu \bigr
]_{\dot{\alpha}}^{\dot{\beta}}=
-2 \eta^{\mu\nu} \delta_{\dot{\alpha}}^{\dot{\beta}}
\qquad{}\label{pauliident}
\eeq
and using the fact that partial derivatives commute $(\partial_\mu\partial_\nu
= \partial_\nu\partial_\mu)$. Equation (\ref{preLpsitrans})
then becomes
\beq
\deltaeps \lagr_{\rm fermion} & =
&\epsilon\partial^\mu\psi\> \partial_\mu\phi^*
+\epsilon^\dagger \partial^\mu\psi^\dagger\> \partial_\mu \phi
\nonumber\\
&& -\partial_\mu \left (
\epsilon \sigma^\nu \sigmabar^\mu \psi \> \partial_\nu \phi^* +
\epsilon \psi\> \partial^\mu \phi^*
+\epsilon^\dagger \psi^\dagger \> \partial^\mu \phi \right ).
\label{Lpsitrans}
\eeq
The first two terms
here 
just cancel against
$\deltaeps\lagr_{\rm scalar}$, while the remaining contribution
is a total derivative. So we arrive at
\beq
\deltaeps S =
\int d^4x \>\>\, (\deltaeps \lagr_{\rm scalar} + \deltaeps
\lagr_{\rm fermion})
= 0,\>
\label{invar}
\eeq
justifying our guess of the numerical multiplicative factor
made in eq.~(\ref{psitrans}).

We are not quite finished in demonstrating that the theory described
by eq.~(\ref{Lwz}) is supersymmetric. We must also show that the
supersymmetry
algebra closes; in other words, that the commutator of two
supersymmetry transformations is another symmetry of the theory.
Using eq.~(\ref{psitrans}) in eq.~(\ref{phitrans}), one finds
\beq
(\delta_{\epsilon_2} \delta_{\epsilon_1} -
\delta_{\epsilon_1} \delta_{\epsilon_2}) \phi =
i (\epsilon_1 \sigma^\mu \epsilon_2^\dagger -
   \epsilon_2 \sigma^\mu \epsilon_1^\dagger)\> \partial_\mu \phi
. \label{coophi}
\eeq
This is a remarkable result; in words, we have found that
the commutator of two supersymmetry transformations gives us back the
derivative of the original field.
Since $\partial_\mu$ just corresponds to
the generator of spacetime translations $P_\mu$, 
eq.~(\ref{coophi}) implies
the form of the supersymmetry algebra which was foreshadowed in
eq.~(\ref{susyalgone}) of the Introduction. (We will make this statement
more explicit before the end of this section.)

All of this will be for naught if we do not find the same result for
the fermion $\psi$, however.
Using eq.~(\ref{phitrans}) in eq.~(\ref{psitrans}), we find
\beq
(\delta_{\epsilon_2} \delta_{\epsilon_1} -
\delta_{\epsilon_1} \delta_{\epsilon_2}) \psi_\alpha =
i(\sigma^\mu\epsilon_1^\dagger)_\alpha\>  \epsilon_2 \partial_\mu\psi -
i(\sigma^\mu\epsilon_2^\dagger)_\alpha \> \epsilon_1 \partial_\mu\psi
{}.
\eeq
We can put this into a more useful form by applying the Fierz
identity
\beq
\chi_{\alpha}\> (\xi\eta) =
- \xi_{\alpha}\> (\eta\chi) - \eta_\alpha\> (\chi\xi)
\label{fierce}
\eeq
with $\chi = \sigma^\mu \epsilon_1^\dagger$,
$\xi = \epsilon_{2} $, $\eta = \partial_\mu \psi$, and again with
$\chi = \sigma^\mu \epsilon_2^\dagger$,
$\xi = \epsilon_{1} $, $\eta = \partial_\mu \psi$, followed
in each case by an application
of the identity eq.~(\ref{yetanotheridentity}). The result is
\beq
(\delta_{\epsilon_2} \delta_{\epsilon_1} -
\delta_{\epsilon_1} \delta_{\epsilon_2}) \psi_\alpha & = &
i (\epsilon_1 \sigma^\mu \epsilon_2^\dagger -
   \epsilon_2 \sigma^\mu \epsilon_1^\dagger) \> \partial_\mu \psi_\alpha
\nonumber \\
& &
-i \epsilon_{1\alpha} \> \epsilon_2^\dagger \sigmabar^\mu \partial_\mu\psi
+i \epsilon_{2\alpha} \> \epsilon_1^\dagger \sigmabar^\mu \partial_\mu\psi
{}.
\label{commpsi}
\eeq
The last two terms in (\ref{commpsi}) vanish on-shell; that is,
if the equation of
motion $\sigmabar^\mu\partial_\mu \psi = 0$ following from
the action is enforced. The remaining piece is exactly the
same spacetime translation that we found for the scalar field.

The fact that the supersymmetry algebra only closes on-shell
(when the classical equations of motion are satisfied) might
be somewhat worrisome, since we would
like the symmetry to hold even quantum mechanically.
This can be fixed by a trick. We invent a new complex scalar field $F$
which does not have a kinetic term. Such fields are called {\it
auxiliary},
and they are really just book-keeping devices which allow the symmetry
algebra to close off-shell.
The lagrangian density for $F$ and its complex conjugate is just
\beq
\lagr_{\rm auxiliary} = F^* F \> .
\label{lagraux}
\eeq
The dimensions of $F$ are (mass)$^2$, unlike an ordinary scalar
field which has dimensions of (mass). Equation (\ref{lagraux})
leads to the not-very-exciting equations of motion
$F=F^*=0$. However, we can use the auxiliary fields to our advantage
by including them in the supersymmetry transformation rules.
In view of eq.~(\ref{commpsi}),
a plausible thing to do is to make $F$ transform
into a multiple of
the equation of motion for $\psi$:
\beq
\deltaeps F = i \epsilon^\dagger \sigmabar^\mu \partial_\mu \psi;
\qquad\qquad
\deltaeps F^* = -i\partial_\mu \psi^\dagger \sigmabar^\mu \epsilon .
\label{Ftrans}
\eeq
Once again we have chosen the overall factor on the right hand side
by virtue of foresight.
Now the auxiliary part of the lagrangian density transforms as
\beq
\delta \lagr_{\rm auxiliary} = i
\epsilon^\dagger \sigmabar^\mu \partial_\mu
\psi \> F^* - i \partial_\mu \psi^\dagger \sigmabar^\mu \epsilon \> F
\eeq
which vanishes on-shell, but not for arbitrary off-shell field
configurations.
It is easy to see that by adding an extra term to
the transformation law for $\psi$ and $\psi^\dagger$:
\beq
\delta \psi_\alpha =
i (\sigma^\mu \epsilon^\dagger)_{\alpha}\> \partial_\mu
\phi + \epsilon_\alpha F;
\qquad\>\>
\delta \psi_{\dot{\alpha}}^\dagger =
-i (\epsilon\sigma^\mu)_{\dot{\alpha}}\> \partial_\mu
\phi^* + \epsilon^\dagger_{\dot{\alpha}} F^*
\label{fermiontrans}
\eeq
one obtains an additional contribution to $\deltaeps \lagr_{\rm fermion}$
which just cancels with
$\deltaeps \lagr_{\rm auxiliary}$, up to a total derivative
term. So
our ``modified" theory with
$\lagr = \lagr_{\rm scalar} +\lagr_{\rm fermion} + \lagr_{\rm auxiliary}$
is still invariant under
supersymmetry transformations. Proceeding as before, one
now obtains
for each of the fields $X=\phi,\phi^*,\psi,\psi^\dagger,F,F^*$,
\beq
(\delta_{\epsilon_2} \delta_{\epsilon_1} -
\delta_{\epsilon_1} \delta_{\epsilon_2}) X =
i (\epsilon_1 \sigma^\mu \epsilon_2^\dagger -
   \epsilon_2 \sigma^\mu \epsilon_1^\dagger) \> \partial_\mu X
\label{anytrans}
\eeq
using eqs.~(\ref{phitrans}), (\ref{Ftrans}),  and (\ref{fermiontrans}),
but without resorting to
any of the equations of motion. So we have succeeded in showing that
supersymmetry is a valid symmetry of the lagrangian off-shell.

In retrospect, one can see why we needed to introduce the auxiliary
field $F$ in order to get the supersymmetry algebra to work off-shell.
On-shell, the complex scalar field $\phi$
has two real propagating degrees of freedom, which
match with the two spin polarization states of $\psi$.
Off-shell, however,
the Weyl fermion $\psi$ is a complex two-component object,
so it has four real degrees of freedom. (Going on-shell
eliminates half of the propagating degrees of freedom
for $\psi$, because the lagrangian is linear in
time derivatives, so that the canonical momenta can be reexpressed in
terms of 
the configuration variables without time derivatives and are not
independent phase space coordinates.)
To make the numbers of bosonic and fermionic degrees of freedom match
off-shell as well
as on-shell, we had to introduce two more real scalar degrees of
freedom in the complex field $F$, which are eliminated when one
goes on-shell. The auxiliary field formulation is especially useful
when discussing spontaneous supersymmetry breaking, as we will see
in section \ref{sec:origins}.

Invariance of the action under a symmetry transformation always
implies the existence of a conserved current, and supersymmetry is no
exception. The {\it supercurrent}
$J^\mu_\alpha$
is an anticommuting four-vector which
also carries a spinor index, as befits the current associated with a symmetry
with fermionic generators.\cite{ref:supercurrent}
By the usual Noether procedure, one finds
for the supercurrent (and its hermitian conjugate) in terms of the
variations of the fields $X=\phi,\phi^*,\psi,\psi^\dagger,F,F^*$:
\beq
\epsilon J_\mu + \epsilon^\dagger J^\dagger_\mu
\equiv \sum_X \, \delta X\>{\delta\lagr\over \delta(\partial^\mu X)} -
K_\mu ,
\label{Noether}
\eeq
where $K_\mu$ is the object whose divergence is the variation of the
lagrangian density under the supersymmetry transformation, $\partial^\mu
K_\mu = \delta \lagr$.
A little work reveals that
\beq
J^\mu_\alpha = (\sigma^\nu\sigmabar^\mu\psi)_\alpha\> \partial_\nu \phi^*
; \qquad\qquad
J^{\dagger\mu}_{\dot{\alpha}}
=  (\psi^\dagger \sigmabar^\mu \sigma^\nu)_{\dot{\alpha}}
\> \partial_\nu \phi .
\label{WZsupercurrent}
\eeq
The supercurrent and its hermitian conjugate are separately
conserved:
\beq
\partial_\mu J^\mu_\alpha = 0;\qquad\qquad
\partial_\mu J^{\dagger\mu}_{\dot{\alpha}} = 0
\eeq
as can be verified by use of the equations of motion. From
these currents one
constructs the conserved charges
\beq
Q_\alpha = {\sqrt{2}}\int d^3x\> J^0_\alpha;\qquad\qquad
Q^\dagger_{\dot{\alpha}} = {\sqrt{2}} \int d^3x \> J^{\dagger
0}_{\dot{\alpha}}
\eeq
which are the generators of supersymmetry transformations.
(The factor of $\sqrt{2}$ normalization is included to agree with
an arbitrary
historical convention.) As
quantum mechanical operators, they satisfy
\beq
\left [ \epsilon Q + \epsilon^\dagger Q^\dagger , X \right ]
= -i{\sqrt{2}} \> \delta X
\eeq
for any field $X$, up to terms which vanish on-shell. This
can be verified explicitly by using the canonical equal-time
commutation and anticommutation relations
\beq
&&[ \phi(x), \pi(y) ] \,=\,
[ \phi^*(x), \pi^*(y) ] \,=\, i \delta^{(3)}(x-y);\qquad\>\>{}\\
&&\{ \psi^\dagger_{\dot{\alpha}}(x), \psi_\alpha(y) \} \,=\,
- \sigma^0_{{\alpha}\dot{\alpha}}\,\delta^{(3)}(x-y)\qquad{}
\eeq
derived from the free field theory lagrangian
eq.~(\ref{Lwz}). Here $\pi = \partial_0 \phi^*$ and
$\pi^* = \partial_0 \phi$ are the momenta conjugate to $\phi$ and
$\phi^*$ respectively.
Now the content of eq.~(\ref{anytrans}) can
be expressed in terms of canonical commutators as
\beq
\Bigl [
\epsilon_2 Q + \epsilon_2^\dagger Q^\dagger,\,
\bigl [
\epsilon_1 Q + \epsilon_1^\dagger Q^\dagger
,\, X
\bigr ] \Bigr ]
-
\Bigl [
\epsilon_1 Q + \epsilon_1^\dagger Q^\dagger,\,
\bigl [
\epsilon_2 Q + \epsilon_2^\dagger Q^\dagger
,\, X
\bigr ] \Bigr ]
=\qquad{}
\nonumber
\\
 2(\epsilon_2 \sigma^\mu \epsilon_1^\dagger -
         \epsilon_1 \sigma^\mu \epsilon_2^\dagger)\, i\partial_\mu X
\qquad\>{}
\label{epsalg}
\eeq
up to terms which vanish on-shell.
The spacetime momentum operator $P^\mu$ is given in terms of the
canonical variables by $P^0 = \pi\pi^* + \partial_j\phi \partial^j\phi^*
+ i \psi^\dagger \sigmabar^j \partial_j \psi$ and
$P^j = - \pi \partial^j\phi - \pi^* \partial^j\phi^* + i \psi^\dagger
\sigmabar^0 \partial^j \psi$, where $j$ is the spacetime vector index
restricted to the three spatial dimensions.
It generates spacetime translations on the fields $X$ according to
\beq
[P_\mu, X ] = i \partial_\mu X.
\label{supergrass}
\eeq
By rearranging the terms in eq.~(\ref{epsalg}) using the Jacobi identity,
we  therefore have
\beq
\Bigl [ \bigl [
\epsilon_2 Q + \epsilon_2^\dagger Q^\dagger,\,
\epsilon_1 Q + \epsilon_1^\dagger Q^\dagger \bigr ]
,\, X
\Bigr ]
=  2(\epsilon_2 \sigma^\mu \epsilon_1^\dagger -
         \epsilon_1 \sigma^\mu \epsilon_2^\dagger)\,  [ P_\mu , X
 ],
\label{epsalg2}
\eeq
for any $X$, so it must be that
\beq
 \bigl [
\epsilon_2 Q + \epsilon_2^\dagger Q^\dagger,\,
\epsilon_1 Q + \epsilon_1^\dagger Q^\dagger \bigr ]
=  2(\epsilon_2 \sigma^\mu \epsilon_1^\dagger -
         \epsilon_1 \sigma^\mu \epsilon_2^\dagger)\,  P_\mu 
\label{epsalg3}
\eeq
up to terms which vanish on-shell.
Now by expanding out eq.~(\ref{epsalg3}), one obtains the
non-schematic form of the supersymmetry
algebra relations
\beq
&&\{ Q_\alpha , Q^\dagger_{\dot{\alpha}} \} =
2\sigma^\mu_{\alpha\dot{\alpha}} P_\mu,
\label{nonschsusyalg1}\\
&&\{ Q_\alpha, Q_\beta\} =
\{ Q^\dagger_{\dot{\alpha}}, Q^\dagger_{\dot{\beta}} \} = 0
\label{nonschsusyalg2}
\eeq
as promised in the Introduction. [The commutator in eq.~(\ref{epsalg3})
turns into anticommutators in eqs.~(\ref{nonschsusyalg1})
and (\ref{nonschsusyalg2}) in the process of 
extracting the anticommuting spinors $\epsilon_1$ and $\epsilon_2$.]
The results $[Q_\alpha, P_\mu ] = 0$ and $[Q^\dagger_{\dot{\alpha}},
P_\mu] = 0$ follow immediately from eq.~(\ref{supergrass})
and the fact that the supersymmetry
transformations are global (independent of position in spacetime).
This demonstration of the supersymmetry algebra
in terms of the canonical generators $Q$ and $Q^\dagger$ requires the use
of the Hamiltonian equations of motion, but the symmetry itself is valid
off-shell at the level of the lagrangian, as we have already shown.

\subsection{Interactions of chiral
supermultiplets}\label{subsec:susylagr.chiral}

In a realistic theory like the
MSSM, there are many chiral supermultiplets which have both gauge
and non-gauge interactions.
In this subsection, our task is to
construct the most general possible theory of masses and
non-gauge interactions
for particles that live in chiral supermultiplets. In the MSSM
these
are the quarks, squarks, leptons, sleptons, Higgs scalars and
higgsino fermions.
We will find that the form of the non-gauge couplings,
including mass terms,
is highly restricted by the requirement that the action is invariant
under supersymmetry transformations. (Gauge interactions will be dealt
with in the following subsections.)

Our starting point is the lagrangian density for a collection
of free chiral supermultiplets labelled by an index $i$ which
runs over all gauge and flavor degrees of freedom.
Since we will
want to construct an interacting theory with supersymmetry closing
off-shell,
each supermultiplet
contains a complex scalar $\phi_i$ and a left-handed Weyl fermion
$\psi_i$ as physical degrees of freedom, plus a complex auxiliary
field $F_i$ which does not propagate. The results of the previous
subsection tell us that the free part of the Lagrangian is
\beq
\lagr_{\rm free} = -\partial^\mu \phi^{*i} \partial_\mu \phi_i
-i \psi^{\dagger i} \sigmabar^\mu \partial_\mu \psi_i
+ F^{*i} F_i
\label{lagrfree}
\eeq
where we sum over repeated indices $i$ (not to be confused
with the suppressed spinor indices), with the convention
that fields $\phi_i$ and $\psi_i$ always carry lowered
indices, while their conjugates always carry raised indices.
It is invariant under the supersymmetry transformation
\beq
&&\!\!\!\!\!\!\!\!\!\!\delta \phi_i = \epsilon\psi_i
\qquad\qquad\qquad\qquad\qquad\>\>\>\>\>\,
\delta \phi^{*i} = \epsilon^\dagger \psi^{\dagger i}
\qquad\qquad\qquad\qquad{}\label{phitran}\\
&&\!\!\!\!\!\!\!\!\!\!\delta (\psi_i)_\alpha =
i (\sigma^\mu \epsilon^\dagger)_{\alpha}\, \partial_\mu
\phi_i + \epsilon_\alpha F_i
\qquad
\delta (\psi^{\dagger i})_{\dot{\alpha}}=
-i (\epsilon\sigma^\mu)_{\dot{\alpha}}\, \partial_\mu
\phi^{*i} + \epsilon^\dagger_{\dot{\alpha}} F^{*i}
\qquad\qquad{}\\
&&\!\!\!\!\!\!\!\!\!\!
\deltaeps F_i = i \epsilon^\dagger \sigmabar^\mu\partial_\mu
\psi_i
\qquad\qquad\qquad\qquad\,
\deltaeps F^{* i} = -i\partial_\mu \psi^{\dagger
i} \sigmabar^\mu \epsilon\> .
\eeq

As we will now argue, the most general set of renormalizable
interactions for these fields can be written in the simple form
\beq
\lagr_{\rm int} = -\half W^{ij} \psi_i \psi_j
+ W^i F_i
+ \conj ,
\label{tryint}
\eeq
where $W^{ij}$ and $W^i$ are some functions of the
bosonic fields
with dimensions of (mass) and (mass)$^2$ respectively, and ``$\conj$"
henceforth stands for complex conjugate.
At this point,
we are not assuming that $W^{ij}$ and $W^i$ are related to each
other in any way whatsoever. However, soon
we will find out that they {\it are}
related, which is why we have chosen the same letter for them.
Notice that eq.~(\ref{xichi}) tells us that $W^{ij}$ is symmetric
under $i\leftrightarrow j$. 
Now, let us require the lagrangian to be renormalizable by power
counting, so that each term has field content with mass dimension
$\leq 4$. It follows immediately that we do not need
to consider the possibility of $W^{ij}$ or $W^i$
being functions of
the fermionic or auxiliary fields.
For the same
reason, we can take $W^i$ to be  at most a quadratic
polynomial, and $W^{ij}$ linear, in the fields $\phi_i$ and $\phi^{*i}$.
Also, we do not need to consider including in $\lagr_{\rm int}$
any term which is a function of the scalar fields $\phi_i,\phi^{*i}$ only.
If there were such a term, then under a supersymmetry
transformation eq.~(\ref{phitran}) it would go into
another function of the scalar fields only, multiplied by $\epsilon\psi_i$
or $\epsilon^\dagger \psi^{\dagger i}$, and with no spacetime
derivatives or $F_i$, $F^{*i}$ fields. It is easy to see from
eqs.~(\ref{phitran})-(\ref{tryint}) that nothing
of this form can possibly be cancelled by the supersymmetry
transformation of any other term in the
lagrangian. So eq.~(\ref{tryint}) is indeed the most general possibility!

We must now require that $\lagr_{\rm int}$ is invariant under the supersymmetry
transformations, since $\lagr_{\rm free}$ was already invariant
by itself. It is easiest to divide the variation of $\lagr_{\rm int}$
into several parts which must cancel separately. First, we consider
the part which contains four spinors:
\beq
\delta \lagr_{\rm int} |_{\rm 4-spinor} \! &=&\!
-\half {\delta W^{ij} \over \delta \phi_k} (\epsilon \psi_k)(\psi_i \psi_j)
-\half {\delta W^{ij} \over \delta \phi^{*k}} (\epsilon^\dagger
\psi^{\dagger k})
(\psi_i \psi_j)
+ \conj \qquad\>
\label{deltafourferm} \eeq
The term proportional to
$(\epsilon \psi_k)(\psi_i\psi_j)$ cannot cancel against any
other
term. Fortunately, however, the Fierz identity eq.~(\ref{fierce}) implies
\beq
(\epsilon \psi_i) (\psi_j \psi_k) + (\epsilon \psi_j) (\psi_k \psi_i)
+ (\epsilon \psi_k) (\psi_i\psi_j) = 0 ,
\eeq
which allows this contribution to $\delta\lagr_{\rm int}$
to vanish identically if and only if $\delta W^{ij}/\delta \phi_k$
is totally symmetric under interchange of $i,j,k$.
There is no such identity available for the term proportional to
$(\epsilon^\dagger \psi^{\dagger k})(\psi_i\psi_j)$. Since it
cannot cancel with
any other term, requiring it to be absent just tells us that $W^{ij}$
cannot contain $\phi^{*k}$.
In other words, $W^{ij}$ is {\it analytic} (or {\it
holomorphic})
in the complex  fields $\phi_k$.

So far, what we have learned is that we can write
\beq
W^{ij} = M^{ij} + y^{ijk} \phi_k
\eeq
where $M^{ij}$ is a symmetric mass matrix for the fermion fields, and
$y^{ijk}$
is a Yukawa coupling of a scalar $\phi_k$ and
two fermions $\psi_i \psi_j$ which must be totally
symmetric under interchange of $i,j,k$.
It is convenient to write
\beq
W^{ij} = {\delta^2 \over \delta\phi_i\delta\phi_j} W
\label{expresswij}
\eeq
where we have introduced a very useful object
\beq
W = {1\over 2} M^{ij} \phi_i \phi_j + {1\over 6} y^{ijk} \phi_i \phi_j
\phi_k
\label{superpotential}
\eeq
which is called the {\it superpotential}. This is not a scalar potential
in the ordinary sense; in fact, it is not even real. It
is instead an analytic function of the scalar fields $\phi_i$ treated
as complex variables.

Continuing on our vaunted quest, we next consider the parts of
$\delta \lagr_{\rm int}$
which contain a spacetime derivative:
\beq
\delta \lagr_{\rm int} |_\partial &=&
- i W^{ij}\partial_\mu \phi_j \> \psi_i \sigma^\mu \epsilon^\dagger
- i W^i\> \partial_\mu \psi_i\sigma^\mu \epsilon^\dagger
+\conj 
\label{wijwi}
\eeq
Here we have used the identity eq.~(\ref{yetanotheridentity})
on the second term, which came from $(\delta F_i)W^i$.
Now we can use
eq.~(\ref{expresswij}) to observe that
\beq
W^{ij} \partial_\mu \phi_j =
\partial_\mu \left ( {\delta W\over \delta{\phi_i}}\right ) .
\label{parttwo}
\eeq
Then it is clear that eq.~(\ref{wijwi}) will be a total derivative
if and only if
\beq
W^i = {\delta W\over \delta \phi_i}
= M^{ij}\phi_j +
{1\over 2} y^{ijk} \phi_j \phi_k\> ,
\label{wiwiwi}
\eeq
which explains why we chose its name the way we did.
The remaining terms in $\delta \lagr_{\rm int}$ are all linear in $F_i$
or $F^{*i}$, and it is easy to show that they cancel, given the
results for
$W^i$ and $W^{ij}$ that we have already found.

To recap, we have found that the most general non-gauge interactions
for chiral supermultiplets are determined by
a single analytic function of the complex scalar fields,
the superpotential $W$.
The auxiliary fields $F_i$ and $F^{*i}$ can be eliminated using their
classical equations of motion.
The part of $\lagr_{\rm free} + \lagr_{\rm int}$
that contains the auxiliary fields is
$
F_i F^{*i} + W^i F_{i} + W^{*}_i F^{*i}$,
leading to the equations of motion
\beq
F_i = -W_i^*;\qquad\qquad F^{*i} = -W^i \> .
\label{replaceF}
\eeq
Thus the auxiliary fields are expressible algebraically
(without any derivatives)
in terms of the scalar fields.
After making the replacement eq.~(\ref{replaceF}) in
$\lagr_{\rm free} + \lagr_{\rm int}$,
we obtain the lagrangian density
\beq
\lagr = -\partial^\mu \phi^{*i} \partial_\mu \phi_i
-i \psi^{\dagger i} \sigmabar^\mu \partial_\mu \psi_i
-\half \left (W^{ij} \psi_i \psi_j  + W^{*ij} \psi^{\dagger i}
\psi^{\dagger j} \right )
- W^i W^{*}_i.
\label{noFlagr}
\eeq
(Since $F_i$ and $F^{*i}$
appear only quadratically in the action, the result of instead
doing a functional
integral over them at the quantum level has precisely the same effect.)
Now that the non-propagating fields $F_i, F^{*i}$ have been eliminated,
it is clear from eq.~(\ref{noFlagr})
that the scalar potential for the theory is just given
in terms of the superpotential by (recall $\lagr$ contains $-V$):
\beq
V(\phi,\phi^*) = W^i W_i^* = F_i F^{*i} =
M^*_{ik} M^{kj} \phi^{*i} \phi_{j}\qquad\qquad\qquad\qquad\qquad
\hfill \label{ordpot}
\\ \hfill+
{1\over 2} M^{in} y_{jkn}^* \phi_i \phi^{*j} \phi^{*k}
+{1\over 2} M_{in}^{*} y^{jkn} \phi^{*i} \phi_j \phi_k
+{1\over 4} y^{ijn} y_{kln}^{*} \phi_i \phi_j \phi^{*k} \phi^{*l}
\> .
\nonumber\eeq
This scalar potential is automatically
bounded from below; in fact, since it is
a sum of squares of absolute values (of the $W^i$),
it is always non-negative.
If we substitute the general form for the superpotential
eq.~(\ref{superpotential}) into eq.~(\ref{noFlagr}), we obtain for the
full lagrangian density
\beq
\lagr &=&
-\partial^\mu \phi^{*i} \partial_\mu \phi_i
-i \psi^{\dagger i} \sigmabar^\mu \partial_\mu \psi_i
\nonumber \\ &&
- \half M^{ij} \psi_i\psi_j - \half M_{ij}^{*} \psi^{\dagger i}
\psi^{\dagger j}
- V(\phi,\phi^*)
\nonumber\\
&& - \half y^{ijk} \phi_i \psi_j \psi_k - \half y_{ijk}^{*} \phi^{*i}
\psi^{\dagger j} \psi^{\dagger k}.
\label{lagrchiral}
\eeq

Now we can compare the masses of the fermions and scalars by looking at
the linearized equations of motion:
\beq
&&\>\>\>\partial^\mu\partial_\mu \phi_i = M_{ik}^{*} M^{kj} \phi_j
+ \ldots;\\
&&-i\sigmabar^\mu\partial_\mu\psi_i = M_{ij}^{*}
\psi^{\dagger j}+\ldots;
\qquad\qquad
-i\sigma^\mu\partial_\mu\psi^{\dagger i} = M^{ij} \psi_j
+\ldots .\qquad\>\>\>{}
\label{linfermiontwo}
\eeq
One can eliminate $\psi$ in terms of
$\psi^\dagger$ and vice versa in eq.~(\ref{linfermiontwo}), obtaining
[after use of the identity eq.~(\ref{pauliident})]
\beq
\partial^\mu\partial_\mu \psi_i =  M_{ik}^{*} M^{kj} \psi_j
+ \ldots ;\qquad\qquad
\partial^\mu\partial_\mu \psi^{\dagger j} =
\psi^{\dagger i}
M_{ik}^{*} M^{kj}+\ldots
\> .
\eeq
Therefore, the fermions and the bosons satisfy the same wave equation
with exactly the same (mass)$^2$ matrix
with real non-negative eigenvalues,
namely ${(M^2)_i}^j = M_{ik}^{*} M^{kj}$.
It follows that diagonalizing this matrix gives a collection of
chiral supermultiplets each of which contains a mass-degenerate complex
scalar and Weyl fermion, in agreement with the general argument in the
Introduction.

\subsection{Lagrangians for gauge
supermultiplets}\label{subsec:susylagr.gauge}

The propagating degrees of freedom in a gauge supermultiplet
are a massless gauge boson field $A_\mu^a$ and a two-component
Weyl fermion gaugino $\lambda^a$. The index $a$ here runs over the
adjoint representation of the gauge group ($a=1\ldots 8$ for
$SU(3)_C$ color gluons and gluinos; $a=1,2,3$ for $SU(2)_L$
weak isospin; $a=1$ for $U(1)_Y$ weak hypercharge).
The gauge transformations of the vector supermultiplet fields are
then
\beq
&&\delta_{\rm gauge} A^a_\mu = -\partial_\mu \Lambda^a + g f^{abc}
A^b_\mu \Lambda^c
\label{Agaugetr}
\\
&&\delta_{\rm gauge} \lambda^a = g f^{abc} \lambda^b \Lambda^c
\label{lamgaugetr}
\eeq
where $\Lambda^a$ is an infinitesimal gauge transformation parameter,
$g$ is the gauge coupling, and $f^{abc}$ are the
totally antisymmetric structure constants
which define the gauge group. (The special case of an abelian
group like $U(1)_Y$ is obtained by just setting $f^{abc}=0$; in particular
the corresponding gaugino is a gauge singlet in that case.)

The on-shell degrees of freedom for $A^a_\mu$ and $\lambda^a_\alpha$
amount
to
two bosonic and two fermionic helicity states (for each $a$), as required
by supersymmetry. However, off-shell $\lambda^a_\alpha$ consists of
two complex, or four real, fermionic degrees of freedom, while
$A^a_\mu$ only has three real bosonic degrees of freedom; one degree of
freedom
is removed by the inhomogeneous gauge transformation eq.~(\ref{Agaugetr}).
So, we will need one real bosonic auxiliary field,
traditionally called $D^a$, in order for supersymmetry to be consistent
off-shell. This field also transforms as an adjoint of the gauge group
[i.e., like eq.~(\ref{lamgaugetr}) with $\lambda \rightarrow D$]
and satisfies $(D^a)^* = D^a$.
Like the chiral auxiliary fields $F_i$, it has dimensions of (mass)$^2$
and thus no kinetic term, so that it can be eliminated on-shell using its
algebraic equation of motion.

Therefore, the lagrangian density for a gauge supermultiplet ought to
be
\beq
\lagr_{\rm gauge} = -{1\over 4} F_{\mu\nu}^a F^{\mu\nu a}
- i \lambda^{\dagger a} \sigmabar^\mu D_\mu \lambda^a
+ {1\over 2} D^a D^a
\label{lagrgauge}
\eeq
where
\beq
F^a_{\mu\nu} = \partial_\mu A^a_\nu - \partial_\nu A^a_\mu -
g f^{abc} A^b_\mu A^c_\nu
\eeq
is the usual Yang-Mills field strength, and
\beq
D_\mu \lambda^a = \partial_\mu \lambda^a - g f^{abc} A^b_\mu \lambda^c
\label{ordtocovlambda}
\eeq
is the covariant derivative of the gaugino field. One can infer the
appropriate form for the supersymmetry transformation of the fields,
up to multiplicative constants, from the requirements that
they
should be linear in the infinitesimal
parameters $\epsilon,\epsilon^\dagger$ with dimensions of
(mass)$^{-1/2}$; that $\delta A^a_\mu$ is real; and that $\delta D^a$
should be real and
proportional to the field equations for the gaugino, in analogy
with the role of the auxiliary field $F$ in the chiral supermultiplet
case.
Thus one can guess, up to multiplicative factors,
\beq
&& \delta A_\mu^a = -{1\over \sqrt{2}} \left [\epsilon^\dagger
\sigmabar_\mu
\lambda^a + \lambda^{\dagger a} \sigmabar_\mu \epsilon \right ]
\label{Atransf}
\\
&& \delta \lambda^a_\alpha =
-{i\over 2\sqrt{2}} (\sigma^\mu \sigmabar^\nu \epsilon)_\alpha
\> F^a_{\mu\nu} + {1\over \sqrt{2}} \epsilon_\alpha\> D^a \\
&& \delta D^a = {i\over \sqrt{2}} \left [
\epsilon^\dagger \sigmabar^\mu D_\mu \lambda^a -
D_\mu \lambda^{\dagger a} \sigmabar^\mu \epsilon \right ] .
\label{Dtransf}
\eeq
The factors of $\sqrt{2}$ are chosen
\footnote{For future convenience in treating the MSSM, we have chosen
complex phases so that our $\lambda^a$, $\lambda^{\dagger a}$ are equal
to $-i$, $i$ times the gaugino spinors in Ref.\cite{WessBaggerbook}}
so that the action obtained by integrating
$\lagr_{\rm gauge}$ is
invariant.
It is now a little bit tedious, but straightforward, to check that
eq.~(\ref{anytrans}) is modified to
\beq
(\delta_{\epsilon_2} \delta_{\epsilon_1} -\delta_{\epsilon_1}
\delta_{\epsilon_2} ) X = i (\epsilon_1\sigma^\mu \epsilon_2^\dagger
-\epsilon_2\sigma^\mu \epsilon_1^\dagger) D_\mu X
\label{joeyramone}
\eeq
for $X$ equal to any of the gauge-covariant fields $F_{\mu\nu}^a$,
$\lambda^a$, $\lambda^{\dagger a}$, $D^a$, as well as arbitrary
covariant derivatives acting on them.
This ensures that the supersymmetry algebra
eqs.~(\ref{nonschsusyalg1})-(\ref{nonschsusyalg2}) is
realized
on gauge-invariant combinations of fields in gauge supermultiplets, as
they
were on the chiral supermultiplets.\footnote{The supersymmetry
transformations
eqs.~(\ref{Atransf})-(\ref{Dtransf}) are
non-linear for non-abelian gauge symmetries,
because of the gauge fields contained in
the covariant derivatives acting on the gaugino
fields and in the field strength
$F_{\mu \nu}^a$. By adding even
more auxiliary fields besides $D^a$, one
can make the supersymmetry transformations linear in the fields.
The version given here in which those extra auxiliary
fields have been
removed by gauge transformations is
called ``Wess-Zumino gauge".\cite{WZgauge}}
These calculations require the use of identities
\beq
&&\xi\sigma^\mu \sigmabar^\nu \chi  =
\chi \sigma^\nu \sigmabar^\mu \xi =
(\chi^\dagger \sigmabar^\nu \sigma^\mu \xi^\dagger)^* =
(\xi^\dagger \sigmabar^\mu \sigma^\nu \chi^\dagger)^*;
\\
&&\sigmabar^\mu \sigma^\nu \sigmabar^\rho =
\eta^{\mu\rho} \sigmabar^\nu -
\eta^{\nu\rho} \sigmabar^\mu -
\eta^{\mu\nu} \sigmabar^\rho - i \epsilon^{\mu\nu\rho\kappa}
\sigmabar_\kappa ;
\\
&&
\sigma_{\alpha\dot{\alpha}}^\mu
\sigmabar_\mu^{\dot{\beta}\beta} = - 2 \delta_\alpha^\beta
\delta_{\dot{\alpha}}^{\dot{\beta}}.
\eeq
If we had not included the auxiliary field
$D^a$, then the supersymmetry
algebra eq.~(\ref{joeyramone})
would hold only after using the equations of motion
for $\lambda^a$ and $\lambda^{\dagger a}$.
The auxiliary fields just satisfy
the equations
of motion $D^a=0$, but this is no longer true if one couples the gauge
supermultiplets
to chiral supermultiplets, as we now do.

\subsection{Supersymmetric gauge
interactions}\label{subsec:susylagr.gaugeinter}

Finally we are ready to consider a general lagrangian density for a
supersymmetric theory with both chiral and gauge supermultiplets.
Suppose that the chiral supermultiplets
transform under the gauge group
in a representation with hermitian matrices ${(T^a)_i}^j$ satisfying
$[T^a,T^b] =i f^{abc} T^c$. [For example, if the gauge group is
$SU(2)$, then
$f^{abc} = \epsilon^{abc}$, and
the $T^a$ are $1/2$ times the Pauli matrices for a chiral supermultiplet
transforming in the fundamental representation.]
Thus
\beq
\delta_{\rm gauge}X_i = ig \Lambda^a (T^a X)_i
\eeq
for $X_i = \phi_i,\psi_i,F_i$; since supersymmetry and gauge transformations
commute, the scalar, fermion, and auxiliary fields must be in the same
representation of the gauge group. To have a gauge-invariant lagrangian,
we need to turn the ordinary derivatives in eq.~(\ref{lagrfree})
into covariant derivatives:
\beq
&&
\partial_\mu \phi_i \rightarrow D_\mu \phi_i =
\partial_\mu \phi_i + i g A^a_\mu (T^a\phi)_i
\label{ordtocovphi}\\
&&
\partial_\mu \phi^{*i} \rightarrow D_\mu \phi^{*i} =
\partial_\mu \phi^{*i} - i g A^a_\mu (\phi^* T^a)^i
\\
&&
\partial_\mu \psi_i \rightarrow D_\mu \psi_i =
\partial_\mu \psi_i + i g A^a_\mu (T^a\psi)_i .
\label{ordtocovpsi}
\eeq
Naively, this simple procedure achieves the goal of coupling
the vector bosons in the gauge supermultiplet to the scalars
and fermions in the chiral supermultiplets. However,
we also have to consider whether there are any other interactions
allowed by gauge invariance involving the gaugino and $D^a$ fields
which might have to be included to make
a supersymmetric lagrangian.

In fact, there are three such possibilities which are renormalizable
(of mass dimension $\leq 4$),
namely
\beq
(\phi^* T^a \psi)\lambda^a,\qquad
\lambda^{\dagger a} (\psi^\dagger T^a
\phi)
\qquad {\rm and} \qquad
(\phi^* T^a \phi) D^a .
\label{extrater}
\eeq
Now one can add them, with arbitrary dimensionless coupling coefficients,
to the lagrangians for the chiral and gauge supermultiplets
and demand that the whole mess be real and invariant under supersymmetry
transformations, up to a total derivative.
Not surprisingly, this is possible only if one modifies the supersymmetry
transformation laws
for the matter fields to include gauge-covariant rather than ordinary
derivatives (and to include
one strategically-chosen extra term in $\delta
F_i$):
\beq&&
\delta \phi_i = \epsilon\psi_i
\label{gphitran}\\
&&\delta (\psi_i)_\alpha =
i (\sigma^\mu \epsilon^\dagger)_{\alpha}\> D_\mu
\phi_i + \epsilon_\alpha F_i
\\
&&\deltaeps F_i = i \epsilon^\dagger \sigmabar^\mu D_\mu \psi_i
\> + \> \sqrt{2} g (T^a \phi)_i\> \epsilon^\dagger \lambda^{\dagger a} .
\eeq
After some algebra one can now fix the coefficients for the terms in
eq.~(\ref{extrater}), so that the full lagrangian density
for a renormalizable supersymmetric theory is
\beq
\lagr & = & \lagr_{\rm gauge} + \lagr_{\rm chiral} \nonumber\\
        && - \sqrt{2} g \left [
(\phi^* T^a \psi)\lambda^a + \lambda^{\dagger a} (\psi^\dagger T^a \phi)
\right ]
\nonumber\\
&& + g  (\phi^* T^a \phi) D^a .
\label{gensusylagr}
\eeq
Here $\lagr_{\rm chiral }$ means the chiral supermultiplet lagrangian
found in section \ref{subsec:susylagr.chiral} [e.g., eq.~(\ref{noFlagr})
or (\ref{lagrchiral})],
but with ordinary derivatives replaced everywhere by gauge-covariant
derivatives, and $\lagr_{\rm gauge}$ was given in eq.~(\ref{lagrgauge}).
To prove that eq.~(\ref{gensusylagr}) is invariant under the
supersymmetry transformations, one must use the identity
\beq
W^i (T^a)_i^j \phi_j = 0.
\label{wgaugeinvar}
\eeq 
This is precisely the condition that must be
satisfied anyway in order for the superpotential (and thus $\lagr_{\rm
chiral}$)
to be gauge invariant, since the left side is proportional to
$\delta_{\rm gauge} W$.

The last two lines in eq.~(\ref{gensusylagr}) are interactions whose
strengths are fixed to be gauge couplings by the requirements of
supersymmetry,
even though they are not gauge interactions from the point of view
of an ordinary field theory. The second line is a direct coupling of
gauginos to matter fields which is the ``supersymmetrization" of
the usual gauge boson coupling to matter fields. The last line combines
with the $(1/2) D^aD^a$ term in $\lagr_{\rm gauge}$ to provide an equation
of motion
\beq
D^a = -g (\phi^* T^a \phi ).
\label{solveforD}
\eeq
Like the auxiliary fields $F_i$ and $F^{*i}$, the $D^a$ are expressible
purely algebraically in terms of the scalar fields.
Replacing the auxiliary fields in eq.~(\ref{gensusylagr}) using
eq.~(\ref{solveforD}),
one finds that
the complete scalar potential is
(recall $\lagr \supset -V$):
\beq
V(\phi,\phi^*) = F^{*i} F_i + \half \sum_a D^a D^a = W_i^* W^i +
\half \sum_a g_a^2 (\phi^* T^a \phi)^2.
\label{fdpot}
\eeq
The two types of
terms in this expression are called ``$F$-term" and
``$D$-term" contributions, respectively.
In the second term in eq.~(\ref{fdpot}), we have now written an
explicit sum $\sum_a$ to cover the case that the gauge group has several
distinct factors with different gauge couplings $g_a$. [For instance,
in the MSSM the three factors $SU(3)_C$, $SU(2)_L$ and $U(1)_Y$ have
different gauge couplings $g_3$, $g$ and $g^\prime$.]
Since $V(\phi,\phi^*)$ is
a sum of squares, it is always greater than or equal to
zero for every field configuration. It is a
very interesting
and unique feature of supersymmetric theories that the scalar potential
is completely determined by the {\it other} interactions in the
theory. The $F$-terms are fixed by Yukawa couplings and fermion mass
terms, and the $D$-terms are fixed by the gauge interactions.

By using Noether's procedure [see eq.~(\ref{Noether})], one finds the
conserved supercurrent
\beq
J_\alpha^\mu &\!\!\!=\!\!\!&
(\sigma^\nu\sigmabar^\mu \psi_i)_\alpha\, D_\nu \phi^{*i}
-i (\sigma^\mu \psi^{\dagger i})_\alpha\, W_i^*
\nonumber
\\ &&- {1\over 2 \sqrt{2}}
(\sigma^\nu \sigmabar^\rho \sigma^\mu
\lambda^{\dagger a})_\alpha\, F^a_{\nu\rho}
- {i\over {\sqrt{2}}} g \phi^* T^a \phi
\> (\sigma^\mu \lambda^{\dagger a})_\alpha , \>\>\>\>{}
\label{supercurrent}
\eeq
generalizing the expression given in eq.~(\ref{WZsupercurrent}) for the
Wess-Zumino model.
This expression will be useful
when we discuss
certain aspects of spontaneous supersymmetry breaking in section
\ref{subsec:origins.gravitino}.

\subsection{Summary: How to build a supersymmetric
model}\label{subsec:susylagr.summary}

In a renormalizable supersymmetric field theory, the interactions
and masses of all particles are determined just by their gauge
transformation properties and by the superpotential $W$.
By construction, we found that $W$ had to be
an analytic function of the complex scalar fields $\phi_i$, which are
always defined to transform under supersymmetry into {\it left}-handed
Weyl fermions. We should mention that
in an equivalent language, $W$ is said to be a function of chiral
{\it superfields}.\cite{superfields}
A superfield is a single object which contains as
components all of the bosonic, fermionic, and auxiliary fields within
the corresponding supermultiplet, e.g.~$\Phi_i \supset (\phi_i,\psi_i,F_i)$.
(This is analogous to the way in which one often describes a weak
isospin doublet or color triplet by a multicomponent field.)
The gauge
quantum numbers and mass dimension of a chiral superfield are the same as
that of its
scalar component.
In
the superfield formulation, one writes instead of
eq.~(\ref{superpotential})
\beq
W = {1\over 2}M^{ij} \Phi_i\Phi_j +{1\over 6} y^{ijk} \Phi_i \Phi_j \Phi_k
\label{superpot}
\eeq
which means exactly the same thing. While this entails no difference in
practical results, the fancier version eq.~(\ref{superpot}) at least serves
to remind us that $W$ determines not only the scalar interactions in the
theory, but the fermion masses and Yukawa couplings as well.
The derivation of all of our preceding results can be obtained somewhat
more elegantly using superfield methods, which have the advantage
of making invariance under supersymmetry
transformations manifest. We have avoided this extra layer of notation
on purpose, in favor of the more pedestrian but hopefully more familiar
component field approach. The latter is at least more appropriate for making
contact with phenomenology in a universe with supersymmetry breaking.
The only (occasional) use we will make of superfield notation is the
purely cosmetic one of following
the common practice of specifying superpotentials like
eq.~(\ref{superpot}) rather than (\ref{superpotential}).
The specification of the
superpotential is really a code for the terms that it implies
in the lagrangian, so the reader may feel free to
think of the superpotential either as a function of the scalar fields
$\phi_i$ or as the same function of
the superfields $\Phi_i$ which contain them.

Given the supermultiplet content of the theory, the form of the
superpotential is restricted by gauge invariance. In any given theory,
only a subset of the
couplings $M^{ij}$ and $y^{ijk}$ will be allowed to be non-zero. The 
entries of the mass
matrix $M^{ij}$
can only be non-zero for $i$ and $j$ such that the supermultiplets
$\Phi_i$ and $\Phi_j$
transform under the gauge group in representations which are
conjugates of each other. (In fact,
in the MSSM there is only one
such term, as we will see.)
Likewise, the Yukawa couplings $y^{ijk}$ can only be non-zero
when $\Phi_i$, $\Phi_j$, and $\Phi_k$ transform in representations which
can combine to form a singlet.

\begin{figure}
\centerline{\psfig{figure=susydim0.ps,height=.96in}}
\caption{The dimensionless non-gauge interaction vertices
in a supersymmetric theory: (a) scalar-fermion-fermion Yukawa
interaction $y^{ijk}$, (b) quartic scalar interaction $y^{ijn}y^*_{kln}$.
\label{fig:dim0}}
\end{figure}
\begin{figure}
\centerline{\psfig{figure=susydim12.ps,height=.96in}}
\caption{Supersymmetric dimensionful couplings: (a)
(scalar)$^3$ interaction vertex $M^*_{in} y^{jkn}$, (b)
fermion mass
term $M^{ij}$, (c) scalar (mass)$^2$ term $M^*_{ik}M^{kj}$.
\label{fig:dim12}}
\end{figure}
The interactions implied by the superpotential eq.~(\ref{superpot})
are shown \footnote{Here, the auxiliary fields have been eliminated using
their equations of motion
(``integrated out") as in eq.~(\ref{lagrchiral}).
It is quite possible instead to give Feynman
rules which include the auxiliary fields, although this tends to be
less useful in phenomenological applications.} in
Figs.~\ref{fig:dim0} and \ref{fig:dim12}.
Those in Fig.~\ref{fig:dim0} are all
determined by the dimensionless parameters $y^{ijk}$.
The Yukawa interaction in Fig.~\ref{fig:dim0}a corresponds to the
next-to-last
term in eq.~(\ref{lagrchiral}). For
each particular Yukawa coupling of $\phi_i \psi_j \psi_k$ with
strength $y^{ijk}$,
there must be equal couplings of $\phi_j \psi_i \psi_k$ and $\phi_k
\psi_i \psi_j$, since $y^{ijk}$ is completely symmetric under interchange
of any two of its indices as shown in section
\ref{subsec:susylagr.chiral}.
There is also a dimensionless coupling for
$\phi_i \phi_j \phi^{*k}\phi^{*l}$, with strength $y^{ijn} y^*_{kln}$
as required by supersymmetry [see the last term in eq.~(\ref{ordpot})].
The arrows on both the fermion and scalar
lines follow the chirality;
i.e., one direction for propagation of $\phi$
and $\psi$ and the other for the propagation of $\phi^*$ and
$\psi^\dagger$. Thus there is a vertex corresponding to the
one in Fig.~\ref{fig:dim0}a but with all arrows reversed, corresponding to
the
complex conjugate [the last term in eq.~(\ref{lagrchiral})].
The relationship between the interactions in Figs.~\ref{fig:dim0}a and
\ref{fig:dim0}b is
exactly of the special type needed to cancel the quadratic divergences in
quantum corrections to scalar masses, as discussed in the Introduction
[compare Fig.~\ref{fig:higgscorr1}].

In Fig.~\ref{fig:dim12}, we show the only
interactions corresponding to renormalizable
and supersymmetric vertices with dimensions of (mass) and (mass)$^2$.
First, there are
(scalar)$^3$ couplings which are entirely determined by the
superpotential mass parameters $M^{ij}$ and
Yukawa couplings $y^{ijk}$, as
indicated by the second and third terms in eq.~(\ref{ordpot}). The
propagators of the fermions and
scalars in the theory are constructed in the usual way using
the fermion mass $M^{ij}$ and scalar (mass)$^2$ $M^*_{in}M^{nj}$.
Of particular interest is the fact that the
fermion mass term $M^{ij}$ leads to a chirality-changing
insertion in the fermion propagator; note the directions of the
arrows in Fig.~\ref{fig:dim12}b. There is no such arrow-reversal for a
scalar
propagator in a theory with exact supersymmetry; as shown in
Fig.~\ref{fig:dim12}c, if one treats the scalar (mass)$^2$ term as an
insertion in the
propagator, the arrow direction is preserved. Again, for each of Figures
\ref{fig:dim12}a and \ref{fig:dim12}b there is an interaction with all
arrows reversed.

\begin{figure}
\centerline{\psfig{figure=susygauge.ps,height=2.1in}}
\caption{Supersymmetric gauge interaction vertices.
\label{fig:gauge}}
\end{figure}
In Fig.~\ref{fig:gauge} we show in a similar manner the gauge interactions
in a supersymmetric theory.
Figures \ref{fig:gauge}a,b,c occur only when the gauge group is
non-abelian
(e.g. for $SU(3)_C$ color and $SU(2)_L$ weak isospin in the MSSM).
Figures \ref{fig:gauge}a and \ref{fig:gauge}b are the interactions of
gauge bosons which derive from
the first term in eq.~(\ref{lagrgauge}).
In the MSSM these are exactly the same as the
well-known QCD gluon and electroweak gauge boson vertices of the Standard
Model.
(We do not show the interactions of ghost fields, which
are necessary only for consistent loop amplitudes.)
Figures \ref{fig:gauge}c,d,e,f are just the standard interactions between
gauge bosons
and fermion and scalar fields
which must
occur in any gauge theory because of the form of the covariant
derivative; they come from eqs.~(\ref{ordtocovlambda}) and
(\ref{ordtocovphi})-(\ref{ordtocovpsi}) inserted in the kinetic part
of the lagrangian.
Figure \ref{fig:gauge}c shows the coupling of a gaugino
to a gauge boson; the gaugino line in a Feynman diagram is
traditionally drawn as a solid fermion line superimposed on a gauge boson
squiggly line. In Fig.~\ref{fig:gauge}g
we have the coupling of a gaugino to a chiral fermion and a complex
scalar [the first term in the second line in eq.~(\ref{gensusylagr})]. One
can think of
this as the ``supersymmetrization" of
Figure \ref{fig:gauge}e or \ref{fig:gauge}f; any of these three vertices
may be obtained from any
other (up to a factor of ${\sqrt{2}}$) by replacing two
of the particles by their supersymmetric partners. There is also an
interaction like Fig.~\ref{fig:gauge}g but with all arrows reversed,
corresponding to the complex conjugate term in the lagrangian
[the second term in the second line in eq.~(\ref{gensusylagr})]. Finally
in Fig.~\ref{fig:gauge}h we have a scalar quartic interaction vertex [the
last term in
eq.~(\ref{fdpot})] which is also determined by
the gauge coupling.

The results of this section can be used as a recipe for constructing
the supersymmetric interactions for any model. In the case of the MSSM,
we already know the gauge group, particle content and the gauge
transformation
properties, so it only remains to decide on the superpotential. This
we will do in section \ref{subsec:mssm.superpotential}.

\section{Soft supersymmetry breaking interactions}\label{sec:soft}
\setcounter{equation}{0}
\setcounter{footnote}{1}

A realistic phenomenological model must contain supersymmetry
breaking. From a theoretical perspective, we expect that supersymmetry,
if it exists at all, should be an exact symmetry which is spontaneously
broken. In other words,
the ultimate model should have a lagrangian density which
is invariant under supersymmetry, but a vacuum state which is not.
In this way, supersymmetry is hidden
at low energies in a manner exactly analogous to
the fate of the electroweak symmetry in the ordinary
Standard Model.

Many models of spontaneous symmetry breaking have indeed been proposed
and we will mention the basic ideas of some of them in section
\ref{sec:origins}.
These always involve extending the MSSM to include new particles
and interactions at very high mass scales, and there is no consensus
on exactly how this should be done. However, from a practical
point of view, it is extremely useful to simply parameterize our ignorance
of these issues by just introducing extra terms
which break supersymmetry explicitly in the effective MSSM lagrangian.
As was argued in the Introduction, the extra supersymmetry-breaking
couplings should be soft (of positive mass dimension) in
order to be able to naturally maintain a hierarchy between the electroweak
scale and the Planck (or some other very large) mass scale. This means in
particular that we
should not consider any dimensionless supersymmetry-breaking couplings.

In the context of a general renormalizable theory, the possible soft
supersymmetry-breaking terms in the lagrangian
are
\beq
\lagr_{\rm soft}\! &=& \!
-\half \left (M_\lambda\, \lambda^a\lambda^a + \conj \right )
- (m^2)_j^i \phi^{j*} \phi_i
\nonumber\\
&& - \left (\half b^{ij} \phi_i\phi_j
+ {1\over 6}a^{ijk} \phi_i\phi_j\phi_k  + \conj \right
),\qquad\>\>\>\>\>{}
\label{lagrsoft}
\\
\lagr_{{\rm maybe}\>\,{\rm soft}}\! &=& \!
-{1\over 2}c_i^{jk} \phi^{*i}\phi_j\phi_k + \conj
\label{lagrsoftprime}
\eeq
They consist of gaugino masses $M_\lambda$ for each gauge group, scalar
(mass)$^2$ terms
$(m^2)_i^j$ and $b^{ij}$, and (scalar)$^3$ couplings $a^{ijk}$ and
$c_i^{jk}$.
One might wonder why we have not included possible soft mass terms for the
chiral supermultiplet fermions. The reason
is that including such terms would be redundant; they can always be
absorbed into a redefinition of the superpotential and the terms
$(m^2)_i^{j}$ and $c_i^{jk}$.
It has been
shown rigorously that a softly-broken supersymmetric theory with
$\lagr_{\rm soft}$ as given by eq.~(\ref{lagrsoft}) is indeed free of
quadratic divergences in quantum corrections to scalar masses, to all
orders in perturbation theory.\cite{softterms}
The situation is slightly more subtle if one tries to include the
non-analytic (scalar)$^3$ couplings in
$\lagr_{{\rm maybe}\>\,{\rm soft}}$. If any of the
chiral supermultiplets in the theory are completely
uncharged under all gauge
symmetries, then non-zero $c_i^{jk}$ terms can lead to quadratic
divergences, despite the fact that they are formally soft.
Now, this constraint need {\it not} apply to the MSSM,
which does not have any gauge-singlet chiral supermultiplets.
Nevertheless, the possibility of $c_i^{jk}$
terms is nearly always neglected.\cite{cterms}
The real
reason for this is that
it is extremely difficult to construct
any model of spontaneous supersymmetry
breaking in which the $c_i^{jk}$ are not utterly negligibly small.
Equation (\ref{lagrsoft}) is therefore usually taken to be the most
general soft supersymmetry-breaking lagrangian.

It should be clear that $\lagr_{\rm soft}$ indeed breaks supersymmetry,
since it involves only scalars and gauginos, and not their respective
superpartners. In fact, the soft terms in $\lagr_{\rm soft}$ are capable
of giving masses to all of the scalars and gauginos in a theory, even if
the gauge bosons and fermions in chiral supermultiplets are massless
(or relatively light).
The gaugino masses $M_\lambda$ are always allowed by gauge symmetry.
The $(m^2)_j^i$ terms are allowed for $i,j$ such that $\phi_i$, $\phi^{j*}$
transform in complex conjugate representations of each other
under all gauge symmetries; in particular this
is true of course when $i=j$, so every scalar is eligible to get
a mass in this way if supersymmetry is broken.
\begin{figure}
\centerline{\psfig{figure=susysoft.ps,height=.88in}}
\caption{Soft supersymmetry-breaking terms:
(a) Gaugino mass insertion $M_\lambda$;
(b) non-analytic scalar (mass)$^2$ $(m^2)_j^i$;
(c) analytic scalar (mass)$^2$ $b^{ij}$;
(d) (scalar)$^3$ coupling $a^{ijk}$.
\label{fig:soft}}
\end{figure}
The remaining soft terms may or may not be allowed by the symmetries. In
this regard it
is useful to note that the
$b^{ij}$ and $a^{ijk}$ terms have the same form as the
$M^{ij}$ and $y^{ijk}$ terms in the superpotential
[compare eq.~(\ref{lagrsoft}) to eq.~(\ref{superpotential}) or
eq.~(\ref{superpot})], so they will be
allowed by gauge invariance
if and only if a corresponding superpotential term is allowed.
The Feynman diagram interactions corresponding to the allowed soft
terms in eq.~(\ref{lagrsoft}) are shown in Fig.~\ref{fig:soft}.
As before, for each of the interactions in Figs.~\ref{fig:soft}a,c,d
there is one
with all arrows reversed, corresponding to the complex conjugate
term in the lagrangian. We will apply these
general results to the specific case of the MSSM in the next section.

\section{The Minimal Supersymmetric Standard Model}\label{sec:mssm}
\setcounter{equation}{0}
\setcounter{footnote}{1}

In sections \ref{sec:susylagr} and \ref{sec:soft}, we have found a general
recipe for constructing
lagrangians for softly broken supersymmetric theories. We
are now ready to apply these general results to the MSSM.
The particle content for the MSSM was described
in the Introduction.
In this section we will complete the model by specifying the
superpotential and the soft-breaking terms.

\subsection{The superpotential and supersymmetric
interactions}\label{subsec:mssm.superpotential}

The superpotential for the MSSM is given by
\beq
W_{\rm MSSM} =
\sbar u {\bf y_u} Q H_u -
\sbar d {\bf y_d} Q H_d -
\sbar e {\bf y_e} L H_d +
\mu H_u H_d \> .
\label{MSSMsuperpot}
\eeq
The objects $H_u$, $H_d$, $Q$, $L$, $\sbar u$, $\sbar d$, $\sbar e$
appearing in eq.~(\ref{MSSMsuperpot}) are chiral superfields corresponding
to the chiral supermultiplets in Table 1. (Alternatively, they can be
just thought of as the corresponding scalar fields, as was done in
section \ref{sec:susylagr}, but we prefer not to put the tildes on
$Q$, $L$, $\sbar u$, $\sbar d$, $\sbar e$
in order to reduce clutter.)
The dimensionless Yukawa coupling parameters
${\bf y_u}, {\bf y_d}, {\bf y_e}$
are 3$\times 3$ matrices in family space.
Here we have suppressed all of the gauge [$SU(3)_C$ color and
$SU(2)_L$ weak isospin]  and family indices.
The ``$\mu$ term", as it is traditionally called, can be written out as
$\mu (H_u)_\alpha (H_d)_\beta \epsilon^{\alpha\beta}$, where
$\epsilon^{\alpha\beta}$ is used to tie together
$SU(2)_L$ weak isospin indices $\alpha,\beta=1,2$ in a gauge-invariant way.
Likewise, the term $\sbar u {\bf y_u} Q H_u$ can be written out
as $\sbar u_{a}^{i}\, {({\bf y_u})_i}^j\, Q^{a}_{j\alpha}\, (H_u)_\beta
\epsilon^{\alpha \beta}$, where
$i=1,2,3$ is a family index, and
$a=1,2,3$ is a
color index which is raised (lowered) in the $\bf 3$ ($\bf \overline 3$)
representation of $SU(3)_C$.

The $\mu$ term in eq.~(\ref{MSSMsuperpot}) is the supersymmetric version
of the Higgs boson mass in the Standard Model. It is unique,
because terms $H_u^* H_u$ or $H_d^* H_d$ are forbidden in the
superpotential, since it must be analytic in the chiral
superfields (or equivalently in the scalar fields) treated as
complex variables, as shown in section \ref{subsec:susylagr.chiral}.
We can also see from the form of eq.~(\ref{MSSMsuperpot}) why both $H_u$
and $H_d$ are needed in order to give Yukawa couplings, and thus masses,
to all of the quarks and
leptons. Since the superpotential must be analytic,
the $\sbar u Q H_u $ Yukawa terms
cannot be replaced by something like $\sbar u Q H_d^*$.
Similarly, the $\sbar d Q H_d$ and $\sbar e L H_d$ terms cannot be
replaced by
something like $\sbar d Q H_u^*$ and $\sbar e L H_u^*$. The analogous
Yukawa couplings would be allowed in a general non-supersymmetric two
Higgs doublet model,
but are forbidden by the structure of supersymmetry. So we need both
$H_u$ and $H_d$, even without invoking the argument based on anomaly
cancellation
which
was mentioned in the Introduction.

The Yukawa matrices determine the masses and CKM mixing angles of the
ordinary quarks and leptons, after the neutral scalar components
of $H_u$ and $H_d$ get VEVs.
Since the top quark, bottom quark and tau lepton are
the heaviest fermions in the Standard Model, it is often useful to make an
approximation that only
the $(3,3)$ family
components of each of ${\bf y_u}$, ${\bf y_d}$ and ${\bf y_e}$
are important:
\beq
{\bf y_u} \approx \pmatrix{0&0&0\cr 0&0&0 \cr 0&0&y_t};\qquad\!\!
{\bf y_d} \approx \pmatrix{0&0&0\cr 0&0&0 \cr 0&0&y_b};\qquad\!\!
{\bf y_e} \approx \pmatrix{0&0&0\cr 0&0&0 \cr 0&0&y_\tau}.\>\>{}
\label{heavytopapprox}
\eeq
In this limit, only the third family and Higgs fields contribute
to the MSSM superpotential. It is instructive to write the superpotential
in terms of the separate $SU(2)_L$ weak isospin components
[$Q_3 = (t\, b)$;
$L_3 = (\nu_\tau\, \tau)$;
$H_u = (H_u^+\, H_u^0)$;
$H_d = (H_d^0\, H_d^-)$;
$\sbar u_3 = \sbar t$;
$\sbar d_3 = \sbar b$;
$\sbar e_3 = \sbar \tau$], so:
\beq
W_{\rm MSSM}\! &\approx & \!
y_t (\sbar t t H_u^0 - \sbar t b H_u^+) -
y_b (\sbar b t H_d^- - \sbar b b H_d^0) -
y_\tau (\sbar \tau \nu_\tau H_d^- - \sbar \tau \tau H_d^0)
\> \nonumber \\
&& +
\mu (H_u^+ H_d^- - H_u^0 H_d^0).
\label{Wthird}
\eeq
The minus signs inside the parentheses
appear because of the antisymmetry
of the
$\epsilon^{\alpha\beta}$ symbol used to tie up the $SU(2)_L$ indices.
The minus signs in eq.~(\ref{MSSMsuperpot}) were chosen so that the
terms $y_t \sbar t t H_u^0$,
$y_b \sbar b b H_d^0$, and $y_\tau \sbar \tau \tau H_d^0$,
which will become the top, bottom and tau
masses when $H_u^0$ and $H_d^0$ get VEVs, have positive signs
in eq.~(\ref{Wthird}).

Since the Yukawa interactions $y^{ijk}$ in a general
supersymmetric theory must be completely symmetric under interchange
of $i,j,k$, we know that ${\bf y_u}$, ${\bf y_d}$ and ${\bf y_e}$
imply not only Higgs-quark-quark and
Higgs-lepton-lepton couplings as in the Standard Model, but also
squark-Higgsino-quark and slepton-Higgsino-lepton interactions.
To illustrate this,
we show in Figs.~{\ref{fig:topYukawa}}a,b,c some of the
interactions which
involve the top-quark Yukawa coupling $y_t$.
\begin{figure}
\centerline{\psfig{figure=susytopYukawa.ps,height=1in}}
\caption{The top-quark Yukawa coupling (a) and its supersymmetrizations
(b),(c), all of strength $y_t$.
\label{fig:topYukawa}}
\end{figure}
Figure \ref{fig:topYukawa}a is the Standard Model-like coupling
of the top quark to the neutral complex scalar Higgs boson,
which follows from
the first term in eq.~(\ref{Wthird}).
For variety, we have used $t_L$ and
$t_R^\dagger$ in place of their
synonyms $t$ and $\sbar t$ in Fig.~\ref{fig:topYukawa};
see the discussion in the final paragraph in section
\ref{sec:notations}.
In Fig.~\ref{fig:topYukawa}b, we have the coupling of the left-handed
top squark $\stilde t_L$ to the neutral higgsino field ${\stilde H}_u^0$
and right-handed top quark,
while in Fig.~\ref{fig:topYukawa}c the right-handed
top-squark field
(known either as $\stilde {\sbar t}$ or $\stilde t_R^*$ depending on
taste) couples to ${\stilde H}^0_u$ and $t_L$.
For each of the three interactions, there is another with
$H_u^0\rightarrow H_u^+$ and $t_L \rightarrow -b_L$ (with tildes where
appropriate), corresponding to
the second part of the first term in eq.~(\ref{Wthird}).
All of these interactions are required by supersymmetry to have
 the same strength $y_t$.
This is also an
incontrovertible prediction of softly-broken supersymmetry at tree-level,
since these interactions are dimensionless and can be modified
by the introduction of soft supersymmetry breaking only through
finite (and small) radiative corrections.
A useful mnemonic is that each of Figs.~{\ref{fig:topYukawa}}a,b,c can be
obtained from any of
the others by changing two of the particles into their superpartners.

There are also scalar quartic interactions with strength
proportional to $y_t^2$,
as can be seen e.g.~from Fig.~\ref{fig:dim0}b or the last term in
eq.~(\ref{ordpot}).
\begin{figure}
\vskip.2cm
\centerline{\psfig{figure=susystop.ps,height=.97in}}
\caption{Some of the (scalar)$^4$ interactions with strength
proportional to $y_t^2$.
\label{fig:stop}}
\end{figure}
Three of them are shown in Fig.~{\ref{fig:stop}}. The reader is invited
to
check, using eq.~(\ref{ordpot}) and eq.~(\ref{Wthird}), that
there are nine
more, which can
be obtained by replacing $\stilde t_L \rightarrow \stilde b_L$ and/or
$H_u^0 \rightarrow H_u^+$ in each vertex. This illustrates the remarkable
economy of supersymmetry; there are many interactions determined by
only a single parameter!
In a similar way, the existence of all the other quark and lepton Yukawa
couplings in the superpotential eq.~(\ref{MSSMsuperpot}) leads not only to
Higgs-quark-quark
and Higgs-lepton-lepton lagrangian terms as in the
ordinary
Standard Model, but also to squark-higgsino-quark and
slepton-higgsino-lepton terms, and
scalar quartic couplings
[(squark)$^4$, (slepton)$^4$,
(squark)$^2$(slepton)$^2$,
(squark)$^2$(Higgs)$^2$, and (slepton)$^2$(Higgs)$^2$].
If needed, these can all be obtained in terms of the Yukawa matrices
$\bf y_u$, $\bf y_d$, and $\bf y_e$ as outlined above.

However, it is
useful to note that the dimensionless interactions determined
by the superpotential are often
not the most important ones of direct
interest for phenomenology. This is because the Yukawa couplings
are already known to be very small, except for those
of the third family (top, bottom, tau).
Instead, decay and especially production processes
for superpartners in the MSSM are typically dominated by the
supersymmetric interactions of gauge-coupling strength, as we will
explore in more detail in sections \ref{sec:decays} and \ref{sec:signals}.
The couplings of the Standard
Model gauge bosons (photon, $W^\pm$, $Z^0$ and gluons) to the MSSM
particles are determined completely by the gauge invariance of
the kinetic terms in the lagrangian. The gauginos also couple to
(squark, quark) and (slepton, lepton)  and (Higgs, higgsino) pairs
as illustrated in the general case in Fig.~\ref{fig:gauge}g and the second
line
in eq.~(\ref{gensusylagr}). For instance, each of the squark-quark-gluino
couplings is given by $\sqrt{2} g_3 (\stilde q \, T^{a} q \stilde g +
\conj)$
where
$T^a$ ($a=1\ldots 8$) are the Gell-Mann
matrices for $SU(3)_C$. The Feynman diagram for this interaction is
\begin{figure}
\centerline{\psfig{figure=susygaugino.ps,height=1.05in}}
\caption{Couplings of the gluino, wino, and bino to MSSM (scalar,
fermion) pairs.
\label{fig:gaugino}}
\end{figure}
shown in Fig.~\ref{fig:gaugino}a. In
Figs.~\ref{fig:gaugino}b,c we show in a similar way the
couplings of (squark, quark), (lepton, slepton) and (Higgs, higgsino)
pairs to the
winos and bino, with strengths proportional
to the electroweak gauge couplings $g$ and $g^\prime$ respectively.
The winos only couple to the left-handed squarks and
sleptons, and the (lepton, slepton) and (Higgs,
higgsino) pairs of course do not couple to
the gluino. The bino couplings for each (scalar, fermion) pair
are also proportional to the weak hypercharges $Y$ as given in Table 1.
The interactions shown in Fig.~\ref{fig:gaugino}
provide for decays $\stilde q \rightarrow q\stilde g$
and $\stilde q \rightarrow \stilde W q^\prime$ and $\stilde q \rightarrow
\stilde B q$ when the final states are kinematically
allowed to be on-shell. However, a complication is that the
$\stilde W$ and $\stilde B$ states are not mass eigenstates, because
of mixing due to electroweak symmetry breaking, as we will see in
section \ref{subsec:MSSMspectrum.inos}.

There are also various scalar quartic interactions in the MSSM which
are uniquely determined by gauge invariance and supersymmetry,
according to the last term in eq.~(\ref{fdpot})
illustrated in Fig.~\ref{fig:gauge}h.
Among them
are (Higgs)$^4$ terms proportional to
$g^2$ and $g^{\prime 2}$ in the scalar potential.
 These are the direct
generalization of the last term in the Standard
Model Higgs potential, eq.~(\ref{higgspotential}), to the case
of the MSSM. We will have occasion to identify them explicitly
when we discuss the minimization of the MSSM Higgs potential in
section \ref{subsec:MSSMspectrum.Higgs}.

The dimensionful terms in the supersymmetric part of the MSSM lagrangian
are all dependent on $\mu$. Following the general result of
eq.~(\ref{lagrchiral}), we find that $\mu$ provides for higgsino
fermion mass
terms
\beq
\lagr \supset -\mu (\stilde H_u^+ \stilde H_d^- - \stilde H_u^0 \stilde
H_d^0)+ \conj,
\label{poody}
\eeq
as well as Higgs (mass)$^2$ terms in the scalar potential
\beq
-\lagr\,\supset\, V\, \supset\, |\mu|^2 \bigl (
|H_u^0|^2 + |H_u^+|^2 + |H_d^0|^2 + |H_d^-|^2 \bigr ).
\label{movie}
\eeq
Since eq.~(\ref{movie}) is positive-definite, it is clear that
we cannot understand electroweak symmetry breaking without including
supersymmetry-breaking (mass)$^2$ soft terms for the Higgs
scalars, which can be negative.
An explicit treatment of the Higgs scalar potential will therefore
have to wait until we have introduced the soft terms for the MSSM.
However, we can already see a puzzle:
we expect that $\mu$ should be roughly of order $10^2$ or $10^3$ GeV,
in order to allow a Higgs VEV of order 174 GeV without too much
miraculous cancellation between $|\mu|^2$ and the negative soft (mass)$^2$
terms that we have not written down yet. But why should $\mu$ be so small
compared to, say,
$\MPlanck$, and in particular why should it be roughly of the same
order as $m_{\rm soft}$? The
scalar potential of the MSSM seems to depend on two types
of dimensionful parameters which are conceptually quite distinct,
namely
the supersymmetry-respecting mass $\mu$ and the supersymmetry-breaking
soft mass terms. Yet the observed value for the electroweak
breaking scale suggests that without miraculous
cancellations, both
of these apparently unrelated mass scales should be within an
order of magnitude or so of 100 GeV. This puzzle is called ``the
$\mu$ problem". Several different solutions to the $\mu$ problem have been
proposed, involving extensions of the MSSM of varying intricacy.
They all work in roughly the same way; the parameter
$\mu$ is required or assumed to be completely absent at tree-level, and is
to be replaced by the
VEV(s) of some new field(s). The latter
are in turn determined by minimizing a potential which depends on
soft supersymmetry-breaking terms. In this way, the value of the effective
parameter
$\mu$ is no longer conceptually distinct from the mechanism of
supersymmetry breaking; if we can explain why $m_{\rm soft} \ll \MPlanck$,
we will also be able to understand why $\mu$ is of the same order. In
section \ref{subsec:variations.NMSSM} we will
describe one such mechanism. Some other attractive
solutions for the $\mu$ problem
are proposed in Refs.\cite{muproblemW,muproblemK,muproblemGMSB} From
the
point of view of the MSSM, however, we can just treat
$\mu$ as an independent parameter.

The $\mu$-term and the Yukawa couplings in the superpotential
eq.~(\ref{MSSMsuperpot})
combine to yield (scalar)$^3$ couplings [see the second and third terms on
the right-hand side of eq.~(\ref{ordpot})] of the form
\beq
\lagr &\! \supset \! &
\mu^* (
 {\stilde{\sbar u}} {\bf y_u} \stilde u H_d^{0*}
+ {\stilde{\sbar d}} {\bf y_d} \stilde d H_u^{0*}
+ {\stilde{\sbar e}} {\bf y_e} \stilde e H_u^{0*}
\cr
&&
{}\>\>\>
+{\stilde{\sbar u}} {\bf y_u} \stilde d H_d^{-*}
+{\stilde{\sbar d}} {\bf y_d} \stilde u H_u^{+*}
+{\stilde{\sbar e}} {\bf y_e} \stilde \nu H_u^{+*}
)
+ \conj
\label{striterms}
\eeq
In Fig.~\ref{fig:stri} we show some of these couplings which are
\begin{figure}
\centerline{\psfig{figure=susystri.ps,height=.97in}}
\caption{Some of the supersymmetric (scalar)$^3$ couplings proportional to
$\mu^* y_t$, $\mu^* y_b$, and $\mu^* y_\tau$.
\label{fig:stri}}
\end{figure}
proportional to $\mu^* y_t$, $\mu^* y_b$, and $\mu^* y_\tau$ respectively.
These play an important role in determining the mixing of top squarks,
bottom squarks, and tau sleptons, as we will see in section
\ref{subsec:MSSMspectrum.sfermions}.

\subsection{$R$-parity (also known as matter parity) and its
consequences}\label{subsec:mssm.rparity}

The superpotential eq.~(\ref{MSSMsuperpot}) is minimal in the sense
that it is sufficient to produce a phenomenologically viable model.
However, there are other terms that one could write down which are
gauge-invariant and analytic in the chiral superfields, but are not
included in the MSSM because
they violate either baryon number ($\Baryon$) or total lepton number
($\Lepton$).
The most general gauge-invariant and renormalizable superpotential would
include not only eq.~(\ref{MSSMsuperpot}), but also the terms
\beq
&&W_{\Delta {\rm L} =1} =
{1\over 2} \lambda^{ijk} L_iL_j{\sbar e_k}
+ \lambda^{\prime ijk} L_i Q_j {\sbar d_k}
+ \mu^{\prime i} L_i H_u
\label{WLviol} \\
&&W_{\Delta {\rm B}= 1} = {1\over 2} \lambda^{\prime\prime ijk}
{\sbar u_i}{\sbar d_j}{\sbar d_k}
\label{WBviol}
\eeq
where we have restored family indices $i=1,2,3$.
The chiral supermultiplets carry baryon number assignments $\Baryon=+1/3$
for $Q_i$;
$\Baryon=-1/3$ for $\sbar u_i, \sbar d_i$; and $\Baryon=0$ for all others.
The total lepton number assignments are
$\Lepton=+1$ for $L_i$, $\Lepton=-1$ for $\sbar e_i$, and $\Lepton=0$ for
all others.
Therefore, the terms
in eq.~(\ref{WLviol}) violate total lepton number by 1 unit (as well as the
individual lepton flavors) and those in eq.~(\ref{WBviol}) violate baryon
number by 1 unit.

The possible existence of such terms might seem rather disturbing, since
$\Baryon$- and $\Lepton$-violating processes have never been seen
experimentally.
The most obvious experimental constraint comes from the non-observation
of proton decay, which would violate both $\Baryon$ and $\Lepton$ by 1
unit.
If both $\lambda^\prime$ and $\lambda^{\prime\prime}$ couplings were
present and of order unity,
then the lifetime of the proton would be measured in minutes or
\begin{figure}
\centerline{\psfig{figure=susyprotondecay.ps,height=1.1in}}
\caption{Squarks can mediate disastrously rapid proton
decay if $R$-parity is violated.
\label{fig:protondecay}}
\end{figure}
hours! For example, the Feynman graph in Fig.~\ref{fig:protondecay} would
lead to $p^+
\rightarrow e^+ \pi^0$ or $e^+ K^0$ or $\mu^+ \pi^0$ or $\mu^+ K^0$
or $\nu \pi^+$
or $\nu K^+$ etc.~depending on which components of
$\lambda^{\prime}$ are largest,
and these processes would seem to be completely  unsuppressed since the
necessary couplings
are all renormalizable.
(The coupling $\lambda^{\prime\prime}$ must be antisymmetric in
its last two flavor
indices,
since the color indices are contracted antisymmetrically. That is why
the squark in Fig.~\ref{fig:protondecay} is $\stilde{\sbar s}$ or
$\stilde{\sbar b}$
but not $\stilde{\sbar d}$, for $u,d$ quarks in the initial state.)
In contrast, the decay time of the proton
into these modes is measured to be in excess of $10^{32}$ years.
Many other processes also give very significant constraints
on the
violation of lepton and baryon numbers; these are reviewed in
Ref.\cite{rparityconstraints}

One could simply try to take $\Baryon$ and $\Lepton$ conservation as a
postulate
in the MSSM. However, this is clearly a
step backwards from the situation in the Standard Model,
where the conservation of these quantum numbers is {\it not} assumed, but
is rather a pleasantly ``accidental"
consequence of the fact that there are no possible
renormalizable lagrangian terms which violate $\Baryon$ or $\Lepton$.
Furthermore, there is a quite general obstacle to treating $\Baryon$ and
$\Lepton$ as
fundamental symmetries of nature, since they are known to be
necessarily violated
by non-perturbative electroweak effects (even though those effects are
calculably negligible for experiments at ordinary energies).
Therefore, in the MSSM one adds a new symmetry which
has the effect of eliminating the possibility of $\Baryon$ and $\Lepton$
violating
terms in the renormalizable superpotential, while allowing the good
terms in eq.~(\ref{MSSMsuperpot}). This new symmetry is called
``$R$-parity" \cite{Rparity}  or equivalently
``matter parity".\cite{matterparity}

Matter parity is a multiplicatively
conserved quantum number
defined as
\beq
P_M = (-1)^{3 (\Baryon-\Lepton)}
\label{defmatterparity}
\eeq
for each particle in the theory.
It is easy to check that the quark and lepton
supermultiplets all have $P_M=-1$, while the Higgs supermultiplets $H_u$
and $H_d$ have $P_M=+1$.
The gauge bosons and gauginos of course do not carry baryon number or
lepton number, so they are assigned matter parity $P_M=+1$.
The symmetry principle to be enforced is
that a term in the Lagrangian (or in the superpotential) is allowed only if
the product of $P_M$ for all of the fields in it is $+1$. It is easy
to see that each of the terms in
eq.~(\ref{WLviol}) and (\ref{WBviol}) is thus forbidden, while the
good and necessary
terms in eq.~(\ref{MSSMsuperpot}) are allowed.
This discrete symmetry commutes with supersymmetry,
as all members of a given supermultiplet have the same matter parity.
The advantage of matter
parity is that it can in principle be an {\it exact} and
fundamental symmetry, which
B and L themselves cannot, since they are known to be violated by
non-perturbative electroweak effects. So even with exact matter parity
conservation in the MSSM,
one expects that baryon number
and total lepton number violation will occur in very tiny amounts, due to
nonrenormalizable
terms in the Lagrangian. However, the
MSSM does not have renormalizable interactions that violate B or
L, with the standard assumption of matter parity conservation.

It is sometimes useful to recast matter parity in terms of $R$-parity,
defined for each particle as
\beq
P_R = (-1)^{3(\Baryon-\Lepton) + 2 s}
\label{defRparity}
\eeq
where $s$ is the spin of the particle. Now,
matter parity conservation and $R$-parity conservation are precisely
equivalent, since the product of $(-1)^{2s}$ is of course equal to $+1$
for the particles involved in any interaction
vertex in a theory that conserves angular momentum.
However, particles within the same supermultiplet do not have the
same $R$-parity. In general, symmetries with the
property that particles within the same multiplet have different
charges are called
$R$ symmetries; they do not commute with
supersymmetry.  Continuous $U(1)$ $R$ symmetries are 
often encountered in the model-building literature; they should not be 
confused with $R$-parity, which is a discrete $Z_2$ symmetry.
In fact, the matter parity version 
of $R$-parity makes clear
that there is really nothing intrinsically
``$R$" about it; in other words it
secretly does commute with supersymmetry, so its name is somewhat
suboptimal. Nevertheless, the $R$-parity assignment
is very useful for phenomenology because all of the Standard Model particles
and the Higgs bosons have even $R$-parity ($P_R=+1$), while all of the
squarks, sleptons, gauginos, and higgsinos have odd $R$-parity ($P_R=-1$).

The $R$-parity odd particles are known as ``supersymmetric
particles" or ``sparticles" for short, and they are distinguished
by a tilde (see Tables 1 and 2). If $R$-parity is exactly conserved,
then
there can be no mixing between the sparticles and the $P_R=+1$
particles. Furthermore, every interaction vertex in the theory contains
an even number of $P_R=-1$ sparticles. This has three extremely important
phenomenological consequences:
\begin{itemize}
\item[$\bullet$]
The lightest sparticle with $P_R=-1$, called the
``lightest supersymmetric particle" or LSP, must be absolutely stable.
If the LSP is electrically neutral, it interacts only weakly with
ordinary matter, and so can make an attractive candidate
\cite{neutralinodarkmatter}
for the non-baryonic dark matter which seems
to be required by cosmology.

\item[$\bullet$]
Each sparticle other than the LSP must eventually
decay into a state which contains an odd number of LSPs
(usually just one).
\item[$\bullet$]
In collider experiments, sparticles can only be
produced in even numbers (usually two-at-a-time).
\end{itemize}

We {\it define} the MSSM to conserve $R$-parity or equivalently matter
parity.
While this decision seems to be well-motivated phenomenologically
by proton decay constraints and the hope that the LSP will provide
a good dark matter candidate, it might appear somewhat {\it ad hoc} from
a theoretical point of view. After all, the MSSM would not suffer
any internal inconsistency if we did not impose matter parity
conservation. Furthermore, it is fair to ask why matter parity
should be exactly conserved, given that the known discrete symmetries
in the Standard Model (ordinary parity $P$, charge conjugation $C$,
time reversal $T$, etc.) are all known to be inexact symmetries.
Fortunately, it {\it is} sensible to formulate matter parity as a discrete
symmetry which is exactly conserved.
In general, exactly conserved, or ``gauged" discrete symmetries
\cite{KW} can
exist provided that they
satisfy certain
anomaly cancellation conditions \cite{discreteanomaly} (much like
continuous gauged symmetries).
One particularly
attractive way this could occur is if B$-$L is a continuous $U(1)$
gauge symmetry which is spontaneously broken at some very high energy
scale. From eq.~(\ref{defmatterparity}), we observe that $P_M$ is
actually a discrete subgroup of the continuous $U(1)_{\Baryon - \Lepton}$
group.
Therefore, if gauged
$U(1)_{\Baryon - \Lepton}$ is broken by scalar VEVs (or other order
parameters) which carry only even integer values of $3($B$-$L$)$, then
$P_M$ will automatically survive as an exactly conserved
remnant.
A variety of extensions of the MSSM in which exact $R$-parity arises
in just this way have been proposed.\cite{Rparityorigin1,Rparityorigin2}
It may also be possible to have gauged
discrete symmetries which do not owe
their exact conservation to an underlying continuous gauged symmetry,
but rather to some other structure such as can occur in string theory.
It is also possible that $R$-parity is broken, or is
replaced by some alternative
discrete symmetry.
We will briefly consider these as
variations on the MSSM in section
\ref{subsec:variations.Rparity}.

\subsection{Soft supersymmetry breaking in the
MSSM}\label{subsec:mssm.soft}

To complete the description of the MSSM, we need to specify the
soft supersymmetry breaking terms. In section \ref{sec:soft}, we learned
how
to write down the most general set of such terms in any supersymmetric
theory. Applying this recipe to the MSSM, we have:
\beq
\lagr_{\rm soft}^{\rm MSSM} &=& -\half\left ( M_3 \stilde g\stilde g
+ M_2 \stilde W \stilde W + M_1 \stilde B\stilde B \right )+\conj
\nonumber
\\
&&
-\left ( \stilde {\sbar u} \,{\bf a_u}\, \stilde Q H_u
- \stilde {\sbar d} \,{\bf a_d}\, \stilde Q H_d
- \stilde {\sbar e} \,{\bf a_e}\, \stilde L H_d
\right ) + \conj
\nonumber
\\
&&
-\stilde Q^\dagger \, {\bf m^2_{Q}}\, \stilde Q
-\stilde L^\dagger \,{\bf m^2_{L}}\,\stilde L
-\stilde {\sbar u} \,{\bf m^2_{{\sbar u}}}\, {\stilde {\sbar u}}^\dagger
-\stilde {\sbar d} \,{\bf m^2_{{\sbar d}}} \, {\stilde {\sbar d}}^\dagger
-\stilde {\sbar e} \,{\bf m^2_{{\sbar e}}}\, {\stilde {\sbar e}}^\dagger
\nonumber \\
&&
- \, m_{H_u}^2 H_u^* H_u - m_{H_d}^2 H_d^* H_d
- \left ( b H_u H_d + \conj \right ) .
\label{MSSMsoft}
\eeq
In eq.~(\ref{MSSMsoft}), $M_3$, $M_2$, and $M_1$ are the gluino, wino,
and bino mass terms.
Here, and from now on, we
suppress the adjoint representation gauge indices on the wino
and gluino fields, and the gauge indices on all of the chiral
supermultiplet fields. The
second line in eq.~(\ref{MSSMsoft}) contains the
(scalar)$^3$ couplings [of the type $a^{ijk}$ in eq.~(\ref{lagrsoft})].
Each of
${\bf a_u}$,
${\bf a_d}$,
${\bf a_e}$
is a complex $3\times 3$ matrix in family space, with dimensions of
(mass). They are in one-to-one correspondence with the Yukawa coupling
matrices in the superpotential.
The third line of eq.~(\ref{MSSMsoft}) consists
of squark and slepton mass terms of the $(m^2)_i^j$ type in
eq.~(\ref{lagrsoft}). Each of
${\bf m^2_{ Q}}$,
${\bf m^2_{{\sbar u}}}$,
${\bf m^2_{{\sbar d}}}$,
${\bf m^2_{L}}$,
${\bf m^2_{{\sbar e}}}$ is a $3\times 3$ matrix
in family space which can have complex entries, but they
must be hermitian
so that the lagrangian is real. (To avoid clutter, we do not
put tildes on the $\bf Q$ in $\bf m^2_Q$, etc.)
Finally, in the last line of
eq.~(\ref{MSSMsoft}) we have supersymmetry-breaking contributions to
the Higgs potential; $m_{H_u}^2$ and $m_{H_d}^2$ are (mass)$^2$ terms
of the $(m^2)_i^j$ type, while $b$ is the only (mass)$^2$ term of the type
$b^{ij}$ in eq.~(\ref{lagrsoft}) which can occur in the
MSSM.\footnote{The parameter we call $b$ is often seen in the
literature as $m_{12}^2$ or $m_3^2$ or $B\mu$.}
Schematically, we can write
\beq
&&\!\!\!\!\! M_1,\, M_2,\, M_3,\, {\bf a_u},\, {\bf a_d},\, {\bf a_e}\,
\sim\, m_{\rm soft};\\
&&\!\!\!\!\! {\bf m^2_{ Q}},\,
{\bf m^2_{L}},\,
{\bf m^2_{{\sbar u}}},\,
{\bf m^2_{{\sbar d}}},\,
{\bf m^2_{{\sbar e}}},\, m_{H_u}^2,\, m_{H_d}^2,\, b\, \sim \,
m_{\rm soft}^2
\eeq
with a characteristic mass scale $m_{\rm soft}$ which is not  much
larger than $10^3$ GeV, as argued in the Introduction.
The expression eq.~(\ref{MSSMsoft}) is the most general soft
supersymmetry-breaking Lagrangian of the form eq.~(\ref{lagrsoft})
which is compatible with gauge invariance and matter parity conservation.

Unlike the supersymmetry-preserving part of the lagrangian,
$\lagr_{\rm soft}^{\rm MSSM}$ introduces many new parameters which were
not present in the ordinary Standard Model.
A careful count \cite{dimsut} reveals that there are 105 masses, phases and 
mixing angles in the MSSM lagrangian which cannot be rotated away
by redefining the phases and flavor basis for the quark and lepton
supermultiplets, and which have no counterpart in the ordinary
Standard Model.
Thus, in principle, supersymmetry (or more precisely,
supersymmetry {\it breaking}) appears to introduce a tremendous
arbitrariness in the lagrangian.

\subsection{Hints of an Organizing Principle}\label{subsec:mssm.hints}

Fortunately, there is already
good experimental evidence that some sort of powerful ``organizing
principle" must govern the soft terms. This is because most of the
new parameters in eq.~(\ref{MSSMsoft}) involve flavor mixing or
CP violation of the type which is already severely restricted by
experiment.\cite{flavorreview}
For example, suppose that $\bf m_{\sbar e}^2$ is not diagonal in a basis
$(\stilde e_R, \stilde \mu_R, \stilde \tau_R)$ of sleptons whose
superpartners are the right-handed pieces of the Standard Model
mass
eigenstates $e,\mu,\tau$.
In that case slepton mixing occurs, and the individual lepton numbers
will not be conserved.
This is true even for processes which only involve the sleptons
as virtual particles. A
particularly strong limit on this possibility
comes from the experimental constraint on $\mu\rightarrow e \gamma$,
\cite{muegamma}
which can occur via the one-loop diagram in Fig.~\ref{fig:flavor}a
\begin{figure}
\centerline{\psfig{figure=susyflavor.ps,height=1.2in}}
\caption{Diagrams which cause flavor violation in models with
arbitrary soft masses.
\label{fig:flavor}}
\end{figure}
featuring a virtual bino and slepton. The cross represents
an insertion of
$\lagr_{\rm soft}^{\rm MSSM} \supset
-({\bf m^2_{\sbar e}})_{21}\stilde e_R \stilde \mu^*_R$, and the
slepton-bino
vertices are determined by the weak hypercharge gauge coupling
[see Fig.~\ref{fig:gauge}g and eq.~(\ref{gensusylagr})]. There are similar
diagrams if the left-handed slepton mass matrix
$\bf m^2_L$ has arbitrary off-diagonal entries. If $\bf m^2_L$ or
${\bf m^2_{\sbar e}}$ were ``random", with all entries of comparable size,
then the contributions to BR($\mu\rightarrow e\gamma)$ would
be about 5 or 6 orders of magnitude larger than the current experimental
upper limit of $5\times 10^{-11}$, even if the sleptons are as heavy as 1
TeV. Therefore the form of the slepton mass matrices must be severely
constrained.

There are also important
experimental constraints on the squark (mass)$^2$ matrices.
The strongest of these come from the neutral kaon system. The
effective hamiltonian for $K^0\leftrightarrow \overline K^0$
mixing gets
contributions from the diagram in
Fig.~\ref{fig:flavor}b, among others, if $\lagr_{\rm soft}^{\rm MSSM}$
contains (mass)$^2$ terms
which mix down squarks and strange squarks.
The gluino-squark-quark vertices in Fig.~\ref{fig:flavor}b are all
fixed by supersymmetry to be of strong interaction strength; there
are similar diagrams in which the bino and winos are
exchanged.\cite{mixwinoex}
If the squark and gaugino masses are of order 1 TeV or less,
one finds that limits on the parameters $\Delta m_K$ and $\epsilon_K$
appearing in the neutral kaon system effective hamiltonian severely
restrict the amount of down-strange squark mixing and
CP-violating complex phases that one can
tolerate in the soft parameters.\cite{morestuff} Considerably weaker, but
still
interesting, constraints come from the
$D^0,
\overline D^0$
and
$B^0,
\overline B^0$
neutral meson systems, and the decay $b\rightarrow s\gamma$.\cite{bsgamma}
 After the
Higgs scalar fields get VEVs, the
$\bf a_u$,
$\bf a_d$,
$\bf a_e$ matrices contribute off-diagonal squark and slepton (mass)$^2$
terms [for example, ${\stilde{\sbar d}} {\bf a_d}
{\stilde Q} H_d + \conj$ $\rightarrow
({\bf a_d})_{12} \langle H_d^0 \rangle \stilde s_L \stilde d_R^* + \conj$,
etc.],
so
their form is also strongly constrained by
flavor-changing neutral current (FCNC) limits.
There are other significant constraints on CP-violating phases in the
gaugino masses and (scalar)$^3$ soft couplings following from
limits on the electric dipole moments of the neutron and
electron.\cite{demon}

All of these
potentially dangerous FCNC and CP-violating effects in the
MSSM
can be evaded if one assumes (or can explain!) that supersymmetry breaking
should be suitably ``universal". In particular, one can
suppose that
the squark and slepton (mass)$^2$ matrices are flavor-blind.
This means that they should each be
proportional to
the $3\times 3$ identity matrix in family space:
\beq
{\bf m^2_{Q}} = m^2_{Q} {\bf 1};
\qquad\!\!\!\!\!\!
{\bf m^2_{\sbar u}} = m^2_{\sbar u} {\bf 1};
\qquad\!\!\!\!\!\!
{\bf m^2_{\sbar d}} = m^2_{\sbar d} {\bf 1};
\qquad\!\!\!\!\!\!
{\bf m^2_{L}} = m^2_{L} {\bf 1};
\qquad\!\!\!\!\!\!
{\bf m^2_{\sbar e}} = m^2_{\sbar e} {\bf 1}
.\>\>\>\>{}
\label{scalarmassunification}
\eeq
If so, then all squark and slepton mixing angles are rendered
trivial, because squarks and sleptons with the same electroweak
quantum numbers will be degenerate in mass
and can be rotated into each other at will.
Supersymmetric contributions to FCNC processes
will therefore be very small in such an idealized limit, modulo the mixing
due to
$\bf a_u$, $\bf a_d$, $\bf a_e$.
One can make the further assumption that the (scalar)$^3$
couplings are each proportional to the corresponding Yukawa coupling
matrix:
\beq
{\bf a_u} = A_{u0} \,{\bf y_u}; \qquad
{\bf a_d} = A_{d0} \,{\bf y_d}; \qquad
{\bf a_e} = A_{e0} \,{\bf y_e}.
\label{aunification}
\eeq
This ensures that only the squarks and sleptons of the third family
can have large (scalar)$^3$ couplings.
Finally, one can avoid disastrously large CP-violating effects with
the assumption that the soft parameters do not introduce new complex
phases. This
is automatic
for $m_{H_u}^2$ and $m_{H_d}^2$, and for $m_Q^2$, $m_{\sbar u}^2$
etc.~if
eq.~(\ref{scalarmassunification}) is assumed; if they were not real
numbers, the
lagrangian would not
be real. One can also fix $\mu$ in the superpotential and $b$ in
eq.~(\ref{MSSMsoft}) to be real, by an appropriate phase rotation of
$H_u$ and $H_d$. If one then assumes that
\beq
\argh (M_1),\, \argh (M_2),\, \argh (M_3),\,
\argh (A_{u0}),\, \argh (A_{d0}),\, \argh (A_{e0}) = 0\,\,{\rm or}\,\,\pi,
\qquad{}
\label{commonphase}
\eeq
then the only CP-violating phase in the theory will be the ordinary
CKM phase found in the ordinary Yukawa couplings.
Together, the conditions
eqs.~(\ref{scalarmassunification})-(\ref{commonphase})
make up a
rather weak version of what
is often called the assumption of {\it soft-breaking universality}.

The soft-breaking universality
relations
eqs.~(\ref{scalarmassunification})-(\ref{commonphase})
(or stronger versions of them)
are presumed to be the result of some specific
model for the origin of supersymmetry breaking, even though there
is considerable disagreement among theorists as to what the specific
model should actually be. In any case, they are indicative of an
underlying simplicity or symmetry of the lagrangian at some very
high energy scale $Q_0$, which we will call the ``input scale".
If we use this lagrangian to compute masses and cross-sections and decay
rates for experiments at ordinary energies near the electroweak scale,
the results will involve large logarithms of order
ln$(Q_0/m_Z)$ coming from loop diagrams. As is usual in quantum field
theory, the
large
logarithms can be conveniently resummed using renormalization group
(RG) equations, by treating the couplings and masses appearing in the
lagrangian as ``running" parameters.
Therefore,
eqs.~(\ref{scalarmassunification})-(\ref{commonphase})
should be interpreted as boundary conditions on the running soft
parameters
at the RG scale $Q_0$ which is very far removed from direct experimental
probes.
We must then RG-evolve all of the soft parameters,
the superpotential parameters, and the gauge couplings down to
the electroweak scale or comparable scales where humans perform
experiments.

At the electroweak scale,
eqs.~(\ref{scalarmassunification}) and (\ref{aunification})
will no longer hold. However, RG corrections due to
gauge interactions will respect eqs.~(\ref{scalarmassunification})
and (\ref{aunification}), while RG corrections due to Yukawa
interactions are quite small except for couplings involving
the top squarks
(stops) and
possibly
the bottom squarks (sbottoms) and tau sleptons (staus). In particular,
the (scalar)$^3$ couplings should be quite negligible for the
squarks and sleptons of the first two families. Furthermore, RG evolution
does not introduce new CP-violating phases. Therefore, if universality can
be arranged to hold at the input scale, supersymmetric contributions to
FCNC and CP-violating observables can be acceptably small in comparison
to present limits (although quite possibly measurable
in future experiments).

One good reason to be optimistic that such a program can succeed
is the celebrated apparent unification of gauge
couplings
in the MSSM. \cite{gaugeunification}
The 1-loop RG equations for the
Standard Model gauge couplings $g_1, g_2, g_3$ are given by
\beq
{d\over dt} g_a =  {1\over 16\pi^2} b_a g_a^3 \qquad
\! \Rightarrow \qquad \! {d\over dt} \alpha_a^{-1} =
-{b_a\over 2\pi} \qquad\qquad (a=1,2,3)
\qquad
\label{mssmg}
\eeq
where $t= {\rm ln} (Q/Q_0)$ with $Q$ the RG scale.
In the Standard Model,
$b^{\rm SM}_a =$ $(41/10,$ $-19/6,$ $-7)$, while in the MSSM
one finds instead
$b^{\rm MSSM}_a =$ $(33/5,$ $1,$ $-3)$.
The latter set of coefficients are
larger because of the virtual effects of the extra MSSM particles in
loops.
The normalization for
$g_1$ here is chosen to agree with the canonical covariant derivative for
grand
unification of the gauge group
$SU(3)_C \times SU(2)_L\times U(1)_Y$
into $SU(5)$ or $SO(10)$.
Thus in terms of the
conventional electroweak gauge couplings $g$ and $g^\prime$ with
$e = g\sin\theta_W = g^\prime \cos\theta_W$,
one has $g_2=g$ and $g_1 = \sqrt{5/3} g^\prime$. The quantities $\alpha_a
= g_a^2/4\pi$ have the nice property that their reciprocals run linearly
\begin{figure}
\centerline{\psfig{figure=susygaugeunification.eps,height=2.7in}}
\caption{RG evolution of the inverse gauge couplings
$\alpha_a^{-1}(Q)$ in the Standard Model (dashed lines) and the MSSM
(solid lines). In the MSSM case, $\alpha_3(m_Z)$ is varied between
$0.113$ and $0.123$, and the sparticle mass thresholds between
250 GeV and 1 TeV. Two-loop effects are included.
\label{fig:gaugeunification}}
\end{figure}
with RG scale at one-loop order.
In Fig.~\ref{fig:gaugeunification} we compare
the RG evolution of the $\alpha_a^{-1}$, including two-loop effects, in
the
Standard Model (dashed lines) and the MSSM (solid lines).
Unlike the Standard Model, the MSSM includes just the right
particle content to ensure that the gauge couplings can unify, at a scale
$M_U \sim 2\times 10^{16}$ GeV.
While the apparent unification of gauge couplings at $M_U$ could be just
an accident,
it may also be taken as a strong hint in favor of a grand unified theory
(GUT) or superstring models, both of which indeed predict gauge coupling
unification below $\MPlanck$. Furthermore, if we take this hint
seriously, then it means that we can reasonably expect to apply a similar
RG analysis to the other MSSM couplings and soft masses as well.

We must mention that there are two other possible
types of explanations for
the suppression of FCNCs in the MSSM, which
could replace the universality hypothesis of
eqs.~(\ref{scalarmassunification})-(\ref{commonphase}).
One might refer to them as
``irrelevancy" and ``alignment" of the soft masses. The ``irrelevancy"
idea is that the sparticles masses are simply {\it extremely}
heavy, so that
their contributions to FCNC and CP-violating diagrams like
Figs.~\ref{fig:flavor}a,b are highly suppressed. In practice, however, the
degree of suppression needed often requires $m_{\rm soft} \gg 1$
TeV for at least some of the scalar masses; this seems to go directly
against the motivation for supersymmetry
as a cure for the hierarchy problem as discussed in the
Introduction. Nevertheless, it is possible to arrange a scheme where
this can work in a sensible way.\cite{Moreminimal} The ``alignment" idea
is that the squark (mass)$^2$ matrices do not have the flavor-blindness
indicated in eq.~(\ref{scalarmassunification}), but are arranged in flavor
space to be aligned with the relevant Yukawa matrices in just such a way
as to avoid large FCNC effects.\cite{cterms,alignmentmodels} The alignment
models typically require
rather special flavor symmetries.
In any case, we will not discuss these possibilities
further.

In practice, a given model for the origin of supersymmetry
breaking may make predictions for the MSSM soft terms
that are even stronger than
eqs.~(\ref{scalarmassunification})-(\ref{commonphase}).
 In the next section we will discuss the ideas that go
into making such predictions, before turning to their implications
for the MSSM spectrum in section \ref{sec:MSSMspectrum}.

\section{Origins of supersymmetry breaking}\label{sec:origins}
\subsection{General considerations for
supersymmetry breaking}\label{subsec:origins.general}
\setcounter{equation}{0}
\setcounter{footnote}{1}

In the MSSM, supersymmetry breaking is simply introduced explicitly.
However, we have seen that the soft parameters cannot be arbitrary.
In order to understand how patterns like
eqs.~(\ref{scalarmassunification}), (\ref{aunification}) and
(\ref{commonphase}) can emerge, it is necessary to consider
models in which supersymmetry is spontaneously broken. By definition,
this means that the vacuum state $\vac$ is not invariant under
supersymmetry transformations, so $Q_\alpha \vac \not= 0$ and
$Q^\dagger_{\dot{\alpha}}\vac \not=0$. Now, in global supersymmetry,
the Hamiltonian operator $H$ can be related to the supersymmetry
generators through the algebra eq.~(\ref{nonschsusyalg1}):
\beq
H=P^0 =
{1\over 4}( Q_1 Q_{{1}}^\dagger + Q_{{1}}^\dagger Q_1 + Q_2
Q_{{2}}^\dagger +
Q_{{2}}^\dagger Q_2 ) .
\eeq
If supersymmetry is unbroken in the vacuum state, it follows that
$H\vac = 0$ and the vacuum has zero energy. Conversely, if supersymmetry
is spontaneously broken in the vacuum state, then
the vacuum must have positive energy, since
\beq
\antivac H \vac = {1\over 4} \Bigl (\| Q_1 \vac \|^2 +
\| Q^\dagger_{{1}} \vac \|^2
+ \| Q_{2} \vac \|^2
+ \| Q^\dagger_{{2}} \vac \|^2
\Bigr ) > 0
\eeq
if the Hilbert space is to have positive norm. If
spacetime-dependent effects and fermion condensates
can be neglected, then
$\antivac H\vac = \antivac V \vac $, where $V$ is the scalar
potential in eq.~(\ref{fdpot}).
Therefore supersymmetry will be spontaneously broken if
$F_i$ and/or $D^a$ does not vanish in the ground state. Note that if
any state exists in which all $F_i$ and $D^a$ vanish, then it will
have zero energy, implying that supersymmetry cannot be
spontaneously broken in the true ground state. Therefore
the way to achieve spontaneous
supersymmetry breaking is to look for models in which the equations
$F_i=0$ and $D^a=0$ cannot be simultaneously satisfied for {\it any}
values of
the fields.

Supersymmetry breaking with non-zero $D$-terms can be achieved through
the Fayet-Iliopoulos mechanism.\cite{FayetIliopoulos} 
If the gauge symmetry includes a $U(1)$ factor, then one can introduce
a term linear in the corresponding auxiliary
field of the gauge supermultiplet:
\beq
\lagr_{\rm Fayet-Iliopoulos}
= \kappa D
\label{FI}
\eeq
where $\kappa$ is a constant parameter with dimensions of (mass)$^2$.
This
term is gauge-invariant and supersymmetric by itself.
[Note that the supersymmetry transformation $\delta D$
in eq.~(\ref{Dtransf})
is a total derivative
for a $U(1)$ gauge symmetry.]
If we include it in the lagrangian, then $D$ may get a non-zero
VEV, depending on the other interactions of the scalar fields
that are charged under the $U(1)$. To see
this, we can write the
relevant part of the scalar potential
using eqs.~(\ref{lagrgauge}) and (\ref{gensusylagr})
as
\beq
V = {1\over 2} D^2 - \kappa D + g D \sum_i q_i \phi^{*i} \phi_i
\eeq
where the $q_i$ are the charges of the scalar fields $\phi_i$
under the $U(1)$ gauge group in question.
The presence of the Fayet-Iliopoulos term %eq.~(\ref{FI}) 
modifies the
equation of motion eq.~(\ref{solveforD}) to
\beq
D = \kappa - g \sum_i q_i \phi^{*i} \phi_i.
\label{booya}
\eeq
Now suppose that the
scalar fields $\phi_i$ have other interactions (such as large
superpotential mass terms) which prevent them from getting VEVs.
Then the auxiliary field $D$ will be forced to get a VEV equal to
$\kappa$,
and supersymmetry will be broken. This mechanism cannot work
for non-abelian gauge groups, however, since the analog of eq.~(\ref{FI})
would not be
gauge-invariant.

In the MSSM, one can imagine that the $D$ term for $U(1)_Y$ has
a Fayet-Iliopoulos term which is the principal source of
supersymmetry breaking.
Unfortunately, this would be an immediate
disaster, because at least some of the squarks and sleptons
would just get non-zero VEVs (breaking color, electromagnetism, and/or
lepton number, but not supersymmetry) in order to satisfy
eq.~(\ref{booya}),
because they do not have superpotential mass terms. This means that
a Fayet-Iliopoulos term for $U(1)_Y$ must be subdominant
compared to other sources of supersymmetry breaking in the MSSM,
if not absent altogether. One could also attempt to trigger
supersymmetry breaking with a Fayet-Iliopoulos term for some other $U(1)$
gauge symmetry which is as yet unknown because it is spontaneously
broken at a very high mass scale or because it does not couple to
the Standard Model particles. However, if this is the ultimate source
for supersymmetry breaking, it
proves difficult to give appropriate masses to all of the MSSM particles,
especially the gauginos. In any case, we will not discuss $D$-term
breaking as the ultimate origin of supersymmetry violation
any further, although it may not be ruled out.\cite{dtermbreakingmaywork}

Models where supersymmetry breaking is due to non-zero $F$-terms,
called O'Rai\-f\-ear\-taigh models,\cite{ORaifeartaigh}
may have brighter phenomenological prospects.
The idea is to pick a set of chiral
supermultiplets $\Phi_i\supset (\phi_i, \psi_i, F_i)$ and a
superpotential $W$ in such a way that
the equations $F_i = -\delta W^*/\delta \phi^{*i} = 0$ have no
simultaneous solution. Then $V=\sum_i |F_i|^2$ will have to be positive at
its
minimum, ensuring that supersymmetry is broken. The simplest example which
does this has three chiral supermultiplets with
\beq
W = -k \Phi_1 + m \Phi_2 \Phi_3 + {y\over 2} \Phi_1 \Phi_3^2 .
\label{oraif}
\eeq
Note that $W$ contains a linear term, with $k$ having dimensions
of (mass)$^2$. This is only possible if $\Phi_1$ is a gauge singlet.
In section \ref{sec:susylagr} we cheated and did not mention such a term,
because we
knew that the MSSM contains no such singlet chiral supermultiplet.
Nevertheless, it should be clear from retracing the derivation in section
\ref{subsec:susylagr.chiral} that
such a term is allowed if a gauge-singlet chiral supermultiplet is
added to the theory. In
fact, a linear term is absolutely
necessary to achieve $F$-term breaking,
since otherwise setting all $\phi_i=0$ will always give a supersymmetric
global
minimum with all $F_i=0$. Without loss of generality, we can choose
$k$, $m$, and $y$ to be real and positive (by a phase rotation of
the fields). The scalar potential following from eq.~(\ref{oraif}) is
\beq
&& V = |F_1|^2 + |F_2|^2 + |F_3|^2; \\
&& F_1 =
k - {y\over 2} \phi_3^{*2} ;\qquad
F_2 = -m \phi^*_3 ;\qquad
F_3 = -m \phi^*_2 - y \phi^*_1 \phi^*_3 .
\eeq
Clearly, $F_1=0$ and $F_2=0$ are not compatible, so supersymmetry
must indeed be broken. If $m^2 > yk$ (which we assume from now on), then
it is easy to show that the absolute minimum of the potential is
at $\phi_2=\phi_3=0$ with $\phi_1$ undetermined, so $F_1 = k$ and
$V=k^2$ at the minimum of the potential.
The fact that $\phi_1$ is undetermined
is an example of a ``flat direction" in the scalar potential; this is
a common feature of supersymmetric models.\footnote{
More generally, ``flat directions" are non-compact lines and surfaces
in the space of scalar fields along which the scalar
potential vanishes. The classical
scalar potential of the MSSM would have
many flat directions if supersymmetry were not broken.}

If we presciently choose to expand $V$ around $\phi_1=0$, the mass
spectrum
of the theory
consists of 6 real scalars with tree-level squared masses
\beq
0,\>\> 0,\>\> m^2,\>\> m^2,\>\> m^2 - yk,\>\> m^2 + yk .
\label{ORscalars}
\eeq
Meanwhile, there are 3 Weyl fermions with masses
\beq
0,\>\> m,\>\> m.
\label{ORfermions}
\eeq
The non-degeneracy of scalars and fermions is a clear sign that
supersymmetry has been spontaneously broken.
The 0 eigenvalues in eqs.~(\ref{ORscalars}) and (\ref{ORfermions})
correspond to the complex scalar $\phi_1$
and its fermionic partner $\psi_1$.
However, $\phi_1$ and $\psi_1$ have different reasons
for being massless. The masslessness of $\phi_1$ corresponds to the
existence of the flat direction, since any value of
$\phi_1$ gives the same energy at tree-level.
This flat direction is an accidental feature of the classical
scalar potential, and in this case it is removed (``lifted") by quantum
corrections. This can be seen by computing the Coleman-Weinberg
one-loop effective potential.\cite{ColemanWeinberg}
After some calculation, one finds
the result that the global minimum is
indeed fixed at
$\phi_1=\phi_2=\phi_3=0$, with the complex scalar
$\phi_1$ receiving a small positive-definite (mass)$^2$ equal to
\beq
m_{\phi_1}^2 = {1\over 32 \pi^2} \left [
\Bigl ({ym^4\over k} + y^3 k \Bigr ) {\rm ln}\Bigl (
{m^2 + yk\over m^2 - yk} \Bigr ) +
2y^2  m^2 \Bigl ( {\rm ln}[1 - {y^2k^2 \over m^4}] -1\Bigr )
\right ].\>\>\>\>\>\>\>{}
\eeq
[In the limit $yk\ll m^2$, this reduces to $m_{\phi_1}^2 = y^4 k^2/(48
\pi^2
m^2)$.]
In contrast, the Weyl fermion $\psi_1$ remains exactly massless
because of a general feature
of all models with spontaneously broken supersymmetry. To understand
this, recall that the
spontaneous breaking of any global symmetry always gives rise
to a massless Nambu-Goldstone mode with the same quantum numbers as the
broken symmetry generator. In the case of supersymmetry, the broken
generator is the fermionic charge $Q_\alpha$, so the
Nambu-Goldstone particle must be a massless neutral Weyl fermion called the
{\it goldstino}. In the O'Rai\-f\-ear\-taigh model example, $\psi_1$ is the
goldstino because it is the fermionic partner of the auxiliary field
$F_1$ which got a VEV. (We will prove these statements in  
%have more to say about the goldstino in
a more general context in section \ref{subsec:origins.gravitino}.)

The O'Rai\-f\-ear\-taigh superpotential determines the mass scale of
supersymmetry breaking $\sqrt{F_1}$ in terms of a dimensionful
parameter $k$ which is put in by hand. This is somewhat {\it ad hoc},
since $\sqrt{k}$ will have to be much less than
$\MPlanck$ in
order to give the right order of magnitude for the MSSM soft terms.
We would like to
have a mechanism which can instead 
generate such scales naturally. This can be
done in models of dynamical supersymmetry breaking. In such theories, the
small (compared to $\MPlanck$) mass scales associated with supersymmetry
breaking arise by dimensional transmutation. In other words, they
generally feature
a new asymptotically-free non-Abelian gauge symmetry with a gauge
coupling $g$ which is perturbative at $\MPlanck$ and which gets
strong in the infrared at some smaller scale
$\Lambda \sim
e^{-8\pi^2/|b| g_0^2} \MPlanck$,
where
$g_0$ is the running gauge coupling at $\MPlanck$
with beta function $- |b| g^3/16 \pi^2$.
Just as in QCD, it is perfectly natural for $\Lambda$ to be
many orders of magnitude below the Planck scale.
Supersymmetry breaking may then be best described in terms of the
effective dynamics of the
strongly coupled theory. One possibility is
that the auxiliary $F$ field for a composite chiral
supermultiplet (built out of the fundamental fields which transform
under the new strongly-coupled gauge group) obtains
a VEV.  Constructing models which actually break
supersymmetry in an acceptable way is a highly non-trivial
business; for more information we refer the
reader to Ref.\cite{dynamicalsusybreaking}

The one thing that is now clear about spontaneous supersymmetry
breaking (dynamical or not) is that it requires us to extend the MSSM.
The ultimate supersymmetry-breaking order parameter cannot belong to any
of the supermultiplets of the MSSM; a $D$-term VEV for $U(1)_Y$ does not
lead to an acceptable spectrum, and there is no candidate gauge-singlet
whose $F$-term could develop a VEV.
Therefore one must ask what effects {\it are} responsible for
spontaneous supersymmetry breaking, and how
supersymmetry breakdown is ``communicated" to the MSSM particles.
It is very difficult to achieve the latter in a phenomenologically viable
way working only
with renormalizable interactions at tree-level.
First, it is problematic
to give masses to the MSSM gauginos, because supersymmetry does not allow
(scalar)-(gaugino)-(gaugino) couplings which could turn into gaugino
mass terms when the scalar gets a VEV.
Second, at least some of the MSSM
squarks and sleptons would have to be unacceptably light, and should have
been discovered already. This can be understood in a general way from
the
existence of a sum rule which governs the tree-level squared
masses of scalars and chiral fermions in theories with spontaneous
supersymmetry breaking:
\beq
{\rm Tr}[ M^2_{\rm real~scalars}] = 2 {\rm Tr}[ M^2_{\rm chiral~fermions}]
{}.
\label{sumrule1}
\eeq
If supersymmetry were not broken, then eq.~(\ref{sumrule1})
would follow immediately from the degeneracy of complex
scalars [with two real scalar components, hence the factor of 2] 
and their Weyl fermion superpartners.
 However, eq.~(\ref{sumrule1}) still
holds at tree-level when supersymmetry is broken spontaneously by
$F$-terms
and $D$-terms, as one can verify in general by explicitly computing
the (mass)$^2$ matrices for arbitrary values of the
fields.\footnote{This assumes only that the trace of the $U(1)$ charges 
over all chiral supermultiplets in
the theory vanishes (${\rm Tr}[ T^a] = 0$). This holds for
$U(1)_Y$
in the MSSM and more generally for any non-anomalous gauge symmetry.}
One can easily see, for example, that with the O'Rai\-f\-ear\-taigh
spectrum of eqs.~(\ref{ORscalars}) and (\ref{ORfermions}),
the sum rule eq.~(\ref{sumrule1}) is indeed satisfied.
This sum rule seems to be bad news for a
phenomenologically viable model, because the masses of all of the
MSSM chiral fermions are already known to be small (except for the top
quark and the higgsinos).
Even if we could succeed in evading this, there is no reason why the
resulting MSSM soft
terms in this type of model should satisfy conditions like
eqs.~(\ref{scalarmassunification}) or (\ref{aunification}).

For these reasons, we expect that the MSSM soft terms arise
indirectly or radiatively, rather than from tree-level renormalizable
couplings to
the supersymmetry-breaking order parameters. Supersymmetry
breaking evidently occurs
in a ``hidden sector" of particles which have no (or only very
small) direct couplings to the ``visible sector"
chiral supermultiplets of the MSSM. However, the two sectors do share
some interactions which are responsible for mediating supersymmetry
breaking from the hidden sector to the visible sector, where they
appear as calculable soft terms.
\begin{figure}
\centerline{\psfig{figure=susystructure.ps,height=.8in}}
\caption{The presumed schematic structure for supersymmetry breaking.
\label{fig:structure}}
\end{figure}
(See Fig.~\ref{fig:structure}.)
In this scenario, the tree-level sum rule eq.~(\ref{sumrule1})
need not hold for the visible sector fields, so that a
phenomenologically viable superpartner mass spectrum is in principle
achievable. As a bonus, if the mediating
interactions are flavor-blind, then the soft terms appearing in the MSSM
may automatically obey conditions like
eqs.~(\ref{scalarmassunification}), (\ref{aunification}) and
(\ref{commonphase}). 

There are two main competing proposals
for what the mediating interactions might be. The first (and historically
the more popular) is that they are gravitational. More
precisely, they are
associated with the new physics, including gravity, which enters at the
Planck scale.
In this {\it gravity-mediated supersymmetry breaking} scenario,
if supersymmetry is
broken in the hidden sector by a VEV $\langle
F\rangle$, then the soft terms in the visible sector should be roughly
of order
\beq
m_{\rm soft} \sim {\langle F \rangle\over \MPlanck},
\label{mgravusual}
\eeq
by dimensional analysis.
This is because we
know that $m_{\rm soft}$
must vanish in the limit $\langle F \rangle \rightarrow 0$
where supersymmetry is unbroken, and also in the
limit $\MPlanck \rightarrow \infinity$ (corresponding to $G_{\rm Newton}
\rightarrow 0$) in which gravity becomes irrelevant.
For $m_{\rm soft}$ of order a few hundred GeV, one would therefore expect
that
the scale
associated with the origin of supersymmetry breaking in the
hidden sector should be roughly ${\sqrt{\langle F\rangle}} \sim 10^{10}$
or $10^{11}$ GeV.
Another possibility is that the supersymmetry breaking order
parameter is a gaugino condensate $\langle 0| \lambda^a \lambda^b
|0 \rangle = \delta^{ab} \Lambda^3\not= 0$.
If the composite field $\lambda^a\lambda^b$ is part of
an auxiliary field $F$ for some (perhaps composite) chiral
superfield,
then by dimensional analysis we expect supersymmetry breaking
soft terms of order
\beq
m_{\rm soft} \sim
{\Lambda^3 \over \MPlanck^2} ,
\label{foofighters}
\eeq
with, effectively, $\langle F \rangle \sim \Lambda^3/\MPlanck$.
In that case, the scale associated with dynamical supersymmetry
breaking should be more like $\Lambda \sim 10^{13} $ GeV.

The second main possibility is that the flavor-blind mediating
interactions
for supersymmetry breaking are the ordinary electroweak and
QCD gauge
interactions.
In this {\it gauge-mediated supersymmetry breaking} 
scenario, the MSSM soft terms
arise from loop diagrams involving
some {\it messenger} particles. 
The messengers 
couple to a supersymmetry-breaking VEV $\langle F\rangle$, and also
have $SU(3)_C \times SU(2)_L \times U(1)_Y$ interactions
which provide a link to the MSSM.
Then, using dimensional analysis, one estimates for the
MSSM soft terms
\beq
m_{\rm soft} \sim {\alpha_a\over 4\pi} {\langle F \rangle
\over
M_{\rm mess}}
\label{mgravgmsb}
\eeq
where the $\alpha_a/4\pi$ is a loop factor for Feynman diagrams
involving gauge interactions, and $M_{\rm mess}$ is a characteristic
scale of the masses of the messenger fields.
So if $M_{\rm mess}$ and $\sqrt{\langle F\rangle}$ are roughly comparable,
then
the scale of
supersymmetry
breaking can be as low as about ${\sqrt{\langle F\rangle}} \sim 10^4$
or $10^5$ GeV
(much lower than in the gravity-mediated case!)
to give $m_{\rm soft}$ of the right order of magnitude.

\subsection{The goldstino and the gravitino}\label{subsec:origins.gravitino}

As explained in the previous section,
the spontaneous breaking of global supersymmetry implies the existence
of a massless Weyl fermion, the goldstino.
In the particular case of the O'Rai\-f\-ear\-taigh model,
the goldstino was identified to be $\psi_1$. More generally, we might
expect that in the case of
$F$-term or $D$-term breaking, the goldstino is the fermionic component
of the supermultiplet whose auxiliary field obtains a VEV.

Let us make this more precise by actually proving that the goldstino
exists and, in the process, identifying it. This is actually rather easy.
Consider a general
supersymmetric model 
with both
gauge and chiral supermultiplets
as in section \ref{sec:susylagr}. 
The fermionic degrees of freedom
consist of gauginos ($\lambda^a$) and chiral fermions
($\psi_i$).
After some of the scalar fields in the theory obtain VEVs, the
fermion mass matrix will have the form:
\beq
{\bf M}_{\rm fermion} =
\pmatrix{
0   &
\sqrt{2} g_a  (\langle \phi^{*}\rangle T^a)^i    \cr
\sqrt{2} g_a  (\langle \phi^{*}\rangle T^a)^j  &
\langle W^{ij} \rangle
}
\eeq
in the $(\lambda^a,\,\psi_i)$ basis. [The off-diagonal entries in this
matrix come from the second line in eq.~(\ref{gensusylagr}), and the 
lower right entry can be seen in eq.~(\ref{noFlagr}).] Now we simply note
that ${\bf M}_{\rm fermion}$ annihilates the vector 
\beq
{\stilde G} = 
%\left ({\langle D^a \rangle \over \sqrt{2}}, \> \langle F_i
% Let's write it as an actual column vector!
\pmatrix{{\langle D^a \rangle /\sqrt{2}} \cr \langle F_i
\rangle
}.
\label{explicitgoldstino}
\eeq
The first row of ${\bf M}_{\rm fermion}$
annihilates $\stilde G$ by virtue of the requirement 
eq.~(\ref{wgaugeinvar}) that the
superpotential is gauge invariant, and the
second row annihilates $\stilde G$ because of the condition
$
\langle {\partial V/ \partial \phi_i} \rangle = 0
$
which must be satisfied at the minimum of the scalar potential.
Eq.~(\ref{explicitgoldstino}) is proportional to the goldstino
wavefunction; it
is non-trivial
if and only if at least one of the auxiliary fields has a VEV, breaking
supersymmetry.
So we have proven that if global supersymmetry is spontaneously broken,
then the goldstino exists and has zero mass, and that its components among
the
various fermions in the theory are just proportional to the corresponding
auxiliary field VEVs.

We can derive another very important property of the goldstino by
considering
the form of the conserved supercurrent eq.~(\ref{supercurrent}).
Suppose for simplicity\footnote{More generally, if supersymmetry is
spontaneously broken by VEVs for several auxiliary fields $F_i$ and
$D^a$, then one should make the replacement $\langle F \rangle
\rightarrow ( \sum_i |\langle F_i \rangle|^2
+ {1\over 2} \sum_a \langle D^a \rangle^2 )^{1/2}$ 
everywhere in the following.}
that the non-vanishing auxiliary field
VEV is
$\langle F \rangle$
and that its goldstino superpartner is $\stilde G$. Then the supercurrent
conservation equation
tells us that
\beq
0 = \partial_\mu J^\mu_\alpha =
i \langle F \rangle (\sigma^\mu \partial_\mu \stilde G^\dagger)_\alpha +
\partial_\mu
j^\mu_\alpha
+ \ldots
\label{beezlebub}
\eeq
where $j^\mu_\alpha$ is the part of the supercurrent which involves
all of the other supermultiplets, and the ellipses represent other
contributions of the goldstino supermultiplet to $\partial_\mu
J^\mu_\alpha$
which we can ignore.
[The first term in eq.~(\ref{beezlebub}) comes from the second term
in eq.~(\ref{supercurrent}), using the equation of motion $F_i =
-W^{*}_i$ for the
goldstino's auxiliary field.]
This
equation of motion for the goldstino
field allows us to write an effective lagrangian
\beq
\lagr_{\rm goldstino}
= -i \stilde G^\dagger \sigmabar^\mu \partial_\mu \stilde G -
{1\over \langle F \rangle}(\stilde G \partial_\mu j^\mu
+ \conj)
\label{goldstinointeraction}
\eeq
which describes the interactions of the goldstino with all of the other
fermion-boson pairs.\cite{Fayetsupercurrent}
In particular, since $j^\mu_\alpha =
(\sigma^\nu\sigmabar^\mu \psi_i)_\alpha \partial_\nu \phi^{*i}
-(1/2\sqrt{2}) \sigma^\nu \sigmabar^\rho \sigma^\mu \lambda^{\dagger a}
F_{\nu\rho}^a + \ldots$, there are goldstino-scalar-chiral fermion and
\begin{figure}
\centerline{\psfig{figure=susygoldstino.ps,height=1in}}
\caption{Goldstino/gravitino interactions with superpartner pairs
$(\phi,\psi)$ and $(\lambda^a,A^a)$.
\label{fig:goldstino}}
\end{figure}
goldstino-gaugino-gauge boson vertices as shown in
Fig.~\ref{fig:goldstino}.
Since this derivation depends only on supercurrent conservation,
eq.~(\ref{goldstinointeraction}) holds independently of the details of how
supersymmetry breaking
is communicated from $\langle F \rangle$ to the MSSM sector fields
$(\phi_i,\psi_i)$ and $(\lambda^a, A^a)$. It may appear strange at first
that the interaction terms in eq.~(\ref{goldstinointeraction}) get larger
as $\langle F \rangle$ goes to zero. However, the
interaction term $\stilde G \partial_\mu j^\mu$ contains two
derivatives which turn out to always
give a kinematic factor proportional to the (mass)$^2$ difference
of the superpartners when they are on-shell,
i.e.~$m_{\phi_i}^2 - m_{\psi_i}^2$
and $m^2_{\lambda} - m_{A}^2$ for Figs.~\ref{fig:goldstino}a and
\ref{fig:goldstino}b respectively. These
can be non-zero only
by virtue of supersymmetry breaking, so they must also vanish as $\langle
F\rangle \rightarrow 0$,
and the interaction is well-defined in that limit.
Nevertheless, for fixed values of $m_{\phi_i}^2 - m_{\psi_i}^2$
and $m^2_{\lambda} - m_{A}^2$, the interaction term in
eq.~(\ref{goldstinointeraction}) can be phenomenologically important
if $\langle F \rangle $ is not too 
large.\cite{Fayetsupercurrent,eeGMSBsignal,DDRT,AKKMM2}

The above remarks apply to the breaking of global supersymmetry.
However, when one takes into account gravity, supersymmetry must
be a local symmetry. This means that the spinor parameter
$\epsilon^\alpha$
which first appeared in section \ref{subsec:susylagr.freeWZ} is no longer
a constant,
but can vary from point to point in spacetime.
The resulting locally supersymmetric theory
is called {\it supergravity}.\cite{supergravity1,supergravity2} It
necessarily unifies the
spacetime symmetries of ordinary general relativity
with local supersymmetry
transformations. In supergravity,
the spin-2 graviton has a spin-3/2 fermion superpartner
called the gravitino, which we will denote $\stilde \Psi_\mu^\alpha$.
The gravitino has odd
$R$-parity ($P_R=-1$),
as can be seen from the definition eq.~(\ref{defRparity}). It carries both
a vector index ($\mu$)
and a spinor index ($\alpha$), and transforms inhomogeneously under local
supersymmetry transformations:
\beq
\delta \stilde\Psi_\mu^\alpha = -\partial_\mu\epsilon^\alpha
+\ldots
\eeq
Thus the gravitino should be thought of as the ``gauge" particle
of local supersymmetry transformations
[compare eq.~(\ref{Agaugetr})].
As long as supersymmetry is unbroken, the graviton and the gravitino
are both massless, each with two spin helicity states. Once
supersymmetry is spontaneously broken, the gravitino acquires a mass
by absorbing (``eating") the goldstino,
which becomes its longitudinal (helicity $\pm 1/2$) components.
This is called the {\it super-Higgs} mechanism. It is
entirely analogous to the ordinary Higgs mechanism for gauge
theories, by which the $W^\pm$ and $Z^0$ gauge bosons
in the Standard Model gain mass by absorbing the
Nambu-Goldstone bosons associated with the spontaneously broken
electroweak gauge
invariance. The counting works, because the massive spin-3/2 gravitino
now
has four helicity states, of which two were originally assigned to the
would-be goldstino.
The gravitino  mass is traditionally called $m_{3/2}$, and in the case of
$F$-term breaking can be estimated as \cite{gravitinomassref}
\beq
m_{3/2} \sim {\langle F \rangle \over \MPlanck},
\label{gravitinomass}
\eeq
This follows simply from dimensional analysis, since $m_{3/2}$ must vanish
in the limits that supersymmetry is restored ($\langle F \rangle
\rightarrow 0$) and that gravity is turned off ($M_P \rightarrow \infty$).
Equation (\ref{gravitinomass}) means that one has very different
expectations for the
mass of the gravitino in gravity-mediated and in gauge-mediated
models, because they usually make very different predictions for
$\langle F \rangle$.

In the gravity-mediated supersymmetry breaking case, the gravitino
mass is comparable to the masses of the MSSM sparticles [compare
eqs.~(\ref{mgravusual}) and (\ref{gravitinomass})]. Therefore $m_{3/2}$
is expected to be at least 100 GeV or so. Its interactions will be of
gravitational strength, so the gravitino will not play any role in
collider physics, but it can be a very important consideration in
cosmology.\cite{cosmogravitino}
If it is the LSP, then it is stable and its
primordial density could easily exceed the critical density,
causing the universe to become matter-dominated too early. Even
if it is not the LSP, the gravitino can cause problems
unless its
density is diluted by inflation at late times, or it decays sufficiently
rapidly.

In contrast, gauge-mediated supersymmetry breaking models predict that the
gravitino
is much lighter than the MSSM sparticles as long as $M_{\rm mess} \ll
\MPlanck$. This can be seen
by comparing eqs.~(\ref{mgravgmsb}) and (\ref{gravitinomass}).
The gravitino is almost certainly the LSP in this case, and all of the
MSSM sparticles will eventually decay into final states that include it.
Naively, one might expect that these decays are extremely slow.
However, this is not necessarily true, because the gravitino
inherits the {\it non}-gravitational interactions of the goldstino
it has absorbed.
This means that the gravitino, or more precisely its longitudinal
(goldstino)
components, can play an important role in collider physics experiments.
The mass of the gravitino can generally be ignored for kinematic
purposes,
as can its transverse (helicity $\pm 3/2$) components which
really do have
only
gravitational interactions. Therefore in collider phenomenology
discussions one may
interchangeably use the same
symbol
$\stilde G$ for the goldstino and for the gravitino of which it is
the longitudinal (helicity $\pm 1/2$) part.
By using
the effective lagrangian eq.~(\ref{goldstinointeraction}),
one can compute that the decay rate of any sparticle $\stilde X$ into
its Standard Model partner $X$ plus a gravitino/goldstino $\stilde G$
is given by
\beq
\Gamma(\stilde X \rightarrow X\stilde G) =
{m^5_{\stilde X} \over 16 \pi \langle F \rangle^2}
\left ( 1 - {m_X^2\over m_{\stilde X}^2} \right )^4 .
\qquad\>\>{}
\label{generalgravdecay}
\eeq
This corresponds to either Fig.~\ref{fig:goldstino}a or
\ref{fig:goldstino}b,
with $(\stilde X,X) = (\phi,\psi)$ or $(\lambda,A)$ respectively.
One factor $(1 - m_X^2/m_{\stilde X}^2 )^2$ came from the derivatives
in the interaction term in eq.~(\ref{goldstinointeraction}) evaluated for
on-shell final states,
and another such factor comes from
the kinematic phase space integral with $m_{3/2} \ll m_{\stilde X}, m_X$.

If the supermultiplet containing the goldstino and $\langle F \rangle$
has canonically-normalized kinetic terms, and one requires the tree-level
vacuum energy
to vanish, then the estimate eq.~(\ref{gravitinomass}) may be sharpened to
\beq
m_{3/2} = {\langle F \rangle \over \sqrt{3} \MPlanck} .
\label{gravitinomassbogus}
\eeq
In that case, one can rewrite eq.~(\ref{generalgravdecay}) as
\beq
\Gamma(\stilde X \rightarrow X\stilde G) =
{m_{\stilde X}^5 \over 48 \pi \MPlanck^2 m_{3/2}^2}
\left ( 1 - {m_X^2\over m_{\stilde X}^2} \right )^4 ,
\qquad\>\>{}
\label{specificgravdecay}
\eeq
and this is how the formula is sometimes presented by those who
prefer to take eq.~(\ref{gravitinomassbogus}) seriously.
Note that the decay width is
larger for smaller $\langle F \rangle$, or equivalently for smaller
$m_{3/2}$, if the other masses are fixed. 
If $\stilde X$ is a mixture
of superpartners of different Standard Model particles $X$, then
eq.~(\ref{generalgravdecay}) should be multiplied by a suppression
factor equal to the square of the cosine of the appropriate
mixing angle.
If $m_{\stilde X}$ is of order 100 GeV or more, and $\sqrt{
\langle F\rangle }
\lsim$ few$\times 10^6$ GeV [corresponding to $m_{3/2}$ less than
roughly 1 keV according to eq.~(\ref{gravitinomassbogus})],
then the decay $\stilde X \rightarrow
X \stilde G$ can occur quickly enough to be observed in a modern
collider detector. This 
gives rise to some very interesting phenomenological signatures,
which we will discuss further in sections
\ref{subsec:decays.gravitino} and \ref{sec:signals}.

We now turn to a slightly more systematic analysis of the
way in which the MSSM soft terms arise, considering in turn
the gravity-mediated and gauge-mediated scenarios.

\subsection{Gravity-mediated supersymmetry breaking
models}\label{subsec:origins.sugra}

The defining feature of
these models is that the hidden sector of the theory
communicates with our MSSM only (or dominantly) through
gravitational-strength interactions. In an effective field theory format,
this means that the supergravity lagrangian contains nonrenormalizable
terms which communicate between the two sectors and which are suppressed
by
powers of the Planck mass,
since the gravitational coupling 
is proportional to
$1/\MPlanck$.
These will include
\beq
\lagr_{\rm NR} \!  &=& \!
-{1\over \MPlanck} F_X \sum_a {1\over 2}f_a \lambda^a \lambda^a
+ \conj
\nonumber
\\
&&- {1\over \MPlanck^2} F_X F_X^*\,
k^i_j \phi_i \phi^{*j}
\nonumber
\\
&&-{1\over \MPlanck} F_X ({1\over 6} y^{\prime ijk}\phi_i \phi_j
\phi_k + {1\over 2} \mu^{\prime ij} \phi_i \phi_j)
+ \conj
\label{hiddengrav}
\eeq
where $F_X$ is the auxiliary field for a chiral supermultiplet $X$
in the hidden sector, and $\phi_i$ and $\lambda^a$ are the scalar and
gaugino fields in the MSSM.
By
themselves, the terms
in eq.~(\ref{hiddengrav}) are not supersymmetric, but it is possible to
show that they are
part of a nonrenormalizable supersymmetric lagrangian (see Appendix) which
contains
other terms that we may ignore.
 Now if one assumes that $\langle F_X \rangle \sim 10^{10}$ or $10^{11}$
GeV,
then $\lagr_{\rm NR}$ will give us nothing other than a lagrangian
of the form $\lagr_{\rm soft}$ in eq.~(\ref{lagrsoft}), with
MSSM soft terms of order a few hundred GeV. [Note that
terms of the form $\lagr_{\rm maybe~soft}$ in eq.~(\ref{lagrsoftprime})
do not arise.]

The dimensionless
parameters $f_a$, $k^i_j$, $y^{\prime ijk}$ and
$\mu^{\prime ij}$
in $\lagr_{\rm NR}$ are to be determined by the underlying theory.
This is a difficult enterprise in general, but a dramatic simplification
occurs if one assumes a ``minimal" form for the normalization of kinetic
terms and gauge interactions in the full, nonrenormalizable
supergravity lagrangian (see Appendix).
In that case, one finds that
there is a common $f_a=f$ for the three gauginos; $k^i_j = k \delta^i_j$
is the same for all scalars; and the other couplings are
proportional to the corresponding superpotential parameters, so that
$y^{\prime ijk} = \alpha y^{ijk}$ and $\mu^{\prime ij} = \beta \mu^{ij}$
with universal dimensionless constants $\alpha$ and $\beta$. Then one
finds that
the soft terms in $\lagr_{\rm soft}^{\rm MSSM}$
can all be written in terms of just four parameters:
\beq
m_{1/2} = f{\langle F_X\rangle\over \MPlanck};\qquad\!\!
m^2_{0} = k{|\langle F_X\rangle|^2\over \MPlanck^2};\qquad\!\!
A_0 = \alpha{\langle F_X\rangle\over \MPlanck};\qquad\!\!
B_0 = \beta{\langle F_X\rangle\over \MPlanck}.\>\>{}
\label{sillyassumptions}
\eeq
In terms of these, one can write for the parameters appearing in
eq.~(\ref{MSSMsoft}):
\beq
&&\!\!\!\! M_3 = M_2 = M_1 = m_{1/2};
\label{gauginounificationsugra}
\\
&&\!\!\!\! {\bf m^2_{Q}} =
{\bf m^2_{{\sbar u}}} =
{\bf m^2_{{\sbar d}}} =
{\bf m^2_{ L}} =
{\bf m^2_{{\sbar e}}} =
m_0^2\, {\bf 1};
\>\>\>\> m_{H_u}^2 = m^2_{H_d} = m_0^2; \>\>\>\qquad{}
\label{scalarunificationsugra}
\\
&&\!\!\!\! {\bf a_u} = A_0 {\bf y_u};\qquad
{\bf a_d} = A_0 {\bf y_d};\qquad
{\bf a_e} = A_0 {\bf y_e};
\label{aunificationsugra}
\\
&&\!\!\!\! b = B_0 \mu .
\label{bsilly}
\eeq
It is a matter of some controversy whether the assumptions going into
this parameterization are completely well-motivated on
purely theoretical
grounds,\footnote{The familiar flavor-blindness of
gravitational interactions expressed in Einstein's equivalence principle
does not, by itself, tell us anything about the form of
eq.~(\ref{hiddengrav}).} but from
a phenomenological perspective they are clearly
very nice. This
framework
successfully evades the most
dangerous types of FCNC and CP-violation as discussed in section
\ref{subsec:mssm.hints}.
In particular, eqs.~(\ref{scalarunificationsugra}) and
(\ref{aunificationsugra}) are just stronger versions of
eqs.~(\ref{scalarmassunification}) and (\ref{aunification}),
respectively. If $m_{1/2}$, $A_0$ and $B_0$
all have the same complex phase, then eq.~(\ref{commonphase})
will also be satisfied.

Equations (\ref{gauginounificationsugra})-(\ref{bsilly})
also have the virtue of being highly predictive. [Of course,
eq.~(\ref{bsilly}) is content-free unless one can relate
$B_0$ to the other parameters in some non-trivial way.]
As discussed in section 5.4, they should be applied as RG boundary
conditions at the
scale $\MPlanck$. The RG evolution of the soft parameters
down to the electroweak scale
will then allow us to predict the entire MSSM
spectrum in terms
of just five parameters $m_{1/2}$, $m_0^2$, $A_0$, $B_0$, and
$\mu$ (plus the already-measured gauge and Yukawa couplings of the MSSM).
In practice, the approximation is usually made
of starting this RG running from
the unification scale $M_U\approx 2\times 10^{16}$ GeV
instead of $\MPlanck$. The reason for this
is that the apparent unification of gauge couplings
gives us a strong hint that we know something about how the RG equations
are behaving up to $M_U$, but gives us little guidance
about what to expect at scales between $M_U$ and $\MPlanck$.
The error made in neglecting these effects is proportional
to a loop suppression factor times ln$(\MPlanck/M_U)$ and can be
partially absorbed
into a redefinition of $m_0^2$, $m_{1/2}$, $A_0$ and $B_0$,
but in some cases can lead to
important effects.\cite{PP}
The framework described in the above few paragraphs
has been the subject of
the bulk of phenomenological
studies of supersymmetry. It is sometimes referred to as the
{\it minimal supergravity} or {\it supergravity-inspired}
scenario for the soft terms. A few examples of the many useful numerical
RG studies of the MSSM spectrum which have been
performed in this framework can be
found in
Ref.\cite{samplespectra}

Particular models of gravity-mediated
supersymmetry breaking can be even more
predictive,
relating some of the parameters $m_{1/2}$, $m_0^2$, $A_0$ and $B_0$ to
each other and to the mass of the gravitino $m_{3/2}$.
For example, three popular
kinds of models for the soft terms are:
\vspace{.05in}

\noindent $\bullet$
Dilaton-dominated: \cite{dilatondominated}~~~$m^2_0 =
m^2_{3/2}$;~~~~$m_{1/2} = -A_0 = {\sqrt 3} m_{3/2}$.
\vspace{.05in}

\noindent $\bullet$
Polonyi: \cite{polonyi}
{}~~~$m^2_0 = m^2_{3/2}$;
{}~~~~$A_0 = (3 -{\sqrt 3}) m_{3/2}$;
{}~~~~$m_{1/2} = {\cal O}(m_{3/2})$.
\vspace{.05in}

\noindent $\bullet$ ``No-scale": \cite{noscale}~~~$m_{1/2} \gg
m_0, A_0, m_{3/2}$.
\vspace{.05in}

The dilaton-dominated scenario arises in a particular limit of
superstring theory. While it appears to be highly predictive, it can
easily be generalized in other limits.\cite{stringsoft} The Polonyi model
has the advantage
of being the simplest possible model for supersymmetry breaking in the
hidden sector, but it is rather {\it ad hoc} and does not seem to
have a special place in grander schemes like superstrings. The ``no-scale"
limit may arise in a low-energy limit of superstrings
in which the gravitino mass scale is undetermined at tree-level (hence the
name).
It implies that only the gaugino masses are appreciable at $\MPlanck$.
As we will see in section \ref{subsec:MSSMspectrum.rges},
RG evolution feeds $m_{1/2}$ into the squark,
slepton and Higgs (mass)$^2$ parameters with sufficient magnitude
to give acceptable phenomenology at the electroweak scale.
More recent versions of the no-scale scenario, however, also can give
significant $A_0$ and $m_0^2$ at $\MPlanck$. In many cases $B_0$ can also
be predicted in terms of the other parameters, but this is quite
sensitive to model assumptions. For phenomenological studies,
$m_{1/2}$, $m_0^2$, $A_0$ and $B_0$ are usually just taken to be
convenient independent
parameters of our ignorance of the supersymmetry breaking mechanism.

\subsection{Gauge-mediated supersymmetry breaking
models}\label{subsec:origins.gmsb}

A strong alternative to the scenario described in the previous
section is provided by the gauge-mediated supersymmetry breaking
proposal.\cite{oldgmsb,newgmsb}
The basic idea is to introduce some new chiral supermultiplets,
called messengers, which couple to the ultimate
source of supersymmetry breaking, and which also couple indirectly
to the (s)quarks and (s)leptons and Higgs(inos) of the MSSM through the
ordinary $SU(3)_C\times
SU(2)_L\times U(1)_Y$ gauge boson and gaugino interactions.
In this way, the ordinary gauge interactions, rather than gravity, are
responsible for the
appearance of soft terms in the MSSM. There is still gravitational
communication between the MSSM and the source of supersymmetry breaking,
of course, but that effect is now relatively unimportant compared to the
gauge interaction effects.

In the simplest such model, the messenger fields are
a  set of chiral supermultiplets
$q$, $\overline q$, $\ell$, $\overline \ell$ which transform under
$SU(3)_C\times SU(2)_L\times U(1)_Y$ as
\beq
q\sim({\bf 3},{\bf 1}, -{1\over 3});\qquad\!\!\!
\overline q\sim({\bf \overline 3},{\bf 1}, {1\over 3});\qquad\!\!\!
\ell \sim({\bf 1},{\bf 2}, {1\over 2});\qquad\!\!\!
\overline \ell\sim({\bf 1},{\bf 2}, -{1\over 2}).\qquad\!\!\!{}
\label{minimalmess}
\eeq
These supermultiplets contain messenger quarks 
$\psi_q, \psi_{\overline q}$ and scalar quarks $q, \overline q$
and messenger leptons
$\psi_\ell, \psi_{\overline \ell}$ and scalar leptons $\ell, \overline
\ell$. All of these particles
must get very large masses so as not to have been
discovered already. They manage to do so by coupling to a gauge-singlet
chiral supermultiplet
$S$ through a superpotential:
\beq
W_{\rm mess} = y_2 S \ell \overline \ell + y_3 S q \overline q .
\eeq
The scalar component of $S$ and its auxiliary ($F$-term) component are
each supposed to acquire VEVs, denoted $\langle S \rangle $ and
$\langle F_S \rangle $ respectively.
This can be accomplished either by putting $S$ into an
O'Rai\-f\-ear\-taigh-type model,\cite{oldgmsb} or by a dynamical
mechanism.\cite{newgmsb}
Exactly how this happens is a very
interesting and important question. Here, we will 
simply parameterize our ignorance of
the precise mechanism of supersymmetry breaking by asserting that
$S$ participates in another part of the superpotential, call it
$W_{\rm breaking}$, which provides for supersymmetry breakdown.

Let us now consider the mass spectrum of the messenger fermions and
bosons. The messenger part of the superpotential now effectively
becomes $W_{\rm mess} = y_2 \langle S \rangle \ell\overline\ell +
y_3 \langle S \rangle q\overline q$. So, 
the fermionic messenger fields
pair up to
get mass terms:
\beq
\lagr &=& 
- (y_2 \langle S \rangle \psi_\ell \psi_{\overline \ell}
 + y_3 \langle S \rangle \psi_q \psi_{\overline q} + \conj )
\label{messfermass}
\eeq
as in eq.~(\ref{lagrchiral}). Meanwhile, their scalar messenger
partners $\ell,\overline\ell$ and $q\overline q$ have a scalar potential
given by (neglecting $D$-term contributions, which do not affect the
following discussion):
\beq
V
&=& 
\left | {\delta \Wmess \over \delta \ell} \right |^2 +
\left | {\delta \Wmess \over \delta \overline\ell} \right |^2 +
\left | {\delta \Wmess \over \delta q} \right |^2 +
\left | {\delta \Wmess \over \delta \overline q} \right |^2 +
\left | {\delta \Wmess\over \delta S} + 
{\delta W_{\rm breaking}\over \delta S}\right |^2\>\phantom{xxx}
\eeq
as in eq.~(\ref{ordpot}). Now, using the supposition that
\beq
\langle \delta W_{\rm breaking}/\delta S \rangle 
= -\langle F_S^* \rangle
\eeq
(with $\langle \delta \Wmess /\delta S \rangle = 0$), and
replacing $S$
and $F_S$ by their VEVs, one finds quadratic mass
terms
in the potential for the messenger scalar leptons:
\beq
V &=& 
|y_2 \langle S \rangle|^2 \left ( |\ell|^2 + 
|\overline \ell |^2 \right ) +
|y_3 \langle S \rangle|^2 \left ( |q|^2 + 
|\overline q|^2 \right )
\nonumber \\
&&-\left (y_2 \langle F_S \rangle \ell\overline \ell 
+ y_3 \langle F_S \rangle q\overline q + \conj \right )
\nonumber
\\&& +\> {\rm quartic}\> {\rm terms}.
\label{nosteenkinlabel}
\eeq
The first line in eq.~(\ref{nosteenkinlabel}) represents supersymmetric
mass terms that go along with eq.~(\ref{messfermass}), while the
second line consists of soft supersymmetry-breaking masses.
The complex scalar messengers $\ell,\overline\ell$ thus obtain a
(mass)$^2$
matrix equal to:
\beq
\pmatrix{ |y_2 \langle S \rangle |^2 & -y^*_2 \langle F^*_S \rangle \cr
-y_2 \langle F_S \rangle & |y_2 \langle S \rangle |^2 }
\eeq
with squared mass eigenvalues $|y_2 \langle S\rangle |^2
\pm |y_2 \langle F_S \rangle |$.
In just the same way, the
scalars $q,\overline q$ get squared masses $|y_3 \langle S\rangle |^2
\pm |y_3 \langle F_S \rangle |$. 

So far, we have found that the effect of supersymmetry breaking 
is to split each messenger supermultiplet pair apart:
\begin{figure}
\centerline{\psfig{figure=susy1loop.ps,height=1.1in}}
\caption{Contributions to the MSSM gaugino masses in gauge-mediated
supersymmetry breaking models arise from one-loop graphs involving
virtual messenger particles.
\label{fig:1loop}}
\end{figure}
\beq
\ell,\overline\ell : \qquad & m_{\rm fermions}^2 = |y_2 \langle S\rangle
|^2\, ,
\qquad & m_{\rm scalars}^2 = |y_2 \langle S\rangle |^2
\pm |y_2 \langle F_S \rangle | \, ; \\
q,\overline q : \qquad & m_{\rm fermions}^2 = |y_3 \langle S\rangle
|^2\, ,
\qquad & m_{\rm scalars}^2 = |y_3 \langle S\rangle |^2
\pm |y_3 \langle F_S \rangle | \> .
\eeq 
The supersymmetry violation apparent
in this messenger spectrum 
for 
$\langle F_S \rangle \not= 0$
is communicated to the MSSM sparticles through
radiative quantum corrections.
The MSSM gauginos obtain
masses from the 1-loop graph shown in Fig.~\ref{fig:1loop}. The
scalar and fermion lines in
the loop are messenger fields. 
Recall that the interaction vertices in Fig.~\ref{fig:1loop} are of
gauge coupling
strength even though they do not involve gauge bosons; compare
Fig.~\ref{fig:gauge}g.
In this way, gauge-mediation provides that $q,\overline q$ messenger loops
give
masses to
the gluino and the bino, and $\ell,\overline \ell$ messenger loops give
masses to the wino and bino fields. By computing
the 1-loop diagrams one finds\cite{newgmsb} that the resulting
MSSM gaugino masses are given by
\beq
M_a = {\alpha_a\over 4\pi} \Lambda , \qquad\>\>\>(a=1,2,3) ,
\label{gauginogmsb}
\eeq
(in the normalization discussed in section \ref{subsec:mssm.hints})
where we have
introduced a mass parameter
\beq
\Lambda \equiv \langle F_S\rangle/\langle S \rangle \> .
\label{defLambda}
\eeq
(Note that if $\langle F_S\rangle$ were 0, then $\Lambda=0$ and
the
messenger
scalars would be degenerate with their fermionic superpartners and there
would be no contribution to the MSSM gaugino masses.)
In contrast, the corresponding MSSM gauge bosons
cannot get a corresponding
mass shift, since they are protected by gauge invariance. 
So supersymmetry breaking has been successfully communicated
to the MSSM (``visible sector").
To a good approximation, eq.~(\ref{gauginogmsb}) holds for the
running gaugino masses at an RG scale $Q_0$ corresponding to the average
characteristic mass of the heavy messenger particles, roughly of order
$M_{\rm mess} \sim y_i \langle S \rangle$.
The running
mass parameters
can then be RG-evolved
down to the electroweak scale to predict the physical masses to be
measured by
future experiments.

\begin{figure}
\centerline{\psfig{figure=susy2loops.ps,height=1.25in}}
\caption{Contributions to MSSM scalar squared masses in
gauge-mediated supersymmetry breaking  models arise in leading order from
these two-loop Feynman graphs.
\label{fig:2loops}}
\end{figure}
The scalars of the MSSM do not get any radiative corrections to
their masses at one-loop order.
The leading contribution to their masses comes from the two-loop
graphs shown in Fig.~\ref{fig:2loops}, with the messenger fermions
(heavy solid
lines) and messenger scalars (heavy dashed lines) and ordinary gauge
bosons and gauginos running around the loops.
By computing these graphs,
%In the same limit $\langle F_S \rangle \ll
%y_i \langle S \rangle^2$, 
one finds that each MSSM scalar $\phi$
gets a
(mass)$^2$ given by:
\beq  m^2_\phi =
2 {\Lambda^2}
\left [ \left ({\alpha_3\over 4\pi}\right )^2 C_3^\phi+
\left ({\alpha_2\over
4 \pi}\right )^2 C_2^\phi +
\left ({\alpha_1\over 4 \pi}\right )^2 C_1^\phi \right ] .
\label{scalargmsb}
\eeq
Here $C^\phi_a$ are the quadratic Casimir group theory invariants for the
scalar $\phi$ for each gauge group. They are defined by
$C^\phi_a \delta_i^j = (T^a T^a)_i^{\,j}$ where the $T^a$ are the group
generators which act on the scalar $\phi$. Explicitly, they are:
\beq
&&C^\phi_3 =
\left \{ \begin{array}{ll}
4/3 & {\rm for}\>\,\phi = \stilde Q_i, \stilde{ \sbar u}_i,\stilde {\sbar
d}_i;\\
           0 & {\rm for}\>\,\phi = \stilde L_i, \stilde{ \sbar e}_i, H_u,
H_d
\end{array}
\right.
\label{defC3}
\\
&&C^\phi_2 =
\left \{ \begin{array}{ll}
3/4 & {\rm for}\>\,\phi = \stilde Q_i, \stilde L_i, H_u, H_d;\\
0 & {\rm for}\>\,\phi = \stilde {\sbar u}_i,\stilde  {\sbar d}_i, \stilde{
\sbar e}_i
\end{array}
\right.
\\
&&
C^\phi_1 = \>
3 Y_\phi^2/5 \>\>\>{\rm for~each}\>\,\phi\>\,{\rm
with~weak~hypercharge}\>\, Y_\phi.
\label{defC1}
\eeq
The squared masses in eq.~(\ref{scalargmsb}) are positive (fortunately!).

The terms $\bf a_u$, $\bf a_d$, $\bf a_e$ arise
first at two-loop order, and are
suppressed by an extra factor of $\alpha_a/(4 \pi)$ compared
to the gaugino masses. So, to a very good approximation one has, at the
messenger scale,
\beq
{\bf a_u} = {\bf a_d} = {\bf a_e} = 0,
\label{aaagmsb}
\eeq
a significantly stronger condition than eq.~(\ref{aunification}).
Again, eqs.~(\ref{scalargmsb}) and (\ref{aaagmsb}) should be applied
at an RG scale equal to the average mass of the messenger fields
running in the loops.
However, after evolving the RG equations down to the
electroweak scale, non-zero $\bf a_u$, $\bf a_d$ and $\bf a_e$ are
generated proportional to the corresponding Yukawa matrices
and the non-zero gaugino masses, as we will see in section
\ref{subsec:MSSMspectrum.rges}.
These will only be large for the third family squarks and sleptons,
in the approximation of eq.~(\ref{heavytopapprox}). The parameter
$b$ may also be taken to vanish near the messenger scale, but this
is quite model-dependent, and in any case $b$ will be non-zero when it
is RG-evolved to the electroweak scale. In practice, $b$ is determined
by the requirement of correct electroweak symmetry breaking,
as discussed below in section \ref{subsec:MSSMspectrum.Higgs}.

Because the gaugino masses arise at {\it one}-loop order and the scalar
(mass)$^2$ contributions appear
at {\it two}-loop order, both
eq.~(\ref{gauginogmsb}) and (\ref{scalargmsb}) correspond to
the estimate eq.~(\ref{mgravgmsb}) for $m_{\rm soft}$, with $M_{\rm
mess} \sim y_i
\langle S \rangle$. Equations (\ref{gauginogmsb}) and (\ref{scalargmsb})
hold in the limit of small
$\langle F_S \rangle /y_i\langle S \rangle^2$, corresponding to
mass splittings within each messenger supermultiplet that are small
compared to the overall messenger mass scale. The subleading corrections
in an expansion in $\langle F_S \rangle /y_i\langle S \rangle^2$
turn out\cite{gmsbcorrections} to be
quite small unless there are very large hierarchies in the messenger
sector.

The model we have described so far is often called the minimal
model of gauge-mediated supersymmetry breaking. Let us now generalize it
to a more complicated messenger sector.
Suppose that $q, \overline q$ and $\ell, \overline \ell $ are replaced
by a collection of messengers $\Phi_i,\overline \Phi_i$ with a
superpotential
\beq
W_{\rm mess} = \sum_i y_i S \Phi_i \overline \Phi_i
. 
\eeq
The
bar means that the chiral
superfields $\overline \Phi_i$ transform as the complex conjugate
representations of the $\Phi_i$ chiral superfields. Together they are said
to form a ``vector-like" (real) representation of the Standard Model gauge
group.
As
before, the fermionic components of each pair $\Phi_i$ and
$\overline\Phi_i$ pair up to get squared masses 
$y_i \langle S \rangle$ and their scalar partners mix to get squared
masses $|y_i \langle S \rangle|^2 \pm |y_i \langle F_S \rangle | $. The
MSSM gaugino mass parameters induced are now
\beq
M_a = {\alpha_a\over 4\pi} \Lambda \sum_i n_a(i) \qquad\>\>\>(a=1,2,3)
\label{gauginogmsbgen}
\eeq
where $n_a(i)$ is the Dynkin index for each $\Phi_i+\overline \Phi_i$,
in a normalization where $n_3 = 1$ for a 
${\bf 3} + {\bf \overline 3}$ of $SU(3)_C$ and $n_2 = 1$ for a
pair of doublets of $SU(2)_L$. For $U(1)_Y$, one has $n_1 = 6Y^2/5$
for each messenger pair with weak hypercharges $\pm Y$. 
In computing $n_1$ one must remember to add up the contributions for each
component of an $SU(3)_C$
or $SU(2)_L$ multiplet. So, for example,
$(n_1, n_2, n_3) = (2/5, 0, 1)$ for $q+\overline q$ and 
$(n_1, n_2, n_3) = (3/5, 1, 0)$ for $\ell+\overline \ell$.
Thus the total is 
$\sum_i (n_1, n_2, n_3) = (1, 1, 1)$ for the minimal model, so that
eq.~(\ref{gauginogmsbgen}) is in agreement
with
eq.~(\ref{gauginogmsb}). On general group-theoretic grounds, $n_2$ and
$n_3$ must be integers,
and $n_1$ is always an integer multiple of $1/5$
if fractional electric charges are confined.

The MSSM scalar masses in this generalized gauge-mediation framework are
now:
\beq  
m^2_\phi =
2 \Lambda^2
\left [ \left ({\alpha_3\over 4\pi}\right )^2 C_3^\phi
\sum_i n_3(i)+
\left ({\alpha_2\over
4 \pi}\right )^2 C_2^\phi 
\sum_i n_2(i)+
\left ({\alpha_1\over 4 \pi}\right )^2 C_1^\phi 
\sum_i n_1(i)
\right ] .
\label{scalargmsbgen}
\eeq
In writing eqs.~(\ref{gauginogmsbgen}) and (\ref{scalargmsbgen}) as
simple sums,
we have implicitly assumed that the messengers are all approximately equal
in
mass, with
\beq
M_{\rm mess} \approx y_i \langle S \rangle .
\eeq
This is a good approximation if the $y_i$ are not too
different from each other, because the dependence of the MSSM mass
spectrum
on the $y_i$ is only logarithmic (due to RG running) for fixed $\Lambda$.
However, if large hierarchies in the
messenger masses are present, then the additive
contributions
to the gaugino and scalar masses from each individual messenger multiplet
$i$ should really instead be incorporated at the mass scale of that
messenger
multiplet.
Then RG evolution is used to run these various contributions down to
the electroweak or TeV scale; the individual messenger contributions
to scalar and gaugino masses as indicated above can be thought of as
threshold corrections to this RG running.

Messengers with masses far below the GUT scale will affect the running of
gauge couplings and might therefore be expected to ruin the apparent
unification shown in Fig.~\ref{fig:gaugeunification}. However, if the
messengers come in complete multiplets of the $SU(5)$
global 
symmetry\footnote{This $SU(5)$ symmetry may or may not be promoted to a
local gauge symmetry at the GUT scale. For our present purposes, it is
used simply as a classification scheme, since the global $SU(5)$ symmetry
is only approximate below the GUT scale at the messenger mass scale where
gauge mediation takes
place.} that contains the Standard Model gauge
group
and are not very different in mass, 
then approximate
unification of gauge couplings will still occur when they are
extrapolated up to the same scale
$M_U$ (but with a larger unified value for the gauge couplings at that
scale).
For this reason, a popular class of models is obtained by taking
the messengers to consist of $\nmess$ copies of the ${\bf 5}+{\bf 
\overline 5}$ of $SU(5)$, resulting in 
\beq
\nmess = \sum_i n_1(i) = \sum_i n_2(i) =\sum_i n_3(i) \> .
\eeq  
In terms of this integer parameter $N_5$,  eqs.~(\ref{gauginogmsbgen}) and
(\ref{scalargmsbgen}) reduce to 
\beq
&&M_a = 
{\alpha_a \over 4 \pi} \Lambda \nmess \\
&&m^2_\phi =   
2 \Lambda^2 \nmess
\sum_{a=1}^3 C_a^\phi \left ({\alpha_a\over 4\pi}\right )^2
,
\label{gmsbnmess}
\eeq
since now there are $\nmess$ copies of the minimal messenger sector 
particles running around the loops.
For example, the minimal model in
eq.~(\ref{minimalmess}) corresponds to $\nmess = 1$.
A single copy of ${\bf 10} + {\bf \overline{ 10}}$ of $SU(5)$ has Dynkin
indices
$\sum_i n_a(i) = 3$, and so can be substituted for 3 copies of ${\bf
5}+{\bf
\overline 5}$.
(Other combinations of messenger multiplets can also preserve the apparent
unification of gauge couplings.) 
Note that the gaugino masses scale like $\nmess$, while the scalar
masses scale like $\sqrt{\nmess}$. This means that sleptons and
squarks will tend to be relatively lighter for larger values of
$\nmess$ in non-minimal models.
However, if $\nmess$ is too
large, then the running gauge couplings will diverge before they can unify
at $M_U$. For messenger
masses of order $10^6$ GeV or less, for example, one needs $\nmess\leq 4$.

There are many other possible generalizations of the basic gauge-mediation
scenario as
described above.
An important general expectation in these models is that the
strongly-interacting
sparticles (squarks, gluino) should be heavier than weakly-interacting
sparticles (sleptons, bino, winos, higgsinos) simply because of the
hierarchy of gauge couplings $\alpha_3 > \alpha_2 > \alpha_1$.
The common feature which makes all of these
models very attractive is that the
masses of the squarks and sleptons depend only on their gauge
quantum numbers, leading automatically
to the degeneracy of squark and slepton
masses needed for suppression of FCNC effects.
But the most distinctive phenomenological prediction of
gauge-mediated models may be the fact that the gravitino is the LSP.
This can have crucial consequences for both cosmology and collider
physics, as we will discuss further in sections
\ref{subsec:decays.gravitino} and \ref{sec:signals}.

\section{The mass spectrum of the MSSM}\label{sec:MSSMspectrum}
\setcounter{equation}{0}
\setcounter{footnote}{1}

In this section, we will study the sparticle and Higgs mass spectrum of
the MSSM. We will pay
special attention to the general classes of models which fit into the
minimal supergravity
eqs.~(\ref{gauginounificationsugra})-(\ref{aunificationsugra})
or gauge-mediated
eqs.~(\ref{gauginogmsb})-(\ref{aaagmsb}) boundary conditions for the soft
terms.
As we have already discussed in section 5.4, the renormalization group
(RG) equations are a crucial tool in determining the lagrangian at the
electroweak scale, given a set of boundary conditions on the theory at
the (much higher) input scale.
Therefore, we will begin by looking at the RG equations for the parameters
of the model, in section \ref{subsec:MSSMspectrum.rges}. 
Of course, the boundary conditions
on soft parameters are quite model-dependent even within the
minimal supergravity and gauge-mediated frameworks, but there
are some important
general
lessons to be learned from the form of the RG equations.
Once the RG equations have been used to determine the effective lagrangian
at the electroweak scale, one can use the results of the earlier sections
to predict the mass spectrum, mixing angles, and interactions of all of
the new particles in the model.
In section \ref{subsec:MSSMspectrum.Higgs} we will discuss electroweak
symmetry breaking and the Higgs scalars. Sections
\ref{subsec:MSSMspectrum.inos},
\ref{subsec:MSSMspectrum.gluino},
\ref{subsec:MSSMspectrum.sfermions}
are devoted to the sparticle masses and mixings. Finally in section
\ref{subsec:MSSMspectrum.summary} we will summarize some of the general
features and expectations for
the MSSM spectrum.

\subsection{Renormalization Group
Equations}\label{subsec:MSSMspectrum.rges}

In order to translate a set of predictions at the input scale into
physically meaningful quantities which describe physics at the electroweak
scale, it is necessary to evolve the gauge couplings,
superpotential parameters, and soft terms using the RG equations.
As a technical aside, we note that when computing RG effects and other
radiative corrections in supersymmetry, it is important to choose
regularization and renormalization schemes that do not violate
supersymmetry.
The most popular regularization method for discussing radiative
corrections within the Standard Model is dimensional regularization
(DREG), in which the number of spacetime dimensions is continued
to $d=4-2\epsilon$. Unfortunately, DREG 
%is horrible for our purposes; it 
violates supersymmetry explicitly because it introduces a
mismatch between
the numbers of gauge boson degrees of freedom and
the gaugino degrees of freedom off-shell.
This mismatch is only $2\epsilon$, but can be multiplied by factors
up to $1/\epsilon^n$ in an $n$-loop calculation.
In DREG, supersymmetric relations between dimensionless coupling
constants (``supersymmetric Ward identities") are therefore disrespected
by radiative corrections involving
the finite parts of one-loop graphs and
by the divergent parts of two-loop graphs. Instead, one may
use the slightly different scheme known as regularization by dimensional
reduction, or DRED, which does respect supersymmetry.\cite{DRED}
In the DRED method, all momentum integrals are still performed in
$d=4-2\epsilon$
dimensions, but the vector index $\mu$ on the gauge boson fields
$A^a_\mu$ now runs over all 4 dimensions. Running couplings are then
renormalized using DRED with modified minimal subtraction ($\drbar$)
rather than the usual DREG with modified minimal subtraction ($\msbar$).
In particular, the boundary conditions at the input scale should be
applied
in the supersymmetry-preserving $\drbar$ scheme. (See Ref.\cite{Shifman}
for an alternative supersymmetric scheme.) One loop $\beta$-functions
are always the same in the two schemes, but it is important to
realize that the $\msbar$ scheme does violate supersymmetry, so that
$\drbar$ is preferred \footnote{Even the DRED
scheme may not provide a supersymmetric regulator, because of
ambiguities which appear at five-loop order at the
latest.\cite{DREDdies}
Fortunately, this does not seem to cause any practical
difficulties.\cite{JJperspective} See also Ref.\cite{Woodard} for a
promising proposal which avoids doing violence
to the number of spacetime dimensions.} from that point of view.
(It is also possible to work consistently within the
$\overline{\rm MS}$ scheme, as long as one is careful to 
correctly translate all
$\overline{\rm DR}$ couplings and masses into their
$\overline{\rm MS}$
counterparts.\cite{%mstodrone,
gluinopolemass,mstodrmore})

The MSSM RG equations in the $\drbar$ scheme are
given in Refs.\cite{rges1}$^{\!-\,}$\cite{threeloops};
they are now known for the gauge couplings and superpotential parameters
up to 3-loop order, and for the soft parameters at 2-loop
order. However, for many purposes including pedagogical ones it suffices
to work in the 1-loop approximation. Here, we will also use the
approximation that only the third family Yukawa couplings are significant;
see eq.~(\ref{heavytopapprox}).
Then the superpotential parameters run with scale according to:
\beq
{d\over dt} y_t\! \!\!&=&\!\!\! {y_t \over 16 \pi^2} \Bigl [ 6 |y_t|^2 +
|y_b|^2
- {16\over 3} g_3^2 - 3 g_2^2 - {13\over 15} g_1^2 \Bigr ];
\\
{d\over dt} y_b \!\!\!&=&\!\!\! {y_b \over 16 \pi^2} \Bigl [ 6 |y_b|^2 +
|y_t|^2 +
|y_\tau|^2
- {16\over 3} g_3^2 - 3 g_2^2 - {7\over 15} g_1^2 \Bigr ];
\\
{d\over dt} y_\tau \!\!\!&=&\!\!\! {y_\tau\over 16 \pi^2} \Bigl [ 4
|y_\tau |^2 +
3
|y_b|^2
- 3 g_2^2 - {9\over 5} g_1^2 \Bigr ];
\\
{d\over dt} \mu \!\!\!&=&\!\!\! {\mu \over 16\pi^2} \Bigl [ 3 |y_t|^2 + 3
|y_b|^2
+ |y_\tau |^2 - 3 g_2^2 - {3\over 5} g_1^2 \Bigr ].
\eeq
The one-loop RG equations for the gauge couplings $g_1,g_2,g_3$
have already been listed in eq.~(\ref{mssmg}).
Note that the $\beta$-functions (the quantities on the right side of
each equation)
for each supersymmetric parameter are proportional
to the parameter itself. This is actually a consequence of a
general and powerful result known as the
{\it supersymmetric nonrenormalization theorem}.\cite{nonrentheo}
This theorem implies that the logarithmically divergent
contributions to a given process can always be written in the form
of a wave-function renormalization, without any vertex
renormalization.\footnote{Actually, there
{\it is} vertex renormalization in the field theory in which auxiliary
fields have been integrated out, but the sum of divergent contributions
for a given process always has the form of
wave-function renormalization. See Ref.\cite{Jonesreview}
for a discussion of this point.}
It is true for any supersymmetric theory, not just the MSSM,
and holds to all orders in perturbation theory. It can be proved most
easily using superfield techniques.
In particular, it means that once we have a theory which can explain
why $\mu$ is of order $10^2$ or $10^3$ GeV at tree-level, we do not have to
worry about $\mu$ being
infected (made very large) by radiative corrections
involving the masses of some very heavy unknown particles; all such
RG corrections to $\mu$ will be directly proportional to $\mu$ itself.

The one-loop RG equations for the three gaugino mass parameters in the
MSSM are determined by the same quantities $b_a^{\rm MSSM}$ which appear in
the gauge coupling RG eqs.~(\ref{mssmg}):
\beq
{d\over dt} M_a = {1\over 8\pi^2} b_a g_a^2 M_a\qquad\>\>\>
(b_a = 33/5,1,-3)
\label{gauginomassrge}
\eeq
for $a=1,2,3$.
It is therefore easy to show that the three ratios $M_a/g_a^2$
are each constant (RG-scale independent) up to small two-loop
corrections. In minimal supergravity models, we can therefore write
\beq
M_a(Q) = {g_a^2(Q)\over g_a^2(Q_0)} m_{1/2} \qquad\>\>\>(a=1,2,3)
\eeq
at any RG scale $Q<Q_0$, where $Q_0$ is the input scale
which is presumably nearly equal to $M_P$.
Since the gauge couplings are observed to unify at $M_U\sim 0.01 M_P$,
one expects \footnote{In a GUT model, it is automatic that the
gauge couplings and gaugino masses are unified at all scales $Q> M_U$
and in particular at $Q\approx M_P$,
because in the unified theory the gauginos all live in the same
representation of the unified gauge group. In many superstring models,
this is also known to be a good approximation.}
%that $g_1^2(M_P) \approx g_2^2(M_P) \approx
%g_3^2 (M_P)$.
that $g_1^2(Q_0) \approx g_2^2(Q_0) \approx
g_3^2 (Q_0)$.
Therefore, one finds that
\beq
{M_1 \over g_1^2} =
{M_2 \over g_2^2} =
{M_3 \over g_3^2}
\label{gauginomassunification}
\eeq
at any RG scale, up to small two-loop effects and possibly larger
threshold
effects near $M_U$ and $M_P$. The common value in
eq.~(\ref{gauginomassunification}) is also equal to $m_{1/2}/g_U^2$
in minimal supergravity models, where $g_U$ is the unified gauge coupling
at the input scale where $m_{1/2}$ is the common gaugino mass.
Interestingly, eq.~(\ref{gauginomassunification}) is {\it also}
the solution to the one-loop RG equations in the case of the
gauge-mediated boundary conditions eq.~(\ref{gauginogmsb})
applied at the messenger mass scale.
This is true even though there is no such thing as a
unified gaugino mass $m_{1/2}$ in the gauge-mediated case,
because of the fact that the gaugino masses are proportional to
the $g^2_a$ times a constant.
So eq.~(\ref{gauginomassunification}) is theoretically well-motivated
(but certainly not inevitable) in both frameworks. The
prediction eq.~(\ref{gauginomassunification})
is particularly useful since the gauge couplings $g_1^2$, $g^2_2$,
and $g_3^2$ are already quite well known at the electroweak scale from
experiment. Therefore they can be extrapolated up to at least $M_U$,
assuming that the apparent unification of gauge couplings is not
a fake.
The gaugino mass parameters feed into the RG equations for all of the
other soft terms, as we will see.

Next we consider the 1-loop RG equations for the analytic
soft parameters ${\bf a_u}$, ${\bf a_d}$, ${\bf a_e}$.
In models obeying eq.~(\ref{aunification}), these matrices
start off proportional to the corresponding Yukawa couplings at the
input scale, and the RG evolution respects this property. With the
approximation of eq.~(\ref{heavytopapprox}), one can therefore also write,
at any RG scale,
\beq
{\bf a_u} \approx \pmatrix{0&0&0\cr 0&0&0 \cr 0&0&a_t};\qquad\!\!
{\bf a_d} \approx \pmatrix{0&0&0\cr 0&0&0 \cr 0&0&a_b};\qquad\!\!
{\bf a_e} \approx \pmatrix{0&0&0\cr 0&0&0 \cr 0&0&a_\tau},\>\>{}
\label{heavyatopapprox}
\eeq
which defines \footnote{We
must warn the reader that rescaled soft parameters
$A_t = a_t/y_t$, $A_b=a_b/y_b$, and
$A_\tau=a_\tau/y_\tau$ are commonly used in the literature.
We do not follow this notation, because
it cannot be generalized beyond the approximation of
eqs. (\ref{heavytopapprox}), (\ref{heavyatopapprox})
without introducing horrible complications such as
non-polynomial RG equations, and because $a_t$, $a_b$ and $a_\tau$
are the couplings that actually appear in the lagrangian anyway.}
running parameters $a_t$, $a_b$, and $a_\tau$.
The RG equations for these parameters
and $b$ are given by
\beq
16\pi^2 {d\over dt} a_t \!\!\!&=&\!\!\! a_t \Bigl [ 18 |y_t|^2 + |y_b|^2
- {16\over 3} g_3^2 - 3 g_2^2 - {13\over 15} g_1^2 \Bigr ]
\nonumber\\ && + 2 a_b y_b^* y_t
+ y_t \Bigl [ {32\over 3} g_3^2 M_3 + 6 g_2^2 M_2 + {26\over 15} g_1^2 M_1
\Bigr
];
\label{atrge}
\\
16\pi^2{d\over dt} a_b \!\!\!&=&\!\!\! a_b \Bigl [ 18 |y_b|^2 + |y_t|^2 +
|y_\tau|^2
- {16\over 3} g_3^2 - 3 g_2^2 - {7\over 15} g_1^2 \Bigr ]
\nonumber \\&&
+ 2 a_t y_t^* y_b + 2 a_\tau y_\tau^* y_b
+ y_b \Bigl [ {32\over 3} g_3^2 M_3 + 6 g_2^2 M_2 + {14 \over 15} g_1^2
M_1 \Bigr
];\qquad{}
\\
16\pi^2{d\over dt} a_\tau \!\!\!&=&\!\!\! a_\tau \Bigl [ 12 |y_\tau|^2 + 3
|y_b|^2
- 3 g_2^2 - {9\over 5} g_1^2 \Bigr ]
\nonumber \\ && + 6 a_b y_b^* y_\tau
+ y_\tau \Bigl [ 6 g_2^2 M_2 + {18\over 5} g_1^2 M_1 \Bigr ];
\\
16\pi^2{d\over dt} b \!\!\!&=&\!\!\! b \Bigl [ 3 |y_t|^2 + 3 |y_b|^2
+ |y_\tau |^2 - 3 g_2^2 - {3\over 5} g_1^2 \Bigr ]
\nonumber \\ && +
\mu \Bigl [ 6 a_t y_t^* + 6 a_b y_b^* + 2 a_\tau y_\tau^* +
6 g_2^2 M_2 + {6\over 5} g_1^2 M_1 \Bigr ]
\label{brge}
\eeq
in this approximation.
The $\beta$-function for each of these
soft parameters is {\it not} proportional
to the
parameter itself; this makes sense because
couplings which violate supersymmetry are not protected
by the supersymmetric nonrenormalization theorem.
In particular, even if $A_0$ and $B_0$ appearing in
eqs.~(\ref{aunificationsugra}) and (\ref{bsilly}) vanish at the input
scale, the RG corrections
proportional to gaugino
masses appearing  in eqs.~(\ref{atrge})-(\ref{brge})
ensure that $a_t$, $a_b$, $a_\tau$ and $b$
will still be non-zero at the electroweak scale.

Next let us consider the RG equations for the scalar masses in the MSSM.
In the approximation of
eqs.~(\ref{heavytopapprox}) and (\ref{heavyatopapprox}), the
squarks and sleptons of the first two families
have only gauge interactions. This means that
if the scalar masses satisfy
a boundary condition like eq.~(\ref{scalarmassunification}) at an input
RG scale, then when renormalized to any other RG scale, they will
still be almost diagonal, with the approximate form
\beq
{\bf m_Q^2} \approx \pmatrix{
m_{Q_1}^2 & 0 & 0\cr
0 & m_{Q_1}^2 & 0 \cr
0 & 0 & m_{Q_3}^2 \cr};\qquad\>\>\>
{\bf m_{\sbar u}^2} \approx \pmatrix{
m_{\sbar u_1}^2 & 0 & 0\cr
0 & m_{\sbar u_1}^2 & 0 \cr
0 & 0 & m_{\sbar u_3}^2 \cr};
\eeq
etc. The first and second family squarks and sleptons with given
gauge quantum numbers remain very nearly degenerate, but the third family
squarks and sleptons feel the effects of the larger Yukawa couplings
and so get renormalized differently. The
one-loop RG equations for the first and second family squark
and slepton squared masses
can be written as
\footnote{There are also terms in
the scalar (mass)$^2$ RG equations which are
proportional to Tr$[Ym^2]$ (the sum of the weak
hypercharge times the soft (mass)$^2$ for all scalars in the theory).
However, these contributions vanish in both the cases of
minimal supergravity and gauge-mediated boundary conditions for the
soft terms, as one can see by explicitly calculating Tr$[Ym^2]$
in each case.
If Tr$[Ym^2]$ is zero at the input scale, then it will remain zero
under RG evolution.
Therefore we neglect such terms
in our discussion, although they can have an important
effect in more general situations.}
\beq
16 \pi^2 {d\over dt} m_{\phi}^2 = - \sum_{a=1,2,3} 8 g_a^2 C_a^\phi
|M_a|^2
\label{easyscalarrge}
\eeq
for each scalar $\phi$,
where the $\sum_a$ is over the three gauge groups $U(1)_Y$,
$SU(2)_L$ and $SU(3)_C$; $M_a$ are the corresponding running gaugino mass
parameters which are known from eq.~(\ref{gauginomassunification});
and the constants $C_a^\phi$ are the
same quadratic Casimir invariants which appeared in
 eqs.~(\ref{defC3})-(\ref{defC1}).
 An important feature of eq.~(\ref{easyscalarrge}) is that
the right-hand sides are strictly negative, so that
the scalar (mass)$^2$ parameters
{\it grow} as they are RG-evolved from the input scale down to
the electroweak scale. Even if the scalars
have zero or very
small masses at the input scale, as in the ``no-scale"
boundary condition limit $m_0^2=0$, they will obtain
large positive squared masses at
the electroweak scale, thanks to the effects of the gaugino masses.

The RG equations for the (mass)$^2$ parameters of the
Higgs scalars and third family squarks
and sleptons get the same gauge contributions as in
eq.~(\ref{easyscalarrge}), but
they also have contributions due to the
large Yukawa ($y_{t,b,\tau}$) and soft
 ($a_{t,b,\tau}$) couplings. At one-loop order, these
only appear in three combinations:
\beq
X_t \!\!\!&=&\!\!\!  2 |y_t|^2 (m_{H_u}^2 + m_{Q_3}^2 + m_{\sbar u_3}^2) +
2
|a_t|^2,
\\
X_b\!\!\! &=& \!\!\! 2 |y_b|^2 (m_{H_d}^2 + m_{Q_3}^2 + m_{\sbar d_3}^2) +
2
|a_b|^2,
\\
X_\tau\!\!\! &=&\!\!\!  2 |y_\tau|^2 (m_{H_d}^2 + m_{L_3}^2 + m_{\sbar
e_3}^2)
+ 2 |a_\tau|^2.
\eeq
In terms of these quantities, the RG equations for the soft Higgs
(mass)$^2$ parameters $m_{H_u}^2$ and $m_{H_d}^2$ are
\beq
16 \pi^2 {d\over dt} m_{H_u}^2 \!\!\!&=&\!\!\!
3 X_t - 6 g_2^2 |M_2|^2 - {6\over 5} g_1^2 |M_1|^2,
\label{mhurge}
\\
16\pi^2{d\over dt} m_{H_d}^2 \!\!\!&=&\!\!\!
3 X_b + X_\tau - 6 g_2^2 |M_2|^2 - {6\over 5} g_1^2 |M_1|^2.
\label{mhdrge}
\eeq
Note that $X_t$, $X_b$, and $X_\tau$ are positive, so their effect
is always to {\it decrease} the Higgs masses as one evolves the RG
equations downward from
the input scale to the electroweak scale. Since $y_t$ is the largest
of the Yukawa couplings because of the experimental fact that the top
quark is heavy, $X_t$ is typically expected to be larger than $X_b$
and $X_\tau$. This can cause the RG-evolved $m_{H_u}^2$ to run negative
near the electroweak scale, helping to destabilize the point $H_u =0$
and so provoking a Higgs VEV which is just what we want.\footnote{One
should think of ``$m_{H_u}^2$" as a parameter unto itself, and not as
the square of some mythical real number $m^{\phantom{2}}_{H_u}$. Thus
there
is nothing strange about having $m_{H_u}^2 < 0$.
However,
strictly speaking $m_{H_u}^2< 0$ is neither necessary nor sufficient
for electroweak symmetry breaking; see section
\ref{subsec:MSSMspectrum.Higgs}.}
Thus a large top Yukawa coupling favors the breakdown
of the electroweak symmetry breaking because it induces negative radiative
corrections to the Higgs (mass)$^2$.

The third family squark and slepton (mass)$^2$ parameters also get
contributions which depend on $X_t$, $X_b$ and $X_\tau$.
Their RG equations are given by
\beq && \!\!\!\!\!\!\!\! 16\pi^2{d\over dt} m_{Q_3}^2 =
X_t +X_b- {32\over 3} g_3^2 |M_3|^2 - 6 g_2^2 |M_2|^2 - {2\over 15} g_1^2
|M_1|^2
\>\>\>\>\>\>\>{}
\label{mq3rge} \\
&&\!\!\!\!\!\!\!\! 16\pi^2 {d\over dt} m_{\sbar u_3}^2 =
2 X_t - {32\over 3} g_3^2 |M_3|^2
- {32\over 15} g_1^2|M_1|^2
\label{mtbarrge}
\\
&& \!\!\!\!\!\!\!\! 16\pi^2 {d\over dt} m_{\sbar d_3}^2 =
2 X_b - {32\over 3} g_3^2 |M_3|^2
- {8\over 15} g_1^2|M_1|^2
\label{md3rge}
\\
&& \!\!\!\!\!\!\!\! 16\pi^2 {d\over dt} m_{L_3}^2 =
X_\tau  - 6 g_2^2 |M_2|^2 - {3\over 5} g_1^2
|M_1|^2
\\
&& \!\!\!\!\!\!\!\! 16\pi^2 {d\over dt} m_{\sbar e_3}^2 =
2 X_\tau - {24\over 5} g_1^2|M_1|^2 .
\label{mstaubarrge}\eeq
In eqs.~(\ref{mhurge})-(\ref{mstaubarrge}), the terms proportional
to $|M_3|^2$, $|M_2|^2$ and $|M_1|^2$ are just the same ones as in
eq.~(\ref{easyscalarrge}). Note that the terms proportional to $X_t$
appear with smaller numerical coefficients in the $m^2_{Q_3}$ and
$m^2_{\sbar u_3}$ RG equations than they did for
the Higgs scalars, and they do not
appear at all in the $m^2_{\sbar d_3}$, $m^2_{L_3}$ and $m^2_{\sbar e_3}$
RG equations.
Furthermore, the third-family squark (mass)$^2$ get a large positive
contribution
proportional to $|M_3|^2$ from the RG evolution, which the Higgs scalars
do not get.
These facts make it easy
to understand why the Higgs scalars in the MSSM can get VEVs, but the
squarks and sleptons, having large positive (mass)$^2$,
do not.
An examination of the RG equations (\ref{atrge})-(\ref{brge}),
(\ref{easyscalarrge}), and (\ref{mhurge})-(\ref{mstaubarrge})
reveals that if the gaugino mass parameters $M_1$, $M_2$, and $M_3$
are non-zero at the input scale, then all of the other soft terms will be
generated. This is why the ``no-scale" limit with
$m_{1/2} \gg m_0, A_0, B_0$ can be phenomenologically viable even though
the squarks and sleptons are massless at tree-level. On the other hand,
if the gaugino masses were to vanish at tree-level, then they would
not get any contributions to their masses at one-loop order;
in that case $M_1$, $M_2$, and $M_3$ would be extremely small.

Now that we have reviewed the effects of RG evolution from the input
scale down to the electroweak or TeV scale, we are ready to
work out the expected features of the MSSM spectrum in some detail. We
will begin with
the Higgs sector in the next section.

\subsection{Electroweak symmetry breaking and the Higgs
bosons}\label{subsec:MSSMspectrum.Higgs}

In the MSSM, the description of electroweak symmetry breaking is
slightly complicated by the fact that there are two complex
Higgs doublets $H_u = (H_u^+,\> H_u^0)$ and $H_d = (H_d^0,\> H_d^-)$
rather than just one in the ordinary Standard Model.
The classical scalar potential for the Higgs scalar fields in the MSSM
is given by
\beq
V\! &=&\!
(|\mu|^2 + m^2_{H_u}) (|H_u^0|^2 + |H_u^+|^2)
+ (|\mu|^2 + m^2_{H_d}) (|H_d^0|^2 + |H_d^-|^2)
\nonumber \\ &&+\, b\, (H_u^+ H_d^- - H_u^0 H_d^0) + \conj
\nonumber \\ &&+ {1\over 8} (g^2 + g^{\prime 2})
( |H_u^0|^2 + |H_u^+|^2 - |H_d^0|^2 - |H_d^-|^2 )^2
\nonumber \\
&& + \half g^2 |H_u^+ H_d^{0*} + H_u^0 H_d^{-*}|^2 .
\label{bighiggsv}
\eeq
The terms proportional to $|\mu |^2$ come from $F$-terms
[see the first term on the right-hand
side of eq.~(\ref{movie})]. The terms proportional to $m_{H_u}^2$,
$m_{H_d}^2$ and $b$ are nothing but a rewriting of the last three
terms of eq.~(\ref{MSSMsoft}). Finally,
the terms proportional to
$g^2$ and $g^{\prime 2}$ are the $D$-term contributions which may be
derived from
the general formula eq.~(\ref{fdpot}), after some
rearranging.
The
full scalar potential of the theory will also
include many terms involving the squark and slepton fields that we
can ignore here, since they do not
get VEVs because they have large positive (mass)$^2$.

We now have to demand that the minimum of this potential
should break electroweak symmetry down to electromagnetism
$SU(2)_L\times
U(1)_Y
\rightarrow U(1)_{\rm EM}$, in accord with
experiment.
We can use the freedom to make gauge transformations to simplify this
analysis. First, the freedom to make $SU(2)_L$ gauge
transformations
allows us to rotate away a possible VEV for one of the weak isospin
components
of one of the scalar fields; so without loss of generality we can take
$H_u^+=0$ at the minimum of the potential. Then one finds that a minimum
of the potential satisfying $\partial V/\partial H_u^+=0$ must also
have $H_d^- = 0$. This is good, because it means that at the minimum of the
potential electromagnetism is necessarily unbroken,
since the charged components of the Higgs scalars cannot get VEVs.
So after setting $H_u^+=H_d^-=0$ we are left to consider the scalar
potential
\beq
V \!&=&\!
(|\mu|^2 + m^2_{H_u}) |H_u^0|^2
+ (|\mu|^2 + m^2_{H_d}) |H_d^0|^2
- (b\, H_u^0 H_d^0 + \conj)
\nonumber \\ && + {1\over 8} (g^2 + g^{\prime 2})
( |H_u^0|^2 - |H_d^0|^2 )^2 .
\label{littlehiggsv}
\eeq
The only term in this potential which depends on the phases of the
fields is the $b$-term. Therefore a redefinition of the phases of $H_u$
and $H_d$ can absorb any phase in $b$, so we can take $b$ to be real
and positive.
Then it is clear that a minimum of the potential $V$ requires that
$H_u^0
H_d^0$ is also real and positive,
so $\langle H_u^0\rangle$ and $\langle H_d^0\rangle$ must have
opposite phases.
We can therefore use a $U(1)_Y$ gauge transformation to
make them both be real and positive without loss
of generality, since $H_u$ and $H_d$
have opposite weak hypercharges ($\pm 1/2$).
It follows that CP cannot be spontaneously broken by the Higgs scalar
potential, since all of the VEVs and couplings can be simultaneously
chosen to be real. This means that
the Higgs scalar mass eigenstates can be assigned well-defined
eigenvalues of CP.

Note that the $b$-term always favors electroweak
symmetry
breaking.
The combination of the $b$ term and the terms
$m_{H_u}^2$ and $m_{H_d}^2$ can allow for one linear combination of
$H_u^0$ and $H_d^0$ to have a negative (mass)$^2$ near
$H_u^0=H_d^0=0$.
This requires that
\beq
b^2 > (|\mu|^2 + m^2_{H_u} )(|\mu|^2 + m^2_{H_d}).
\label{destabilizeorigin}
\eeq
If this
inequality is not
satisfied, 
 then $H_u^0 = H_d^0 = 0$
will be a stable minimum of the potential, and
electroweak symmetry breaking will not occur.
A negative value
for $|\mu|^2 + m_{H_u}^2$ will help
eq.~(\ref{destabilizeorigin}) to be satisfied, but it is not necessary.
Furthermore, even
if $m_{H_u}^2<0$, there may be no electroweak symmetry breaking if
$|\mu|$ is too large or if $b$ is too small. Still, the large negative
contributions to $m_{H_u}^2$ from the RG equation (\ref{mhurge}) discussed
in the previous section are an
important factor in ensuring that electroweak symmetry breaking can
occur in models with minimal supergravity or gauge-mediated boundary
conditions for the soft terms.

In order for the MSSM scalar potential to be viable, it is not enough that
the point $H_u^0=H_d^0=0$ is
destabilized by a negative (mass)$^2$ direction; we must also
make sure that the potential is bounded from below for
arbitrarily large values of the scalar fields, so that $V$ will really
have a minimum.
(Recall from the discussion in sections
\ref{subsec:susylagr.chiral} and \ref{subsec:susylagr.gaugeinter} that
scalar potentials
in purely supersymmetric theories are automatically positive and
so  clearly bounded from below. But,
now that we have introduced supersymmetry breaking, we must be careful.)
The scalar quartic interactions in $V$ will stabilize the potential
for almost all arbitrarily large values of $H_u^0$ and $H_d^0$.
However, there
are special directions in
field space with $|H_u^0| = |H_d^0|$,
along which the quartic contributions to $V$
[the second line in eq.~(\ref{littlehiggsv})] are identically zero.
Such directions in field space are called $D$-flat directions, because
along them the part of the scalar potential coming from $D$-terms vanishes.
In order for the potential to be bounded from below, we need the
quadratic part of the scalar potential to be positive along the
$D$-flat directions. This requirement amounts to
\beq
2 b< 2 |\mu |^2 + m^2_{H_u} + m^2_{H_d}.
\label{boundedfrombelow}
\eeq
 Interestingly, if $m_{H_u}^2 = m_{H_d}^2$,
the constraints eqs.~(\ref{destabilizeorigin}) and
(\ref{boundedfrombelow}) cannot both be satisfied.
In models derived from the minimal supergravity or gauge-mediated
boundary
conditions,
$m_{H_u}^2 = m_{H_d}^2$ holds at tree-level at the input scale, but
the $X_t$ contribution to the RG equation for $m_{H_u}^2$ naturally
pushes it to negative or small values $m_{H_u}^2 < m_{H_d}^2$
at the electroweak scale,
as we saw in section \ref{subsec:MSSMspectrum.rges}.
Unless this effect is large, the parameter space in which
the electroweak symmetry is broken would be quite small.
So in these models
electroweak symmetry breaking is actually driven purely by
quantum corrections; this mechanism is therefore
known as {\it radiative electroweak symmetry breaking}.
The realization that this works most
naturally with a large top-quark
Yukawa coupling provides additional motivation for these
models.\cite{rges1,rewsb}

Having established the conditions necessary for $H_u^0$ and $H_d^0$
to get non-zero VEVs, we can now require that they are compatible with
the observed phenomenology of electroweak symmetry breaking
$SU(2)_L \times U(1)_Y \rightarrow U(1)_{\rm EM}$.
Let us write
$\langle H_u^0\rangle = v_u$ and
$\langle H_d^0\rangle = v_d$ for the VEVs at the minimum of the
potential. These VEVs can be connected to the known mass of the $Z^0$ boson
and the electroweak gauge couplings:
\beq
v_u^2 + v_d^2 = v^2 = 2 m_Z^2/(g^2 + g^{\prime 2}) \approx (174\>{\rm
GeV})^2.
\label{vuvdcon}
\eeq
The ratio of the two VEVs is traditionally written as
\beq
\tan\beta \equiv v_u/v_d.
\label{deftanbeta}
\eeq
The value of $\tan\beta$ is not fixed by present experiments, but it
depends on the lagrangian
parameters of the MSSM in a calculable way.
Since $v_u = v \sin\beta$ and $v_d = v \cos\beta$ 
were taken to be real and 
positive, we have
$0 < \beta < \pi/2$, a requirement that will be
sharpened below.
Now one can write down the conditions $\partial V/\partial H_u^0=
\partial V/\partial H_d^0 = 0$ under
which the potential
eq.~(\ref{littlehiggsv}) will have a minimum satisfying
eqs.~(\ref{vuvdcon}) and (\ref{deftanbeta}):
\beq
&&|\mu |^2 + m_{H_d}^2 =  b \tan\beta - (m_Z^2/2) \cos 2\beta;
\label{mubsub1}
\\
&&|\mu |^2 + m_{H_u}^2 =  b \cot\beta + (m_Z^2/2) \cos 2\beta .
\label{mubsub2}
\eeq
It is easy to check that these equations
indeed satisfy the necessary conditions eqs.~(\ref{destabilizeorigin}) and
(\ref{boundedfrombelow}).
They allow us to eliminate two of the lagrangian parameters
$b$ and $|\mu|$ in favor of $\tan\beta$, but do not determine the
phase of $\mu$.

As an aside, we note that eqs.~(\ref{mubsub1}) and (\ref{mubsub2})
highlight the ``$\mu$ problem" already mentioned in section
\ref{subsec:mssm.superpotential}.
If we view
$|\mu|^2$, $b$, $m_{H_u}^2$ and $m_{H_d}^2$ as input parameters,
and $m_Z^2$ and $\tan\beta$ as output parameters obtained by solving
these two equations, then without
miraculous cancellations we expect that all of the input parameters ought
to be within an order of magnitude or two of $m^2_Z$. However, in the
MSSM,
$\mu$ is a supersymmetry-respecting parameter appearing in the
superpotential, while $b$,
$m_{H_u}^2$, $m_{H_d}^2$ are supersymmetry-breaking parameters. This has
lead to a widespread belief that the MSSM must be extended at very high
energies to include a mechanism which relates the effective value of
$\mu$
to the supersymmetry-breaking mechanism in some way; see section
\ref{subsec:variations.NMSSM}
and Refs.\cite{muproblemW,muproblemK,muproblemGMSB} for examples.

The Higgs scalar fields in the MSSM consist of two complex
$SU(2)_L$-doublet,
or eight real, scalar degrees of freedom.
When the electroweak symmetry is broken, three of them are the
would-be Nambu-Goldstone bosons $G^0$, $G^\pm$
which become the longitudinal modes of the $Z^0$ and $W^\pm$ massive
vector bosons. The remaining five Higgs scalar mass eigenstates consist of
one CP-odd neutral scalar
$A^0$,
a charge $+1$ scalar $H^+$ and its conjugate charge $-1$ scalar
$H^-$,
and two CP-even neutral scalars $h^0$ and $H^0$.
In terms of the original gauge-eigenstate fields,
the mass eigenstates and would-be Nambu-Goldstone bosons
are given by
\beq
\pmatrix{G^0 \cr A^0 } = \sqrt{2}
\pmatrix{\sin\beta & -\cos\beta\cr
          \cos\beta &  \sin\beta } \pmatrix{{\rm Im}[ H_u^0] \cr
                                              {\rm Im}[ H_d^0] },
\eeq
\beq
\pmatrix{G^+ \cr H^+ } =
\pmatrix{\sin\beta & -\cos\beta\cr
          \cos\beta &  \sin\beta } \pmatrix{H_u^+ \cr
                                              H_d^{-*} },
\eeq
with $G^- = G^{+*}$ and $H^- = H^{+*}$, and
\beq
\pmatrix{h^0 \cr H^0 } = \sqrt{2}
\pmatrix{\cos\alpha & -\sin\alpha\cr
          \sin\alpha &  \cos\alpha } \pmatrix{{\rm Re}[ H_u^0] - v_u \cr
                                              {\rm Re}[H_d^0] - v_d}.
\label{defalphahH}
\eeq
which defines a mixing angle $\alpha$.
The tree-level masses of these fields can be found by expanding the
scalar potential around the minimum. One obtains
\beq
m_{A^0}^2 \!\!\!&=&\!\!\! 2 b/\sin 2\beta
\\
m^2_{H^\pm} \!\!\!&=&\!\!\! m^2_{A^0} + m_W^2
\\
m^2_{h^0, H^0} \!\!\!\!&=\!\!\!& \half
\Bigl (
m^2_{A^0} + m_Z^2 \mp \sqrt{(m_{A^0}^2 + m_Z^2)^2 - 4 m_Z^2 m_{A^0}^2
\cos^2 2\beta} \Bigr ).\>\>\>\>\>{}
\label{m2hH}
\eeq
In terms of these masses, the mixing angle $\alpha$ appearing in
eq.~(\ref{defalphahH}) is determined at tree-level by
\beq
{\sin 2\alpha\over \sin 2\beta} =
-{m_{A^0}^2 + m_{Z}^2 \over m_{H^0}^2 - m^2_{h^0}};\qquad\>\>
{\cos 2\alpha\over \cos 2\beta} =
-{m_{A^0}^2 - m_{Z}^2 \over m_{H^0}^2 - m^2_{h^0}}.\>\>\>
\eeq
The Feynman rules for couplings of the mass eigenstate Higgs scalars to the
Standard Model
quarks and leptons and the electroweak vector bosons, as well as to
the various sparticles, have been worked out in detail in
Ref.\cite{GunionHaber,HHG}

The masses of $A^0$, $H^0$ and $H^\pm$ can in principle
be arbitrarily large since they all grow with $b/\sin 2\beta$.
In contrast, the mass of $h^0$ is bounded from above. It
is not hard to show from eq.~(\ref{m2hH}) that
\beq
m_{h^0} <  |\cos 2\beta | m_Z
\eeq
at tree-level.\cite{treelevelhiggsbound} If this inequality were robust,
it would guarantee
that the lightest Higgs boson of the MSSM would be kinematically
accessible to LEP2, with large regions of parameter space already
ruled out. However, the tree-level mass formulas given above for
the Higgs mass eigenstates are subject to quite significant quantum
corrections which are especially important to take into
account in the case of $h^0$. The largest such contributions typically
come from top-stop loop corrections to the terms in the scalar
potential.
In the limit of stop squark masses
$m_{\stilde t_1}$, $m_{\stilde t_2}$ much greater than the top quark mass
$m_t$,
one finds a one-loop radiative correction to eq.~(\ref{m2hH}):
\beq
\Delta (m^2_{h^0}) =
{3\over 4 \pi^2} v^2 y_t^4 \sin^4\!\beta
\>\> {\rm ln}\left (m_{\stilde t_1} m_{\stilde t_2} \over m_t^2 \right )
{}.
\label{hradcorr}
\eeq
Including this and other corrections,\cite{hcorrections,HHH} one
can obtain only a considerably weaker, but still very interesting, bound
\beq
m_{h^0} \lsim 130\>{\rm GeV}
\label{mssmhiggsbound}
\eeq
in the MSSM. This assumes that all of the sparticles that can contribute
to $\Delta (m_{h^0}^2)$ in loops have masses that do not exceed 1 TeV.
By adding extra supermultiplets to the MSSM, this
bound can be made even weaker.
However, assuming that none of the MSSM sparticles have masses exceeding
1 TeV and that all of the couplings in the theory remain perturbative
up to the unification scale, one still finds \cite{KKW}
\beq
m_{h^0} \lsim 150\>{\rm GeV}.
\label{generalhiggsbound}
\eeq
This bound is also weakened if, for example, the top squarks are heavier
than
1 TeV, but the upper bound rises only logarithmically with the soft
masses, as can be seen from eq.~(\ref{hradcorr}).
Thus it is a fairly robust prediction of
supersymmetry at the electroweak scale
that at least one of the Higgs scalar bosons must be light.

An interesting limit occurs when $m_{A^0} \gg m_Z$. In that
case, $m_{h^0}$ can saturate the upper bound just mentioned
with $m_{h^0} \approx m_Z |\cos 2\beta|$ at tree-level, but subject
to large positive quantum corrections.
The particles
$A^0$, $H^0$, and $H^\pm$ are much heavier and nearly degenerate,
forming an isospin doublet which decouples from
sufficiently low-energy experiments.
The angle $\alpha$ is
fixed to be approximately $\beta-\pi/2$.
In this limit,
$h^0$ has the same couplings to quarks and leptons and electroweak
gauge bosons as would
the physical Higgs boson of the
ordinary Standard Model without supersymmetry.
Indeed, model-building experiences
have
shown that it is quite common for $h^0$
to behave in a way nearly indistinguishable from a Standard
Model-like Higgs boson, even if $m_{A^0}$ is not
too huge. On the other hand, it is important to keep in mind that
the couplings of $h^0$ might turn out
to deviate in important ways from those of a Standard Model Higgs boson.
For a given set of model parameters, it is very important to take into
account the complete set of one-loop corrections and
even the dominant two-loop effects in a leading logarithm approximation
in order to get accurate predictions for the Higgs masses and mixing
angles.\cite{hcorrections,HHH}

In the MSSM, the masses and CKM mixing angles of the quarks and leptons
are determined by the Yukawa couplings of the superpotential and the
parameter $\tan\beta$. This is because the top, charm and up quarks
get masses proportional to $v_u = v \sin\beta$ and the bottom, strange, and
down
quarks and the charge leptons get masses proportional to $v_d = v
\cos\beta$.
Therefore one finds at tree-level
\beq
y_t = {g m_t\over \sqrt{2} m_W \sin\beta};
\qquad
y_b = {g m_b\over \sqrt{2} m_W \cos\beta};
\qquad
y_\tau = {g m_\tau\over \sqrt{2} m_W \cos\beta}
.\qquad{}
\label{ytbtau}
\eeq
These relations hold for the running masses of $t,b,\tau$ rather than
the physical pole masses which are significantly larger.\cite{polecat}
Including those corrections, one can
relate the Yukawa couplings to $\tan\beta$ and the known fermion masses
and CKM mixing angles. It is now clear why we have not neglected
$y_b$ and $y_\tau$, even though $m_b,m_\tau\ll m_t$. To a first
approximation,
$y_b/y_t =  (m_b/m_t)\tan\beta$ and $y_\tau/y_t =
(m_\tau/m_t)\tan\beta$, so
that $y_b$ and $y_\tau$ cannot be neglected if $\tan\beta$ is much larger
than 1. In fact, there are good theoretical motivations for considering
models with large $\tan\beta$. For example, models based on the
GUT gauge group $SO(10)$ (or certain of its subgroups) can unify the
running top, bottom and tau Yukawa couplings at the unification scale;
this requires $\tan\beta$ to be very roughly of order
$m_t/m_b$.\cite{so10,copw}

Note that if one tries to make $\sin\beta$ too small, $y_t$ will become
nonperturbatively
large. Requiring that $y_t$ does not blow up above the electroweak scale,
one finds that $\tan\beta \gsim 1.2$ or so, depending on the mass of
the top quark, the QCD coupling, and other fine details. In principle,
one can also determine a lower bound on $\cos\beta$ and thus an upper bound
on $\tan\beta$ by requiring that $y_b$ and $y_\tau$ are not
nonperturbatively large.
This gives a rough upper
bound of $\tan\beta \lsim$ 65.
However, this is complicated slightly by the fact
that the bottom quark mass gets significant one-loop corrections in the
large $\tan\beta$ limit.\cite{copw} One can obtain a slightly
stronger upper bound on $\tan\beta$ in models where
$m_{H_u}^2 = m_{H_d}^2$ at the input scale, by requiring that $y_b$
does not significantly exceed $y_t$. [Otherwise,
$X_b$ would be larger than $X_t$ in eqs.~(\ref{mhurge})
and (\ref{mhdrge}), so one would find $m_{H_d}^2 <
m_{H_u}^2$ at the electroweak scale,
and the minimum of the potential would
have to be at $\langle H_d^0 \rangle >  \langle H_u^0 \rangle$ which
would be a contradiction
with the supposition that $\tan\beta$ is large.]
In the following, we will see that the parameter $\tan\beta$ has an
important effect on the masses and mixings of the MSSM sparticles.

\subsection{Neutralinos and charginos}\label{subsec:MSSMspectrum.inos}

The higgsinos and electroweak gauginos
mix with each other because of the effects of electroweak symmetry
breaking.
The neutral higgsinos ($\stilde H_u^0$ and $\stilde H_d^0$) and the
neutral gauginos ($\stilde B$, $\stilde W^0$) combine to form four
neutral mass eigenstates called {\it neutralinos}.
The charged higgsinos ($\stilde H_u^+$ and $\stilde H_d^-$)
and winos ($\stilde W^+$ and $\stilde W^-$) mix to form two
mass eigenstates with charge $\pm 1$ called {\it charginos}.
We will denote
\footnote{Other common notations use
$\stilde \chi_i^0$ or $\stilde Z_i$ for neutralinos,
and $\stilde \chi^\pm_i$ or $\stilde W^\pm_i$ for charginos.}
the neutralino and chargino mass eigenstates by
$\stilde N_i$ ($i=1,2,3,4$) and $\stilde C^\pm_i$ ($i=1,2$).
By convention, these are labelled in ascending order, so that 
$m_{\stilde N_1}
< m_{\stilde N_2} <m_{\stilde N_3} <m_{\stilde N_4}$ and
$m_{\stilde C_1} < m_{\stilde C_2}$. The lightest neutralino,
$\stilde N_1$, is usually assumed to
be the LSP, unless there is a lighter
gravitino or unless $R$-parity is not conserved, because it is the
only MSSM particle which can make a good cold dark matter candidate.
In
this subsection,
we will describe the mass spectrum and mixing of the neutralinos
and charginos in the MSSM.

In the gauge-eigenstate basis
$\psi^0 = (\stilde B, \stilde W^0, \stilde H_d^0, \stilde H_u^0)$,
the neutralino mass terms in the lagrangian are
\beq
\lagr \supset -\half (\psi^{0})^T {\bf M}_{\stilde N} \psi^0 + \conj
\eeq
where
\beq
{\bf M}_{\stilde N} = \pmatrix{M_1 & 0 & - \cbeta\, \sW\, \mZ &
\sbeta\, \sW \, \mZ\cr
0 & M_2 & \cbeta\, \cW\, \mZ & - \sbeta\, \cW\, \mZ \cr
-\cbeta \,\sW\, \mZ & \cbeta\, \cW\, \mZ & 0 & -\mu \cr
\sbeta\, \sW\, \mZ & - \sbeta\, \cW \, \mZ& -\mu & 0 \cr }.
\label{neutralinomassmatrix}
\eeq
Here we have introduced abbreviations $\sbeta = \sin\beta$,
$\cbeta = \cos\beta$, $\sW = \sin\theta_W$, and $\cW = \cos\theta_W$.
The entries $M_1$ and $M_2$ in this matrix come directly from the MSSM
soft Lagrangian [see eq.~(\ref{MSSMsoft})] while the entries
$-\mu$ are the supersymmetric higgsino mass terms [see
eq.~(\ref{poody})]. The terms proportional
to $m_Z$ are the result of Higgs-higgsino-gaugino couplings
[see eq.~(\ref{gensusylagr}) and Fig.~\ref{fig:gauge}g], with the Higgs
scalars
getting
their VEVs [eqs.~(\ref{vuvdcon}),(\ref{deftanbeta})].
The mass matrix ${\bf M}_{\stilde N}$ can be diagonalized
by a unitary matrix ${\bf N}$ with
$\stilde N_i = {\bf N}_{ij} \psi^0_j$, so that
\beq
{\bf M}^{\rm diag}_{\stilde N} =
{\bf N}^* {\bf M}_{\stilde N} {\bf N}^{-1}
\label{diagmN}
\eeq
has positive real entries $m_{\stilde N_1}$, $m_{\stilde N_2}$,
$m_{\stilde N_3}$, $m_{\stilde N_4}$ on the diagonal.
These are the absolute values of the eigenvalues of ${\bf M}_{\stilde N}$,
or equivalently the square roots of the eigenvalues of
${\bf M}^\dagger_{\stilde N}{\bf M}_{\stilde N}$.
The indices $(i,j)$ on ${\bf N}_{ij}$ are (mass, gauge) eigenstate labels.
The mass eigenvalues and the mixing matrix ${\bf N}_{ij}$ can be given in
closed form in terms of the parameters $M_1$, $M_2$, $\mu$ and
$\tan\beta$, but the results are very complicated and not very
illuminating.

In general, the parameters $M_1$, $M_2$, and $\mu$ can have
arbitrary complex phases.
In the broad class of minimal supergravity or gauge-mediated
models
satisfying the gaugino unification conditions
eq.~(\ref{gauginounificationsugra}) or (\ref{gauginogmsb}),
$M_2$ and $M_1$ will have the same complex phase which is preserved
by RG evolution eq.~(\ref{gauginomassrge}). In
that case, a
redefinition
of the phases of $\stilde B$ and $\stilde W$
allows us to make $M_1$ and $M_2$ both real and positive.
The phase of $\mu$ is then really a physical parameter which cannot be
rotated
away. [We have already
used up the freedom to redefine the phases of the Higgs fields, since
we have picked $b$ and $\langle H_u^0\rangle$ and $\langle H_d^0 \rangle$
to be real and positive, to guarantee that the off-diagonal
entries in eq.~(\ref{neutralinomassmatrix}) proportional to $m_Z$ are
real.]
However, if $\mu$ is not real, then there can be potentially disastrous
CP-violating effects in low-energy physics, including electric dipole
moments for both the electron and the neutron. Therefore, it is usual
(although not mandatory because of the possibility of nontrivial
cancellations) to
assume that $\mu$ is real
in the same set of phase conventions which make $M_1$, $M_2$,
$b$, $\langle H_u^0\rangle$ and $\langle H_d^0 \rangle$ real and positive.
The sign of $\mu$ is still undetermined by this constraint.

In models which satisfy eq.~(\ref{gauginomassunification}), one has the
nice prediction
\beq
M_1 \approx {5\over 3}\tan^2\theta_W \, M_2 \approx 0.5 M_2
\label{usualm1m2}
\eeq
at the electroweak scale.
If so, then the neutralino masses and mixing angles
depend on only three unknown parameters. This assumption is
sufficiently theoretically compelling that it has been made in almost all
phenomenological studies; nevertheless it should be recognized as an
assumption, to be tested someday by experiment.

Specializing further,
there is an interesting and not unlikely limit in which
electroweak symmetry breaking effects can be viewed as a small
perturbation on the neutralino mass matrix. If
\beq
m_Z \ll |\mu \pm M_{1}|, |\mu \pm M_{2}|
\label{gauginolike}
\eeq
then the
neutralino mass eigenstates are very nearly
$\stilde N_1 \approx \stilde B$; $\stilde N_2 \approx \stilde W^0$;
$\stilde N_3, \stilde N_4 \approx (\stilde H_u^0 \pm
\stilde H_d^0)/\sqrt{2}$, with mass eigenvalues:
\beq
m_{{\stilde N}_1}\!\!\! &=&\!\!\! M_1 -
{ m_Z^2 s^2_W (M_1 + \mu \sin 2 \beta ) \over \mu^2 - M_1^2 }
+\ldots
\\
m_{{\stilde N}_2}\!\!\! &=&\!\!\! M_2 -
{ m_W^2 (M_2 + \mu \sin 2 \beta ) \over \mu^2 - M_2^2 }
+\ldots \qquad {}\\
m_{{\stilde N}_3}, m_{{\stilde N}_4}\!\!\! &=&\!\!\! |\mu|  +
{ m_Z^2  (1-\epsilon \sin 2 \beta) (|\mu| + M_1 c^2_W +M_2 s^2_W
)
\over 2 (|\mu| + M_1) (|\mu| + M_2) }
+\ldots, \\
&&\!\!\! |\mu|  +
{ m_Z^2  (1+\epsilon \sin 2 \beta) (|\mu| - M_1 c^2_W - M_2 s^2_W
)
\over 2 (|\mu| - M_1) (|\mu| - M_2) }
+\ldots \qquad {}
\eeq
where we have assumed $\mu$ is real with sign $\epsilon = \pm 1$.
The labeling of the mass eigenstates $\stilde N_1$ and $\stilde
N_2$ assumes
$M_1< M_2 < |\mu|$; otherwise the subscripts may need to be rearranged.
It turns out that a ``bino-like" LSP $\stilde N_1$
can very easily have the right cosmological abundance to make a good
dark matter candidate, so the large $|\mu |$ limit may be
preferred from that point of view. In addition, this
limit tends to emerge from minimal
supergravity boundary conditions on the soft parameters,
which often require $|\mu |$ to be larger than $M_1$
and $M_2$ in order
to get correct
electroweak symmetry breaking.

The chargino spectrum can be analyzed in a similar way.
In the gauge-eigenstate basis $\psi^\pm =
(\stilde W^+,\, \stilde
H_u^+,\, \stilde W^- ,\, \stilde H_d^- )$, the chargino mass terms in
the lagrangian are
\beq
\lagr \supset -\half (\psi^\pm)^T {\bf M}_{\stilde C} \psi^\pm
+\conj
\eeq
where, in $2\times 2$ block form,
\beq
{\bf M}_{\stilde C}
= \pmatrix{{\bf 0}&{\bf X}^T\cr
         {\bf X} &{\bf 0}};\qquad\>\>
{\bf X} = \pmatrix{M_2 & \sqrt{2} \sbeta\, m_W\cr
                              \sqrt{2} \cbeta\, m_W & \mu \cr }.
\label{charginomassmatrix}
\eeq
The mass eigenstates are related to the gauge eigenstates by two unitary
2$\times$2 matrices
$\bf U$ and $\bf V$ according to
\beq
\pmatrix{\stilde C^+_1\cr
         \stilde C^+_2} = {\bf V}
\pmatrix{\stilde W^+\cr
         \stilde H_u^+};\qquad\>\>\>
\pmatrix{\stilde C^-_1\cr
         \stilde C^-_2} = {\bf U}
\pmatrix{\stilde W^-\cr
         \stilde H_d^-}.\qquad\>\>\>
\eeq
Note that there are different mixing matrices for the positively
charged states and for the negatively charged states.
They are to be chosen so that
\beq
{\bf U}^* {\bf X} {\bf V}^{-1} =
\pmatrix{m_{\stilde C_1} & 0\cr
              0   & m_{\stilde C_2}}.
\eeq
Because these are only 2$\times$2 matrices, it is not hard to solve
for the masses explicitly:
\beq
m^2_{{\stilde C}_{1}},
m^2_{{\stilde C}_{2}}
& =
& {1\over 2} 
\Bigl [ (|M_2|^2 + |\mu|^2 + 2m_W^2)
\nonumber
\\
&&\mp
\sqrt{(|M_2|^2 + |\mu |^2 + 2 m_W^2 )^2 - 4 | \mu M_2 - m_W^2 \sin 2
\beta |^2 }
\Bigr ] .
\eeq
It should be noted that these are the (doubly degenerate) eigenvalues
of the $4\times 4$ matrix ${\bf M}_{\stilde C}^\dagger {\bf M}_{\stilde
C}$, or equivalently the eigenvalues of ${\bf X}^\dagger {\bf X}$, but
they are
{\it
not} the squares of the eigenvalues of $\bf X$. 
In the limit of eq.~(\ref{gauginolike}) with real $M_2$ and $\mu$, one
finds that the charginos
mass eigenstates consist of a wino-like $\stilde C_1^\pm$ and
and a higgsino-like $\stilde C_2^\pm$, with masses
\beq
m_{{\stilde C}_1} &=& M_2 -
{ m_W^2 (M_2 + \mu \sin 2 \beta ) \over \mu^2 - M_2^2 } +\ldots
\\
m_{{\stilde C}_2}
&=& |\mu | + {m_W^2 (|\mu |+ \epsilon
M_2 \sin 2 \beta) \over \mu^2 -
M^2_2 }+\ldots .
\eeq
Here again the labeling assumes $M_2<|\mu|$, and $\epsilon$ is the
sign of $\mu$.
Amusingly,
the lighter chargino
$\stilde C_1$ is nearly degenerate with the second lightest
neutralino $\stilde N_2$
in this limit,
but this is not an exact result.
Their higgsino-like colleagues
$\stilde N_3$, $\stilde N_4$ and
$\stilde C_2$  have masses of order $|\mu|$.
The case of $M_1 \approx 0.5 M_2 \ll |\mu|$ is not uncommonly found in
viable models following from the boundary conditions in section
\ref{sec:origins},
and it has been elevated to the status of a benchmark scenario in
many phenomenological studies. However it cannot be overemphasized
that such expectations are not mandatory.

In practice, the masses and mixing angles for the neutralinos and
charginos are best computed numerically. The corresponding Feynman rules
may be inferred in terms of $\bf N$, $\bf U$ and $\bf V$ from the
MSSM lagrangian as discussed above; they are collected in
Refs.\cite{HaberKanereview}$^{\!,\,}$\cite{GunionHaber}

\subsection{The gluino}\label{subsec:MSSMspectrum.gluino}

The gluino is a color octet fermion, so it cannot mix with any other
particle in the MSSM, even if $R$-parity is violated. In this
regard, it is unique among all of the MSSM sparticles.
In the models
following from minimal supergravity or gauge-mediated
boundary conditions, the gluino mass
parameter $M_3$ is related to the bino and wino mass parameters $M_1$ and
$M_2$ by eq.~(\ref{gauginomassunification})
\beq
M_3 = {\alpha_S\over \alpha} \sin^2\theta_W\, M_2 =
{3\over 5} {\alpha_S \over \alpha} \cos^2\theta_W\, M_1
\eeq
at any RG scale, up to small two-loop corrections. If we
use values $\alpha_S = 0.118$, $\alpha=1/128$, $\sin^2\theta_W=0.23$,
then one finds the rough prediction
\beq
M_3 : M_2 : M_1 \approx 7:2:1
\eeq
at the electroweak scale. In particular, we suspect that the gluino should
be
much heavier than the lighter neutralinos and charginos.

For more precise estimates, one must take into account the
fact that
the parameter $M_3$ is really a running mass which has an implicit
dependence on the RG scale $Q$.
Because the gluino is a strongly interacting particle,
$M_3$ runs rather quickly with $Q$ [see eq.~(\ref{gauginomassrge})].
A more
useful
quantity
physically is the RG scale-independent mass $m_{\stilde g}$ at which the
renormalized gluino propagator has a pole. Including one-loop
corrections to the gluino propagator due to
gluon exchange and quark-squark loops,
one finds that
the pole mass is given in terms of the running mass
in the $\drbar$ scheme by \cite{gluinopolemass}
\beq
m_{\stilde g} = M_3(Q) \Bigl ( 1 + {\alpha_S\over 4 \pi}
\bigl [ 15 + 6\> {\rm ln}(Q/ M_3)
+ \sum
A_{\stilde q}\bigr ] \Bigr )
\label{gluinopole}
\eeq
where
\beq
A_{\stilde q} =
\int_0^1 \> dx \> x \> {\rm ln}
\bigl [
x m_{\stilde q}^2/M_3^2 + (1-x) m_{q}^2/M_3^2 - x(1-x) 
\bigr ].
\eeq
The sum in eq.~(\ref{gluinopole})
is over all 12 squark-quark
supermultiplets, and we have neglected
small effects due to squark mixing.
It is easy to check that requiring $m_{\stilde g}$ to be independent
of $Q$
in eq.~(\ref{gluinopole}) reproduces the one-loop RG equation for $M_3(Q)$
in eq.~(\ref{gauginomassrge}).
The correction terms proportional to $\alpha_S$ in eq.~(\ref{gluinopole})
can be quite significant, so that $m_{\stilde g}/M_3(M_3)$ can exceed
unity by 25\% or more. The reasons for this are that the gluino is strongly
interacting, with a large group theory factor [the 15 in
eq.~(\ref{gluinopole})] due to its color octet nature, and that
it couples to all the squark-quark pairs.
Of course, there are similar
corrections which relate the running masses of all the other
MSSM particles to their physical masses. These
have been systematically evaluated at one-loop order in Ref.\cite{PBMZ}
They are more complicated in form and usually numerically
smaller than for the gluino, but in some cases they
could be quite
important in future efforts to connect a given candidate model for
the soft terms to experimentally measured
masses and mixing angles of the MSSM particles.

\subsection{The squark and slepton mass
spectrum}\label{subsec:MSSMspectrum.sfermions}

In principle, any scalars with the same
electric charge, $R$-parity, and color quantum numbers can mix with each
other.
This means that with completely arbitrary soft terms, the
mass eigenstates of the squarks and sleptons
of the MSSM should be obtained by diagonalizing three $6\times 6$
(mass)$^2$ matrices
for up-type squarks ($\stilde u_L$, $\stilde c_L$, $\stilde t_L$, $\stilde
u_R$, $\stilde c_R$, $\stilde t_R$),
down-type squarks ($\stilde d_L$, $\stilde
s_L$, $\stilde b_L$, $\stilde d_R$, $\stilde s_R$, $\stilde b_R$), and
charged
sleptons ($\stilde e_L$, $\stilde \mu_L$, $\stilde \tau_L$, $\stilde
e_R$, $\stilde \mu_R$, $\stilde \tau_R$),
and one $3\times 3$ matrix for sneutrinos ($\stilde \nu_e$, $\stilde
\nu_\mu$, $\stilde \nu_\tau$). Fortunately,
the general hypothesis of flavor-blind soft parameters
eqs.~(\ref{scalarmassunification}) and (\ref{aunification})
predicts that most of these mixing angles are very small.
The third-family squarks and sleptons can have very different masses
compared to their first- and second-family counterparts, because
of the effects of large Yukawa ($y_t$, $y_b$, $y_\tau$) and
soft ($a_t$, $a_b$, $a_\tau$) couplings in the RG
equations (\ref{mq3rge})-(\ref{mstaubarrge}). Furthermore, they
can have substantial mixing in pairs ($\stilde t_L$, $\stilde t_R$),
 ($\stilde b_L$, $\stilde b_R$) and ($\stilde \tau_L$, $\stilde \tau_R$).
In contrast, the first- and second-family squarks and sleptons have
negligible Yukawa couplings, so they end up in 7 very nearly
degenerate, unmixed pairs
$(\stilde e_R, \stilde \mu_R)$,
$(\stilde \nu_e, \stilde \nu_\mu)$,
$(\stilde e_L, \stilde \mu_L)$,
$(\stilde u_R, \stilde c_R)$,
$(\stilde d_R, \stilde s_R)$,
$(\stilde u_L, \stilde c_L)$,
$(\stilde d_L, \stilde s_L)$.
As we have
already discussed in section \ref{subsec:mssm.hints}, this avoids the
problem of
disastrously large virtual sparticle contributions to FCNC processes.

Let us first consider the spectrum of first- and second-family
squarks and sleptons.
In models fitting into both of the broad categories of minimal
supergravity
[eq.~(\ref{scalarunificationsugra})] or gauge-mediated
[eq.~(\ref{scalargmsb})] boundary conditions,
their running masses can be conveniently parameterized in the following
way:
\beq
m_{Q_1}^2 = m_{Q_2}^2  \!\!\!&=&\!\!\! m_0^2 + K_3 + K_2 + {1\over 36}K_1,
\label{mq1form} \\
m_{\sbar u_1}^2 = m_{\sbar u_2}^2 \!\!\! &=&\!\!\! m_0^2 + K_3
\qquad\>\>\>
+ {4\over 9} K_1,
\\
m_{\sbar d_1}^2 = m_{\sbar d_2}^2  \!\!\!&=&\!\!\! m_0^2 + K_3
\qquad\>\>\>
+ {1\over 9} K_1,
\\
m_{L_1}^2 = m_{L_2}^2 \!\!\!&=&\!\!\! m_0^2 \qquad\>\>\> + K_2 + {1\over
4} K_1,
\\
m_{\sbar e_1}^2 = m_{\sbar e_2}^2 \!\!\!&=&\!\!\! m_0^2
\qquad\qquad\>\>\>\>\>\> \, +\, K_1.
\label{me1form}
\eeq
In minimal supergravity models, $m_0^2$ is the common scalar (mass)$^2$
which appears in eq.~(\ref{scalarunificationsugra}).
It can be 0 in the ``no-scale" limit, but it could also be the dominant
source of the scalar masses.
The contributions $K_3$, $K_2$ and $K_1$
are due to the RG running proportional to the
gaugino masses; see eq.~(\ref{easyscalarrge}). They are strictly positive.
A key point is that the same $K_3$, $K_2$ and $K_1$ appear everywhere in
eqs.~(\ref{mq1form})-(\ref{me1form}), since all of the chiral
supermultiplets couple to the same gauginos with the same gauge couplings.
The different coefficients in front of $K_1$ just correspond to the
various values of weak hypercharge squared for each scalar.
The quantities $K_1$, $K_2$, $K_3$ depend on the RG scale $Q$ at which
they are evaluated.
Explicitly, they are found by solving eq.~(\ref{easyscalarrge}):
\beq
K_a(Q) = \left\lbrace \matrix{{3/5}\cr {3/4} \cr {4/3}}
\right \rbrace \times
{1\over 2 \pi^2} \int^{{\rm ln} Q_{0}}_{{\rm ln}Q}dt\>\,
g^2_a(t) \,|M_a(t)|^2\qquad (a=1,2,3).
\label{kintegral}
\eeq
Here $Q_{0}$ is the input RG scale at which the boundary condition
eq.~(\ref{scalarunificationsugra}) is applied, and $Q$ should be
taken to be evaluated near the squark and slepton mass
under consideration, presumably less than about 1 TeV or so.
The values of the running parameters
$g_a(Q)$ and $M_a(Q)$ can be found using
eqs.~(\ref{mssmg}) and (\ref{gauginomassunification}).
If the input scale is approximated by the apparent scale of
gauge coupling unification $Q_0 = M_U \approx 2 \times 10^{16}$ GeV,
one finds that numerically
\beq
K_1 \approx 0.15 m_{1/2}^2;\qquad
K_2 \approx 0.5 m_{1/2}^2;\qquad
K_3 \approx (4.5\>{\rm to}\> 6.5) m_{1/2}^2.
\label{k123insugra}
\eeq
for $Q$ near 1 TeV. Here $m_{1/2}$ is the common gaugino mass parameter
at the unification scale. Note that
$K_3 \gg K_2 \gg K_1$; this is a direct consequence
of the relative sizes of the gauge couplings $g_3$, $g_2$, and $g_1$.
The large uncertainty in $K_3$ is due in part to the
experimental uncertainty in the
QCD coupling constant, and in part to the uncertainty in where to
choose $Q$, since $K_3$ runs rather quickly below 1 TeV.
If the gauge couplings and gaugino masses are unified between $M_U$ and
$M_P$, as would occur in a GUT model,
then the effect of RG running for $M_U < Q < M_P$ can be absorbed into
a redefinition of $m_0^2$. Otherwise, it adds a further uncertainty
which is roughly proportional to ln$(M_P/M_U)$, compared to the
larger contributions in eq.~(\ref{kintegral}) which go roughly
like ln$(M_U/1$~TeV).

In gauge-mediated models, the same parameterization
 eqs.~(\ref{mq1form})-(\ref{me1form}) holds, but
$m_0^2$ is always 0. At the input scale $Q_0$,
each MSSM scalar gets contributions to its (mass)$^2$
 which depend only
on its
gauge interactions, as in eq.~(\ref{scalargmsb}).
It is not hard to see that in general
these contribute in exactly the same pattern as $K_1$, $K_2$, and $K_3$
in eq.~(\ref{mq1form})-(\ref{me1form}). The subsequent evolution
of the scalar squared masses down to the electroweak scale
again just yields more contributions
to the $K_1$, $K_2$, and $K_3$ parameters. It is somewhat more difficult
to give meaningful numerical estimates for these parameters in
gauge-mediated models than in the minimal supergravity models,
because of uncertainties in the messenger mass scale(s)
and in the multiplicities of the messenger fields.
However, in the gauge-mediated case one quite generally expects
that the numerical values of the
ratios $K_3/K_2$, $K_3/K_1$ and $K_2/K_1$ should be
even larger than in eq.~(\ref{k123insugra}).
There are two reasons for this.
First, the running squark squared masses start off larger
than slepton squared masses
already at the input
scale in gauge-mediated models, rather than having a common value $m_0^2$.
Furthermore, in the gauge-mediated
case, the input scale $Q_0$
is typically much lower than
$M_P$ or $M_U$, so that the RG evolution gives relatively more weight
to smaller RG scales where the hierarchies
$g_3>g_2>g_1$ and $M_3>M_2>M_1$ are already in effect.

In general, one therefore expects that the squarks should be
considerably heavier than the sleptons, with the effect being more
pronounced in gauge-mediated supersymmetry breaking models
than in minimal supergravity models.
For any specific
choice of model, this effect can be easily quantified with an RG analysis.
The hierarchy $m_{\rm squark} > m_{\rm slepton}$ tends to hold even
in models which do not really fit into any of the categories outlined in
section \ref{sec:origins}, because the RG contributions to squark
masses from the gluino
are always present and usually quite large, since QCD has a larger gauge
coupling than the electroweak interactions.

There is also a
``hyperfine" splitting in the squark and slepton mass spectrum
produced by electroweak symmetry breaking.
Each squark and slepton $\phi$ will get a contribution
$\Delta_\phi$ to its (mass)$^2$, coming from
the $SU(2)_L$ and $U(1)_Y$ $D$-term
quartic
interactions [see the last term in eq.~(\ref{fdpot})] of the form
(squark)$^2$(Higgs)$^2$ and (slepton)$^2$(Higgs)$^2$,
when the neutral Higgs scalars $H_u^0$ and $H_d^0$ get VEVs.
They are model-independent for a given value of $\tan\beta$,
and are given by
\beq
\Delta_\phi= (T^\phi_3 - Q^\phi_{\rm EM}\sin^2\theta_W)
\cos 2\beta\, m_Z^2 ,
\label{defDeltaphi}
\eeq
where $T^\phi_3$ and $Q^\phi_{\rm EM}$ are the third component of
weak isospin and the electric charge of the %left-handed
chiral supermultiplet to which $\phi$ belongs.
[For example, $\Delta_{u} = ({1\over 2} - {2\over 3}
\sin^2\theta_W)\cos 2\beta
\, m_Z^2$ and $\Delta_{\sbar u} = ({2\over 3}
\sin^2\theta_W)\cos 2\beta
\, m_Z^2$]. These $D$-term contributions are typically smaller than
the $m_0^2$ and $K_1$, $K_2$, $K_3$ contributions, but should not be
neglected. They split apart the components of the $SU(2)_L$-doublet
sleptons and squarks $L_1=(\stilde \nu_e, \stilde e_L)$, etc. Including
them, the first-family squark and slepton masses are now given by:
\beq
m_{\stilde d_L}^2 \!\!\!&=&\!\!\! m_0^2 + K_3 + K_2 + {1\over 36} K_1 +
\Delta_{d},
\label{msdlform}
\\
m_{\stilde u_L}^2 \!\!\!&=&\!\!\! m_0^2 + K_3 + K_2 + {1\over 36} K_1 +
\Delta_{u},
\\
m_{\stilde u_R}^2\!\!\! &=&\!\!\! m_0^2 + K_3 \qquad\>\>\>  + {4\over 9}
K_1
+
\Delta_{\sbar u},
\\
m_{\stilde d_R}^2 \!\!\!&=&\!\!\! m_0^2 + K_3 \qquad\>\>\>  + {1\over 9}
K_1 +
\Delta_{\sbar d},
\label{msdrform}
\\
m_{\stilde e_L}^2 \!\!\!&=&\!\!\! m_0^2 \qquad\>\>\> + K_2 + {1\over 4}
K_1 +
\Delta_{e},
\label{mselform}
\\
m_{\stilde \nu}^2 \!\!&=& \!\!\! m_0^2 \qquad\>\>\> + K_2 + {1\over 4}
K_1 +
\Delta_{\nu},
\\
m_{\stilde e_R}^2 \!\!\!&=&\!\!\! m_0^2 \qquad\qquad\>\>\>\>\>\> \, +\,
K_1
\, + \Delta_{\sbar e},
\label{mserform}
\eeq
with identical formulas for the second-family squarks and sleptons.
The mass splittings
for the left-handed squarks and sleptons are governed by
model-independent sum rules
\beq
m_{\stilde e_L}^2 -m_{\stilde \nu_e}^2 = m_{\stilde d_L}^2 -m_{\stilde
u_L}^2 = -\cos 2\beta\> m_W^2  .
\eeq
Since $\cos 2\beta<0$ in the allowed range $\tan\beta>1$, it follows
that $m_{\stilde e_L} > m_{\stilde \nu_e}$ and
$m_{\stilde d_L} > m_{\stilde u_L}$, with the magnitude of the splittings
constrained by electroweak symmetry breaking.

Let us next consider the masses of the top squarks, for which there are
several non-negligible
contributions. First, there are (mass)$^2$ terms for
$\stilde t^*_L \stilde t_L$ and $\stilde t_R^* \stilde t_R$
which are just equal to $m^2_{Q_3} + \Delta_u$ and $m^2_{\sbar u_3} +
\Delta_{\sbar u}$, respectively, just as for the first- and second-family
squarks. Second, there are contributions equal to $m_t^2$ for
each of $\stilde t^*_L \stilde t_L$ and $\stilde t_R^* \stilde t_R$.
These come from $F$-terms in the scalar potential of the form $y_t^2
H_u^{0*}
H_u^0 \stilde t_L^* \stilde t_L$ and
$y_t^2 H_u^{0*}
H_u^0 \stilde t_R^* \stilde t_R$ (see Figs.~\ref{fig:stop}b and
\ref{fig:stop}c), with
the Higgs
fields replaced by their VEVs. These contributions are of course
present for all of the squarks and sleptons, but they are much too small
to worry about except in the case of the top squarks. Third, there are
contributions to the scalar potential from $F$-terms
of the form $-\mu y_t \stilde{\sbar t} \stilde t H_d^{0*}
+\conj$; see eqs.~(\ref{striterms}) and Fig.~\ref{fig:stri}a. These
become $-\mu v y_t \cos\beta\, \stilde t^*_R \stilde t_L + 
\conj$ when $H_d^0$ is replaced by its VEV.
Finally, there are contributions to the scalar potential from the
soft (scalar)$^3$ couplings $a_t
\stilde{\sbar t}
\stilde Q_3 H_u^0
+ \conj$ [see the first term of the second line of
eq.~(\ref{MSSMsoft}) and eq.~(\ref{heavyatopapprox})], which become $
a_t v \sin\beta\, \stilde t_L \stilde t_R^* + \conj$ when $H_u^0$
is replaced by its VEV. Putting
these all together,
we have a (mass)$^2$ matrix for the top squarks,
which in the
gauge-eigenstate
basis ($\stilde t_L$, $\stilde t_R$) is given by
\beq
-\lagr \supset \pmatrix{\stilde t_L^* & \stilde t_R^*}
{\bf m_{\stilde t}^2} \pmatrix{\stilde t_L \cr \stilde t_R}
\eeq
where
\beq
{\bf m_{\stilde t}^2} =
\pmatrix{
m^2_{Q_3} + m_t^2 + \Delta_{u} & v(a_t \sin\beta - \mu
y_t\cos\beta )\cr
v (a_t \sin\beta - \mu y_t\cos\beta ) & m^2_{\sbar u_3} + m_t^2 +
\Delta_{\sbar u}
} .
\label{mstopmatrix}
\eeq
This matrix can be diagonalized to give mass eigenstates
\beq
\pmatrix{\stilde t_1\cr\stilde t_2} =
\pmatrix{
\cos\theta_{\stilde t} &
\sin\theta_{\stilde t} \cr
-\sin\theta_{\stilde t} &
\cos\theta_{\stilde t}}
\pmatrix{\stilde t_L \cr \stilde t_R}
\label{pixies}
\eeq
with $m^2_{\stilde t_1}< m^2_{\stilde t_2}$
being the eigenvalues of eq.~(\ref{mstopmatrix})
and $0\leq \theta_{\stilde t} \leq \pi$.
Because of the large RG effects proportional
to $X_t$ in eq.~(\ref{mq3rge}) and eq.~(\ref{mtbarrge}),
at the electroweak
scale one
finds that $m_{\sbar u_3}^2 < m_{Q_3}^2$,
and
both of these quantities are usually significantly smaller
than the squark squared masses for the first two families.
The diagonal terms $m_t^2$ in eq.~(\ref{mstopmatrix}) tend to
mitigate this effect somewhat, but the off-diagonal entries will
typically induce a significant mixing which always reduces
the lighter top-squark (mass)$^2$ eigenvalue. For this reason, it is often
found
in models that $\stilde t_1$ is the lightest squark of all.

A very similar analysis can be performed for the bottom squarks and
charged tau
sleptons, which in their respective gauge-eigenstate bases
($\stilde b_L$, $\stilde b_R$) and ($\stilde \tau_L$, $\stilde
\tau_R$) have (mass)$^2$ matrices:
\beq
{\bf m_{\stilde b}^2} =
\pmatrix{
m^2_{Q_3} + \Delta_{d} & v (a_b \cos\beta - \mu
y_b\sin\beta )\cr
v (a_b \cos\beta - \mu y_b\sin\beta ) & m^2_{\sbar d_3}  +
\Delta_{\sbar d}
}
\label{msbottommatrix}
;
\eeq
\beq
{\bf m_{\stilde \tau}^2} =
\pmatrix{
m^2_{L_3} + \Delta_{e} & v (a_\tau \cos\beta -\mu
y_\tau\sin\beta )\cr
v (a_\tau \cos\beta - \mu y_\tau\sin\beta ) & m^2_{\sbar e_3}
+ \Delta_{\sbar e}
}.
\label{mstaumatrix}
\eeq
These can be diagonalized to give mass eigenstates $\stilde b_1, \stilde
b_2$ and $\stilde \tau_1, \stilde \tau_2$ in exact analogy with
eq.~(\ref{pixies}).

The magnitude and importance of mixing in the sbottom and stau sectors
depends
on how large $\tan\beta$ is.
If $\tan\beta$ is not too large
(in practice, this usually
means less than about $10$ or so,
depending on the situation under study), the sbottoms and staus
do not get a very large effect from the mixing terms and the
RG effects due to $X_b$ and $X_\tau$, because
$y_b,y_\tau \ll y_t$ from
eq.~(\ref{ytbtau}).
In that case the mass eigenstates are very nearly the same as
the gauge eigenstates $\stilde b_L$, $\stilde b_R$, $\stilde \tau_L$
and $\stilde \tau_R$. The latter three, and $\tilde \nu_\tau$,
will be nearly degenerate with
their first- and second-family counterparts with the same
$SU(3)_C \times SU(2)_L \times U(1)_Y$ quantum numbers.
However, even in the case of small $\tan\beta$,
$\stilde b_L$ will feel the effects of the large top Yukawa coupling
because it is part of the doublet $\stilde Q_3$ which contains
$\stilde
t_L$.
In particular, from eq.~(\ref{mq3rge}) we see that $X_t$ acts to decrease
$m_{\stilde Q_3}^2$ as it is RG-evolved down from the input scale
to the electroweak scale.
Therefore the mass of ${\stilde b_L}$ can be significantly
less than the masses of $\stilde d_L$
and $\stilde s_L$.

For larger values of $\tan\beta$, the mixing in
eqs.~(\ref{msbottommatrix}) and (\ref{mstaumatrix}) can be quite
significant, because $y_b$, $y_\tau$ and $a_b$, $a_\tau$
are non-negligible.
Just as in the case of the top squarks, the
lighter sbottom and stau mass eigenstates (denoted $\stilde b_1$ and
$\stilde \tau_1$)
can be significantly lighter than their first- and second-family
counterparts. Furthermore, ${\stilde \nu_\tau}$ can be significantly 
lighter than
the nearly degenerate ${\stilde \nu_e}$, $\stilde \nu_\mu$.

The requirement that the third-family squarks and sleptons should all
have positive (mass)$^2$ implies limits on the sizes of $a_t\sin\beta
-\mu y_t \cos\beta$, $a_b\cos\beta - \mu y_b \sin\beta$,
and $a_\tau \cos\beta - \mu y_\tau \sin\beta$. If they are too large, the
smaller eigenvalue
of eq.~(\ref{mstopmatrix}), (\ref{msbottommatrix}) or (\ref{mstaumatrix})
will be driven
negative, implying that a squark or charged slepton gets a VEV, breaking
$SU(3)_C$ or electromagnetism. Since this is clearly unacceptable,
one can put bounds on the (scalar)$^3$ couplings, or equivalently
on the parameter $A_0$ in minimal supergravity models. Even
if all of
the (mass)$^2$ eigenvalues are positive, the presence of large
(scalar)$^3$ couplings can yield global minima of the scalar potential
with non-zero squark and/or charged slepton VEVs
which are disconnected from the vacuum which conserves $SU(3)_C$ and
electromagnetism.\cite{badvacua}
However, it is not always clear whether the non-existence of such
disconnected global
minima should really be taken as a constraint, 
because the tunneling rate from our ``good" vacuum to the
``bad"  vacua can easily be much longer than the age of the
universe.\cite{kusenko}

\subsection{Summary: the MSSM sparticle
spectrum}\label{subsec:MSSMspectrum.summary}

In the MSSM there are 32 distinct masses corresponding to
undiscovered particles, not including the gravitino.
In this section we have explained how
the masses and mixing angles for these particles can be computed,
given an underlying model for the soft terms at some input scale.
Assuming only that the mixing of first- and second-family squarks and
sleptons is negligible,
the mass eigenstates of the MSSM are listed in Table
\renewcommand{\arraystretch}{1.4}
\begin{table}[tb]
\caption{
Undiscovered particles in the Minimal Supersymmetric Standard
Model\label{tab:undiscovered}}
\vspace{0.4cm}
\begin{center}
\begin{tabular}{|c|c|c|c|c|}
\hline
Names & Spin & $P_R$ & Mass Eigenstates & Gauge Eigenstates \\
\hline\hline
Higgs bosons& 0 &$+1$& $h^0\>\> H^0\>\> A^0 \>\> H^\pm$& $
H_u^0\>\> H_d^0\>\> H_u^+ \>\> H_d^-$ \\
\hline
& & &${\stilde u}_L\>\> {\stilde u}_R\>\> \stilde d_L\>\> \stilde d_R$&`` "
\\
squarks& 0&$-1$& ${\stilde s}_L\>\> {\stilde s}_R\>\> \stilde c_L\>\>
\stilde
c_R$& `` " \\
& & &${\stilde t}_1\>\> {\stilde t}_2\>\> \stilde b_1\>\> \stilde
b_2$&$\stilde
t_L \>\>
\stilde t_R \>\>\stilde b_L\>\> \stilde b_R$ \\
\hline
& & &${\stilde e}_L\>\> {\stilde e}_R \>\>\stilde \nu_e$&`` " \\
sleptons& 0&$-1$&${\stilde \mu}_L \>\>{\stilde \mu}_R\>\> \stilde
\nu_\mu$&
`` "
\\
& & &${\stilde \tau}_1 \>\>{\stilde \tau}_2 \>\>\stilde \nu_\tau$&
$\stilde \tau_L\>\> \stilde \tau_R \>\>\stilde \nu_\tau$ \\
\hline
neutralinos & $1/2$&$-1$ & $\stilde N_1\>\> \stilde N_2 \>\>\stilde N_3\>\>
\stilde
N_4$ &
$\stilde B^0 \>\>\>\stilde W^0\>\>\> \stilde H_u^0\>\>\> \stilde H_d^0$   \\
\hline
charginos & $1/2$&$-1$ & $\stilde C_1^\pm\>\>\>\stilde C_2^\pm $ &
$\stilde W^\pm\>\>\> \stilde H_u^+ \>\>\>\stilde H_d^-$ \\
\hline
gluino & $1/2$&$-1$ &$\stilde g$  &`` " \\
\hline
${\rm gravitino/}\atop{\rm goldstino}$ & $3/2$&$-1$&$\stilde G$  &`` " \\
\hline
\end{tabular}
\end{center}
\end{table}
\ref{tab:undiscovered}.
A complete set of Feynman rules for the interactions of these particles
with each other and with the Standard Model quarks, leptons, and gauge
bosons can be found in
Refs.\cite{HaberKanereview,GunionHaber}
Specific models for the soft terms typically predict the masses
and the mixing angles angles for the MSSM in terms of far fewer
parameters. For example, in the minimal supergravity models,
one has only the parameters $m_0^2$, $m_{1/2}$,
$A_0$, $\mu$, and $b$ which are not already measured
 by experiment. 
On the other hand, in gauge-mediated supersymmetry breaking
models, the
free
parameters include at least the scale $\Lambda$, the typical messenger
mass scale $M_{\rm mess}$, the integer number $\nmess$ of copies of
the minimal messengers, the goldstino decay constant $\langle F \rangle $,
and the Higgs mass parameters $\mu$ and $b$.
After RG evolving the soft terms down to the
electroweak scale, one can impose that the scalar potential
gives correct electroweak symmetry breaking. This allows us
to trade $|\mu|$ and $b$ (or $B_0$) for one parameter $\tan\beta$,
as in eqs.~(\ref{mubsub1})-(\ref{mubsub2}). So, to a reasonable
approximation, the entire mass spectrum in minimal supergravity
models is
determined by only five
unknown parameters:
$m_0^2$, $m_{1/2}$, $A_0$, $\tan\beta$, and Arg($\mu$), while in
the simplest gauge-mediated supersymmetry breaking models one can pick
parameters
$\Lambda$, $M_{\rm mess}$, $\nmess$, $\langle F \rangle $, $\tan\beta$,
and Arg($\mu$). 
Both frameworks are highly
predictive.
Of course, it is easy to imagine that the essential physics of
supersymmetry breaking is not captured by either of these two scenarios
in their minimal forms.

While it would be a mistake to underestimate the uncertainties in
the MSSM mass and mixing spectrum, it is also useful to keep in mind
some general lessons that recur in various different scenarios.
Indeed, there has emerged a sort of folklore concerning likely
features of the MSSM spectrum, which is partly based on theoretical
bias and partly on the constraints inherent in any supersymmetric
theory. We remark on these features mainly because they represent
the prevailing prejudice among supersymmetry theorists, which is
certainly a useful thing for the reader to know even if
he or she wisely decides to remain skeptical.
For example, it is perhaps not unlikely that:
\begin{itemize}
\item[$\bullet$] The LSP is the lightest neutralino $\stilde N_1$, unless
the gravitino is
lighter or $R$-parity is not conserved. If $\mu > M_1, M_2$,
then $\stilde N_1$ is likely to be bino-like, with a mass roughly
0.5 times the masses of $\stilde N_2$ and $\stilde C_1$.
In the opposite case $\mu < M_1,M_2$, then $\stilde N_1$ has a large
higgsino content and $\stilde N_2$ and $\stilde C_1$ are not much heavier.
\item[$\bullet$] The gluino will be much heavier than the lighter
neutralinos and charginos. This is certainly true in the case
of the ``standard" gaugino mass relation eq.~(\ref{gauginomassunification});
more
generally,
the running gluino mass parameter grows relatively quickly as it is
RG-evolved into
the infrared because the QCD coupling is larger than the electroweak
gauge couplings. So
even if there are
big corrections to
the gaugino mass boundary conditions eqs.~(\ref{gauginounificationsugra})
or (\ref{gauginogmsb}), the gluino mass parameter $M_3$ is likely to come
out larger than $M_1$ and $M_2$.
\item[$\bullet$] The squarks of the first and second families are
nearly degenerate and much
heavier than the sleptons. This is because each squark mass gets the same
large positive-definite
radiative corrections from loops involving the gluino.
The left-handed squarks $\stilde u_L$, $\stilde d_L$, $\stilde s_L$
and $\stilde c_L$ are likely to be heavier than their right-handed
counterparts $\stilde u_R$, $\stilde d_R$, $\stilde s_R$
and $\stilde c_R$,  because of the effect of $K_2$ in
eqs.~(\ref{msdlform})-(\ref{mserform}).
\item[$\bullet$] The squarks of the first two families cannot be
lighter than about 0.8 times the mass of the gluino in minimal
supergravity models, and about 0.6 times the mass of the gluino
in the simplest gauge-mediated models as discussed in section
\ref{subsec:origins.gmsb} if the number
of messenger squark pairs is
$\nmess
\leq 4$.
 In the minimal supergravity case this is because the gluino mass feeds
into the squark masses through RG evolution; in the gauge-mediated case it
is because the gluino and squark masses are tied together by
eqs.~(\ref{gauginogmsb}) and (\ref{scalargmsb}) [multiplied by $\nmess$,
as explained at the end of section \ref{subsec:origins.gmsb}].
\item[$\bullet$] The lighter stop $\stilde t_1$ and the lighter
sbottom $\stilde b_1$ are probably the lightest squarks. This is
because stop and sbottom mixing effects and the effects of $X_t$ and
$X_b$ in eqs.~(\ref{mq3rge})-(\ref{md3rge}) both tend to decrease the
lighter stop and sbottom masses.
\item[$\bullet$] The lightest charged slepton is probably
a stau $\stilde
\tau_1$.
The mass difference
$m_{\stilde e_R}-m_{\stilde \tau_1}$ is likely to be significant
if $\tan\beta$
is large, because of the effects of a large tau Yukawa coupling.
For smaller $\tan\beta$, $\stilde \tau_1$ is predominantly $\stilde
\tau_R$
and it is not so much lighter than $\stilde e_R$, $\stilde \mu_R$.
\item[$\bullet$] The left-handed charged sleptons $\stilde e_L$ and
$\stilde \mu_L$ are likely to be heavier than their right-handed
counterparts
$\stilde e_R$ and $\stilde \mu_R$. This is because of the effect of
$K_2$ in eq.~(\ref{mselform}). (Note also that
$\Delta_e - \Delta_{\sbar e}$ is positive
but very small because of the numerical
accident $\sin^2\theta_W \approx 1/4$.)
\item[$\bullet$] The lightest neutral Higgs boson $h^0$
should be lighter than about 150 GeV, and may be much lighter than the
other Higgs scalar mass eigenstates
$A^0$, $H^\pm$, $H^0$.
\end{itemize}
\begin{figure}
\centerline{\psfig{figure=susysample.ps,height=2.5in}}
\caption{A schematic sample spectrum for the undiscovered
particles in the MSSM. This spectrum is presented for entertainment
purposes only. No warranty, expressed or implied, guarantees that this
spectrum looks anything like the real world.
\label{fig:sample}}
\end{figure}
In Figure \ref{fig:sample} we show a qualitative sketch
of a sample MSSM mass spectrum which illustrates these features.
Variations in the model parameters can have important and
predictable effects.
For example, taking larger (smaller) $m_0^2$ in minimal supergravity
models
will tend to move the entire spectrum of squarks, sleptons and the
Higgs
scalars $A^0$, $H^\pm$, $H^0$ higher (lower) compared to the neutralinos,
charginos and gluino; taking larger values of $\tan\beta$ with other
model parameters held fixed will usually tend to lower
$\stilde b_1$ and $\stilde \tau_1$ masses compared to those of the other
sparticles, etc.
The important point is that by measuring the masses and mixing
angles of the MSSM
particles we will be able to gain a great deal of information
which can rule out or bolster evidence
for competing proposals for the origin of supersymmetry breaking.
Testing the various possible organizing
principles will provide the high-energy physicists of the next millennium
with an exciting challenge.


\section{Sparticle decays}\label{sec:decays}
\setcounter{equation}{0}
\setcounter{footnote}{1}

In this section we will give a brief qualitative overview of the
decay patterns of sparticles in the MSSM, assuming that $R$-parity
is exactly conserved. We will
consider in turn the possible decays of neutralinos, charginos,
sleptons, squarks, and the gluino. If, as is most often assumed, the
lightest
neutralino
$\NI$ is the LSP, then all decay chains will end up containing it
in the final state. In section \ref{subsec:decays.gravitino} we
consider the alternative
possibility that the gravitino/goldstino $\G$ is the LSP.

\subsection{Decays of neutralinos and
charginos}\label{subsec:decays.inos}

Let us first consider the possible two-body decays.
Each neutralino and chargino contains
at least a small admixture of the electroweak gauginos $\stilde B$,
$\stilde W^0$
or $\stilde W^\pm$, as we saw in section \ref{subsec:MSSMspectrum.inos}.
So $\stilde N_i$ and $\stilde C_i$ inherit couplings of weak
interaction strength to (scalar, fermion) pairs, as shown in
Fig.~\ref{fig:gaugino}b,c.
If sleptons or squarks are sufficiently light, a neutralino or chargino
can therefore decay into lepton+slepton or quark+squark. (We will
often not distinguish between particle and antiparticle names and
labels in this section.) Since sleptons are probably lighter than
squarks, the lepton+slepton final states are more likely to be open.
A neutralino or chargino may also decay into any lighter
neutralino or chargino plus a Higgs scalar or an electroweak gauge boson,
because they inherit the
gaugino-higgsino-Higgs (see Fig.~\ref{fig:gaugino}b,c) and $SU(2)_L$
gaugino-gaugino-vector boson (see Fig.~\ref{fig:gauge}c)
couplings of their components. 
So, the possible two-body decay modes
for neutralinos and charginos in the MSSM are:
\beq
\stilde N_i \rightarrow
Z\stilde N_j,\>\>\, W\stilde C_j,\>\>\, h^0\stilde N_j,\>\>\, \ell \stilde
\ell,\>\>\,
\nu \stilde \nu,\>\>\,
[A^0 \stilde N_j,\>\>\, H^0 \stilde N_j,\>\>\, H^\pm
\stilde C_j^\mp,\>\>\,
q\stilde q];
\qquad\>\>\>{}
\label{nino2body}
\\
\stilde C_i \rightarrow
W\stilde N_j,\>\>\, Z\stilde C_1,\>\>\, h^0\stilde C_1,\>\>\, \ell \stilde
\nu,\>\>\,
\nu \stilde \ell,\>\>\,
[A^0 \stilde C_1,\>\>\, H^0 \stilde C_1,\>\>\, H^\pm \stilde N_j,\>\>\,
q\stilde q^\prime],
\qquad\>\>\>{}
\label{cino2body}
\eeq
using a generic notation $\nu$, $\ell$, $q$ for neutrinos,
charged leptons, and quarks. The final states in brackets are
the more kinematically-implausible ones.
(Since $h^0$
is required to be light, it is the most likely of the Higgs scalars
to appear in these decays.) 
For the heavier neutralinos and chargino ($\stilde N_3$, $\stilde N_4$
and $\stilde C_2$), one or more of the
decays in eqs.~(\ref{nino2body}) and (\ref{cino2body})
is likely to be
kinematically allowed. However, it
may be that all of these two-body modes are kinematically forbidden
for a given chargino or neutralino,
especially in the case of $\stilde C_1$ and $\stilde N_2$ decays.
If so, then one has three-body decays
\beq
\stilde N_i \rightarrow f f \stilde N_j,\>\>\>\,
\stilde N_i \rightarrow f f^\prime \stilde C_j,\>\>\>\,
\stilde C_i \rightarrow f f^\prime \stilde N_j,\>\>\>\,{\rm and}\>\>\>\,
\stilde C_2 \rightarrow f f^\prime \stilde C_1,\qquad\>\>\>\>\>{}
\label{cino3body}
\eeq
through the same (but now off-shell) gauge bosons, Higgs scalars,
sleptons, and squarks that appeared in the two-body
decays eqs.~(\ref{nino2body}) and (\ref{cino2body}).
Here $f$ is generic notation for a lepton or quark,
with $f$ and $f^\prime$ belonging to the same $SU(2)_L$
multiplet. The chargino and neutralino decay widths into the various
final states can be found in Ref.\cite{inodecays,epluseminuscrosssections}
The decays
\beq
\stilde C_1^\pm \rightarrow \ell^\pm \nu \stilde N_1,\qquad\>\>\>
\stilde N_2 \rightarrow \ell^+\ell^- \stilde N_1
\label{trileptonbabies}
\eeq
can be particularly important for phenomenology, because the leptons in the
final state often will result in clean signals.
In certain regions of parameter space, the above decays can be suppressed
by kinematics or by coupling, and one-loop decays (notably $\stilde N_2
\rightarrow \gamma \NI$) might play an important
role.\cite{AmbrosanioMele}

\subsection{Slepton decays}\label{subsec:decays.sleptons}

Sleptons have two-body decays into a lepton and a chargino or
neutralino, because of the gaugino admixture of the latter,
as can be seen directly from the couplings in
Figs.~\ref{fig:gaugino}b,c.
The two-body decays
\beq
\stilde \ell \rightarrow \ell \stilde N_i,\>\>\,
\stilde \ell \rightarrow \nu \stilde C_i,\>\>\,
\stilde \nu \rightarrow \nu \stilde N_i,\>\>\,
\stilde \nu \rightarrow \ell \stilde C_i
\eeq
are therefore of weak interaction strength. In particular,
the direct decays
\beq
\stilde \ell \rightarrow \ell \stilde N_1
\>\>\>{\rm and}\>\>\>
\stilde \nu \rightarrow \nu \stilde N_1
\label{sleptonrightdecay}
\eeq
are (essentially\footnote{An exception
occurs if the mass difference
$m_{\stilde \tau_1} - m_{\stilde N_1}$ is less than $m_{\tau}$.}) 
always kinematically allowed if $\stilde
N_1$ is the LSP.
However, if the sleptons are sufficiently heavy, then the two-body
decays to charginos and heavier neutralinos can be important,
especially
\beq
\stilde \ell \rightarrow \nu \stilde C_{1}
,\>\>\>
\stilde \ell \rightarrow \ell \stilde N_{2}
,\>\>\>{\rm and}\>\>\>
\stilde \nu \rightarrow \ell \stilde C_{1}.
\label{sleptonleftdecay}
\eeq
The right-handed sleptons do not have a coupling to the $SU(2)_L$
gauginos, so they typically prefer the direct decay $\stilde
\ell_R \rightarrow \ell\NI$, if $\NI$ is bino-like. In contrast, the
left-handed sleptons may
prefer to decay as in eq.~(\ref{sleptonleftdecay})
rather than the direct decays to the LSP as in
eq.~(\ref{sleptonrightdecay}), if the former is kinematically open and if
$\stilde C_1$ and $\stilde N_2$ are mostly wino.
This is because the slepton-lepton-wino interactions in
Fig.~\ref{fig:gaugino}b are
proportional to the $SU(2)_L$ gauge coupling $g$,
whereas the slepton-lepton-bino interactions in Fig.~\ref{fig:gaugino}c
are
proportional to
the much smaller $U(1)_Y$ coupling $g^\prime$. General results for
these decay widths can be found in Ref.~\cite{epluseminuscrosssections}

\subsection{Squark decays}\label{subsec:decays.squarks}

If the decay
$
\stilde q \rightarrow q\stilde g
$
is kinematically allowed, it will always dominate, because the
quark-squark-gluino vertex in Fig.~{\ref{fig:gaugino}}a has QCD strength.
Otherwise, the
squarks can decay into a quark plus neutralino or chargino:
 $
\stilde q \rightarrow q \stilde N_i$ or $
q^\prime \stilde C_i
$.
The direct decay to the LSP $\stilde q \rightarrow
q \stilde N_1$ is always kinematically favored,
and for right-handed squarks it can dominate because
$\stilde N_1$ is mostly bino.
However, the left-handed squarks may strongly prefer to decay into
heavier charginos or neutralinos instead, for example
$\stilde q \rightarrow  q \stilde N_2$ or $q^\prime \stilde C_1$,
because the relevant squark-quark-wino couplings
are much bigger than the squark-quark-bino couplings.
Squark decays to higgsino-like charginos and neutralinos are less
important, except in the cases of stops and sbottoms which have
sizeable Yukawa couplings.
The gluino, chargino or neutralino resulting from the squark
decay will
in turn decay, and so on, until a final state containing
$\stilde N_1$ is reached. This can result in
very numerous and complicated decay chain possibilities
called cascade decays.\cite{cascades}
Special attention must be payed to the top squark, because it
is possible that the decays $\stilde t_1 \rightarrow t\stilde g$
and $\stilde t_1 \rightarrow t \stilde N_1$ are both kinematically
forbidden. If so, then the stop may decay only into charginos,
by $\stilde t_1 \rightarrow b \stilde C_1$. If even this
decay is kinematically closed, then the stop has only the
flavor-suppressed decay to a charm quark:
$
\stilde t_1\rightarrow c \stilde N_1
$.
This decay can be very slow,\cite{stoptocharmdecay} so that the lightest
stop can be
quasi-stable on the time scale relevant for collider physics,
and can hadronize and form bound states inside the detector.

\subsection{Gluino decays}\label{subsec:decays.gluino}

The decay of the gluino can only proceed through an on-shell or a virtual
squark. If two-body decays
$
\stilde g \rightarrow q\stilde q
$
are open, they will dominate, again because the relevant
gluino-quark-squark coupling in Fig.~\ref{fig:gaugino}a has QCD strength.
Since the top and bottom squarks can easily be much
lighter than all of the other squarks, it is quite possible that
$
\stilde g \rightarrow t \stilde t_1$ and/or
$\stilde g \rightarrow b \stilde b_1$
are the only available two-body decay mode(s) for the gluino,
in which case they will dominate over all others.
If instead all of the squarks are heavier than the gluino, the gluino
will decay only through off-shell squarks, so
$
\stilde g
\rightarrow
q q^\prime \stilde N_i$ and $
q q^\prime \stilde C_i
$.
The squarks, neutralinos and charginos in these final states
will then decay as discussed above,
so there can be very many competing gluino decay chains.
These cascade decays can have final-state branching
fractions that are individually small and quite sensitive to the
parameters of the model.

\subsection{Decays to the
gravitino/goldstino}\label{subsec:decays.gravitino}

Most phenomenological studies of supersymmetry assume explicitly
or implicitly that the lightest neutralino is the LSP.
This is typically the case in gravity-mediated models for the soft terms.
However, in gauge-mediated models (and in ``no-scale" models), the LSP
is instead the gravitino.
As we saw in section \ref{subsec:origins.gravitino}, a very light
gravitino may be relevant
for collider phenomenology, because it contains as its longitudinal
component the goldstino, which has a non-gravitational coupling
to all sparticle-particle pairs $(\stilde X, X$). The decay rate found in
eq.~(\ref{generalgravdecay}) for $\stilde X\rightarrow X\G$ is usually
not fast enough to compete with the other decays of sparticles $\stilde X$
as mentioned above,
{\it except} in the case that $\stilde X$ is the
next-to-lightest supersymmetric particle (NLSP). Since the NLSP has no
competing
decays, it should always decay into its superpartner and the LSP
gravitino.

In principle, any of the MSSM superpartners could be the NLSP in models
with a light goldstino,
but most models with gauge-mediation of supersymmetry breaking 
have either a neutralino
or a charged lepton playing this role. The argument for this can be seen
immediately from eqs.~(\ref{gauginogmsbgen}) and (\ref{scalargmsbgen});
since $\alpha_1 <
\alpha_2,\alpha_3$, those superpartners which have only $U(1)_Y$
interactions will tend to get the smallest masses. The gauge-eigenstate
sparticles with this property are the bino and the right-handed sleptons
$\stilde e_R$, $\stilde \mu_R$, $\stilde \tau_R$, so the appropriate
corresponding mass eigenstates should be plausible candidates for the
NLSP.

First suppose that $\stilde N_1$ is the NLSP in light
goldstino models. Since $\stilde N_1$ contains an admixture of
the photino (the linear combination of bino and neutral wino whose
superpartner is the photon), from eq.~(\ref{generalgravdecay})
it should then decay into
photon + goldstino/gravitino
with a width given by
\beq
\Gamma (\NI \rightarrow \gamma \G ) \,=\,
2\times 10^{-3} \> \kappa_{1\gamma}\left ({m_{\NI}\over 100\>\rm{
GeV}}\right )^5
\left ( {\sqrt{\langle F \rangle}\over 100\>{\rm TeV}} \right )^{-4} \>
{\rm eV}.\qquad{}
\label{neutralinodecaywidth}
\eeq
Here
$\kappa_{1\gamma}
\equiv |{\bf N}_{11}\cos\theta_W + {\bf N}_{12}\sin \theta_W |^2$
is the ``photino content" of $\stilde N_1$, in terms of the neutralino
mixing matrix ${\bf N}_{ij}$ defined by eq.~(\ref{diagmN}).
We have normalized $m_{\NI}$ and $\sqrt{\langle F \rangle}$ to (very
roughly)
minimum expected values in gauge-mediated models.
This width is
much smaller than for a typical flavor-unsuppressed weak
interaction decay, but it is still large enough to
allow $\stilde N_1$ to decay before it has left a collider detector,
if $\sqrt{\langle F\rangle}$ is less than a few thousand TeV
in gauge-mediated models, or equivalently if
$m_{3/2}$ is less than a keV or so when eq.~(\ref{gravitinomass})
holds. In fact, from
eq.~(\ref{neutralinodecaywidth}), the mean
decay length of an $\NI$ with energy
$E$ in the lab frame is
\beq
d = 9.9 \times 10^{-3}\> {1\over \kappa_{1\gamma}}\,
({E^2/ m_{\NI}^2} - 1)^{1/2}
\left ({m_{\NI}\over 100\>\rm{
GeV}}\right )^{-5}
\left({\sqrt{\langle F \rangle}\over 100\>{\rm TeV}} \right )^{4}
\>{\rm cm},
\label{neutralinodecaylength}
\eeq
which could be anywhere from sub-micron to multi-kilometer depending on
the scale of supersymmetry
breaking $\sqrt{\langle F \rangle}$.
(In other models with a gravitino LSP which are not described by
$F$-term breaking of global supersymmetry, including certain ``no-scale"
models,\cite{noscalephotons}
the same formulas may be applied with 
${\langle F \rangle} \rightarrow \sqrt{3} m_{3/2} \MPlanck$.)

Of course, $\stilde N_1$ is not a pure photino, but contains also
admixtures of the superpartner of the $Z$ boson and the neutral
Higgs scalars. So, one can also have \cite{DDRT}
$\NI\rightarrow Z\G$, $h^0\G$, $A^0\G$, or $H^0\G$,
with decay widths given in Ref.\cite{AKKMM2}
Of these decays, the last two are unlikely to be kinematically
allowed, and only the $\NI \rightarrow \gamma\G$
mode is guaranteed to be kinematically allowed for a gravitino LSP.
Furthermore, even if
they
are open, the
decays $\stilde N_1 \rightarrow Z\G$ and $\stilde N_1 \rightarrow h^0 \G$
are subject to strong kinematic suppressions proportional to
$(1-m_Z^2/m_{\stilde N_1}^2)^4$ and $(1 - m_{h^0}^2/m_{\stilde N_1}^2)^4$,
respectively, in view of eq.~(\ref{generalgravdecay}). Still, these decays
may play an important role in phenomenology if 
$\sqrt{\langle F\rangle }$ is not too large, $\stilde N_1$
has a sizeable zino or higgsino content, and $m_{\stilde N_1}$ is
significantly greater than $m_Z$ or $m_{h^0}$.

A charged slepton makes another likely candidate for the NLSP.
Actually, it is important to note that more than one
slepton can act effectively as the NLSP, even though one of them is
slightly lighter, if they are sufficiently degenerate in mass so that each
has no kinematically allowed decays except to
the goldstino.
In GMSB models, the squared masses obtained by $\widetilde e_R$,
$\widetilde \mu_R$ and $\widetilde \tau_R$ are equal because of the
flavor-blindness of the gauge couplings. However, this is not the whole   
story, because
one must take into account mixing with
$\widetilde e_L$,
$\widetilde \mu_L$, and $\widetilde \tau_L$ and renormalization group
running.  These effects are very small for $\widetilde e_R$ and  
$\widetilde \mu_R$ because of the tiny electron and muon Yukawa couplings,
so we can quite generally treat them as degenerate, unmixed mass
eigenstates. In
contrast, $\widetilde \tau_R$ usually has a quite significant mixing with
$\widetilde \tau_L$, proportional to the tau Yukawa coupling. This means
that the lighter stau mass eigenstate $\widetilde \tau_1$ is pushed lower
in mass than $\widetilde e_R$ or $\widetilde \mu_R$, by an amount that
depends most strongly on $\tan\beta$.  If $\tan\beta$ is not too
large then the stau mixing effect leaves the slepton mass
eigenstates $\widetilde e_R$, $\widetilde \mu_R$, and $\widetilde \tau_1$
degenerate to within less than $m_\tau \approx 1.8 $ GeV, so they act
effectively as
co-NLSPs.  In particular, this means that even though the stau is slightly
lighter, the three-body slepton decays $\widetilde e_R \rightarrow
e\tau^\pm\widetilde \tau_1^\mp$ and $\widetilde \mu_R \rightarrow
\mu\tau^\pm\widetilde \tau_1^\mp$ are not kinematically allowed; the only
allowed decays for the three lightest sleptons are $\widetilde
e_R\rightarrow e \G$ and $\widetilde \mu_R \rightarrow \mu\G$ and
$\widetilde \tau_1 \rightarrow \tau \G$.
This situation is called the ``slepton co-NLSP"
scenario.

For larger values of $\tan\beta$, the lighter stau eigenstate
$\stilde \tau_1$ is
more than $1.8$ GeV lighter than $\widetilde e_R$ and $\widetilde \mu_R$
and $\NI$.  This means that the decays $\NI \rightarrow \tau\stilde
\tau_1$ and
$\widetilde e_R \rightarrow e \tau \stilde \tau_1$ and $\widetilde \mu_R
\rightarrow \mu \tau \stilde\tau_1$ are open.  Then $\widetilde
\tau_1$ is   
the sole NLSP, with all other MSSM supersymmetric particles having
kinematically allowed decays into it. This is called the ``stau NLSP"
scenario.

In any case, a slepton NLSP can decay like $\stilde \ell \rightarrow
\ell \G$ according to eq.~(\ref{generalgravdecay}), with a width and
decay length just given by eqs.~(\ref{neutralinodecaywidth})
and (\ref{neutralinodecaylength}) with the replacements $\kappa_{1\gamma}
\rightarrow 1$ and $m_{\stilde N_1} \rightarrow m_{\stilde \ell}$.
So, just as for the neutralino NLSP case, the decay $\stilde \ell
\rightarrow
\ell\G$ can be either fast or very slow, depending on the
scale of supersymmetry breaking.

If $\sqrt{\langle F \rangle}$ is
larger than roughly $10^3$ TeV
(or the gravitino is heavier than a keV or so), 
then the NLSP is so long-lived that
it will usually escape a typical collider detector.
If $\NI$ is the NLSP, then, it might as well be
the LSP from the point of
view of collider physics.
However, the decay of $\NI$ into the gravitino is obviously still crucial
for cosmology, since an unstable $\NI$ is clearly not a good dark matter
candidate while the gravitino LSP conceivably could be.
On the other hand, if the NLSP is a long-lived charged slepton,
then one can see its tracks (or possibly decay kinks) inside a collider
detector.\cite{DDRT} The presence of a massive charged NLSP can be
established by measuring its anomalously high ionization rate or its
time-of-flight in the detector.

\section{Experimental signals for supersymmetry}\label{sec:signals}
\setcounter{equation}{0}
\setcounter{footnote}{1}

So far, the experimental study of supersymmetry has unfortunately
been confined to setting limits. As we have already remarked in section
\ref{subsec:mssm.hints}, there can be indirect signals
for supersymmetry from processes that are
rare or forbidden in the Standard Model
but can have contributions from loops
involving virtual sparticles.
These include $\mu\rightarrow e\gamma$, $b\rightarrow s\gamma$,
neutral meson mixing,
electric dipole moments for the neutron and the electron, etc.
There are also virtual sparticle effects on Standard Model predictions
like $R_b$ (the fraction of $b\overline b$ pairs in hadronic $Z$
decays).\cite{Rb}
Extensions of the MSSM (GUT and otherwise) can quite easily predict
proton decay and neutron-antineutron oscillations at low but observable
rates, even if $R$-parity is exactly conserved. However, it would be quite
difficult to ascribe a positive result for any of these processes to
supersymmetry in an unambiguous way. There is no substitute for the
direct detection of sparticles. In this section we will give
an incomplete and entirely qualitative review of some of the possible
signals for direct detection of supersymmetry. The reader
is encouraged to consult
Refs.\cite{DPFpheno,Tatareview,snowmass96} 
for recent reviews which cover the subject more systematically.

\subsection{Signals at $e^+e^-$ colliders}\label{subsec:signals.LEPNLC}

At $e^+e^-$ colliders, sparticles (other than the
gluino) can be pair-produced through tree-level processes:
\beq
e^+e^- \rightarrow
\stilde C_i^+ \stilde C_j^-,\>\>\, \stilde N_i \stilde N_j,
\>\>\, \stilde \ell\stilde \ell,\>\>\, \stilde \nu \stilde \nu,
\>\>\, \stilde q \stilde q.
\label{eesignals}
\eeq
with cross-sections
determined just by the electroweak gauge couplings
and the sparticle mixings. All of the processes in eq.~(\ref{eesignals})
get contributions from the $s$-channel exchange of the $Z$ boson
and (for charged sparticle pairs) of the photon.
In the cases of $\stilde C_i^+ \stilde C_j^-$,
$\stilde N_i \stilde N_j$, $\stilde e_R \stilde e_R$,
$\stilde e_L \stilde e_L$ and  $\stilde \nu_e \stilde \nu_e$
production, there are also $t$-channel contributions from
the exchanges of a virtual sneutrino, selectron, neutralino,
neutralino and chargino, respectively. The $t$-channel
contributions are quite significant if the exchanged sparticle
is not too heavy, and interference between
the $s$- and $t$-channel contributions can be either destructive
or constructive. For example, the production of wino-like
$\stilde C_1^+ \stilde C_1^-$ pairs typically suffers a destructive
interference between the $s$-channel graphs with $\gamma,Z$ exchange
and the $t$-channel graphs with $\stilde \nu_e$ exchange, if the
sneutrinos are not too heavy.
In the case of sleptons, the pair-production of
smuons and staus
proceeds only through the $s$-channel diagrams of
Fig.~\ref{fig:sleptonprod}a, while selectron production
also has a contribution from the $t$-channel exchanges of the
neutralinos, as shown in Fig.~\ref{fig:sleptonprod}b.
\begin{figure}
\centerline{\psfig{figure=susysleptonprod.ps,height=1.2in}}
\caption{Diagrams contributing to slepton  pair-production at $e^+e^-$
colliders.
\label{fig:sleptonprod}}
\end{figure}
[We have drawn the neutralino line as if it were a pure gaugino,
since the gaugino components of $\stilde N_i$ are responsible
for the coupling to electron-selectron.]
For this reason, selectron production may be significantly larger
than smuon or stau production at $e^+e^-$ colliders.
The important interactions for sparticle production processes
are always of electroweak interaction strength, namely the ones
shown in
Figs.~\ref{fig:gaugino}b,c and the ordinary gauge interactions.
The cross sections are too complicated to be listed here,
but can be found in Ref.\cite{epluseminuscrosssections}

The pair-produced sparticles will decay as discussed in section
\ref{sec:decays}.
If the LSP is the lightest neutralino, it will always escape the
detector because it has no strong or electromagnetic interactions.
Therefore every event will have two LSPs leaving the detector,
so there will be at least $2m_{\NI}$ of missing energy ($\Etot$).
For example, in the case of $\stilde C_1^+ \stilde C_1^-$ production, the
possible signals
include a pair of acollinear leptons plus $\Etot$,
one lepton and a pair of jets plus $\Etot$, and multiple jets
plus $\Etot$.
The relative importance of these signals depends on the branching fraction
of the chargino into the competing channels $\stilde C_1 \rightarrow
\ell\nu \NI$ and $qq^\prime\NI$. In the case of slepton pair-production,
the signal should be two energetic, acollinear, same-flavor leptons plus
$\Etot$.
It is not difficult to construct the other possible signatures
for sparticle pairs, which can become quite complicated for the
heavier charginos, neutralinos and squarks.

At the CERN LEP $e^+e^-$ collider, one has a reasonable possibility
of seeing neutralino, chargino, charged slepton, sneutrino, or top-squark
pairs. In the LEP1 runs at $\sqrt{s} = m_Z$,
the measurement of the invisible decay width of the $Z$ boson placed
a lower bound on sneutrino masses of about 40 GeV,
even though they can decay completely invisibly like
$\stilde \nu \rightarrow \nu\NI$. Similarly, the contribution
of $\NI\NI$ to the invisible width of the $Z$ rules out a significant
region of parameter space, with a lower bound on
$m_{\NI}$ which unfortunately depends strongly on the other parameters.
Model-independent lower bounds
have been set on the charged sparticle masses of roughly $m_Z/2$.
At this writing, LEP2 upgrades at $\sqrt{s} =$ 130-140, 161, 172, 183, 189 
GeV
and beyond are continuing to raise the lower bounds on the lightest
``visible" sparticles.
It is worth noting that in all future $e^+e^-$ collider
searches, there
will be a
large background for
the acollinear leptons plus $\Etot$ and the lepton plus jets
plus $\Etot$ signals from $W^+W^-$ production with one or both
of the $W$ bosons decaying leptonically. However,
these and other Standard Model backgrounds
can be kept under control with clever cuts.
It should also be mentioned that LEP2 is 
conducting a promising search for the lightest Higgs boson of
supersymmetry
through $e^+e^-\rightarrow h^0Z$ or perhaps $e^+e^-\rightarrow h^0A^0$.
However, observation
of the Higgs at LEP2 would be only a powerful clue that we are on the
right track in pursuing supersymmetry, and not a proof. Conversely,
the non-observation of $h^0$ at LEP2 should not be construed as
evidence against supersymmetry, in view of
eqs.~(\ref{mssmhiggsbound}) and (\ref{generalhiggsbound}).

At a future linear $e^+e^-$ collider
with $\sqrt{s} = $ a few hundred GeV to
1.5 TeV, the processes in eq.~(\ref{eesignals}) should be
probed close to the kinematic limit, given sufficient integrated
luminosity.\cite{NLCsusy}
In the case of $\stilde \nu\stilde \nu$ production, this
assumes that some of the decays are visible, rather than just
$\stilde\nu\rightarrow\nu\NI$. In the cases of the heavier sparticles,
the cascade decays mentioned in the previous section will provide a rich
set of signals to study. By making use of polarized beams and the
relatively clean $e^+e^-$ collider environment, one can
disentangle the sparticle spectrum. For example, measuring the maximum
and minimum energy endpoints of the leptons produced in $e^+e^-
\rightarrow \stilde\ell_R\stilde\ell_R$ with
$\stilde\ell_R \rightarrow \ell\NI$,
one can precisely determine both $m_{\stilde\ell_R}$ and
$m_{\stilde N_1}$. By varying the polarization of the electron beam,
one can control the $W^+W^-$ background and simultaneously check
that the contributions to the sparticle production cross-sections have the
correct magnitude and vary in the right
way. This will allow one to check
the spin and the ``handedness" of the produced squarks and sleptons.
Similar precision studies of chargino and neutralino production
can also be performed. In general, a high-energy linear lepton collider
will provide an excellent way of testing supersymmetric relations.
It is also worth noting that searches for $e^+e^- \rightarrow$
$h^0Z$, $h^0A^0$, $H^0Z$, $H^0A^0$ and $H^+H^-$ should be able to
definitively test the Higgs sector of the MSSM.

If the gravitino is the LSP as in gauge-mediated models, then one must
take into account the possibilities mentioned in section
\ref{subsec:decays.gravitino}.
If the lightest neutralino is the NLSP and the decay
$\NI\rightarrow\gamma\G$ occurs within the detector, then even the
process $e^+e^-\rightarrow \NI\NI$ leads to a dramatic signal of
two energetic photons plus missing energy.\cite{eeGMSBsignal,DDRT,AKKMM2}
There are
significant
backgrounds to the $\gamma\gamma\Etot$ signal, but they are
easily removed by cuts. Each of the other sparticle pair-production
modes eq.~(\ref{eesignals}) will lead to the same signals
as in the neutralino LSP case, but now with two additional
energetic photons which should make the experimentalists' tasks
quite easy. If the decay length for $\NI\rightarrow\gamma\G$ is much
larger than the
size of a detector, then the signals revert back to those found in
the neutralino LSP scenario. In an intermediate regime for
the $\NI\rightarrow\gamma\G$
decay length, one may see events with one or both
photons displaced from the event vertex by a macroscopic distance.

If the NLSP is a charged slepton $\stilde \ell$, then
$e^+e^-\rightarrow \stilde \ell^+\stilde \ell^-$ followed by prompt
decays $\stilde \ell\rightarrow \ell \G$ will yield
two energetic same-flavor leptons in every event,
and with a different energy distribution than the acollinear leptons
that would follow from either $\stilde C_1^+\stilde C_1^-$
or $\stilde \ell^+\stilde \ell^-$ production in the neutralino
LSP scenario. The $W^+W^-$ background can be a problem here, but can
be defeated with angular cuts at LEP2 or polarized beams at future
$e^+e^-$ colliders. Pair-production of non-NLSP sparticles will yield
unmistakable signals which are the same as those found in the neutralino
NLSP case but with two additional energetic leptons (not necessarily
of the same flavor). A perhaps even more exciting possibility is that the
NLSP is
a slepton which decays very slowly.\cite{DDRT} If the slepton NLSP is
so long-lived that it decays outside the detector, then slepton
pair-production
will lead to events featuring a pair of charged particle tracks
with a high ionization rate which betrays their very large mass.
If the sleptons decay within the detector, then one can look for kinks
in the charged particle tracks, or a macroscopic impact
parameter. The pair-production of any of the other
heavy charged sparticles will also yield heavy charged particle
tracks or decay kinks, plus leptons and/or jets, but no
$\Etot$ unless the decay chains happen to include neutrinos.
It may also be possible to identify the presence of a heavy charged
NLSP by measuring its anomalously long time-of-flight through the
detector.

\subsection{Signals at hadron colliders}\label{subsec:signals.FNALLHC}

At hadron colliders, the most important channels for
sparticle production are typically expected to be
\beq
&&\stilde C_i^+ \stilde C_j^-,\>\>\,
\stilde N_i \stilde C_j^\pm,\>\>\,
\stilde N_i \stilde N_j,\>\>\,{\rm and}
\label{charneutproduction}
\\
&&\stilde g \stilde g,\>\>\,
\stilde g \stilde q,\>\>\,
\stilde q \stilde q.
\label{gluinosquarkproduction}
\eeq
At the Fermilab Tevatron $p\overline p$ collider with
$\sqrt{s} = 2$ TeV, the chargino and
neutralino production processes (through valence quark
annihilation into virtual weak bosons)
tend to
have the larger cross-sections, unless the squarks or gluino are rather
light (less than 300 GeV or so).
In a typical scenario where $\stilde C_1$ and $\stilde N_2$
are mostly $SU(2)_L$ gauginos and $\stilde N_1$ is mostly bino,
the largest production cross-sections in eq.~(\ref{charneutproduction})
belong to the
$\stilde C_1\stilde C_1$
and 
$\stilde N_2\stilde C_1$
channels, because they have significant couplings to $W$ and
$\gamma,Z$ bosons,
respectively.
At the future CERN
LHC $p p$ collider with $\sqrt{s} \sim 14$ TeV,
the situation is typically reversed, with production of gluinos and
squarks by gluon fusion and gluon-quark fusion usually dominating, unless
the gluino and squarks are heavier
than 1 TeV or so. At both colliders, one can also have associated
production of a chargino or neutralino together with a squark or gluino, but
the cross-sections for such processes are probably significantly lower
than
for the ones in eqs.~(\ref{charneutproduction}) and
(\ref{gluinosquarkproduction}). Slepton pair production may be
rather small at the Tevatron, but might be observable there or at the
LHC.\cite{sleptonLHC}
Cross-sections for sparticle production at hadron colliders
can be found in Ref.\cite{gluinosquarkproduction}

The decays of the produced sparticles result in final states with
two neutralino LSPs which escape the detector.
The LSPs again carry away at least $2 m_{\NI}$ of missing energy, but
at hadron colliders only the
component of the missing energy which is manifest in momenta
transverse to the colliding beams (denoted $\Et$) is
observable. Therefore
in general the observable signals for supersymmetry at hadron colliders
are
$n$ leptons + $m$ jets + $\Et$, where either $n$ or $m$ might be 0. There
are important Standard Model backgrounds to
many of these signals, especially from processes involving
production of $W$ and $Z$ bosons which can decay to neutrinos, yielding
$\Et$. Therefore it is important to identify specific signals for
which the backgrounds can be reduced.
Of course, this depends on which sparticles are being produced and how they
are decaying.
For example, the ``classic" $\Et$ signal for supersymmetry at hadron
colliders is events
with jets and $\Et$ but no energetic isolated leptons. The latter
requirement
reduces backgrounds from Standard Model processes with leptonic $W$
decays,
and is obviously most effective if the relevant sparticle decays have
sizeable
branching fractions into channels with no leptons in the final state.

Another type of signal arises if
the gluino decays with a significant branching fraction
to hadrons plus a chargino, which can subsequently
decay into a final state with a charged
lepton, a neutrino, and $\stilde N_1$. Since the gluino doesn't know
anything about
electric charge, the
single charged lepton produced from each gluino decay can have either sign
with equal probability. This means that
gluino pair production will often
lead to events with two leptons with the same charge
(but possibly different flavors) plus jets and $\Et$.
This signal can also arise from $\stilde q\stilde q$ and $\stilde q
\stilde g$ production, e.g. if the squarks decay like
$\stilde q \rightarrow q\stilde g$.
This same-sign
dilepton signal \cite{likesigndilepton} has small
physics backgrounds from
the Standard Model both at the Tevatron and the LHC.
The reason is that the largest background sources for isolated
lepton pairs, namely $W^+W^-$, Drell-Yan and $t\overline t$ production,
can only yield opposite-charge dileptons.

Despite the backgrounds just mentioned, opposite-charge dilepton signals,
e.g.~from slepton pair production with subsequent decays $\stilde \ell
\rightarrow \ell \NI$, can give an observable signal especially at the
LHC.

Another useful possibility is the trilepton signal,\cite{trilepton} which
features three leptons plus $\Et$ and possibly jets.
This can come about from $\stilde C_1\stilde N_2$ production
followed by the decays indicated in eq.~(\ref{trileptonbabies}),
in which case one expects little hadronic activity
in the event. It could also come from $\stilde g\stilde g$,
$\stilde q\stilde g$, or $\stilde q \stilde q$ production,
with one of the gluinos or squarks decaying through a $\stilde C_1$
and the other through a $\stilde N_2$. In that case, there
will be jets from the decays, in addition to the three leptons and $\Et$.
These signatures rely on the $\stilde N_2$ having a significant
branching fraction for the three-body decay to leptons in
eq.~(\ref{trileptonbabies}).
For this reason, the two-body decay modes in
eq.~(\ref{nino2body})
are sometimes called ``spoiler" modes, since if they are kinematically
allowed they can dominate, spoiling the trilepton signal.
This is because if
the $\stilde N_2$ decay is through
an on-shell $h^0$, then the final state will very likely include
bottom-quark jets rather than isolated leptons, while if
the decay is through
an on-shell $Z$, then there can still be two leptons but there are Standard
Model backgrounds with unfortunately
similar kinematics from processes
involving $Z\rightarrow \ell^+\ell^-$. Either way,
the trilepton signal can be spoiled, but other leptons + jets + $\Et$
signals may be observable above Standard Model backgrounds, especially
if bottom quark jets can be tagged with high efficiency.

The single lepton plus jets plus $\Et$ signal \cite{singlelepton}
has large
Standard
Model backgrounds from processes with $W\rightarrow\ell\nu$. However,
it also can have a large rate from various superpartner production modes,
and may still give the best signal at the LHC. One should also be
aware of very interesting signals which can arise for particular
ranges of parameters. For example, in a scenario studied in
Ref.\cite{snowmass96}, the only two-body decay channel for the gluino is
$\stilde g\rightarrow b\stilde b_1$, with subsequent decays
$\stilde b_1 \rightarrow b \stilde N_2$ and $\stilde N_2 \rightarrow
\ell^+\ell^-\NI$ or $\stilde N_2 \rightarrow q q \NI$. In
that case, gluino pair production
gives a spectacular signal of four bottom jets plus
up to four leptons plus $\Et$. In general, production of relatively
light $\stilde t_1$ and $\stilde b_1$ can give hadron collider
signals rich in bottom jets, either through direct production or cascade
decays.

If the gravitino is the LSP, these signals can be significantly
modified. If the NLSP is a neutralino with a prompt decay
$\NI\rightarrow \gamma\G$, then one expects events with two energetic,
isolated photons plus $\Et$ from the escaping gravitinos,
rather than just $\Et$.
So at a hadron collider the signal is $\gamma\gamma+X+\Et$ where $X$ is
any collection of
leptons plus jets. The Standard Model backgrounds
relevant for such events
are quite small. If
the $\NI$ decay length is long enough, then it may be
measurable because the photons will not point
back to the event vertex. If the $\NI$ decay is outside of the
detector, then one just has the usual leptons + jets + $\Et$ signals
as discussed above in the neutralino LSP scenario.

In the case that
the NLSP is a charged slepton, then the decay $\stilde \ell
\rightarrow \ell\G$ can provide two extra leptons in each event,
compared to the signals with a neutralino LSP. If the $\stilde \tau_1$
is sufficiently
lighter than the other charged sleptons $\stilde e_R$,
$\stilde \mu_R$,
and so is effectively the sole NLSP, then events will always have a pair
of taus. If the slepton NLSP
is long-lived, one can look for events with
a pair of very heavy charged particle tracks or a long time-of-flight
in the detector.
Since slepton
pair-production usually has a much smaller cross-section than
the processes in eq.~(\ref{charneutproduction}) and
(\ref{gluinosquarkproduction}), this will typically be accompanied by
leptons and/or jets from the same event vertex, which may be of crucial
help in identifying candidate events.
It is also quite possible
that the decay length of $\stilde \ell \rightarrow \ell\G$ is measurable
within the detector, seen as a macroscopic kink in the charged
particle track.

\subsection{Dark matter detection}\label{subsec:signals.darkmatter}

One of the major successes of supersymmetry with
exact $R$-parity conservation is that an electrically neutral LSP can be
a good candidate for the dark matter.
There are three obvious candidates: the gravitino, the lightest sneutrino,
and the lightest neutralino.
If the gravitino is the LSP, as in gauge-mediated models, then
relic gravitinos left over from the early universe would be essentially
impossible to detect even if they can be arranged
to have the right cosmological density today.
The possibility of a sneutrino LSP
making up the dark matter with a cosmologically interesting
density has now been
ruled out by direct
searches.\cite{sneutrinonotLSP}
The most attractive prospects for direct
detection of supersymmetric dark matter, therefore,
are based on the idea that the lightest neutralino $\NI$ is the LSP,
as happens quite naturally in the minimal supergravity models.

In the early universe, sparticles
existed in thermal equilibrium with the ordinary Standard Model particles.
As the universe cooled and expanded, the sparticles could no longer
be produced and they all annihilated or decayed into $\NI$. The
remaining $\NI$ can annihilate through processes
$\NI\NI \rightarrow f\overline f$ with $t$-channel exchange
of squarks and sleptons or the $s$-channel exchange of
Higgs scalars or a $Z$ boson. Depending on the mass of $\NI$,
other processes like $\NI\NI\rightarrow$$W^+W^-$, $ZZ$,
$Zh^0$, $h^0h^0$ or even $W^\pm H^\mp$,
$Z A^0$, $h^0 A^0$, $h^0 H^0$, $H^0 A^0$, $H^0H^0$, $A^0 A^0$,
or $H^+H^-$ could also have been important. Eventually,
as the density of LSPs decreased, the annihilation rate became very small,
and the $\NI$ relic density is determined by this small rate
and the
subsequent dilution
due to the expansion of the universe.

It is a remarkable coincidence that the predicted density of
a bino-like (or perhaps higgsino-like) neutralino LSP obtained by
doing these calculations carefully can be in the right range to
make up a significant fraction of the critical density of the universe, 
and perhaps to
explain the rotation curves of galaxies.\cite{darkmatterreviews} (A
wino-like $\NI$ would
have only a tiny relic density, but given the gaugino mass hierarchy
eq.~(\ref{usualm1m2}), there is little motivation for such a thing
anyway.) It is also necessary to require that the density of
surviving LSPs not be too large, so that the universe could have
reached its present size and age of at least $10^{10}$ years.
This tends to put an upper limit on the LSP mass, but unfortunately
it is difficult to make a general, parameter-independent bound
out of this because if
the masses are arranged just right, the LSP may happen to
annihilate very efficiently through a resonance.
If neutralino LSPs really make up the cold dark matter,
then their mass density in our neighborhood ought to be
at least about 0.1 GeV/cm$^3$ in order to explain the rotation
curves of galaxies. In principle, they should be detectable through
their weak interactions with ordinary matter, or by their ongoing
annihilations.

The direct detection of $\NI$ depends on their elastic scattering
off of heavy nuclei in a detector. At a fundamental level, $\NI$
can scatter off of a quark by virtual exchange of squarks,
a $Z$ boson, or Higgs scalars, or can scatter off of gluons
through one-loop diagrams. The energy transferred to the nucleus
in these collisions is typically of order tens of keV. However,
there are important backgrounds from radioactivity and cosmic rays.
The optimal detector material (e.g. germanium, silicon, or
niobium) depends on the details of the $\NI$-nucleus interaction.
Present detectors are still not sensitive to most regions of parameter
space, but there is hope that this can change in the future.

Another, more indirect, way to detect neutralino LSPs is through
ongoing annihilations. This can occur
in regions of space where the
density
is greatly enhanced compared to our own neighborhood. This can
occur if the LSPs lose energy by repeated scattering off of nuclei,
eventually becoming concentrated inside massive astronomical bodies
like the Earth or the Sun. In this case the annihilation of neutralino
pairs into neutrinos is the most important process,
since no other particles can escape from the center of the object
where the annihilation is going on. In particular, muon neutrinos
and antineutrinos from
$\NI\NI \rightarrow \nu_\mu\overline\nu_\mu$
will travel large distances, finally undergoing a charged-current
interaction leading to energetic muons pointing back to the center of the
Earth or Sun. There are also interesting possible signatures from
neutralino LSP annihilation in the galactic halo which might produce
detectable quantities of high-energy photons, positrons, and
antiprotons.\cite{darkmatterreviews}

\section{Some miscellaneous variations}\label{sec:variations}
\setcounter{equation}{0}
\setcounter{footnote}{1}

In this section we will briefly consider a few variations on the
simple picture of the MSSM that has been outlined above. First, we will 
consider
the possibility of $R$-parity violation
in section \ref{subsec:variations.Rparity}. Another obvious way to
extend the MSSM is to introduce new chiral supermultiplets,
corresponding to scalars and fermions that are all sufficiently heavy to
have avoided discovery so far. In general, this requires that the new
chiral supermultiplets must form a real representation of the Standard
Model gauge group. The simplest such possibility is that the new particles
live in just one gauge-singlet chiral supermultiplet; this
possibility is discussed in section \ref{subsec:variations.NMSSM}. One can
also extend the MSSM
by introducing new gauge interactions which are spontaneously broken
at very high energies. The possibilities here include GUT models
like $SU(5)$ and $SO(10)$ and $E_6$ which unify the Standard Model gauge
interactions, with important implications for rare processes like
proton decay. Superstring models also quite generically imply that
the Standard Model gauge group should be extended at high energies.
There is
a vast literature on these possibilities, but we will
concentrate instead on the implications of just adding a single
additional abelian factor to the gauge group, in section
\ref{subsec:variations.Dterms}.

\subsection{Models with $R$-parity
violation.}\label{subsec:variations.Rparity}

So far we have assumed that $R$-parity (or equivalently
matter parity) is an exact symmetry of the MSSM.
This assumption precludes renormalizable proton decay and
predicts that the LSP should be stable, but despite these virtues
$R$-parity is not inevitable.
Because of
the threat of proton decay, we expect that if $R$-parity is violated,
then in the renormalizable lagrangian either B-violating or L-violating
couplings are allowed, but not
both, as explained in section \ref{subsec:mssm.rparity}.

One proposal is that matter parity can be replaced by an alternative
discrete symmetry which still manages to forbid proton decay
at the level of the renormalizable lagrangian. The possibilities
have been cataloged in Ref.\cite{baryonparity}, where it was
found that provided no new particles are to be added to the
MSSM, that the discrete symmetry is family-independent, and that it can be
defined at the level of the superpotential,
there is only one other candidate discrete symmetry besides matter parity.
That other possibility is a $Z_3$ discrete symmetry
\cite{baryonparity} which was originally called
``baryon parity", but is more appropriately referred to as ``baryon
triality". The baryon triality of any particle with baryon number B and
weak hypercharge $Y$ is defined to be
\beq
Z_3^{\rm B} = {\rm exp}\left ( {2\pi i\over 3} [{\rm B}-2Y] \right ).
\eeq
It is easy to check that this is always a cube root of unity for
the MSSM particles, since B$-2Y$ is always an integer.
The symmetry principle to be enforced is that the product of the
baryon trialities of the particles in any term in the lagrangian
(or superpotential) must be 1.
This symmetry conserves baryon number at the renormalizable
level while allowing lepton number violation; in other words,
it allows the superpotential terms in eq.~(\ref{WLviol}) but forbids
those in eq.~(\ref{WBviol}). In
fact, baryon triality conservation
has the remarkable property that it absolutely forbids proton 
decay.\cite{noprotondecay}
The reason for this is simply that baryon triality requires
that B can
only be violated in multiples of 3 units (even in nonrenormalizable
interactions), while any kind of
proton decay would have to violate
B by 1 unit. So it is eminently falsifiable. Similarly, baryon triality
conservation predicts that
experimental
searches for neutron-antineutron
oscillations will be negative, since they would violate baryon number by 2
units. However, baryon triality conservation does allow the LSP to decay.
If one adds some new chiral supermultiplets to the MSSM (corresponding
to particles which are presumably very heavy), one can concoct a variety
of new candidate discrete symmetries besides matter parity and baryon
triality. Some of these will allow B violation in the superpotential,
while forbidding the lepton number violating superpotential
terms in eq.~(\ref{WLviol}).

Another idea is that matter parity is an exact symmetry of the underlying
superpotential, but it is spontaneously broken by the VEV of a scalar
with $P_R=-1$. 
One possibility is that an MSSM
sneutrino gets a VEV,\cite{sneutrinovevRparityviolation} since sneutrinos
are scalars carrying L=1. However,
there are strong bounds \cite{nonsneutrinovevRparityviolation} on
$SU(2)_L$-doublet sneutrino
VEVs
$\langle \stilde \nu \rangle \ll m_Z$ coming from the
requirement
that the corresponding
neutrinos do not have large masses. It is somewhat
difficult
to understand why such a small VEV should occur, since the scalar
potential which produces it must include soft sneutrino (mass)$^2$
terms of order $m^2_{\rm soft}$. One can get around this
by instead introducing a new gauge-singlet chiral supermultiplet with
L=$-1$. The scalar component can get a large VEV, which can induce
L-violating terms (and in general B-violating terms also) in the
low-energy effective superpotential of the
MSSM.\cite{nonsneutrinovevRparityviolation}

In any case, if $R$-parity is violated, then the LSP will decay,
completely
altering the signals for supersymmetry. The type of signal
to look for depends on the form of $R$-parity violation.
If there are L-violating
terms of the type $\lambda$ and/or $\lambda^\prime$ as in
eq.~(\ref{WLviol}), then the final states from
$\stilde N_1$ decay will always involve a charged lepton or a neutrino
plus either a pair of additional charged leptons or a pair of jets.
Two such decays are shown in Fig.~\ref{fig:rparityviolation}a,b, but
there are others. Signals for
supersymmetry
\begin{figure}
\centerline{\psfig{figure=susyrparityviolation.ps,height=.926in}}
\caption{Decays of the $\NI$ LSP in models with $R$-parity violation
[see eqs.~(\ref{WLviol}) and (\ref{WBviol})].
\label{fig:rparityviolation}}
\end{figure}
will therefore always include leptons or large missing energy, or both.
On the other hand,
if terms of the form $\lambda^{\prime\prime}$ in eq.~(\ref{WBviol}) are
present instead, then there are baryon-number violating decays
$\stilde N_1 \rightarrow q q^\prime q^{\prime\prime}$ from
graphs like the
one shown in Fig.~\ref{fig:rparityviolation}c. In
that case, supersymmetric
events will always have lots of hadronic activity, and will only have
missing energy signatures when the other parts of the decay chains
happen to include neutrinos. This could make the discovery
and study
of supersymmetry very difficult.
There are other possibilities, too, because if $R$-parity is violated,
then the decaying LSP need not be $\NI$,
and sparticles which are not the LSP can in principle decay directly
to Standard Models quarks and leptons. If $\lambda^{\prime}$
is non-zero, then squarks can be produced as resonances at
the $e^\pm p$ collider at HERA.
A complete survey of the
possibilities would be far too complicated to present here.

\subsection{The next-to-minimal supersymmetric standard
model}\label{subsec:variations.NMSSM}

The simplest possible extension of the particle content of the MSSM
is to add a new gauge-singlet chiral supermultiplet. The resulting
model is often called the next-to-minimal supersymmetric standard model
(NMSSM).\cite{NMSSM}
The most general possible
superpotential for this model is given by
\beq
W_{\rm NMSSM} =
{1\over 6} k S^3 + {1\over 2} \mu_S S^2 + \lambda S H_u H_d
+ W_{\rm MSSM},
\label{NMSSMwww}
\eeq
where $S$ stands for both the new chiral supermultiplet and its
scalar component. (There could also be a term linear in $S$ in $W_{\rm
NMSSM}$, but this can always be removed by redefining $S$ by a constant
shift.)

One of the virtues of the NMSSM is that it can
provide a solution to the $\mu$ problem mentioned in sections
\ref{subsec:mssm.superpotential} and \ref{subsec:MSSMspectrum.Higgs}.
To understand this, suppose we set $\mu_S = \mu = 0$ so that there
are no mass terms or dimensionful parameters in the superpotential
at all.
Then an effective $\mu$-term for $H_uH_d$ will still arise from
the third term in eq.~(\ref{NMSSMwww}) if $S$ gets a
VEV, with $\mu = \lambda \langle S \rangle$. The absence of
dimensionful terms in $W_{\rm NMSSM}$ can be enforced by introducing
a new symmetry (in various different ways).
The soft terms in the lagrangian give a contribution to the
scalar potential which can be written as
\beq
V_{\rm soft}^{\rm NMSSM} =
({1\over 6} a_k S^3 + a_{\lambda} S H_u H_d + \conj) +
m_S^2 |S|^2
+ V_{\rm soft}^{\rm MSSM},
\eeq
where $a_k$ and $a_{\lambda}$ have dimensions of mass. One
may now set $b=0$ in $V_{\rm soft}^{\rm MSSM}$, because
an effective value for $b$ will be generated, equal to
$a_{\lambda} \langle S \rangle$.
If the new parameters
$k$, $\lambda$, $a_k$ and $a_\lambda$ are chosen correctly, then
phenomenologically acceptable VEVs will be induced for $S$, $H_u^0$, and
$H_d^0$. A correct treatment of this requires the inclusion of one-loop
radiative corrections. But the important point is that
the scale of the VEV $\langle S \rangle$, and therefore the
effective value of $\mu$, is then 
determined by
the soft terms of order $m_{\rm soft}$, instead of being
a free parameter which is conceptually independent of supersymmetry
breaking.

The NMSSM contains, besides the particles of the MSSM, a real
$P_R=+1$ scalar, a real $P_R=+1$ pseudoscalar, and a $P_R=-1$ Weyl fermion
``singlino". These fields have no gauge couplings of their own,
so they can only interact with Standard Model particles by mixing
with the neutral MSSM fields with the same spin and charge.
The real scalar mixes with the MSSM particles $h^0$ and $H^0$, and
the
pseudo-scalar mixes with $A^0$.
One of the effects of replacing the $\mu$ term by the dynamical field $S$
is to raise the upper bound on the lightest Higgs mass, for a given set
of the other parameters in the theory.
However, the bound in
eq.~(\ref{generalhiggsbound}) is still respected in the NMSSM (and any
other perturbative extension of the MSSM), provided only that the
sparticles that contribute in loops to the Higgs mass are lighter
than 1 TeV or so.
The odd $R$-parity singlino mixes
with the four MSSM neutralinos, so there are really five neutralinos now.
In many regions of parameter space, mixing effects involving
the singlet
fields are small, and they essentially just decouple.
In that case, the phenomenology of the NMSSM is nearly 
indistinguishable from that of the MSSM.
However, if any of the five NMSSM neutralinos
(and especially the LSP)
has a large mixing between the  singlino
and the usual gauginos and higgsinos,
then the signatures for sparticles can be altered
in important ways.\cite{NMSSMpheno}

\subsection{Extra $D$-term contributions to scalar
masses}\label{subsec:variations.Dterms}

Another way to generalize the MSSM is to include additional gauge
interactions.
The simplest gauge extension of the MSSM introduces just an additional
abelian gauge symmetry, which we can call $U(1)_X$.
As long as $U(1)_X$ is broken at a very high mass scale, then the
corresponding vector gauge boson and gaugino fermion will be very heavy
and will decouple from physics
at the TeV scale and below. If so, one
might
suppose
that all effects following from the existence of $U(1)_X$ will be
completely negligible for collider experiments in the
foreseeable future. However, this is not necessarily so, because
as long as the MSSM fields carry $U(1)_X$ charges, the breaking of
$U(1)_X$ at a very high energy scale can leave its imprint on the soft
terms of the
MSSM.\cite{Dterms}

To see how this works, let us consider the scalar potential for a model
in which $U(1)_X$ is broken. Suppose that the MSSM scalar fields,
denoted generically by $\phi_i$, carry $U(1)_X$ charges $x_i$.
In order to break $U(1)_X$, we also introduce a pair of chiral
supermultiplets with $U(1)_X$ charges $\pm 1$, denoted $S_+$ and
$S_-$.
These fields are singlets under the Standard Model gauge group
$SU(3)_C \times SU(2)_L \times U(1)_Y$, so that when they get VEVs,
they will just break $U(1)_X$. An obvious guess for the
superpotential containing $S_+$ and $S_-$ is $W = M S_+ S_-$,
where $M$ is a supersymmetric mass. However, unless $M$ vanishes or
is very small, it will yield positive-semidefinite quadratic terms in the
scalar potential
of the form $V = |M|^2 (|S_+|^2 + |S_-|^2)$ which will force the minimum
to be at $S_+ = S_- = 0$. Since we want $S_+$ and $S_-$ to obtain VEVs,
this is unacceptable. Therefore we assume that $M$ is 0 (or
very small) and that the leading contribution
to  the superpotential comes instead from a nonrenormalizable term, say:
\beq
W = {\lambda\over 2 \MPlanck} S_+^2 S_-^2.
\label{wfordterms}
\eeq
(Non-renormalizable terms in the superpotential obey the same rules
as we found before; in particular they must be analytic functions of the
chiral superfields. See the Appendix for more details on
non-renormalizable
lagrangians in supersymmetric theories.)
The equations of motion for the auxiliary fields are then
$F^*_{S_+} = -\partial W/\partial S_+ = -(\lambda/\MPlanck)S_+ S_-^{2}$
and
$F^*_{S_-} = -\partial W/\partial S_- = -(\lambda/\MPlanck)S_- S_+^{2}$,
and the
corresponding contribution to the scalar potential is
\beq
V_F \>=\> |F_{S_+}|^2 + |F_{S_-}|^2 + \ldots \>=\>
{|\lambda|^2\over \MPlanck^2}
\Bigl (
|S_+|^4 |S_-|^2 + |S_+|^2 |S_-|^4 \Bigr ) +\ldots .
\eeq
Here the ellipses represent other terms that are higher order in
$1/\MPlanck$ (see Appendix), which we can safely ignore.
In addition, there are soft terms which must be taken into account:
\beq
V_{\rm soft} = m_+^2 |S_+|^2 + m_-^2 |S_-|^2 -
\left ({a\over 2\MPlanck} S_+^2 S_-^2 + \conj\right ).
\eeq
The terms with $m_+^2$ and
$m_-^2$ are soft masses for $S_+$ and $S_-$. We can
assume that they come from a minimal supergravity framework at the Planck
scale, but in general they will be renormalized
differently, due to different interactions for $S_+$ and $S_-$
which we have not bothered to write down in eq.~(\ref{wfordterms})
because they involve fields that will not get VEVs.
The last term is a ``soft" term exactly analogous to the ones
appearing in the second line of eq.~(\ref{lagrsoft}), with
$a$ of order $m_{\rm soft}$.
The coupling $a/2\MPlanck$ is
actually dimensionless, but should be treated as soft because of
its origin and
its tiny magnitude.
Such terms arise from the
supergravity lagrangian in an exactly analogous way to the
usual soft terms.
Usually one can just ignore them,
but this one plays a crucial role in the gauge symmetry breaking
mechanism.
The scalar potential for terms containing $S_+$ and $S_-$ is now:
\beq
V =
{1\over 2} g_X^2 \Bigl ( |S_+|^2 - |S_-|^2 + \sum_i x_i |\phi_i|^2
\Bigr )^2 +
V_F + V_{\rm soft}.
\label{xpotential}
\eeq
The first term is the square of the $U(1)_X$ $D$-term
[see eqs.~(\ref{solveforD}) and (\ref{fdpot})], and
$g_X$ is the $U(1)_X$ gauge coupling.
The scalar potential eq.~(\ref{xpotential}) has a nearly $D$-flat
direction, because the $D$-term part vanishes for $\phi_i=0$ and
any $|S_+| = |S_-|$.
Therefore, if
$m_+^2 + m_-^2 < 0$, the point $S_+ = S_- = 0$
will be destabilized and $S_+$ and $S_-$ can obtain large VEVs.
Without loss of generality, we can take $a$ and $\lambda$ to both be
real and positive for purposes of minimizing the scalar potential.
As long as $a^2 - 8 \lambda^2 (m_+^2 + m_-^2) > 0$,
the global minimum of the potential occurs for
\beq
\langle S_+ \rangle^2 \approx \langle S_- \rangle^2
\approx {a\MPlanck \over 6 \lambda^2} \Bigl [
1 + \sqrt{ 1 - 6 \lambda^2 (m_+^2 + m_-^2)/a^2 } \Bigr ]
\eeq
(with $\langle \phi_i\rangle = 0$), so $\langle S_+ \rangle
\approx \langle S_- \rangle \sim {\cal O}(\sqrt{m_{\rm soft} \MPlanck})$.
The $V_F$ contribution is what stabilizes the scalar potential
at very large field strengths.
The VEVs of $S_+$ and $S_-$ will be much larger than 1 TeV
as long as $a$ is not too small. Therefore the $U(1)_X$ gauge
boson and
gaugino can be very heavy, with masses of order $g_X \langle
S_\pm\rangle$,
and play no role in collider physics.

However, there is also a small deviation from $\langle S_+\rangle =
\langle S_- \rangle$, as long
as $m_+^2 \not= m_-^2$. At the minimum of the potential
with $\partial V/\partial S_+ = \partial V/\partial S_- = 0$,
the leading order difference in the VEVs is given by
\beq
 \langle S_+ \rangle^2 - \langle S_- \rangle^2 =
-{1\over g_X}\langle D_X \rangle \approx {1\over 2 g_X^2}(m_-^2 - m_+^2)
\eeq
assuming that
$\langle S_+ \rangle$ and $\langle S_- \rangle$ are much larger
than their difference. After integrating out
the $S_+$ and $S_-$ by replacing them
with their equations of motion expanded around the minimum of the
potential, one finds that the MSSM scalars
$\phi_i$ each
receive a correction to their (mass)$^2$ given by
\beq
\Delta m_i^2 =  -x_i g_X \langle D_X \rangle\, ,
\label{dxtermcorrections}
\eeq
in addition to the usual soft terms derived from the minimal supergravity
boundary conditions and RG equations.
The $D$-term corrections eq.~(\ref{dxtermcorrections}) can be roughly of
the order of $m_{\rm soft}^2$
at most, since they are all proportional to $m_-^2- m_+^2$.
Note that the result eq.~(\ref{dxtermcorrections}) does not actually
depend on our choice of
the nonrenormalizable superpotential, 
as long as  it produces the required symmetry
breaking with large VEVs; this is a general feature. In a
sense, the soft supersymmetry-breaking terms $m_+^2$ and $m_-^2$
have been recycled into a non-zero $D$-term for $U(1)_X$,
which then leaves its ``fingerprint" on the spectrum of MSSM scalar
masses.
The most important feature of the correction eq.~(\ref{dxtermcorrections})
is that each MSSM scalar (mass)$^2$ obtains a correction just
proportional to its charge $x_i$ under the spontaneously broken
gauge group, with a universal factor $g_X \langle D_X \rangle$. From
the
point of view of TeV scale physics, the quantity
$g_X \langle D_X \rangle$ can simply be taken to parameterize
our ignorance of how $U(1)_X$ got broken.
Typically, the charges $x_i$ are rational numbers
and do not all have the same sign, so that a particular candidate
$U(1)_X$ can leave a quite distinctive pattern of mass splittings
on the squark and slepton spectrum.

The additional gauge symmetry $U(1)_X$ in the above discussion
can stand alone, or may perhaps be embedded in a larger non-abelian
gauge group. If the gauge group for the underlying theory at the Planck
scale contains more than one new $U(1)$ factor, then each such
factor can make a contribution exactly analogous to
eq.~(\ref{dxtermcorrections}). Additional $U(1)$ gauge groups are
quite common in superstring models, so from that point of view
one may be optimistic about the existence of the corresponding $D$-term
corrections. Once one merely assumes the existence
of additional $U(1)$ gauge groups at very high energies, it is quite
unnatural to assume that such $D$-term contributions to the MSSM scalar
masses should vanish, unless there is an {\it exact} symmetry which will
enforce $m_+^2 = m_-^2$. The only question is
whether or not the magnitude of the $D$-term contributions is significant
compared to the usual minimal supergravity and RG contributions;
it may very well not be.
Note also that as long as the charges $x_i$ are family-independent,
then from eq.~(\ref{dxtermcorrections}) the squarks and sleptons with the
same electroweak quantum numbers remain degenerate, maintaining the
natural suppression of FCNC effects. So it is quite possible that
efforts to understand the sparticle spectrum of the MSSM will
need to take into account the possibility of $D$-terms from
additional gauge groups.

\section{Concluding remarks}\label{sec:outlook}
\setcounter{equation}{0}
\setcounter{footnote}{1}

In this primer, I have attempted to convey some of the more essential
features of supersymmetry as it is known so far. One of the most amazing
qualities of supersymmetry is that so much is known about it already,
despite the present lack of direct experimental data. Even the terms and
stakes of many of the important outstanding questions, especially the
paramount issue ``How is supersymmetry broken?", are already rather clear.
That this can be so is a testament to the unreasonably predictive quality
of the symmetry itself.

We have seen that sensible and economical models for supersymmetry at
the
TeV scale can be used as convenient templates for experimental searches.
As summarized in section \ref{subsec:MSSMspectrum.summary}, two of the
simplest possibilities are the ``minimal supergravity" scenario with new
parameters $m^2_0$, $m_{1/2}$, $A_0$, $\tan\beta$ and Arg$(\mu )$, and the
``gauge-mediated" scenario with new parameters $\Lambda$, $M_{\rm mess}$,
$\nmess$, $\langle F \rangle$, $\tan\beta$, and ${\rm Arg}(\mu )$.
However, one should not lose sight of the fact that the only indispensable
idea of supersymmetry is simply that of a symmetry between fermions and
bosons. Nature may or may not be kind enough to realize this beautiful
idea within one of the specific frameworks that have already been
explored well by theorists.

The experimental verification of supersymmetry will not be an end, but
rather a revolution in high energy physics. It seems likely to present us
with questions and challenges which we can only guess at presently. The
measurement of sparticle masses, production cross-sections, and decays
modes will rule out some models for supersymmetry breaking and lend
credence to others. We will be able to test the principle of $R$-parity
conservation, the idea that supersymmetry has something to do with the
dark matter, and possibly make connections to other aspects of cosmology
including baryogenesis and inflation. Other fundamental questions, like
the origin of the $\mu$ parameter and the rather peculiar hierarchical
structure of the Yukawa couplings may be brought into sharper focus with
the discovery of the MSSM spectrum. Understanding the precise connection
of supersymmetry to the electroweak scale will surely open the window to
even deeper levels of fundamental physics.

\section*{Acknowledgments}
I am grateful to G.~Kane and J.~Wells for many helpful comments
on this primer.
I am also indebted to my other collaborators
on supersymmetric matters,
S.~Ambrosanio,
N.~Arkani-Hamed,
%G.~Blanston,
D.~Casta\~no,
M.~Dine,
T.~Gherghetta,
I.~Jack, 
D.R.T.~Jones,
C.~Kolda,
G.~Kribs,
S.~Mrenna,
M.~Vaughn, 
Y.~Yamada
and especially P.~Ramond, for countless illuminating and
inspiring conversations on the subjects discussed here.
I thank the Aspen Center for Physics and the Stanford Linear
Accelerator Center for their hospitality.
This work was supported in part by the U.S. Department of Energy.

\addcontentsline{toc}{section}{Appendix: Nonrenormalizable supersymmetric
lagrangians}
\section*{Appendix: Nonrenormalizable supersymmetric
lagrangians}\label{appendix}
\renewcommand{\theequation}{A.\arabic{equation}}
\setcounter{footnote}{1}

In section \ref{sec:susylagr}, we discussed only renormalizable
supersymmetric lagrangians. However, like all known theories that
include general relativity, supergravity is nonrenormalizable as a
quantum field theory. It is therefore clear that nonrenormalizable
interactions must be present in any low-energy effective description
of the MSSM. Fortunately, these can be neglected for most
phenomenological purposes, because nonrenormalizable
interactions have couplings of negative mass dimension, proportional to
powers of 
$1/\MPlanck$ (or perhaps $1/\Lambda_{\rm UV}$, where $\Lambda_{\rm UV}$ is
some other
cutoff scale associated with new physics). This means that their
effects at ordinary energy scales $E$ accessible to experiment
are typically suppressed by powers of ${E/\MPlanck}$ (or by
powers of $E/\Lambda_{\rm UV}$). For energies $E\lsim 1$ TeV, the
effects of nonrenormalizable interactions are therefore usually too small
to be interesting.

Still, there are several reasons
why one might be interested in nonrenormalizable contributions to
supersymmetric lagrangians. First, some very rare processes (like proton
decay) can only be described
using an effective MSSM lagrangian which includes nonrenormalizable
terms.
Second, one may be interested in understanding physics at very high energy
scales where the suppression associated with nonrenormalizable terms is
not enough to stop them from being important. For example, this could be
the case in the study of the very early universe, or in
understanding how additional gauge symmetries get broken. Third, the
nonrenormalizable interactions may play a crucial role in understanding
how supersymmetry breaking is transmitted to the MSSM.
Finally, it is sometimes useful to treat strongly-coupled
supersymmetric gauge theories using nonrenormalizable effective
lagrangians, in the same way that chiral effective lagrangians are
used to study hadron physics in QCD. Unfortunately,
we will not be able to treat these rather complicated subjects in any sort
of systematic way.
Instead, we will merely sketch for the reader a few of the key
elements that go into defining a nonrenormalizable supersymmetric
lagrangian, so that they may hopefully seem slightly less mysterious when
encountered in other works. More detailed treatments may be found in
Refs.\cite{BailinLovebook,Nillesreview}

Let us consider a supersymmetric theory containing gauge and chiral
supermultiplets whose lagrangian may contain terms that are
nonrenormalizable.
It turns out that the part of the lagrangian containing terms up to two
spacetime derivatives is completely
determined by
specifying
three independent functions of the scalar fields (or 
equivalently,\footnote{The reader will lose nothing here by considering
them as functions of the scalar fields;
however, in a more sophisticated treatment some value would be lost.}
of the chiral
superfields).
They are:
\begin{itemize}
\item[$\bullet$]
The superpotential $W(\phi_i)$, which we have already
encountered in the case of renormalizable
supersymmetric lagrangians.
It must be an analytic function
of the superfields treated as complex variables; in other words it
depends only on the $\phi_i$ and not on the $\phi^{* i}$. It has
dimensions of (mass)$^3$.
\item[$\bullet$]
The {\em K\"ahler potential} $K(\phi_{i},\phi^{*i})$.
Unlike the superpotential, the K\"ahler potential is a function
of both $\phi_i$ and $\phi^{* i}$. It is real, and has dimensions of
(mass)$^2$. In the special case
of renormalizable theories, we did not have to discuss the
K\"ahler potential explicitly, because at tree-level there is only one
possibility for
it: $K = \phi^{i*}\phi_i$ (with the index $i$ summed over as usual).
\item[$\bullet$]
The {\it gauge kinetic function} $f_{ab}(\phi_i)$. Like the
superpotential, this
is an analytic function of the $\phi_i$ treated as complex
variables. It is
dimensionless and symmetric under interchange of its two
indices $a,b$, which run over the adjoint representations of the
gauge groups of the model. In the special case of renormalizable
supersymmetric lagrangians, it is just a constant (independent of the
$\phi_i$), and is equal to the identity matrix divided by the gauge
coupling squared:
$f_{ab} = \delta_{ab}/g_a^2$.
More generally, it also determines the nonrenormalizable couplings of
the gauge supermultiplets.
\end{itemize}
The whole lagrangian with up to two derivatives can now be written
down in terms of these functions.
This is a non-trivial consequence of supersymmetry,
because many different individual couplings in the lagrangian are
determined by the same three functions. 
This applies not only to theories
with an ultraviolet cutoff like supergravity, but also to effective
theories where one has integrated out
ultraviolet degrees of freedom.


For example,
in supergravity models the part of the scalar potential which does not
depend on the
gauge kinetic function can be found as follows.
First, one may define the
real, dimensionless ``K\"ahler function":
\beq
G = {K\over \MPlanck^2} + {\rm ln}{W\over \MPlanck^3} +
{\rm ln}{W^*\over \MPlanck^3}.
\eeq
(Just to maximize the confusion, $G$ is also sometimes referred to
as the K\"ahler potential.
%, and sometimes what we call $K$ is
%written as $-3 \MPlanck^2 {\rm ln}(-K/3\MPlanck^2)$, so one must be
%careful.
Also, many authors work in units with $\MPlanck=1$, which
simplifies the expressions but can slightly obscure the
correspondence with the global superymmetry limit of large $\MPlanck$.) From
$G$, one can construct its derivatives with respect to
the scalar fields and their complex conjugates:
$G^i = {\delta G/\delta \phi_i}$;
$G_i = {\delta G/\delta \phi^{* i}}$; and
$G_i^j = {\delta^2 G/\delta\phi^{* i}\delta\phi_j}$.
Note that $G_i^j$ really only depends on $K$. So using the same convention
in which raised (lowered) indices $i$ correspond to derivatives with
respect to $\phi_i$ ($\phi^{*i}$), we have $G_i^j = K_i^j/\MPlanck^2$,
which is sometimes called the K\"ahler metric.
The inverse of this matrix is denoted
$(G^{-1})_i^j$, or equivalently
$\MPlanck^2
(K^{-1})_i^j$,
so that
$(G^{-1})^k_i G_k^j =
(G^{-1})^j_k G_i^k = \delta_i^j$.
In terms of these objects,
the direct generalization of the $F$-term contribution
to the scalar potential in ordinary renormalizable global
supersymmetry turns out to be, after a complicated
derivation: \cite{supergravity1,supergravity2}
\beq
V = \MPlanck^4 \, e^G \Bigl [ G^i (G^{-1})_i^j G_j -3 \Bigr ]
\label{vsugra}
\eeq
in supergravity. It can be rewritten in a slightly less compact form:
\beq
V = e^{K/\MPlanck^2} \left [
(K^{-1})_j^i
\Bigl (W^i + {1\over\MPlanck^2} W K^i \Bigr )
\Bigl (W^*_j + {1\over\MPlanck^2} W^* K_j \Bigr )
 - {3\over \MPlanck^2} W W^*\right ]
\label{compactvsugra}
\eeq
where $K^i = \delta K/\delta \phi_i$ and $K_j = \delta K/\delta
\phi^{*j}$.
The order parameters for supersymmetry breaking (analogous to the
auxiliary fields in the renormalizable, global supersymmetry
case) turn out to be
\beq
F_i \, =\, -\MPlanck^2\, e^{G/2} \,
(G^{-1})_i^j
G_j \,=\, -e^{K/2 \MPlanck^2}\,
(K^{-1})_i^j
\Bigl ( W^*_j + {1\over \MPlanck^2} W^* K_j \Bigr )
\label{fisugra}
\eeq
in supergravity. In other words, local supersymmetry will be broken if
one or more of the $F_i$ obtain a VEV. The
gravitino then absorbs the would-be goldstino and obtains a mass given by
\beq
m^2_{3/2} = {1\over 3\MPlanck^2}\langle K_j^i F_i F^{*j}\rangle.
\label{sugragravitinomass}
\eeq
Now if one assumes a ``minimal" K\"ahler potential
$ K = \phi^{*i} \phi_i $,
then $K_i^j=(K^{-1})_i^j = \delta_i^j$,
so that expanding eqs.~(\ref{compactvsugra}) and (\ref{fisugra})
to lowest order in $1/\MPlanck$ just reproduces
the results $F_i = -W^*_i$ and
$V = W^i W_i^*$ which were found in section
\ref{subsec:susylagr.chiral}
for renormalizable global supersymmetric theories [see
eqs.~(\ref{replaceF})-(\ref{ordpot})]. Equation~(\ref{sugragravitinomass})
also reproduces
the expression for the gravitino mass that was quoted in
eq.~(\ref{gravitinomass}).

The scalar potential
eq.~(\ref{vsugra}) does not yet include contributions from gauge
interactions. The
$D$-term contributions to the scalar potential are given by
\beq
V = {1\over 2}{\rm Re}\,f^{-1}_{ab}\, {\widehat D}^a
{\widehat D}^b;\qquad\>\>\> {\widehat D}^a = -K^i
(T^a)_i^j \phi_j,\qquad\>\>{}
\eeq
where ${\rm Re}\,f^{-1}_{ab}$ is the inverse of the real part of the
gauge kinetic function matrix. In the case that $f_{ab} =
\delta_{ab}/g_a^2$ and
$K^i = \phi^{*i}$, this just reproduces the result of section
\ref{subsec:susylagr.gaugeinter} for the renormalizable
global supersymmetry scalar potential, with $\widehat{D}^a = D^a/g^a$
being the
$D$-term order parameter for supersymmetry breaking.
If supersymmetry breaking takes place through $F$-term breaking,
then it is often not necessary to include the supergravity effects on the
$D$-terms
explicitly. There are also many contributions to the lagrangian
other than the scalar potential which depend on the three functions
$W$, $K$ and $f_{ab}$, which can be found in Ref.\cite{supergravity2}

It should be noted that unlike in the case of global supersymmetry, the
scalar potential in supergravity is {\it not}
necessarily non-negative, because of the $-3$ term in
eq.~(\ref{vsugra}). This means that in principle, one can have
supersymmetry breaking with a positive, negative, or zero vacuum energy.
The last option might seem to be preferred phenomenologically by the
absence of a cosmological constant, although it is not clear why
the terms in the scalar potential should conspire to
have $\langle V \rangle = 0$ at the minimum. Furthermore, it
is not at all clear that $\langle V
\rangle = 0$ really corresponds to
the requirement of a vanishing observable, quantum-corrected cosmological
constant.\cite{cosmock}
In any case, with $\langle V \rangle = 0$ imposed as a
constraint,\footnote{We do this only to follow a popular example; as just
noted we cannot endorse this imposition.}
eqs.~(\ref{compactvsugra})-(\ref{sugragravitinomass}) tell us that
$ \langle K_j^i F_i F^{*j} \rangle = 3 \MPlanck^4 e^{\langle G \rangle}
=
3 e^{\langle K \rangle/\MPlanck^2} |\langle W \rangle|^2/\MPlanck^2$,
and an equivalent formula for the gravitino mass is therefore
$m_{3/2} = e^{\langle G\rangle/2} \MPlanck$.

An instructive special case arises if we assume a ``minimal"
K\"ahler potential and divide the fields
$\phi_i$ into a visible sector including the MSSM fields
$\varphi_i$ and a
hidden sector containing a field $X$ which breaks supersymmetry for
us (and other fields that we need not treat explicitly).
In other words,
suppose that the superpotential and the K\"ahler potential have
the form
\beq
W &=& W_{\rm vis}(\varphi_i) + W_{\rm hid}(X);
\label{minw}\\
K &=& \varphi^{*i} \varphi_i + X^* X .
\label{mink}
\eeq
Now let us further assume that the dynamics of the hidden sector fields
gives rise to non-zero VEVs
\beq
\langle X \rangle = x \MPlanck;\qquad
\langle W_{\rm hid}\rangle = w \MPlanck^2;\qquad
\langle \delta W_{\rm hid}/\delta X \rangle = w^\prime \MPlanck .
\eeq
which defines
a dimensionless quantity $x$ and
$w$, $w^\prime$
with dimensions of (mass).
Requiring $\langle V \rangle = 0$ yields
$|w^\prime + x^* w|^2 = 3 |w|^2$, and
\beq
m_{3/2} = {|\langle F_X \rangle | \over \sqrt{3} \MPlanck} =
e^{|x|^2/2} |w|.
\eeq
Now we suppose that it is valid
to expand the scalar potential in powers of the dimensionless
quantities $w/\MPlanck$, $w^\prime/\MPlanck$, $\varphi_i/\MPlanck$, etc.,
keeping only terms that depend on the visible sector fields $\varphi_i$.
It is not a difficult exercise to
show that in leading order the result is:
\beq
V &=& (W^*_{\rm vis})_i (W_{\rm vis})^i + m_{3/2}^2
\varphi^{*i}\varphi_{i}
\nonumber \\ && \!\!
+ e^{|x|^2/2} \left [w^* \varphi_i (W_{\rm vis})^i\, +\,
(x^* w^{\prime *} + |x|^2 w^* - 3 w^*) W_{\rm vis} + \conj \right
].\qquad{}
\label{yapot}
\eeq
A tricky point here is that we have rescaled the visible sector
superpotential $W_{\rm vis} \rightarrow e^{-|x|^2/2} W_{\rm vis}$
everywhere, in order that
the first term in eq.~(\ref{yapot}) is the usual, properly normalized,
$F$-term contribution in global supersymmetry.
The next term is a universal soft scalar (mass)$^2$
of the form eq.~(\ref{scalarunificationsugra})
with
\beq
m_0^2 \,=\, {|\langle F_X \rangle|^2\over 3 \MPlanck^2}\,=\,
m_{3/2}^2.
\eeq
The second line of eq.~(\ref{yapot}) just yields soft
(scalar)$^3$ and (scalar)$^2$ analytic couplings of the form
eqs.~(\ref{aunificationsugra}) and (\ref{bsilly}), with
\beq
A_0 \,=\, -{\langle F_X \rangle \over \MPlanck} x^*
;\qquad\>\>
B_0 \,=\, {\langle F_X \rangle \over \MPlanck}\Bigl (
-x^* +{1\over x + w^{\prime *}/w^*}\Bigr )
\qquad{}
\label{a0b0x}
\eeq
since $\varphi_i (W_{\rm vis})^i$ is equal to $3 W_{\rm vis}$ for the
cubic part of $W_{\rm vis}$, and to $2 W_{\rm vis}$
for the quadratic part. [If the complex phases of $x$, $w$, $w^\prime$
can be rotated away,
then eq.~(\ref{a0b0x}) implies $B_0 = A_0 - m_{3/2}$, but there
are many effects which can ruin this prediction.]
The Polonyi model mentioned in section \ref{subsec:origins.sugra}
is just the special case of this exercise in which $W_{\rm hid}$ is
assumed
to be linear in $X$.

However, there is no particular reason why $W$ and $K$ must have
the simple form eq.~(\ref{minw}) and eq.~(\ref{mink}).
Furthermore, we have not yet explained how gaugino masses arise from
nonrenormalizable terms. This
requires a non-minimal gauge kinetic function $f_{ab}$.
If the gauge kinetic function can be expanded in powers
of $1/\MPlanck$ as
\beq
f_{ab} = \delta_{ab}\Bigl [{1\over g_a^2} + {1\over \MPlanck} f^i_{a}
\phi_i +
\ldots \Bigr ],
\label{uglyfab}
\eeq
then it is possible to show that the gaugino mass
induced by supersymmetry breaking is
\beq
m_{\lambda^a} =
{1\over 2\MPlanck}
{\rm Re}[f_{a}^i]
{\langle F_i \rangle} .
\label{sugragauginomasses}
\eeq
The assumption of
universal gaugino masses therefore follows if
the dimensionless quantities $f_a^i$
are the same for each of the three MSSM gauge groups; this can be
automatic in certain GUT and superstring models.
Similarly, the superpotential can be expanded with the schematic form
\beq
W = W_{\rm ren} + {1\over \MPlanck} \phi^4 +
{1\over \MPlanck^2} \phi^5 + \ldots
\label{uglyw}\eeq
where $W_{\rm ren}$ is the renormalizable superpotential
with terms up to $\phi^3$.
It may also be possible to expand the
K\"ahler potential like
\beq
K = \phi_i \phi^{* i} + {1\over \MPlanck} \bigl (
\phi^{* 3} + \phi^{* 2}\phi +\phi^* \phi^2 +  \phi^{3} \bigr ) +\ldots,
\label{uglyk}
\eeq
If one now plugs eqs.~(\ref{uglyw}) and
(\ref{uglyk}) with arbitrary
hidden sector fields and VEVs into eq.~(\ref{vsugra}), one obtains
a general form like eq.~(\ref{hiddengrav}) for the soft
terms. It is only when special
assumptions are made [like eqs.~(\ref{minw}),(\ref{mink})]
that one gets the
phenomenologically desirable results in
eqs.~(\ref{sillyassumptions})-(\ref{bsilly}).
This is why it is often said that supergravity by itself does not
guarantee universality of the soft terms.
Furthermore, there is no guarantee that expansions in
$1/\MPlanck$ of the form given above are valid or appropriate.
%In ``no-scale" models, the K\"ahler potential
%is assumed to contain terms of the form $K= -3 M_{\rm P}^2{\rm
%ln}[(\phi+\phi^*)/\MPlanck]$.
In superstring models,
the ``dilaton" and ``moduli" fields have 
K\"ahler potential terms proportional to $M_{\rm P}^2{\rm
ln}[(\phi+\phi^*)/\MPlanck]$.
(The moduli are massless fields which do not appear in the
tree-level perturbative superpotential. The
dilaton is a special modulus field whose VEV determines the gauge
couplings
in the theory.)

Finally, let us mention how gaugino condensates can give rise
to supersymmetry breaking in supergravity models. This requires that the
gauge
kinetic function has a non-trivial dependence on the scalar fields, as in
eq.~(\ref{uglyfab}).
Then
eq.~(\ref{fisugra}) is modified to
\beq
F_i \, =\, -\MPlanck^2\, e^{G/2} \,
(G^{-1})_i^j G_j - {1\over 4}(K^{-1})_i^j
{\partial f_{ab}\over \partial \phi_j}
\lambda^a \lambda^b
+ \ldots .
\eeq
Now if there is a gaugino condensate $\langle \lambda^a \lambda^b \rangle
= \delta ^{ab} \Lambda^3$
and $\langle (K^{-1})_i^j{\partial f_{ab}/ \partial \phi_j}
\rangle\sim 1/\MPlanck$,
then $\langle F_i \rangle \sim\Lambda^3/\MPlanck$. Then as above, the
non-vanishing
$F$-term gives rise to soft parameters of order $m_{\rm soft} \sim
\langle F_i \rangle/\MPlanck \sim \Lambda^3/\MPlanck^2$, as
in eq.~(\ref{foofighters}).

\addcontentsline{toc}{section}{References}
\section*{References}

\begin{thebibliography}{99}

\bibitem{hierarchyproblem} S.~Weinberg, \Journal{\PRD}{13}{974}{1976};
\Journal{\PRD}{19}{1277}{1979};
L.~Susskind, \Journal{\PRD}{20}{2619}{1979};
G.~'t Hooft, in {\it Recent developments in gauge theories},
Proceedings of the NATO Advanced Summer Institute, Cargese 1979,
ed. G.~'t Hooft {\it et al.} (Plenum, New York 1980).

\bibitem{quadscancel} E.~Witten, \Journal{\NP}{B188}{513}{1981};
N.~Sakai, \Journal{\ZPC}{11}{153}{1981};
S.~Dimopoulos and H.~Georgi, \Journal{\NP}{B193}{150}{1981};
R.K.~Kaul and P.~Majumdar, \Journal{\NP}{B199}{36}{1982}.

\bibitem{HLS} S.~Coleman and J.~Mandula, \Journal{\PR}{159}{1251}{1967};
R.~Haag, J.~Lopuszanski, and M.~Sohnius, \Journal{\NP}{B88}{257}{1975}.

\bibitem{FayetHsnu} P.~Fayet,
\Journal{\PLB}{64}{159}{1976}.

\bibitem{FayetMSSM} P.~Fayet,
\Journal{\PLB}{69}{489}{1977};
\Journal{\PLB}{84}{416}{1979}.

\bibitem{Rparity} G.R.~Farrar and P.~Fayet,
\Journal{\PLB}{76}{575}{1978}.

\bibitem{RNS} P.~Ramond, \Journal{\PRD}{3}{2415}{1971};
A.~Neveu and J.H.~Schwarz, \Journal{\NP}{B31}{86}{1971};
J.L. Gervais and B.~Sakita, \Journal{\NP}{B34}{632}{1971}.

\bibitem{Golfand} Yu.~A.~Gol'fand and E.~P.~Likhtman,
\Journal{\JETPLett}{13}{323}{1971}.

\bibitem{WessZumino} J.~Wess and B.~Zumino,
\Journal{\NP}{B70}{39}{1974}.

\bibitem{Volkov} D.V.~Volkov and V.P.~Akulov,
\Journal{\PLB}{46}{109}{1973}.

\bibitem{WessBaggerbook} J.~Wess and J.~Bagger,
{\em Supersymmetry and Supergravity},
(Princeton University Press, Princeton NJ, 1992).

\bibitem{Rossbook} G.G.~Ross, {\em Grand Unified Theories},
(Addison-Wesley, Redwood City CA, 1985).

\bibitem{Srivastavabook} P.P.~Srivastava, {\em Supersymmetry and
superfields}, (Adam-Hilger, Bristol England, 1986).

\bibitem{Freundbook}
P.G.O.~Freund, {\em Introduction to Supersymmetry},
(Cambridge University Press, Cambridge England, 1986).

\bibitem{Westbook} P.C.~West, {\em Introduction to Supersymmetry
and Supergravity}, (World Scientific, Singapore, 1990).

\bibitem{Mohapatrabook}
R.N.~Mohapatra,
{\em Unification and Supersymmetry: The Frontiers of
Quark-Lepton Physics}, Springer-Verlag, New York 1992.

\bibitem{BailinLovebook} D.~Bailin and A.~Love,
{\em Supersymmetric Gauge Field Theory and String Theory},
(Institute of Physics Publishing, Bristol England, 1994).

\bibitem{Ramondbook} P.~Ramond, {\em Beyond the Standard Model},
Frontiers in Physics series, Addison-Wesley, to appear.

\bibitem{HaberKanereview} H.E.~Haber and G.L.~Kane,
\Journal{\PRP}{117}{75}{1985}.

\bibitem{Nillesreview} H.P.~Nilles, \Journal{\PRP}{110}{1}{1984}.

\bibitem{1001}
{\em Superspace or One Thousand and One Lessons in Supersymmetry},
S.J.
Gates, M.T. Grisaru, M. Rocek and W.~Siegel,
Benjamin/Cummings 1983.

\bibitem{ACNreview} R.~Arnowitt, A.~Chamseddine and P.~Nath,
{\em N=1 Supergravity}, World Scientific, Singapore, 1984.

\bibitem{Jonesreview} D.R.T.~Jones, ``Supersymmetric gauge theories",
in {\em TASI Lectures in Elementary Particle Physics 1984},
ed.~D.N.~Williams, TASI publications, Ann Arbor 1984.

\bibitem{HaberTASI} H.E.~Haber, ``Introductory low-energy
supersymmetry", TASI-92 lectures, in {\em
Recent Directions in Particle Theory},
eds.~J.~Harvey and J.~Polchinski, World Scientific, 1993.

\bibitem{Ramondreview} P.~Ramond, ``Introductory lectures on low-energy
supersymmetry", TASI-94 lectures, hep-th/9412234.

\bibitem{Baggerreview} J.A.~Bagger,
``Weak-scale supersymmetry: theory and practice",
 TASI-95 lectures,
hep-ph/9604232.

\bibitem{DPFpheno} H.~Baer et al.,
``Low energy supersymmetry phenomenology",
hep-ph/9503479, in {\em Report of the Working Group on Electroweak
Symmetry Breaking and New Physics} of the 1995 study of the future of
particle physics in the USA, to be published by World Scientific.

\bibitem{DPFtheory} M.~Drees and S.P.~Martin,
``Implications of SUSY model building", as in Ref.\cite{DPFpheno}

\bibitem{Lykkenreview} J.D.~Lykken, ``Introduction to Supersymmetry",
TASI-96 lectures, hep-th/9612114

\bibitem{Dawsonreview} S.~Dawson, ``SUSY and such",
Lectures given at NATO
Advanced Study Institute on Techniques and Concepts of High-energy
Physics,
hep-ph/9612229.

\bibitem{Dinereview} M.~Dine, ``Supersymmetry Phenomenology
(With a Broad Brush)", hep-ph/9612389.

\bibitem{Gunionreview} J.F.~Gunion, ``A simplified summary of
supersymmetry", to appear in {\em Future High Energy Colliders},
%proceedings of the ITP Symposium,
AIP Press, hep-ph/9704349.

\bibitem{Shifmanreview} M.~Shifman, ``Non-perturbative dynamics
in supersymmetric gauge theories", hep-ph/9704114.

\bibitem{Tatareview}
X.~Tata, ``What is supersymmetry and how do we find it?",
Lectures presented at the IX Jorge A.~Swieca Summer School,
Campos do Jord\~ao, Brazil, hep-ph/9706307.

\bibitem{reprints} S.~Ferrara, editor, {\em Supersymmetry},
(World Scientific, Singapore, 1987).

\bibitem{superfields} A.~Salam and J.~Strathdee,
\Journal{\NP}{B76}{477}{1974};  S.~Ferrara, J.~Wess and B.~Zumino,
\Journal{\PLB}{51}{239}{1974}.  See Ref.\cite{WessBaggerbook} for a
pedagogical introduction to the superfield formalism.

\bibitem{ref:supercurrent} J.~Wess and B.~Zumino,
\Journal{\PLB}{49}{52}{1974}; J.~Iliopoulos and B.~Zumino,
\Journal{\NP}{B76}{310}{1974}.

\bibitem{WZgauge} J.~Wess and B.~Zumino,
\Journal{\NP}{B78}{1}{1974}.

\bibitem{softterms} L.~Girardello and M.T.~Grisaru
\Journal{\NP}{B194}{65}{1982}.

\bibitem{cterms} See, however, L.J.~Hall and L.~Randall,
\Journal{\PRL}{65}{2939}{1990}.

\bibitem{muproblemW}
J.E.~Kim and H.~P.~Nilles, \Journal{\PLB}{138}{150}{1984}
J.E.~Kim and H.~P.~Nilles, \Journal{\PLB}{263}{79}{1991};
E.J.~Chun, J.E.~Kim and H.P.~Nilles, \Journal{\NP}{B370}{105}{1992}.

\bibitem{muproblemK}
G.F.~Giudice and A.~Masiero, \Journal{\PLB}{206}{480}{1988};
J.A.~Casas and C.~Mu\~noz, \Journal{\PLB}{306}{288}{1993}.

\bibitem{muproblemGMSB} G.~Dvali, G.F.~Giudice and A.~Pomarol,
\Journal{\NP}{B478}{31}{1996}.

\bibitem{rparityconstraints}
See, for example,
F.~Zwirner, \Journal{\PLB}{132}{103}{1983};
R.~Barbieri and A.~Masiero, \Journal{\NP}{B267}{679}{1986};
S.~Dimopoulos and L.~Hall, \Journal{\PLB}{207}{210}{1987};
V.~Barger, G.~Giudice, and T.~Han, \Journal{\PRD}{40}{2987}{1989};
R.~Godbole, P.~Roy and X.~Tata, \Journal{\NP}{B401}{67}{1992};
G.~Bhattacharya and D.~Choudhury, \Journal{\MPL}{A10}{1699}{1995};
G.~Bhattacharya, ``$R$-parity-violating supersymmetric
Yukawa couplings: a Mini-review", hep-ph/9608415,  in {\em Supersymmetry
'96}, ed. R.N.~Mohapatra and A.~Rasin;
H.~Dreiner, ``An Introduction to explicit $R$-parity violation",
hep-ph/9707435,
\perspectives.

\bibitem{matterparity} S.~Dimopoulos and H.~Georgi,
\Journal{\NP}{B193}{150}{1981};
S.~Weinberg, \Journal{\PRD}{26}{2878}{1982};
N.~Sakai and T.~Yanagida, \Journal{\PRD}{B197}{533}{1982};
S.~Dimopoulos, S.~Raby and F.~Wilczek, \Journal{\PLB}{112}{133}{1982}.

\bibitem{neutralinodarkmatter}
H.~Goldberg, \Journal{\PRL}{50}{1419}{1983};
J.~Ellis, J.~Hagelin, D.V.~Nanopoulos, K.~Olive, and M.~Srednicki,
\Journal{\NP}{B238}{453}{1984}.

\bibitem{KW}
L.~Krauss and F.~Wilczek, \Journal{\PRL}{62}{1221}{1989}.

\bibitem{discreteanomaly} L.E.~Ib\'a\~nez and G.~Ross,
\Journal{\PLB}{260}{291}{1991};
T.~Banks and M.~Dine, \Journal{\PRD}{45}{1424}{1995};
L.E.~Ib\'a\~nez, \Journal{\NP}{B398}{301}{1993}.

\bibitem{Rparityorigin1}
R.N.~Mohapatra, \Journal{\PRD}{34}{3457}{1986};
A.~Font, L.E.~Ib\'a\~nez and F.~Quevedo,
\Journal{\PLB}{228}{79}{1989};
S.P.~Martin
\Journal{\PRD}{46}{2769}{1992}; \Journal{\PRD}{54}{2340}{1996}.

\bibitem{Rparityorigin2}
R.~Kuchimanchi and R.N.~Mohapatra, \Journal{\PRD}{48}{4352}{1993};
\Journal{\PRL}{75}{3989}{1995};
C.S.~Aulakh, K.~Benakli and G.~Senjanovi\'c, hep-ph/9703434;
\Journal{\PRL}{79}{2188}{1997};
C.S.~Aulakh, A.~Melfo and G.~Senjanovi\'c, \Journal{\PRD}{57}{4174}{1998}.

\bibitem{dimsut} S.~Dimopoulos and D.~Sutter,
\Journal{\NP}{B452}{496}{1995}.

\bibitem{flavorreview} For a comprehensive analysis, see F.~Gabbiani,
E.~Gabrielli,
A.~Masiero and L.~Silvestrini, \Journal{\NP}{B477}{321}{1996}
and references therein.

\bibitem{muegamma} L.J.~Hall, V.A.~Kostalecky and S.~Raby,
\Journal{\NP}{B267}{415}{1986}; F.~Gabbiani and A.~Masiero,
\Journal{\PLB}{209}{289}{1988}; R.~Barbieri and L.J.~Hall,
\Journal{\PLB}{338}{212}{1994}.

\bibitem{mixwinoex}
J.~Ellis and D.V.~Nanopoulos, \Journal{\PLB}{110}{44}{1982};
R.~Barbieri and R.~Gatto, \Journal{\PLB}{110}{343}{1982};
B.A.~Campbell, \Journal{\PRD}{28}{209}{1983}.

\bibitem{morestuff} M.J.~Duncan, \Journal{\NP}{B221}{285}{1983};
J.F.~Donahue, H.P.~Nilles and D.~Wyler, \Journal{\PLB}{128}{55}{1983};
A.~Bouquet, J.~Kaplan and C.A.~Savoy, \Journal{\PLB}{148}{69}{1984};
M.~Dugan, B.~Grinstein and L.J.~Hall, \Journal{\NP}{B255}{413}{1985};
F.~Gabbiani and A.~Masiero, \Journal{\NP}{B322}{235}{1989};
J.~Hagelin, S.~Kelley and T.~Tanaka, \Journal{\NP}{B415}{293}{1994}.

\bibitem{bsgamma}
S.~Bertolini, F.~Borzumati, A.~Masiero and G.~Ridolfi,
\Journal{\NP}{B353}{591}{1991};
R.~Barbieri and G.F.~Giudice,
\Journal{\PLB}{309}{86}{1993};
J.~Hewett and J.D.~Wells, \Journal{\PRD}{55}{5549}{1997}.

\bibitem{demon}
J.~Ellis, S.~Ferrara and D.V.~Nanopoulos, \Journal{\PLB}{114}{231}{1982};
W.~Buchm\"uller and D.~Wyler, \Journal{\PLB}{121}{321}{1983};
J.~Polchinski and M.B.~Wise, \Journal{\PLB}{125}{393}{1983};
F.~del Aguila, M.B.~Gavela, J.A.~Grifols and A.~M\'endez,
\Journal{\PLB}{126}{71}{1983};
D.V.~Nanopoulos and M.~Srednicki, \Journal{\PLB}{128}{61}{1983}.

\bibitem{gaugeunification} P.~Langacker, in Proceedings of the
PASCOS90 Symposium, Eds.~P.~Nath
and S.~Reucroft, (World Scientific, Singapore 1990)
J.~Ellis, S.~Kelley, and D.~Nanopoulos, \Journal{\PLB}{260}{131}{1991};
U.~Amaldi, W.~de Boer, and H.~Furstenau, \Journal{\PLB}{260}{447}{1991};
P.~Langacker and M.~Luo, \Journal{\PRD}{44}{817}{1991};
C.~Giunti, C.W.~Kim and U.~W.~Lee, \Journal{\MPL}{A6}{1745}{1991}.

\bibitem{Moreminimal}
A.G.~Cohen, D.B.~Kaplan and A.E.~Nelson,
\Journal{\PLB}{388}{588}{1996}.

\bibitem{alignmentmodels}
Y.~Nir and N.~Seiberg, \Journal{\PLB}{309}{337}{1993}.

\bibitem{FayetIliopoulos} P.~Fayet and J.~Iliopoulos,
\Journal{\PLB}{51}{461}{1974}; P.~Fayet, \Journal{\NP}{B90}{104}{1975}.

\bibitem{dtermbreakingmaywork} However, see for example
P.~Bin\'etruy and  E.~Dudas, \Journal{\PLB}{389}{503}{1996};
G.~Dvali and A.~Pomarol, \Journal{\PRL}{77}{3728}{1996};
R.N.~Mohapatra and A.~Riotto, \Journal{\PRD}{55}{4262}{1997}.
A non-zero Fayet-Iliopoulos term for an anomalous $U(1)$ symmetry
is commonly found in superstring models:
M.~Green and J.~Schwarz, \Journal{\PLB}{149}{117}{1984};
M.~Dine, N.~Seiberg and E.~Witten, \Journal{\NP}{B289}{589}{1987};
J.~Atick, L.~Dixon and A.~Sen, \Journal{\NP}{B292}{109}{1987};
This may even help to explain the
observed structure of the Yukawa couplings:
L.E.~Ib\'a\~nez, \Journal{\PLB}{303}{55}{1993};
L.E.~Ib\'a\~nez and G.G.~Ross,
\Journal{\PLB}{332}{100}{1994};
P.~Bin\'etruy, S.~Lavignac, P.~Ramond,
\Journal{\NP}{B477}{353}{1996};
P.~Bin\'etruy, N.Irges, S.~Lavignac and P.~Ramond,
\Journal{\PLB}{403}{38}{1997};
N.~Irges, S.~Lavignac, P.~Ramond,
\Journal{\PRD}{58}{035003}{1998}.
%``Predictions from an Anomalous U(1) Model of Yukawa Hierarchies",
%hep-ph/9802334.

\bibitem{ORaifeartaigh}
L.~O'Raifeartaigh, \Journal{\NP}{B96}{331}{1975}.

\bibitem{ColemanWeinberg} S.~Coleman and E.~Weinberg,
\Journal{\PRD}{7}{1888}{1973}.

\bibitem{dynamicalsusybreaking}
E.~Witten, \Journal{\NP}{B202}{253}{1982};
I.~Affleck, M.~Dine and N.~Seiberg, \Journal{\NP}{B241}{493}{1981};
\Journal{\NP}{B256}{557}{1986}.
For reviews, see
L.~Randall, ``Models of Dynamical Supersymmetry Breaking", hep-ph/9706474;
A.~Nelson, ``Dynamical Supersymmetry Breaking", hep-ph/9707442;
G.F.~Giudice and R.~Rattazzi, ``Theories with Gauge-Mediated Supersymmetry
Breaking", hep-ph/9801271.

\bibitem{Fayetsupercurrent} P.~Fayet, \Journal{\PLB}{70}{461}{1977};
\Journal{\PLB}{86}{272}{1979}; and in {\em Unification of the fundamental
particle interactions} (Plenum, New York, 1980).

\bibitem{eeGMSBsignal} N.~Cabibbo, G.R.~Farrar and L.~Maiani,
\Journal{\PLB}{105}{155}{1981}; M.K.~Gaillard, L.~Hall and
I.~Hinchliffe, \Journal{\PLB}{116}{279}{1982};
J.~Ellis and J.S.~Hagelin, \Journal{\PLB}{122}{303}{1983},
D.A.~Dicus, S.~Nandi and J.~Woodside, \Journal{\PLB}{258}{231}{1991};
D.R.~Stump, M.~Wiest, and C.P.~Yuan, \Journal{\PRD}{54}{1936}{1996};
S.~Ambrosanio, G.~Kribs, and S.P.~Martin, \Journal{\PRD}{56}{1761}{1997}.

\bibitem{DDRT} S.~Dimopoulos, M.~Dine, S.~Raby and S.Thomas,
\Journal{\PRL}{76}{3494}{1996}; S.~Dimopoulos, S.~Thomas and J.~D.~Wells,
\Journal{\PRD}{54}{3283}{1996}.

\bibitem{AKKMM2} S.~Ambrosanio et al.,
\Journal{\PRL}{76}{3498}{1996}; 
\Journal{\PRD}{54}{5395}{1996}.

\bibitem{supergravity1} S.~Ferrara, D.Z.~Freedman and P.~van
Nieuwenhuizen,
\Journal{\PRD}{13}{3214}{1976}; S.~Deser and B.~Zumino,
\Journal{\PLB}{62}{335}{1976}; D.Z.~Freedman and  P.~van
Nieuwenhuizen, \Journal{\PRD}{14}{912}{1976};
E.~Cremmer et al., \Journal{\NP}{B147}{105}{1979};
J.~Bagger, \Journal{\NP}{B211}{302}{1983}.

\bibitem{supergravity2}
E.~Cremmer, S.~Ferrara, L.~Girardello, and A.~van Proeyen,
\Journal{\NP}{B212}{413}{1983}.

\bibitem{gravitinomassref} S.~Deser and B.~Zumino,
\Journal{\PRL}{38}{1433}{1977};
E.~Cremmer et al., \Journal{\PLB}{79}{1978}{231}.

\bibitem{cosmogravitino}
H.~Pagels, J.R.~Primack, \Journal{\PRL}{48}{1982}{223};
T.~Moroi, H.~Murayama, M.~Yamaguchi, \Journal{\PLB}{303}{1993}{289}.

\bibitem{PP}
P.~Moxhay and K.~Yamamoto, \Journal{\NP}{B256}{130}{1985};
K.~Grassie, \Journal{\PLB}{159}{32}{1985};
B.~Gato, \Journal{\NP}{B278}{189}{1986};
N.~Polonsky and A.~Pomarol, \Journal{\PRL}{73}{2292}{1994}.

\bibitem{samplespectra}
G.G.~Ross and R.G.~Roberts,
\Journal{\NP}{B377}{571}{1992}.
H.~Arason et al., \Journal{\PRL}{67}{2933}{1991};
R.~Arnowitt and P.~Nath,
\Journal{\PRL}{69}{725}{1992} and \Journal{\PRD}{46}{3981}{1992};
D.J.~Casta\~no, E.J.~Piard, P.~Ramond,
\Journal{\PRD}{49}{4882}{1994};
V.~Barger, M.~Berger and P.~Ohmann, \Journal{\PRD}{49}{4908}{1994};
G.L.~Kane, C.~Kolda, L.~Roszkowski, J.D.~Wells,
\Journal{\PRD}{49}{6173}{1994};
B.~Ananthanarayan, K.S.~Babu and Q.~Shafi, \Journal{\NP}{B428}{19}{1994};
M. Carena, M. Olechowski, S. Pokorski, C.E.M.~Wagner,
\Journal{\NP}{B419}{213}{1994};
W.~de Boer, R.~Ehret and D.~Kazakov, \Journal{\ZPC}{67}{647}{1995};
M.~Carena, P.~Chankowski, M.~Olechowski, S.~Pokorski, C.E.M.~Wagner,
\Journal{\NP}{B491}{103}{1997}.

\bibitem{dilatondominated}
V.~Kaplunovsky and J.~Louis, \Journal{\PLB}{306}{269}{1993};
R.~Barbieri, J.~Louis and M.~Moretti, \Journal{\PLB}{312}{451}{1993};
A.~Brignole, L.E.~Ib\'a\~nez and  C.~Mu\~noz,
\Journal{\NP}{B422}{125}{1994}, erratum \Journal{\NP}{B436}{747}{1995}.

\bibitem{polonyi} J.~Polonyi, Hungary Central Research Institute
report KFKI-77-93 (1977) (unpublished). See Ref.\cite{BailinLovebook}
for a pedagogical account.

\bibitem{noscale} For a review, see
A.B.~Lahanas and D.V.~Nanopoulos, \Journal{\PRP}{145}{1}{1987}.

\bibitem{stringsoft} For a review, see A.~Brignole, L.E.~Ib\'a\~nez and
C.~Mu\~noz, ``Soft supersymmetry-breaking terms from supergravity
and supersring models",
hep-ph/9707209,
\perspectives.

\bibitem{oldgmsb} M.~Dine and W.~Fischler,
\Journal{\PLB}{110}{227}{1982}; C.R.~Nappi and B.A.~Ovrut,
\Journal{\PLB}{113}{175}{1982}; L.~Alvarez-Gaum\'e, M. Claudson,
and M.~B.~Wise, \Journal{\NP}{B207}{96}{1982}.

\bibitem{newgmsb} M.~Dine, A.~E.~Nelson,
\Journal{\PRD}{48}{1277}{1993};
M.~Dine, A.E.~Nelson, Y.~Shirman, \Journal{\PRD}{51}{1362}{1995};
M.~Dine, A.E.~Nelson, Y.~Nir, Y.~Shirman, \Journal{\PRD}{53}{2658}{1996}.

\bibitem{gmsbcorrections} S.~Dimopoulos, G.F.~Giudice and
A.~Pomarol, \Journal{\PLB}{389}{37}{1996};
S.P.~Martin \Journal{\PRD}{55}{3177}{1997};
E.~Poppitz and S.P.~Trivedi, \Journal{\PLB}{401}{38}{1997}.

\bibitem{DRED} W.~Siegel, \Journal{\PLB}{84}{193}{1979};
D.M.~Capper, D.R.T.~Jones and P.~van~Nieuwenhuizen,
\Journal{\NP}{B167}{479}{1980}.

\bibitem{Shifman} V.~Novikov, M.~Shifman, A.~Vainshtein and
V.~Zakharov, \Journal{\NP}{B229}{381}{1983};
\Journal{\PLB}{166}{329}{1986}; J.~Hisano and M.~Shifman,
\Journal{\PRD}{56}{5475}{1997}.
%``Exact results for soft supersymmetry breaking parameters in
%supersymmetric gauge theories", hep-ph/9705417

\bibitem{DREDdies} W.~Siegel, \Journal{\PLB}{94}{37}{1980};
L.V.~Avdeev, G.A.~Chochia and A.A.~Vladimirov, 
\Journal{\PLB}{105}{272}{1981};
L.V.~Avdeev and A.A.~Vladimirov, \Journal{\NP}{B219}{262}{1983}.

\bibitem{JJperspective} I.~Jack and D.R.T.~Jones, ``Regularisation
of supersymmetric theories", hep-ph/9707278,
\perspectives.

\bibitem{Woodard} D.~Evans, J.W.~Moffat, G.~Kleppe, and R.P.~Woodard,
\Journal{\PRD}{43}{499}{1991};
G.~Kleppe and R.P.~Woodard, \Journal{\PLB}{253}{331}{1991};
\Journal{\NP}{B388}{81}{1992};
G.~Kleppe, \Journal{\PLB}{256}{431}{1991}.

%\bibitem{mstodrone} I.~Antoniadis, C.~Kounnas and K.~Tamvakis,
%\Journal{\PLB}{119}{377}{1982}; G.A.~Schuler, S.~Sakakibara and 
%J.G. Korner, \Journal{\PLB}{194}{125}{1987}; Y.~Yamada,
%\Journal{\PLB}{316}{109}{1993}.

\bibitem{gluinopolemass} S.P.~Martin and M.T.~Vaughn,
\Journal{\PLB}{318}{331}{1993}.

\bibitem{mstodrmore} I.~Jack, D.R.T.~Jones and K.L.~Roberts,
\Journal{\ZPC}{62}{161}{1994} and
\Journal{\ZPC}{63}{151}{1994}.

\bibitem{rges1}
K.~Inoue, A.~Kakuto, H.~Komatsu and H.~Takeshita,
\Journal{\PTP}{68}{927}{1982} and {\bf 71}, 413 (1984);
N.~K.~Falck \Journal{\ZPC}{30}{247}{1986}.

\bibitem{rges2} V.~Barger, M.S.~Berger, and P.Ohmann,
\Journal{\PRD}{47}{1093}{1993}.

\bibitem{twoloopsoft}
S.P.~Martin and M.T.~Vaughn, \Journal{\PRD}{50}{2282}{1994};
Y.~Yamada, \Journal{\PRD}{50}{3537}{1994};
I.~Jack and D.R.T.~Jones, \Journal{\PLB}{333}{372}{1994};
I.~Jack, D.R.T.~Jones, S.P.~Martin, M.T.~Vaughn
and Y.~Yamada, \Journal{\PRD}{50}{5481}{1994}.

\bibitem{threeloops} P.M.~Ferreira, I.~Jack, D.R.T.~Jones,
\Journal{\PLB}{387}{80}{1996}

\bibitem{nonrentheo}
A.~Salam and J.~Strathdee, \Journal{\PRD}{11}{1521}{1975};
M.T.~Grisaru, W.~Siegel and M.~Rocek, \Journal{\NP}{B159}{429}{1979}.

\bibitem{rewsb}
L.E.~Ib\'a\~nez and G.G.~Ross, \Journal{\PLB}{110}{215}{1982};
L.E.~Ib\'a\~nez, \Journal{\PLB}{118}{73}{1982};
J.~Ellis, D.V.~Nanopoulos and K.~Tamvakis,
\Journal{\PLB}{121}{123}{1983};
L.~Alvarez-Gaum\'e, J.~Polchinski, and M.~Wise,
\Journal{\NP}{B221}{495}{1983}.
%radiative electroweak symmetry breaking

\bibitem{GunionHaber} J.F.~Gunion and H.E.~Haber,
\Journal{\NP}{B272}{1}{1986}; \Journal{\NP}{B278}{449}{1986};
\Journal{\NP}{B307}{445}{1988}.
%(Errata hep-ph/9301205).
(E: \Journal{\NP}{B402}{567}{1993}).

\bibitem{HHG}
J.F.~Gunion, H.E.~Haber, G.L.~Kane and S.~Dawson,
{\em The Higgs Hunter's
Guide} Addison-Wesley 1991,
errata: hep-ph/9302272.

\bibitem{treelevelhiggsbound} K.~Inoue, A.~Kakuto, H.~Komatsu and
S.~Takeshita, \Journal{\PTP}{67}{1889}{1982}; R.A.Flores and
M.~Sher {\it Ann.~Phys.} (NY) {\bf 148}, 95, 1983.

\bibitem{hcorrections} 
H.E.~Haber and R.~Hempfling, \Journal{\PRL}{66}{1815}{1991};
Y.~Okada, M.~Yamaguchi and T.~Yanagida,
\Journal{\PTP}{85}{1}{1991}, \Journal{\PLB}{262}{54}{1991}; 
J.~Ellis, G.~Ridolfi and F.~Zwirner,
\Journal{\PLB}{257}{83}{1991}, \Journal{\PLB}{262}{477}{1991}.

\bibitem{HHH} 
G.~Gamberini, G.~Ridolfi and F.~Zwirner,
\Journal{\NP}{B331}{331}{1990}; 
R.~Barbieri, M.Frigeni and F.~Caravaglio, \Journal{\PLB}{258}{167}{1991};
A.~Yamada, \Journal{\PLB}{263}{233}{1991} and
\Journal{\ZPC}{61}{247}{1994};
J.R.~Espinosa and M.~Quiros, \Journal{\PLB}{266}{389}{1991};
A.~Brignole, \Journal{\PLB}{281}{284}{1992};
M.~Drees and M.M.~Nojiri, \Journal{\PRD}{45}{2482}{1992} and
\Journal{\NP}{B369}{54}{1992};
H.E.~Haber and R.~Hempfling, \Journal{\PRD}{48}{4280}{1993};
P.H.~Chankowski, S.~Pokorski and J.~Rosiek,
\Journal{\PLB}{274}{191}{1992} and \Journal{\NP}{B423}{437}{1994};
R.~Hempfling and A.H.~Hoang, \Journal{\PLB}{331}{99}{1994};
J.~Kodaira, Y.~Yasui and K.~Sasaki, \Journal{\PRD}{50}{7035}{1994};
J.A.~Casas, J.R.~Espinosa, M.~Quiros and A.~Riotto,
\Journal{\NP}{B436}{3}{1995} [E: \Journal{\NP}{B439}{466}{1995}];
M.~Carena, M.~Quir\'os and C.~Wagner, \Journal{\NP}{B461}{407}{1996}.

\bibitem{KKW} See G.L. Kane, C.~Kolda and J.D. Wells,
\Journal{\PRL}{70}{2686}{1993}; J.R.~Espinosa and M.~Quir\'os,
\Journal{\PLB}{302}{51}{1993},
 and references therein.

\bibitem{polecat} R.~Tarrach, \Journal{\NP}{B183}{384}{1980};
H.~Gray, D.J.~Broadhurst, W.~Grafe and K.~Schilcher,
\Journal{\ZPC}{48}{673}{1990};
H.~Arason et al., \Journal{\PRD}{46}{3945}{1992}.

\bibitem{so10}
L.E.~Ib\'a\~nez and C.~Lopez, \Journal{\PLB}{126}{54}{1983};
H.~Arason et al., \Journal{\PRL}{67}{2933}{1991};
V.~Barger, M.S.~Berger and P. Ohmann, \Journal{\PRD}{47}{1093}{1993};
P.~Langacker and N.~Polonsky, \Journal{\PRD}{49}{1454}{1994};
P.~Ramond, R.G.~Roberts, G.G.~Ross,
\Journal{\NP}{B406}{19}{1993};
M.~Carena, S.~Pokorski and C.~Wagner,
\Journal{\NP}{B406}{59}{1993};
G.~Anderson et al, \Journal{\PRD}{49}{3660}{1994}.

\bibitem{copw}
L.J.~Hall, R.~Rattazzi and U.~Sarid,
\Journal{\PRD}{50}{7048}{1994};
M.~Carena, M.~Olechowski, S.~Pokorski and
C.E.M.~Wagner, \Journal{\NP}{B426}{269}{1994};
R.~Hempfling, \Journal{\PRD}{49}{6168}{1994};
R.~Rattazzi and U.~Sarid, \Journal{\PRD}{53}{1553}{1996}.

\bibitem{PBMZ}
D.~Pierce and A.~Papadopoulos,
\Journal{\PRD}{50}{565}{1994};
\Journal{\NP}{B430}{278}{1994};
D.~Pierce, J.A.~Bagger, K.~Matchev, and R.-J.~Zhang,
\Journal{\NP}{B491}{3}{1997}.

\bibitem{badvacua} J.~Fr\`ere, D.R.T.~Jones and S.~Raby,
\Journal{\NP}{B222}{11}{1983}; M.~Claudson, L.J.~Hall,
and I.~Hinchliffe, \Journal{\NP}{B228}{501}{1983};
J.A.~Casas, A.~Lleyda, C.~Mu\~noz, \Journal{\NP}{B471}{3}{1996}.
T.~Falk, K.A.~Olive, L.~Roszkowski, and M.~Srednicki
\Journal{\PLB}{367}{183}{1996}.
H.~Baer, M.~Brhlik and D.J.~Casta\~no,
\Journal{\PRD}{54}{6944}{1996}. For a review,
see J.A.~Casas, hep-ph/9707475,
\perspectives.

\bibitem{kusenko}
A.~Kusenko, P.~Langacker and G.~Segre, \Journal{\PRD}{54}{5824}{1996}.

\bibitem{inodecays}  A.~Bartl, H.~Fraas and W.~Majerotto,
\Journal{\ZPC}{30}{411}{1986};
\Journal{\ZPC}{41}{475}{1988};
\Journal{\NP}{B278}{1}{1986};
A.~Bartl, H.~Fraas, W. Majerotto and B.~M\"osslacher,
\Journal{\ZPC}{55}{257}{1992}. For large $\tan\beta$ results,
see H.~Baer, C.-h.~Chen, M.~Drees, F.~Paige and X.~Tata,
\Journal{\PRL}{79}{986}{1997}.
%preprint hep-ph/9704457.

\bibitem{epluseminuscrosssections} H.~Baer 
%, A.~Bartl, D.~Karatas, W.~Majerotto, and X.~Tata,
et al., 
\Journal{\IJMP}{A4}{4111}{1989}.

\bibitem{AmbrosanioMele} H.E.~Haber and D.~Wyler,
\Journal{\NP}{B323}{267}{1989}; S.~Ambrosanio and B.~Mele,
\Journal{\PRD}{55}{1399}{1997}; 
S.~Ambrosanio et al.,
\Journal{\PRD}{55}{1372}{1997}
and references therein.

\bibitem{cascades} H.~Baer 
%, J.~Ellis, G.~Gelmini, D.V.~Nanopoulos, and X.~Tata,
et al., 
\Journal{\PLB}{161}{175}{1985};
G.~Gamberini, \Journal{\ZPC}{30}{605}{1986};
H.A.~Baer, V.~Barger, D.~Karatas and X.~Tata,
\Journal{\PRD}{36}{96}{1987};
R.M.~Barnett, J.F.~Gunion amd H.A.~Haber, \Journal{\PRD}{37}{1892}{1988}.

\bibitem{stoptocharmdecay} K.~Hikasa and M.~Kobayashi,
\Journal{\PRD}{36}{724}{1987}.

\bibitem{noscalephotons} J.~Ellis, J.~L.~Lopez, D.V.~Nanopoulos 
\Journal{\PLB}{394}{354}{1997}; J.~L.~Lopez, D.V.~Nanopoulos 
\Journal{\PRD}{55}{4450}{1997}. 

\bibitem{Rb} M.~Boulware and D.~Finnell, \Journal{\PRD}{44}{2054}{1991};
G.~Altarelli, R.~Barbieri and F.~Caravaglios,
\Journal{\PLB}{314}{357}{1993};
J.D.~Wells, C.~Kolda and G.L.~Kane \Journal{\PLB}{338}{219}{1994};
G.~Kane, R.~Stuart, and J.D.~Wells, \Journal{\PLB}{354}{350}{1995};
D.~Garcia and J.~Sola, \Journal{\PLB}{357}{349}{1995};
J.~Erler and P.~Langacker, \Journal{\PRD}{52}{441}{1995};
X.~Wang, J.~Lopez and D.V.~Nanopoulos,
\Journal{\PRD}{52}{4116}{1995};
P.~Chankowski and S.~Pokorski, \Journal{\NP}{B475}{3}{1996}.

\bibitem{snowmass96} J.~Bagger, U.~Nauenberg, X.~Tata, and A.~White,
``Summary of the Supersymmetry Working Group",
1996 DPF/DPB Summer Study on New Directions for High-Energy Physics
(Snowmass 96), hep-ph/9612359; J.~Amundson et al., Report of
the Snowmass Supersymmetry Theory Working Group,  hep-ph/9609374.

\bibitem{NLCsusy} T.~Tsukamoto et al.,
%K.~Fujii, H.~Murayama, M.~Yamaguchi and Y.~Okada, 
\Journal{\PRD}{51}{3153}{1995};
H.~Baer, R.~Munroe and X.~Tata, \Journal{\PRD}{54}{6735}{1996}.

\bibitem{sleptonLHC} F.~del Aguila and L.~Ametller,
\Journal{\PLB}{261}{326}{1991};
H.~Baer, C-H.~Chen, F.~Paige and X.~Tata, \Journal{\PRD}{49}{3283}{1994}.
\Journal{\PRD}{52}{2746}{1995}.

\bibitem{gluinosquarkproduction} P.~Harrison and C.~Llewellyn-Smith,
\Journal{\NP}{B213}{223}{1983};
G.~Kane and J.L.~Leveille,
\Journal{\PLB}{112}{227}{1982};
S.~Dawson, E.~Eichten and C.~Quigg,
\Journal{\PRD}{31}{1581}{1985};
H.~Baer and X.~Tata, \Journal{\PLB}{160}{159}{1985}.
Significant next-to-leading order corrections have
been computed by W.~Beenakker, R.~Hopker, M.~Spira and P.M.~Zerwas,
\Journal{\PRL}{74}{2905}{1995};
\Journal{\ZPC}{69}{163}{1995};
\Journal{\NP}{B492}{51}{1997}.

\bibitem{likesigndilepton}
V.~Barger, Y.~Keung and R.J.N.~Phillips, \Journal{\PRL}{55}{166}{1985};
R.M.~Barnett, J.F.~Gunion, and H.E.~Haber, \Journal{\PLB}{315}{349}{1993};
H.~Baer, X.~Tata and J.~Woodside, \Journal{\PRD}{41}{906}{1990}.

\bibitem{trilepton} R.~Arnowitt and P.~Nath, {\em Mod.~Phys.~Lett.}
{\bf A2}, 331, (1987); H.~Baer and X.~Tata,
\Journal{\PRD}{47}{2739}{1993}; H.~Baer, C.~Kao and X.~Tata
\Journal{\PRD}{48}{5175}{1993}; T.~Kamon, J.~Lopez, P.~McIntyre and
J.T.~White, \Journal{\PRD}{50}{5676}{1994};
H.~Baer, C.-h.~Chen, C.~Kao and X.~Tata, \Journal{\PRD}{52}{1565}{1995};
S.~Mrenna, G.L.~Kane, G.D.~Kribs and J.D.~Wells,
\Journal{\PRD}{53}{1168}{1996}.

\bibitem{singlelepton}
H.~Baer, C-H.~Chen, F.~Paige and X.~Tata,
\Journal{\PRD}{52}{2746}{1995}.

\bibitem{sneutrinonotLSP}
D.O.~Caldwell et al., \Journal{\PRL}{61}{510}{1988} and
\Journal{\PRL}{65}{1305}{1990};
D.~Reuner et al., \Journal{\PLB}{255}{143}{1991};
M.~Mori et al. (The Kamiokande Collaboration),
\Journal{\PRD}{48}{5505}{1993}.

\bibitem{darkmatterreviews} For reviews, see
G.~Jungman, M.~Kamionkowski and K.~Griest,
\Journal{\PRP}{267}{195}{1996};
M.~Drees, ``Recent developments in Dark Matter Physics",
hep-ph/9703260; J.D.~Wells, ``Mass density of neutralino dark matter",
hep-ph/9708285, \perspectives.

\bibitem{baryonparity} L.E.~Ib\'a\~nez and G.~Ross,
\Journal{\NP}{B368}{3}{1992}.

\bibitem{noprotondecay} D.J.~Casta\~no and S.P.~Martin,
\Journal{\PLB}{340}{67}{1994}. 

\bibitem{sneutrinovevRparityviolation} C.~Aulakh and R.~Mohapatra,
\Journal{\PLB}{119}{136}{1983}; G.G.~Ross and J.W.F.~Valle,
\Journal{\PLB}{151}{375}{1985}; J.~Ellis et al.,
\Journal{\PLB}{150}{142}{1985}; D.~Comelli, A.~Masiero, M.~Pietroni, and
A.~Riotto, \Journal{\PLB}{324}{397}{1994}.

\bibitem{nonsneutrinovevRparityviolation} A.~Masiero and J.W.F.~Valle,
\Journal{\PLB}{251}{273}{1990}; J.C.~Romao, C.A.~Santos and J.W.F.~Valle,
\Journal{\PLB}{288}{311}{1992}.


\bibitem{NMSSM} H.P.~Nilles, M.~Srednicki and D.~Wyler,
\Journal{\PLB}{120}{346}{1983}; J.~Ellis et al.,
\Journal{\PRD}{39}{844}{1989}.

\bibitem{NMSSMpheno} See, for example,
U.~Ellwanger, M.~Rausch de Traubenberg,
and C.A.~Savoy,
\Journal{\PLB}{315}{331}{1993},
\Journal{\NP}{B492}{21}{1997};
S.A.~Abel, S.~Sarkar and I.B.~Whittingham, \Journal{\NP}{B392}{83}{1993};
F.~Franke, H.~Fraas and A.~Bartl, \Journal{\PLB}{336}{415}{1994};
S.F.~King and P.L.~White, \Journal{\PRD}{52}{4183}{1995};

\bibitem{Dterms} M.~Drees, \Journal{\PLB}{181}{279}{1986};
J.S.~Hagelin and S.~Kelley, \Journal{\NP}{B342}{85}{1990};
A.E.~Faraggi, J.S.~Hagelin, S.~Kelley, and D.V.~Nanopoulos,
\Journal{\PRD}{45}{3272}{1992};
Y.~Kawamura and M.~Tanaka, \Journal{\PTP}{91}{949}{1994};
Y.~Kawamura, H.~Murayama and M.~Yamaguchi,
\Journal{\PLB}{324}{52}{1994}; \Journal{\PRD}{51}{1337}{1995};
H.-C.~Cheng and L.J.~Hall, \Journal{\PRD}{51}{5289}{1995};
C.~Kolda and S.P.~Martin, \Journal{\PRD}{53}{3871}{1996};
T.~Gherghetta, T.~Kaeding and G.L.~Kane, hep-ph/9701343.

\bibitem{cosmock} A.~Brignole, L.E.~Ib\'a\~nez and C.~Mu\~noz,
\Journal{\NP}{B422}{125}{1994}, [erratum \Journal{\NP}{B436}{747}{1995}]; 
K.~Choi, J.E.~Kim and H.P.~Nilles, \Journal{\PRL}{73}{1758}{1994};
K.~Choi, J.E.~Kim and G.T.~Park, \Journal{\NP}{B442}{3}{1995}. See also
N.C.~Tsamis and R.P.~Woodard, \Journal{\PLB}{301}{351}{1993};
\Journal{\NP}{B474}{235}{1996};  {\it Annals Phys.} {\bf 253}, 1 (1996);
%hep-ph/9602316;  
{\it Annals Phys.} {\bf 267}, 145 (1997),
%hep-ph/9712331,
and references therein, for an exposition of nonperturbative
infrared quantum gravitational effects on the effective cosmological
constant. This work implies that it is quite unlikely that requiring the
tree-level vacuum energy to vanish is correct or meaningful. Moreover,
naive
supergravity or superstring predictions for the vacuum energy
need not have any relevance to the question of whether the observed
cosmological constant is sufficiently small.

\end{thebibliography}

\end{document}
