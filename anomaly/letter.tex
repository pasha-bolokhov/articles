%\documentclass[12pt]{article}
\documentclass[12pt]{revtex4}
%\documentclass[draft,12pt]{article}
\usepackage{amsmath}
\usepackage{amsfonts}
\usepackage{amsbsy}
\usepackage{lscape} 
\usepackage{color}
\usepackage{graphicx,epsfig}
\usepackage[english]{babel}
\usepackage{latexsym}
\usepackage{amssymb}
\usepackage{palatino}
%\usepackage{chancery}
%\usepackage{newcent}
%\usepackage{charter}
%\usepackage{zapfchan}
%\usepackage{bookman}
%\usepackage{sparticles} 	%Package for displaying sparticle names. 
%\usepackage{feynmf}		%Package for feynman diagrams. 

%% slashed symbols
%\newcommand{\slashed}[1]{\hbox{{$#1$}\llap{$/$}}}
%\newcommand{\sslashed}[1]{\hbox{{$#1$}\llap{$/\,$}}}

%%%%%%%%%%%%%%%%%%%%%%%%%%%%%%%%%%%%%%%
%  Slash character...
\def\slashed#1{\setbox0=\hbox{$#1$}             % set a box for #1
   \dimen0=\wd0                                 % and get its size
   \setbox1=\hbox{/} \dimen1=\wd1               % get size of /
   \ifdim\dimen0>\dimen1                        % #1 is bigger
      \rlap{\hbox to \dimen0{\hfil/\hfil}}      % so center / in box
      #1                                        % and print #1
   \else                                        % / is bigger
      \rlap{\hbox to \dimen1{\hfil$#1$\hfil}}   % so center #1
      /                                         % and print /
   \fi}                                        %

%%EXAMPLE:  $\slashed{E}$ or $\slashed{E}_{t}$



\newcommand{\beq}{\begin{equation}}
\newcommand{\eeq}{\end{equation}}

\newcommand{\p}{\partial}
\newcommand{\wt}{\widetilde}
\newcommand{\ov}{\overline}
\newcommand{\md}{\mathcal{D}}

\newcommand{\suc}{{\rm SU}_{\rm C}(3)}
\newcommand{\sul}{{\rm SU}_{\rm L}(2)}
\newcommand{\ue}{{\rm U}(1)}
\newcommand{\GeV}{{\rm GeV}}
%\newcommand{\su3}{{\rm SU}_{\rm C}(3)}

%\newcounter{dim5}
%\setcounter{dim5}{5}


\begin{document}
    Hi Stefan,

  I have been trying to apply that trick which we used in our SQED
paper for calculation of anomaly in generic QCD theory, but unfortunately,
not successful.
  First, it's highly desirably to be able to reproduce the anomaly 
in a regular theory. Then it hopefully will be easy to generalize it.
  Here's the main trouble:

 The covariant variation of the effective action is

\[
    \delta \Gamma    ~~~\propto~~~
 Tr \left[~  \delta A^\mu  \gamma_\mu\,  e^{-\slashed{\md}^2/M^2}\, \slashed{\md}^{-1} 
	\delta (x_2 - x_1) ~\right]
\]

  My hope was as follows:  as you see here 
\[
           \delta A^\mu  ~=~  [  \md_\mu, \phi ]
\]
  where $ \phi $ is the infinitesimal parameter of gauge variation,
and therefore
\[
	\delta A^\mu \gamma_\mu  ~=~  [ \slashed{\md}, \phi ]
\]

And I hoped that this $ \slashed{\md} $ can be integrated by parts
and hopefully would cancel the $ \slashed{\md}^{-1} $ from the propagator.
Furthermore! It can be shown that in the case of a generalized kinetic
term  $ \ov{\psi} {\mathcal O} \psi $, the variation looks approximately
as
\[
  \delta \Gamma    ~~~\propto~~~
 Tr \left[~  [{\mathcal O}, \phi]\,  e^{-\slashed{\md}^2/M^2}\, 
{\mathcal O}^{-1} \delta (x_2 - x_1) ~\right]
\]

The striking fact is that if that indeed happened, the result would
be more or less correct: one would have again to regularize the delta-function,
apply the regularizational exponent, take the trace over gamma-matrices
and obtain the conventional $ F_* F $. Up to a coefficient, this is evident.

However, this cancellation does not happen completely, since there is
a commutator $ [ \slashed{\md}, \phi ] $. Basically, here $ \md $ 
is applied in adjoint representation, whereas in the propagator
$ \slashed{\md}^{-1} \delta (x_2 - x_1) $
it is applied in the fundamental representation. 
The extra undesired term is obviously
\[
 Tr \left[~  \slashed{A} \;\phi \; e^{-\slashed{\md}^2/M^2}\; \slashed{\md}^{-1} 
	\delta (x_2 - x_1) ~\right]
\]
which I don't see how one can get rid of..

***

I've been trying to look in the literature, since definitely Gates wasn't
the one who invented this trick. I found the application of this trick
in supersymmetry explained in more detail in earlier literature, but did 
not find it applied to regular QCD in exactly the same way. People do 
calculate the propagator using more or less the same trick,
but then they always expand the measure in the eigenfunctions of Dirac
operator or so (maybe that's the way I should try it too..) to actually
compute the covariant anomaly..

~~~regards

Pasha

\end{document}
