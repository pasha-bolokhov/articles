\typeout{CVSId: $Id: prosper-doc.tex,v 1.7 2001/01/12 08:58:34 exupery Exp $}
\documentclass[a4paper]{article}
\def\Version{1.0i}

\usepackage{color}
\usepackage{times}
\usepackage[latin1]{inputenc}
\usepackage{textcomp}
\usepackage[ps2pdf,bookmarks,%
                urlcolor=blue,citecolor=blue,linkcolor=blue,%
                pagecolor=blue,%hyperindex,%
                colorlinks,hyperfigures,
           ]{hyperref}
\usepackage{index}

\makeindex

% Registered logo
\newcommand{\Registered}{\ensuremath{^{\hbox{\fontsize{5pt}{5pt}\selectfont%
      \textregistered}}}}
% Code to be put in the index
\newcommand{\code}[1]{{\bfseries\texttt{#1}}%
  \index{#1@\texttt{#1}}}
% Macro without argument
\newcommand{\codeM}[1]{{\bfseries\texttt{\textbackslash #1}}%
  \index{#1@\texttt{\textbackslash #1}}}
% Macro with argument(s)
\newcommand{\codeA}[2]{{\bfseries\texttt{\textbackslash #1%
    #2}\index{#1@\texttt{\textbackslash #1}}}}
% Code entry in the index
\newcommand{\cindex}[1]{\index{#1@\texttt{\textbackslash #1}}}

\hoffset=-1.8cm
\textwidth=16cm
\voffset=-1.5cm
\textheight=23cm


% Trademark
\newcommand{\tm}{\ensuremath{^{\hbox{\usefont{T1}{phv}{m}{n}%
        \fontsize{5pt}{5pt}\selectfont TM}}}}

\usepackage{graphicx}
\usepackage{fancybox,amssymb}
\usepackage{pstricks}

\title{Manual for the \texttt{prosper} class%
  \footnote{This documentation describes version \Version\ of the class.
  {\tiny CVS id $=$ \$Id: prosper-doc.tex,v 1.7 2001/01/12 08:58:34 exupery Exp $ $}}}
\author{Fr�d�ric Goualard \\ \textit{Centrum voor Wiskunde en Informatica}\\
  Amsterdam, The Netherlands}
\date{}

\begin{document}
\maketitle
\begin{abstract}
  The \texttt{prosper} class permits producing high quality
  slides; it is also easily extendable.  This documentation is meant
  to be a user manual for the predefined slide styles as well as a
  technical note describing how to create your own styles.
\end{abstract}

\section{Using the class}
%========================
\LaTeX\ files using the \texttt{prosper} class may be eventually translated
into two different formats:
\begin{itemize}
\item the Adobe\Registered\ \emph{PostScript}\tm\ format for printing
  transparencies;
\item the Adobe\Registered\ \emph{Portable Document Format} (PDF) for
  displaying slides on computers with Acrobat\Registered\ Reader in
  full-screen mode.
\end{itemize}

When translated into PDF files, \texttt{prosper} slides benefit from
additional possibilities such as transition effects between slides and
incremental displaying of a slide with several animation effects. The
currently supported transitions are: 
\index{transitions!supported}%
\begin{itemize}
\item \code{Split}: two lines sweep across the screen revealing the 
  new slide;
\item \code{Blinds}: multiple lines, evenly distributed across the
  screen, appear and synchronously sweep in the same direction to
  reveal the new slide;
\item \code{Box}: a box sweeps from the center, revealing the new slide;
\item \code{Wipe}: a single line sweeps across the screen from one
  edge to the other, revealing the new slide;
\item \code{Dissolve}: the old page image dissolves to reveal the new slide;
\item \code{Glitter}: similar to \texttt{Dissolve}, except the
  effect sweeps across the image in a wide band moving from one side
  of the screen to the other;
\item \code{Replace}: the effect is simply to replace the old page
  with the new page.
\end{itemize}

Figure~\ref{fig:structure} presents a bird's-eye view of the structure
of a \LaTeX\ file using the \texttt{prosper} class.


\section{Options of the class}
%=============================

The \texttt{prosper} class supports the following options (default
options are preceded by a black triangle $\blacktriangleright$, while
the others are preceded by a black square $\blacksquare$):

%--------------------------------------------------------------------- Figure -
\begin{figure}[htbp]
  \begin{center}
    \includegraphics[width=.6\textwidth]{prosper-structure.eps}
    \caption{Structure for a \LaTeX\ file using {\normalfont\texttt{prosper}}}
    \label{fig:structure}
  \end{center}
\index{slide@\texttt{slide} environment}%
\end{figure}
%------------------------------------------------------------------------------

\begin{description}
\item [$\blacksquare$~\code{draft}.] The file is compiled in draft
  mode: figures are replaced by bounding boxes; the caption at the
  bottom of every slide displays the date and time of the compiling
  together with the file name;
\item [$\blacktriangleright$~\code{final}.] The file is compiled
  in final mode: figures are inserted at their position; the caption
  on every slide contains the text  given (optionally) by the
  user with the macro \verb$\slideCaption$, except if the
\cindex{slideCaption}%
  macro \verb$\displayVersion$ appears in the preamble (in that case,
\index{displayVersion@\texttt{\textbackslash displayVersion}}%
  the same caption as in the draft mode is used);
\item [$\blacksquare$~\code{slideColor}.] Slides will
  use many colors. To be used with caution when the slides are to be printed
  on a black \& white device;
\item [$\blacktriangleright$~\code{slideBW}.] Slides will use
  a restricted set of colors. Should be used whenever the presentation is
  meant to be printed in black \& white;
\item [$\blacktriangleright$~\code{total}.] The caption at the bottom of
  every slide displays the number of the current slide along with 
  the total number of slides;
\item [$\blacksquare$~\code{nototal}.] Only the number of the current slide
  appears in the caption;
\item [$\blacktriangleright$~\code{nocolorBG}.] The background of the
  slide is white whatever the style may be. It is a good idea to 
  use this option for printing slides in black \& white;
\item [$\blacksquare$~\code{colorBG}.] The color of
  the background depends on the current style;
\item [$\blacktriangleright$~\code{ps}.] The \LaTeX\ file is
  compiled to produce a PostScript\tm\ file for printing;
\item [$\blacksquare$~\code{pdf}.] The \LaTeX\ file is compiled to
  produce a PDF file for a presentation with a video projector;
\item [$\blacksquare$~\code{accumulate}.] Macros \verb$\onlySlide$,
\cindex{untilSlide}\cindex{fromSlide}\cindex{onlySlide}%
  \verb$\untilSlide$ and \verb$\fromSlide$ interpret their argument in
  \verb$ps$ mode. Note that it is possible to locally modify the option setting
  by using macros \codeM{Accumulatetrue} and \codeM{Accumulatefalse};
\item [$\blacktriangleright$~\code{noaccumulate}.] Macros \verb$\onlySlide$,
  \verb$\untilSlide$ and \verb$\fromSlide$ do not interpret their argument in
  \verb$ps$ mode;
\item [$\blacksquare$~\code{distiller}.] Use this option if the
  PostScript\Registered\ file is to be translated into a PDF file using
  Adobe\Registered\ Distiller.
\end{description}

\section{Predefined macros and environments} 
%===========================================

\subsection{Macros to appear in the preamble}
%--------------------------------------------
The \texttt{prosper} class (re-)defines some standard macros. Those
given hereunder are to be put in the preamble (that is, before
\index{preamble}%
\verb$\begin{document}$):

\begin{description}
\item \codeM{title}. Title of the presentation;
\item \codeM{subtitle}. Subtitle of the presentation;
\item \codeM{author}. Author(s) of the presentation;
\item \codeM{email}. E-mail address of the author(s);
\item \codeM{institution}. Name of the institute/company the author(s) 
  come(s) from;
\item \codeA{slideCaption}{\{c\}}. Caption to be put at the bottom
  of every slide (name of the event/conference\dots). 
  The title of the presentation is used as the default caption whenever 
  the author
  do not override it by providing his own caption by using this macro;
\item \codeA{Logo}{(x,y)\{mylogo\}} or \codeA{Logo}{\{mylogo\}}. 
  The logo given by \verb$mylogo$ will be put at the position \verb$(x,y)$
  on each slide (resp.\ at a default position defined by each slide style).
  The reference point is bottom left. An example of use is: \\
  \verb$\Logo(2,5){\includegraphics[width=1cm]{irinLOGO.eps}}$)
\item \codeM{displayVersion}. Displays a draft caption (with
  the name of the file, the title of the presentation, the name of the
  author(s), and the date/time of the last \LaTeX\ compiling) instead
  of the caption defined by the user even when in final mode;
\item \codeA{DefaultTransition}{\{trans\}}: definition of the default
  transition mode between slides. By default, the \texttt{Replace} mode is
\index{transition!default}%
  used;
\item \codeA{NoFrenchBabelItemize}. To be used when loading the babel 
  style with the ``french'' option in order to have the ability to choose
  ones own items. The french itemize glue is preserved.
\end{description}

\subsection{The \texttt{slide} environment}
%------------------------------------------
Figure~\ref{fig:structure} describes the \texttt{slide} environment. An
optional argument is the transition effect for displaying the slide. 
The default transition is \texttt{R} (Replace).

\subsection{Some \texttt{itemize} environments}
%----------------------------------------------
The \texttt{Itemize} environment corresponds to the \LaTeX\ 
\index{Itemize@\texttt{Itemize} environment}%
\texttt{itemize} environment where the text is justified. In
\index{itemize@\texttt{itemize} environment}%
\texttt{prosper}, the \texttt{itemize} environment has been redefined
such that text is not justified in it (a better choice for slides).

There also exist an \texttt{itemstep} environment where each item is displayed
\index{itemstep@\texttt{itemstep} environment}%
incrementally (in PDF mode).

\subsection{Macros to be used out of any \texttt{slide} environment}
%-------------------------------------------------------------------

\begin{description}
\item \codeA{part}{[transition]\{xx\}}. Creates a slide only containing the
  text \texttt{xx} vertically and horizontally centered in the 
  font title. The transition \texttt{transition}---if given---will be used 
  for this slide.
\end{description}

\subsection{Macros that may appear in a \texttt{slide} environment}
%------------------------------------------------------------------
\begin{description}
\item \codeA{FontTitle}{\{C\}\{BW\}}. Use this macro to change the
  font/color to be used for slide titles. The first argument is for
  color slides, the second for black and white ones;
\item \codeA{FontText}{\{C\}\{BW\}}.  Use this macro to change the
  font/color to be used for slide text. The first argument is for
  color slides, the second for black and white ones;
\item \codeA{fontTitle}{\{xx\}}. Writes its argument using the
  title font and color;
\item \codeA{fontText}{\{xx\}}.  Writes its argument using the
  text font and color;
\item \codeA{ColorFoot}{\{col\}}. The footer is to be written with color
  \texttt{col};
\item \codeA{PDFtransition}{\{tr\}}. Uses \texttt{tr} as the transition
  effect from the previous slide to the current slide;
\item \codeA{myitem}{\{lvl\}\{def\}}. Defines the item of level
  \verb$lvl$ (where \verb$lvl$ may be 1, 2 or 3) to be \verb$def$. By
  default, it is a green lozenge for all levels. The following code
  define the items to be 3D bullets of different size and color (the
  corresponding PostScript\tm\ files are provided in the \texttt{img/}
  directory of the \texttt{prosper} distribution):
\begin{verbatim}
\myitem{1}{\includegraphics[width=.4cm]{red-bullet-on-blue.ps}}
\myitem{2}{\includegraphics[width=.3cm]{green-bullet-on-blue.ps}}
\myitem{3}{\includegraphics[width=.3cm]{yellow-bullet-on-blue.ps}}
\end{verbatim}  
\end{description}

\subsection{Overlays}
%--------------------

Overlays add animated effects to slides in PDF mode. They may be used
to display a slide incrementally (in several steps), for making appear and
disappear some elements on a slide\dots\ To use overlays, one has to embed
the corresponding \texttt{slide} environment into an \codeM{overlays} macro
as follows:

\begin{verbatim}
\overlays{n}{
\begin{slide}{...}
...
\end{slide}}
\end{verbatim}

The first argument (\verb$n$) of the \texttt{overlays} macro stands
for the number of steps composing the animation.

The following macros may be used to control what should appear on each
slide composing a \verb$n$ slides overlay:
\begin{itemize}
\item \codeA{fromSlide}{\{p\}\{mat\}}. Puts \verb$mat$ on slides
  \verb$p$ through \verb$n$;
\item \codeA{onlySlide}{\{p\}\{mat\}}. Puts \verb$mat$ on slide \verb$p$
  only;
\item \codeA{untilSlide}{\{p\}\{mat\}}. Puts \verb$mat$ on slides
  \verb$1$ through \verb$p$;
\item \codeA{FromSlide}{\{p\}}. All the material after the occurrence
  of the macro will be put on slides \verb$p$ through \verb$n$;
\item \codeA{OnlySlide}{\{p\}}. All the material after the occurrence
  of the macro will be put on slide \verb$p$ only;
\item \codeA{UntilSlide}{\{p\}}. All the material after the
  occurrence of the macro will be put on slides \verb$1$ through
  \verb$p$.
\end{itemize}

All those macros are only really meaningful in \verb$pdf$ mode; in
\verb$ps$ mode, they do nothing or interpret their argument, depending
on the option \verb$no/accumulate$.  Macros \verb$\fromSlide$,
\verb$\onlySlide$, and \verb$\untilSlide$ accept stared versions which
\index{fromSlide@\texttt{\textbackslash fromSlide}!stared}%
\index{onlySlide@\texttt{\textbackslash onlySlide}!stared}%
\index{untilSlide@\texttt{\textbackslash untilSlide}!stared}%
typeset the \verb$mat$ material in a zero dimension box (i.e.\ the
position pointer is not moved). The stared versions should be used for
replacement purposes. For example, the piece of code:
\begin{verbatim}
\onlySlide*{1}{\includegraphics{example-1.eps}}%
\onlySlide*{2}{\includegraphics{example-2.eps}}%
\onlySlide*{3}{\includegraphics{example-3.eps}}%
\end{verbatim}
would put image \verb$example-1.eps$ on the first slide; this image
would then be replaced by \verb$example-2.eps$ on the second slide,
and by \verb$example-3.eps$ on the third slide. Note the \verb$%$ 
comment sign at the end of each line: it prevents \LaTeX\ from inserting
some space---due to the carriage return---which would induce a slight
displacement between each image on the slides.


\noindent\textbf{Important note}: keep in mind that macros \verb$\FromSlide$, 
\verb$\OnlySlide$, \verb$\UntilSlide$, and the un-stared versions of
\verb$\fromSlide$, \verb$\onlySlide$, and \verb$\untilSlide$ interpret
the argument \verb$mat$ \emph{for each slide composing the overlay}
even if it is not displayed (this is mandatory in order to know the size of the
box that needs be reserved for the un-displayed material).

The following macros permit choosing the material to put on a slide depending 
on the chosen mode (ps or pdf):
\begin{itemize}
\item \codeA{PDForPS}{\{ifpdf\}\{ifps\}}. Interprets material \verb$ifpdf$ 
  if the chosen mode is \verb$pdf$, otherwise interprets \verb$ifps$;
\item \codeA{onlyInPS}{\{mat\}}. Interprets material \verb$mat$ only if
  the mode is \verb$ps$;
\item \codeA{onlyInPDF}{\{mat\}}. Interprets material \verb$mat$ only if
  the mode is \verb$pdf$.
\end{itemize}

These macros may be used as follows:
\begin{verbatim}
\overlays{3}{%
\begin{slide}{Example}
\onlySlide*{1}{\includegraphics{example-1.eps}}%
\onlySlide*{2}{\includegraphics{example-2.eps}}%
\onlySlide*{3}{\includegraphics{example-3.eps}}%
\onlyInPS{\includegraphics{example.eps}}%
\end{slide}}
\end{verbatim}
This slide will be displayed in three steps with three different
figures in \verb$pdf$ mode; in \verb$ps$ mode, there will be only one
slide containing figure \verb$example.eps$.

\section{Predefined styles}
%==========================

Figures hereunder present the currently predefined styles with options
\texttt{colorBG} and \texttt{slideColor}. The next section presents some
contributed styles by users (available in the \texttt{contrib/} directory).

\begin{figure}[htbp]
\begin{center}
\begin{minipage}[t]{.45\textwidth}
  \centering 
  \shadowbox{\includegraphics[angle=-90,width=.9\textwidth]{./doc-examples/ExampleFrames.ps}}
\caption{Style {\normalfont\texttt{frames}} in color}
\label{fig:style-frames}
\end{minipage}\hskip1cm
\begin{minipage}[t]{.45\textwidth}
  \centering 
  \shadowbox{\psclip{\psframe(-.1,-4.7)(6.6,0.11)}%
    {\includegraphics[angle=-90,width=.9\textwidth]%
    {./doc-examples/ExampleLignesbleues.ps}}\endpsclip}
\caption{Style {\normalfont\texttt{lignesbleues}} in color}
\label{fig:style-lignesbleues}
\end{minipage}

\vskip1cm
\begin{minipage}[t]{.45\textwidth}
  \centering 
  \shadowbox{\includegraphics[angle=-90,width=.9\textwidth]{./doc-examples/ExampleAzure.ps}}
\caption{Style {\normalfont\texttt{azure}} in color}
\label{fig:style-azure}
\end{minipage}\hskip1cm
\begin{minipage}[t]{.45\textwidth}
  \centering 
  \shadowbox{\includegraphics[angle=-90,width=.9\textwidth]%
    {./doc-examples/ExampleTroisPoints.ps}}
\caption{Style {\normalfont\texttt{troispoints}} in color}
\label{fig:style-troispoints}
\end{minipage}

\vskip1cm
\begin{minipage}[t]{.45\textwidth}
  \centering 
  \shadowbox{\includegraphics[angle=-90,width=.9\textwidth]%
    {./doc-examples/ExampleContemporain.ps}}
\caption{Style {\normalfont\texttt{contemporain}} in color}
\label{fig:style-contemporain}
\end{minipage}\hskip1cm
\begin{minipage}[t]{.45\textwidth}
  \centering 
  \shadowbox{\includegraphics[angle=-90,width=.9\textwidth]%
    {./doc-examples/ExampleNuanceGris.ps}}
\caption{Style {\normalfont\texttt{nuancegris}} in color}
\label{fig:style-nuancegris}
\end{minipage}
\end{center}
\end{figure}

\begin{figure}[htbp]
\begin{center}
\begin{minipage}[t]{.45\textwidth}
  \centering 
  \shadowbox{\includegraphics[angle=-90,%
                              width=.9\textwidth]{./doc-examples/ExampleDarkblue.ps}}
\caption{Style {\normalfont\texttt{darkblue}} in color}
\label{fig:style-darkblue}
\end{minipage}\hskip1cm
\begin{minipage}[t]{.45\textwidth}
  \centering 
  \shadowbox{\psclip{\psframe(-.1,-4.7)(6.6,0.11)}%
    {\includegraphics[angle=-90,width=.9\textwidth]%
    {./doc-examples/ExampleAlienglow.ps}}\endpsclip}
\caption{Style {\normalfont\texttt{alien glow}} in color}
\label{fig:style-alienglow}
\end{minipage}
\end{center}
\end{figure}

\begin{figure}[htbp]
\begin{center}
\begin{minipage}[t]{.45\textwidth}
  \centering 
  \shadowbox{\includegraphics[angle=-90,%
                              width=.9\textwidth]{./doc-examples/ExampleAutumn.ps}}
\caption{Style {\normalfont\texttt{autumn}} in color}
\label{fig:style-autumn}
\end{minipage}\hskip1cm
\begin{minipage}[t]{.45\textwidth}
%  \centering 
%  \shadowbox{\psclip{\psframe(-.1,-4.7)(6.6,0.11)}%
%    {\includegraphics[angle=-90,width=.9\textwidth]%
%    {ExampleAlienglow.ps}}\endpsclip}
%\caption{Style {\normalfont\texttt{alien glow}} in color}
%\label{fig:style-alienglow}
\end{minipage}
\end{center}
\end{figure}

\section{Contributed styles}
%===========================
This section may grow as the number of contributed styles increases. Note 
that the authors of these styles are completely responsible for supporting
them. 

The persons listed below (alphabetical order) have supported
\texttt{prosper} by contributing additional great styles. Many thanks
to them.

\begin{center}
\begin{tabular}{lll}
\hline
Name & E-mail & Style \\
\hline
\'Eric Langu�nou & \texttt{Eric.Languenou@irin.univ-nantes.fr} & 
\textsf{rico}\\
Guillaume Raschia & \texttt{Guillaume.Raschia@irin.univ-nantes.fr} & 
\textsf{gyom}
\end{tabular}
\end{center}


\begin{figure}[htbp]
\begin{center}
\begin{minipage}[t]{.45\textwidth}
  \centering 
  \shadowbox{\includegraphics[angle=-90,width=.9\textwidth]{../contrib/gyom.ps}}
\caption{Contributed style {\normalfont\texttt{gyom}} in color}
\end{minipage}\hskip1cm
\begin{minipage}[t]{.45\textwidth}
  \centering 
  \shadowbox{\psclip{\psframe(-.1,-4.7)(6.6,0.11)}%
    {\includegraphics[angle=-90,width=.9\textwidth]%
    {../contrib/rico.ps}}\endpsclip}
% File only used to circumvent the PDF options of the previously inserted files
\includegraphics[bb=0 0 1 1]{rotation.ps}
\caption{Contributed style {\normalfont\texttt{rico}} in color}
\end{minipage}
\end{center}
\end{figure}

\section{Warning}\label{sec:avertissement}
%=================

The \texttt{prosper} slide styles are not bound to provide the same
display area. Consequently, using different styles may require some
\index{slide!display area}%
adjustment in the text and graphics positioning.

\section{The Compilation Process}
%================================

The compilation process slightly differs depending on the intended use of
the slides. It is sketched in Fig.~\ref{fig:compilation}. If you plan
to print slides on transparencies, you should select the \verb$ps$
option and create a PostScript\tm\ file, while if you want to display them
with a computer and an overhead projector, you should select the
\verb$pdf$ option and create a PDF file from the PostScript\tm\ file.
Translation of a PostScript\tm\ file into a PDF file is done by the
program \texttt{ps2pdf} included in the GhostScript distribution.

%--------------------------------------------------------------------- FIGURE -
\begin{figure}[htbp]
  \begin{center}
    \includegraphics[width=\textwidth]{compilation.eps}
    \caption{Compilation process}
    \label{fig:compilation}
  \end{center}
\end{figure}
%------------------------------------------------------------------------------

\noindent\textbf{Important note}: PDF file should be made resolution 
independent by using vectorial fonts only (no \TeX\ bitmap fonts). To
do so, you have to use a GhostScript version at least equal to 6.0.
You also need to create a \verb$.dvipsrc$ file in your home directory
with the following lines:
\begin{verbatim}
p +psfonts.cmz
p +psfonts.amz
\end{verbatim}
Last, \texttt{prosper} styles have been devised to be used with A4 European
paper format. Consequently, you will have to instruct GhostScript to
use the appropriate format by defining the \code{GS\_OPTIONS}
environment variable to \verb$"-sPAPERSIZE=a4"$. If you use bash as
\cindex{PAPERSIZE}%
your main shell, this is done by adding the line
\begin{verbatim}
export GS_OPTIONS="-sPAPERSIZE=a4"
\end{verbatim}
in your \verb$.bash_profile$ file.

You will need Adobe\Registered\ Acrobat\Registered\ Reader
(\texttt{acroread}) to display PDF files. It is available for free on
the Adobe\Registered\ \href{http://www.adobe.com/}{web site}.
Acrobat\Registered\ Reader provides a full-screen mode that is
particularly handy for presentations.

\section{Devising new slide styles}
%===================================

Devising new \texttt{prosper} styles is an easy task provided you know
the basics of Van Zandt's \texttt{PSTricks} package (refer to
\emph{PSTricks: PostScript\tm\ macros for Generic \TeX}, User's Guide,
Timothy Van Zandt).  In order to devise your own style named
\texttt{foo}, you first have to create a file
\texttt{PPR\textbf{foo}.sty} which will contain its definition. Refer
to predefined styles for some examples and to
Section~\ref{sec:definition-example}.

\noindent\textbf{A word of caution:} you are free to create a new 
style by modifying an existing one. In that case, \textbf{it is
  MANDATORY} renaming your file; do NEVER EVER modify a style without
renaming it (is that clear enough?).  You should also write your name
and email address in any of your styles such that users know who to
get in touch with when they use the style. Please choose a name for
your style that is unique in the \texttt{prosper} distribution (with
respect to both predefined and contributed styles so far).

\medskip Please send slide styles you are proud of. I will add them to
the distribution in the \texttt{contrib/} directory. Note that I will
only consider for addition styles that are indeed original. Modifying
the colors or the fonts of an existing one is definitely not
sufficient since this can be done by users in their \LaTeX\ file by using
the provided hooks for customization.

\subsection{Predefined tests}
%-----------------------------
The following tests may be used in your style file in order to modify
its behaviour according to the active options. The general scheme is:
\begin{verbatim}
       \ifxxxx%
           % The ``then'' part
       \else%
           % The ``else'' part
       \fi
\end{verbatim}

\begin{description}
\item \codeM{ifDVItoPS}. True when the DVI file will be eventually 
  translated into a PostScript\tm\ file, false when the final target is
  the PDF format;
\item \codeM{ifisDraft}. True when the file is compiled in draft mode;
\item \codeM{ifinColor}. True when the option \texttt{slideColor} 
  has been chosen;
\item \codeM{ifallPages}. True when the option \texttt{total} has
  been chosen;
\item \codeM{ifcolorBG}. True when the option \texttt{colorBG} has
  been chosen;
\item \codeM{ifshowVersion}. True whenever the macro 
  \verb$\displayVersion$ appears in the preamble;
\item \codeM{ifInOverlays}. True if the current \texttt{slide} 
  environment is embedded into an \texttt{overlays} macro.
\end{description}

\subsection{Macros to customize or create a style}
%-------------------------------------------------
\begin{description}
\item \codeA{slideCaption}{\{cap\}}. Definition of a caption
  to appear on every slide;
\item \codeA{PDFCroppingBox}{\{lx ly ux uy\}}. Definition of a
  PostScript\tm\ \emph{bounding box} to crop slides for enhancing
\index{bounding box}%
  their appearance on $4/3$ devices such as monitors (only used in PDF
  mode);
\item \codeA{NewSlideStyle}{[width]\{anchor\}\{pos\}\{defin\}}.
  Defines a new slide style whose definition is given by the macro
  \verb$\defin$ and whose contents area has width \verb$width$ and is
  put at position \verb$(pos)$ with anchor \verb$anchor$. If no width
  is given, a default width of 11\,cm is used;
\item \codeA{LogoPosition}{\{pos\}}. Default position for a logo if none
  is given by the user;
\item \codeM{PutLogo}. A macro to be put at the end of the macro
  that defines your own style.
\end{description}

\subsection{Lengths}
%-------------------

\begin{description}
\item \codeM{slideWidth}. Defines the width of the text area in the slide. 
  Should not be modified by the user. Corresponds to the firth argument of
  macro \code{NewSlideStyle}.
\end{description}

\subsection{Example: the \texttt{troispoints} style}
%----------------------------------------------------
\label{sec:definition-example}

\begin{small}
\begin{verbatim}
\NeedsTeXFormat{LaTeX2e}[1995/12/01]
\ProvidesPackage{PPRtroispoints}[2000/04/17]
\typeout{`Trois points' style for Prosper ---}
\typeout{(c) 2000 Frederic Goualard, CWI, The Netherlands}
\typeout{CVSId: $Id: prosper-doc.tex,v 1.7 2001/01/12 08:58:34 exupery Exp $}
\typeout{ }

\RequirePackage{amssymb}
% Loading packages necessary to define this slide style.
\IfFileExists{pst-grad}{\RequirePackage{pst-grad}}{\RequirePackage{gradient}}

\newgray{mygrey}{.5}
\newrgbcolor{mellow}{.847 .72 .525}
\newrgbcolor{orange}{1.00 0.65 0.00}

\FontTitle{%
  \usefont{T1}{ptm}{m}{sl}\fontsize{22pt}{20pt}\selectfont\orange}{%
  \usefont{T1}{ptm}{m}{sl}\fontsize{22pt}{20pt}\selectfont\blue}
\FontText{%
  \mellow\usefont{T1}{phv}{m}{n}\fontsize{14.4pt}{14pt}\selectfont}{%
  \black\usefont{T1}{phv}{m}{n}\fontsize{14.4pt}{14pt}\selectfont}

\ColorFoot{\mellow}

% Positionning of the title of a slide.
\newcommand{\slidetitle}[1]{%
  \rput[l](-0.4,3.7){\parbox{10cm}{\fontTitle{#1}}}
}

% Positionning for a logo
\LogoPosition{-1,-1.1}

% Definition of this style for slides.

\newcommand{\TPFrame}[1]{%
  \ifinColor
  \ifcolorBG
  \psframe[linestyle=none,fillstyle=solid,fillcolor=black](-2,-1.4)(12.5,9)
  \fi
  \fi
  \psframe[linestyle=dotted,dotsep=5pt,linewidth=2pt,linecolor=mellow]%
  (-1,-.5)(11.6,8.3)
  \pspolygon[linestyle=none,fillstyle=solid,%
  fillcolor=mygrey](8.4,8.4)(9.6,8.4)(9,7.4)
  \pspolygon[linestyle=none,fillstyle=solid,%
  fillcolor=red](8.2,8.5)(9.4,8.5)(8.8,7.5)
  \pspolygon[linestyle=none,fillstyle=solid,%
  fillcolor=mygrey](1.4,-1.1)(2.6,-1.1)(2,-.1)
  \pspolygon[linestyle=none,fillstyle=solid,%
  fillcolor=red](1.1,-.9)(2.3,-.9)(1.7,.1)
  \PutLogo % Mandatory
 {#1}}

\NewSlideStyle{t}{5.3,2.9}{TPFrame}
\PDFCroppingBox{10 40 594 800}
\RequirePackage{semhelv}

\endinput
\end{verbatim}
\end{small}


\section{Copyright information}
%===============================
Copyright \copyright{} 2000 by Fr\'ed\'eric Goualard, all rights reserved.
 
\medskip\noindent
Permission is hereby granted, without written agreement and without
license or royalty fees, to use, copy, modify, and distribute this
software and its documentation for any purpose, provided that the
above copyright notice and the following two paragraphs appear in all
copies of this software.
 
\medskip\noindent 
IN NO EVENT SHALL THE AUTHOR BE LIABLE TO ANY PARTY
FOR DIRECT, INDIRECT, SPECIAL, INCIDENTAL, OR CONSEQUENTIAL DAMAGES
ARISING OUT OF THE USE OF THIS SOFTWARE AND ITS DOCUMENTATION, EVEN IF
THE AUTHOR HAS BEEN ADVISED OF THE POSSIBILITY OF SUCH DAMAGE.
 
\medskip\noindent 
THE AUTHOR SPECIFICALLY DISCLAIMS ANY WARRANTIES,
INCLUDING, BUT NOT LIMITED TO, THE IMPLIED WARRANTIES OF
MERCHANTABILITY AND FITNESS FOR A PARTICULAR PURPOSE.  THE SOFTWARE
PROVIDED HEREUNDER IS ON AN ``AS IS'' BASIS, AND THE AUTHOR
HAS NO OBLIGATION TO PROVIDE MAINTENANCE, SUPPORT, UPDATES,
ENHANCEMENTS, OR MODIFICATIONS.

\section{The Prosper homepage}
%=============================

The official Prosper homepage is located at Source Forge (tm):

\begin{center}
\texttt{\href{http://prosper.sourceforge.net/}{http://prosper.sourceforge.net/}}
\end{center}

You will find there additional information, CVS tarballs, news, up to
date distributions of Prosper\dots\ If you plan using Prosper on a
regular basis, you should consider subscribing to the lists
\texttt{prosper-users} and \texttt{prosper-announce}.  Directions to
subscribe to them are available on the homepage.

\bigskip
Prosper is also available on the CTAN repository though the version there
might sometimes be slightly out of date:

\begin{center}
\texttt{\href{http://www.ctan.org/tex-archive/macros/latex/contrib/supported/prosper/}{http://www.ctan.org/tex-archive/macros/latex/contrib/supported/prosper/}}
\end{center}


\section{Troubleshootings}
%=========================

If you experience some problem when installing or using Prosper,
please go first to the Prosper homepage to check whether there is some
hint on how to solve it in one of the list archives. If you do not
find any answer to your problem, send a mail to the
\texttt{prosper-users} list.  Only in the last resort should you send
a mail directly to me. I am a subscriber of this list anyway, so I
will certainly answer to you if nobody else does it.

There is also a file \texttt{TROUBLESHOOTINGS} in the distribution listing
solutions to commonly encountered problems.

Prosper relies on some recent versions of some packages and software
(mainly \texttt{hyperref} and Aladdin GhostScript). Check the homepage
to find links to the required versions.

\section{Bugs reports}
%=====================

Bugs are to be reported by filling the appropriate forms available at
the Prosper homepage.


\section{Getting in touch with the author}
%=========================================

You can send me flames or praises at the following address.

\begin{center}
  \texttt{exupery@users.sourceforge.net}
\end{center}

You might find different email addresses scattered in some files of
the distribution. However, you should not rely on them to get in touch
with me since they are very likely to be out of date at the time you use them.

\printindex
\end{document}

%%% Local Variables: 
%%% mode: latex
%%% TeX-master: t
%%% End: 
