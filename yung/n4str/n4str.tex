%\documentstyle[12pt,epsf]{article}
\documentclass[12pt,epsf]{article}
%\usepackage{graphicx}
\setlength{\unitlength}{1mm}
\textwidth 15.0 true cm
\textheight 22.0 true cm
\headheight 0 cm
\headsep 0 cm
\topmargin 0.4 true in
\oddsidemargin 0.25 true in
\def\lsim{\mathrel{\rlap{\lower3pt\hbox{\hskip0pt$\sim$}}
    \raise1pt\hbox{$<$}}}
%less than or approx. symbol
\def\gsim{\mathrel{\rlap{\lower4pt\hbox{\hskip1pt$\sim$}}
    \raise1pt\hbox{$>$}}}
%greater than or approx. symbol
\begin{document}
%\renewcommand{\theequation}{\thesection.\arabic{equation}}
%
\newcommand{\beq}{\begin{equation}}
\newcommand{\eeq}{\end{equation}}
%
\def\beqn{\begin{eqnarray}}
\def\eeqn{\end{eqnarray}}
%
\newcommand{\qt}{\tilde q}
\newcommand{\Tr}{{\rm Tr}\,}
\newcommand{\E}{{\cal E}}
\newcommand{\qtu}{\tilde q_{1}}
\newcommand{\qtd}{\tilde q_{2}}
\newcommand{\ntwo}{${\cal N}=2\;$}
\newcommand{\none}{${\cal N}=1\;$}
\newcommand{\nfour}{${\cal N}=4\;$}
\newcommand{\vp}{\varphi}
\newcommand{\pt}{\partial}
\renewcommand{\theequation}{\thesection.\arabic{equation}}


%%%%%%%%%%%%%%%%%%%%%%%%%%%%%%%
%%%%%%%%%%%%%%%%%%%%%%%%%%%%%%%
\begin{titlepage}
\renewcommand{\thefootnote}{\fnsymbol{footnote}}











\begin{flushright}
FIAN/TD-??/03\\
ITEP/TH-??/03\\
PNPI/????/03
\end{flushright}


\vfil

\begin{center}
\baselineskip20pt
{\bf \LARGE Non-Abelian Strings in \none * gauge theory}
\end{center}
\bigskip
%\begin{center}
%\baselineskip12pt
%{\large A.~Marshakov}\\
%\medskip
%{\em Theory Department, Lebedev Physics Institute\\
%Institute of Theoretical and Experimental Physics\\ Moscow, Russia}\\
%{\sf e-mail:\ mars@lpi.ru, mars@itep.ru}\\
%\bigskip
%{\large A.~Yung}\\
%\medskip
%{\em Petersburg Nuclear Physics Institute, Gatchina, St.
%Petersburg\\
%Institute of Theoretical and Experimental Physics, Moscow\\ Russia}\\
%{\sf e-mail:\ yung@thd.pnpi.spb.ru}\\
%\end{center}
%\bigskip\bigskip\medskip


\begin{center}
{\large\bf Abstract} \vspace*{.2cm}
\end{center}

\begin{quotation}
We investigate the low energy effective theory and string solutions in
supersymmetric QCD.
\end{quotation}



\vfil
\end{titlepage}

\newpage







\setcounter{footnote}{0}
\setcounter{equation}{0}


\section{Introduction}





\section{The model}

In terms of \none superfields   \nfour gauge theory with
SU(2) gauge group has a vector
multiplet which consist of gauge field $A_{\mu}^a$ and gaugino
$\lambda^{\alpha a}$ and three chiral multiplets $\Phi^a_A$, $A=1,2,3$
all in the adjoint representation of the gauge group, $a=1,2,3$ is the
colour index.
The superpotential of the \nfour gauge theory  reads
\beq
W_{{\cal N}=4}= -\frac{\sqrt{2}}{g^2}
\varepsilon_{abc}\Phi_1^a\Phi_2^b\Phi_3^c.
\label{n4sup}
\eeq

We can deform this theory breaking \nfour supersymmetry down to \ntwo
by adding mass terms
with equal mass $m$ for first two flavors of the adjoint matter
\beq
\label{N2mass}
W_{{\cal N}=2} = {m\over 2g^2} \Phi_1^{a2} +   {m\over 2g^2}
\Phi_2^{a2}.
\eeq
Then the third flavor combines with the vector
multiplet to form a \ntwo vector supermultiplet while first two flavors
represent the adjoint matter.

We can further break the supersymmetry down to \none adding a mass
term  to $\Phi_3$ multiplet
\beq
\label{N1mass}
W_{{\cal N}=1} = {m_3\over 2g^2} \Phi_3^{a2}.
\eeq

The bosonic action of the model with above  mass terms  reads
$$
{1\over g^2}\int d^4 x\left\{ \frac14F_{\mu\nu}^2 + \sum_A \left| D^{\rm
adj}_\mu\ \Phi^a_A\right|^2 +\right.
$$
$$
+\frac12\left[(\bar{\Phi}_A\bar{\Phi}_B)(\Phi^A\Phi^B)-
(\bar{\Phi}_A\Phi^B)(\bar{\Phi}_B\Phi^A)\right]
$$
\beq
\left.
+\left|\frac1{\sqrt{2}}\varepsilon_{abc}\varepsilon^{ABC}\Phi^b_B\Phi^c_C
-m_A\Phi^a_A\right|^2\right\},
\label{su2}
\eeq
where we we use the same notations $\Phi^a_A$ for the scalar components
of the corresponding chiral superfields.

In this paper we are mostly interested  in a particular point
of the parameter space of the theory for which $m_3=m$.
For this value of mass the theory (\ref{su2}) has a symmetric
vacuum
\beq
\label{svac}
\Phi_A^a =\frac{m}{\sqrt{2}}\left(
\begin{array}{ccc}
  1 &
 0& 0 \\
 0 &
1 & 0 \\
  0 & 0 & 1
\end{array}
\right),
\eeq
where $\Phi^a_A$
most conveniently (for the $SU(2)$ gauge group and three \nfour
"flavours") written in terms of the $3\times 3$ colour-flavour matrix.


This vacuum respect global O(3)$_{C+F}$ symmetry
\beq
\Phi\rightarrow O\Phi O^{-1}
\label{c+f}
\eeq
which combine transformations from global color  and flavor groups.
As we will see later this symmetry is responsible for the presence of
non-Abelian strings in the vacuum (\ref{svac}).

Now let us study the mass spectrum of the theory in the vacuum
(\ref{svac})...

\section{U(1) truncation and \ntwo limit}


In fact in the limit of small $m_3$ the mass term  (\ref{N1mass}
does not break \ntwo
supersymmetry \cite{HSZ,VY}. The model reduces to \ntwo QED with
a Fayet-Iliopoulos (FI) term
\cite{FI}. At zero $m_3$ the theory has a Coulomb branch parametrized
by arbitrary VEV of $\Phi^a_3$ which by gauge rotation can
be directed along the third axis in color space,
$<\Phi^a_3>=\delta^{a3}<a>$. Thus the SU(2) group is broken down to
U(1) and the theory becomes essentially Abelian.

The Coulomb branch has a singular points
where some   matter adjoint fields or monopoles/dyons become massless
\cite{SW1,SW2}.  These singular points become isolated \none vacua
once small parameter $m_3$ is introduced.

In this paper we will concentrate at the vacuum in
which adjoint matter becomes massless. At large $m$ this vacuum is
in the weak coupling regime. If we increase $m_3$
and eventualy come to the point $m_3=m$ this vacuum will
evolve to the one in (\ref{svac}).
Now let  see which matter fields become massless in this vacuum
at $m_3=0$. To do so
we analyse the mass matrix of the matter fields given by
superpotentials (\ref{n4sup}), (\ref{N2mass})...


Thus our low energy effective QED besides U(1) gauge multiplet
includes the following mater fields
\beq
\label{phi}
\phi_A^a =\left(
\begin{array}{ccc}
  {g\over 2\sqrt{2}}\left(\chi+\tilde\chi\right) &
-{g\over 2\sqrt{2}i}\left(\chi-\tilde\chi\right) & 0 \\
 {g\over 2\sqrt{2}i}\left(\chi-\tilde\chi\right) &
{g\over 2\sqrt{2}}\left(\chi+\tilde\chi\right) & 0 \\
  0 & 0 & a
\end{array}\right)
\eeq
The normalisation here is chosen to ensure standard kinetic terms
for the charged chiral fields   $\chi$ and $\tilde{\chi}$ which form a
\ntwo matter hypermultiplet of \ntwo QED. The superpotential
(\ref{n4sup}), (\ref{N2mass}) is given by
\beq
W_{QED}=-\sqrt{2}{\chi}a\tilde\chi +m{\chi}\tilde\chi.
\label{qedsup}
\eeq
We see that two flavors of adjoint matter
of \nfour theory form one flavor of  standard
charged hypermultiplet (with charge equal one) in the low energy
effective \ntwo QED.

Now let us introduce small mass term for $a$-field. The bosonic action
of the model takes the form
\beq
\label{qed}
\int d^4 x\left\{{1\over 4g^2}F_{\mu\nu}^2 +
|D_\mu\chi|^2 + | D_\mu\bar{\tilde{\chi}}|^2 +
{1\over g^2}|\pt_\mu a|^2 +
V(\chi,\tilde\chi,a)\right\}
\eeq
with $D_\mu = \pt_\mu - iA^3_\mu$, and the potential
$$
V(\chi,\tilde\chi,a) =
2g^2\left|\chi\tilde\chi -
\frac1{\sqrt{2}g^2}m_3 a\right|^2 + 2\left|a-\frac{m}{\sqrt{2}}
\right|^2\left(
|\chi|^2+|\tilde\chi|^2\right)
$$
\beq
\label{pot}
 +const g^2 \left[|\chi|^2-|\bar{\chi}|^2\right],
\eeq
where the last line comes from the D-terms.

The vacuum of this theory is given by
$$
<a>=\frac{m}{\sqrt{2}},
$$
\beq
<\chi>=<\bar{\tilde{\chi}}>=\frac1{g}\sqrt{\frac{mm_3}{2}}
\label{vac}.
\eeq

To study the mass spectrum of the low energy theory let us analyse
the mass matrix in (\ref{pot}) in the vacuum (\ref{vac})...

This  model possess standard Abrikosov-Nielsen-Olesen
(ANO) U(1) strings \cite{ANO}. In the limit of small
$m_3$, $m_3\ll m$ the addition of the mass term for $\Phi_3$
reduces to FI term \cite{VY} with FI parameter
\beq
\xi=\frac1{g^2}mm_3.
\label{xi}
\eeq
This limit corresponds to expansion of $a$ in the first term in the
potential (\ref{pot}) near its VEV and truncating the series keeping
only the constant term
\beq
\frac1{\sqrt{2}g^2}m_3 a\rightarrow \frac1{2g^2}mm_3=\frac12 \xi.
\label{atoxi}
\eeq

With this truncation the theory (\ref{qed}) is a bosonic part of \ntwo
QED. It has a BPS Abelian ANO  strings. To find the string
with winding number $n=1$ one uses the
{\em ansatz}
$$
\chi = \frac1{\sqrt{2}}\phi(r)e^{i\alpha},
$$
$$
\tilde\chi = \frac1{\sqrt{2}}\phi(r)e^{-i\theta},
$$
$$
a=\frac{m}{\sqrt{2}},
$$
\beq
\label{bpsstr}
A_i = \pt_i\alpha\left(1-f(r)\right),
\eeq
where $r$ and $\alpha$ are polar coordinates in the plane orthogonal
to the string axis. The profile functions in (\ref{bpsstr}) satisfy
the first order equations \cite{B}
$$
r\phi' - f\phi=0,
$$
\beq
-\frac1r f' +g^2(\phi^2-\xi)=0,
\label{foe}
\eeq
where primes denote derivatives with respect to $r$. These equations
 should be supplemented by the boundary conditions
$$
\phi(0) = 0, \ \ \ \ \phi(\infty) = \sqrt{\xi}=\frac1{g^2}\sqrt{
mm_3},
$$
\beq f(0) = 1, \ \ \ \ f(\infty) = 0.
\label{bpsbc}
\eeq

Now let us increase $m_3$. The \ntwo supersymmetry gets broken down to
\none. Still it is clear that the deformed theory has ANO strings.
However they  are not  BPS saturated any longer. In fact what happen is
that short BPS string multiplet of \ntwo SUSY becomes a long
non-BPS string multiplet of \none theory \cite{VY}. The number of
states in the string multiplet remains the same (two bosonic + two
fermionic).

As $m_3$ increases and becomes of order of $m$ the "light" fields which
enter QED (\ref{qed}) become havier.  Eventually their masses ($\sim
\sqrt{mm_3}$) become of the same order as the masses of heavy
non-Abelian fields ( of order of $m$).  Thus the QED
description becomes no longer valid and we should study the full
non-Abelian theory (\ref{su2}). Still we can use the QED truncation
(\ref{qed}) on the classical level to look for the Abelian strings
embedded into non-Abelian theory.

The only modification we have to make is to introduce a new profile
function for the field $a$. This is because the replacement
(\ref{atoxi})
is no longer valid and field $a$ cannot be constant on the string
solution. Our modified {\em ansatz} for the Abelian string
embedded in the non-Abelian theory is
$$
\chi = \frac1{\sqrt{2}}\phi(r)e^{i\alpha},
$$
$$
\tilde\chi = \frac1{\sqrt{2}}\phi(r)e^{-i\theta},
$$
$$
a=a_0(r),
$$
\beq
\label{str}
A_i = \pt_i\alpha\left(1-f(r)\right).
\eeq

The profile functions here satisfy now the second order equations
$$
\phi''+{1\over r}\phi'-{1\over r^2}f^2\phi =
g^2\phi\left(\phi^2-\sqrt{2}\frac{m_3}{g^2} a\right)+4\phi\left(a-
\frac{m}{\sqrt{2}}\right)^2 ,
$$
$$
a''+ {1\over r}a' = \sqrt{2}m_3\left(
\phi^2-\sqrt{2}\frac{m_3}{g^2} a\right)
+4\left(a-\frac{m}{\sqrt{2}}\right)^2 \phi^2,
$$
\beq \label{streq}
f''- {1\over r}f' = 2g^2f\phi^2 .
\eeq
and boundary conditions
$$
\phi(0) = 0, \ \ \ \ \phi(\infty) = \sqrt{\xi}=\frac1{g^2}\sqrt{
mm_3},
$$
$$
a(\infty) = \frac{m}{\sqrt{2}},
$$
\beq
f(0) = 1, \ \ \ \ f(\infty) = 0.
\label{bc}
\eeq

It is worth noting that only elementary strings with winding number
$n=1$ are stable when embedded in the non-Abelian theory. They are
so called $Z_2$ strings which are associated with the center of the
SU(2) gauge group
 \beq \pi_1(SO(3))=Z_2.
\label{top}
\eeq
All strings with multiple winding numbers become unstable
at $m_3\sim m$ \cite{SYmeta}.




\section{Non-Abelian strings}

When $m_3\rightarrow m$ the theory acquires additional symmetry
In this
case a new orientational zero
modes arises, which can be introduced in the
following way.



%\section*{References}
\addcontentsline{toc}{section}{References}

\begin{thebibliography}{99}

\bibitem{Shifman:2002jm}
M.~Shifman and A.~Yung,
%{\em Domain walls and flux tubes in N = 2 SQCD: D-brane prototypes,}
Phys.\ Rev.\ D {\bf 67}, 125007 (2003)
[hep-th/0212293].
%%CITATION = HEP-TH 0212293;%%

\bibitem{SYsu3wall}
M.~Shifman and A.~Yung,
{\em Localization of non-Abelian gauge fields on domain walls at weak coupling
(D-brane prototypes II),}
hep-th/0312257, (Phys. Rev. D, submitted).
%%CITATION = HEP-TH 0312257;%%

\bibitem{Auzzi:2003fs}
R.~Auzzi, S.~Bolognesi, J.~Evslin, K.~Konishi and A.~Yung,
%{\em Nonabelian superconductors: Vortices and confinement in N = 2
%SQCD,}
Nucl.\ Phys.\ B {\bf 673}, 187 (2003)
hep-th/0307287.  %%CITATION = HEP-TH 0307287;%%

\bibitem{Hanany:2003hp}
A.~Hanany and D.~Tong,
%``Vortices, instantons and branes,''
JHEP {\bf 0307}, 037 (2003)
[hep-th/0306150].
%%CITATION = HEP-TH 0306150;%%

\bibitem{Tong:2003pz}
D.~Tong,
{\em Monopoles in the Higgs phase,}
hep-th/0307302.
%%CITATION = HEP-TH 0307302;%%

\bibitem{AdSCFT}
J.~M.~Maldacena,
%``The large N limit of superconformal field theories and supergravity,''
Adv.\ Theor.\ Math.\ Phys.\  {\bf 2}, 231 (1998)
[Int.\ J.\ Theor.\ Phys.\  {\bf 38}, 1113 (1999)]
[hep-th/9711200];
%%CITATION = HEP-TH 9711200;%%
S.~S.~Gubser, I.~R.~Klebanov and A.~M.~Polyakov,
%``Gauge theory correlators from non-critical string theory,''
Phys.\ Lett.\ B {\bf 428}, 105 (1998)
[hep-th/9802109];
%%CITATION = HEP-TH 9802109;%%
E.~Witten,
%``Anti-de Sitter space and holography,''
Adv.\ Theor.\ Math.\ Phys.\  {\bf 2}, 253 (1998)
[hep-th/9802150].
%%CITATION = HEP-TH 9802150;%%

\bibitem{P}
J.~Polchinski,
%``Dirichlet-Branes and Ramond-Ramond Charges,''
Phys.\ Rev.\ Lett.\  {\bf 75}, 4724 (1995)
[hep-th/9510017];
%%CITATION = HEP-TH 9510017;%%
see also the excellent text by J.~Polchinski, {\em String Theory}, Vols. 1 and 2
(Cambridge University Press,  Cambridge, 1998).

\bibitem{DS}
G.~R.~Dvali and M.~A.~Shifman,
%``Domain walls in strongly coupled theories,''
Phys.\ Lett.\ B {\bf 396}, 64 (1997)
(E)\ B {\bf 407}, 452 (1997)
[hep-th/9612128].
%%CITATION = HEP-TH 9612128;%%

\bibitem{Witten:1997ep}
E.~Witten,
%``Branes and the dynamics of {QCD},''
Nucl.\ Phys.\ B {\bf 507}, 658 (1997)
[hep-th/9706109].
%%CITATION = HEP-TH 9706109;%%




\bibitem{SW1}
N.~Seiberg and E.~Witten,
 %``Electric - magnetic duality, monopole
 %condensation, and confinement in N=2
%supersymmetric Yang-Mills theory,''
Nucl. Phys. {\bf B426}, 19 (1994),
(E) {\bf B430},  485 (1994) [hep-th/9407087].

\bibitem{SW2}
N.~Seiberg and E.~Witten,
%``Monopoles, duality and chiral symmetry breaking in N=2
%supersymmetric QCD,''
Nucl. Phys. {\bf B431}, 484  (1994)
[hep-th/9408099].

\bibitem{FI}
P.~Fayet and J.~Iliopoulos,
%``Spontaneously Broken Supergauge Symmetries And Goldstone Spinors,''
Phys.\ Lett.\ B {\bf 51}, 461 (1974).
%%CITATION = PHLTA,B51,461;%%



\bibitem{APS}
P.~Argyres, M.~Plesser and N.~Seiberg,
%``The Moduli Space of N=2 SUSY {QCD} and Duality in
%N=1 SUSY {QCD},'
Nucl. Phys. {\bf B471}, 159  (1996)
[hep-th/9603042].

\bibitem{CKM}
G.~Carlino, K.~Konishi and H.~Murayama,
 %``Dynamical symmetry breaking in supersymmetric
 %SU(n(c)) and USp(2n(c))  gauge
%theories,''
Nucl.\ Phys.\ B {\bf 590}, 37 (2000)
[hep-th/0005076].
%%CITATION = HEP-TH 0005076;%%

\bibitem{MY}
A.~Marshakov and A.~Yung,
%``Non-Abelian confinement via Abelian
%flux tubes in softly broken N = 2  SUSY QCD,''
Nucl.\ Phys.\ B {\bf 647}, 3 (2002)
[hep-th/0202172].
%%CITATION = HEP-TH 0202172;%%

\bibitem{ABEK}
R.~Auzzi, S.~Bolognesi, J.~Evslin and  K.~Konishi,
{\em Non-Abelian monopoles and vortices that confine them,}
hep-th/0312233.

\bibitem{We}
E.~J.~Weinberg,
 %``Fundamental Monopoles And Multi -
 %Monopole Solutions For Arbitrary Simple
%Gauge Groups,''
Nucl.\ Phys.\ B {\bf 167}, 500 (1980);
%%CITATION = NUPHA,B167,500;%%
%``Fundamental Monopoles In Theories With
%Arbitrary Symmetry Breaking,''
Nucl.\ Phys.\ B {\bf 203}, 445 (1982).
%%CITATION = NUPHA,B203,445;%%

\bibitem{thopo}
G.~'t Hooft,
%``Magnetic Monopoles In Unified Gauge Theories,''
Nucl.\ Phys.\ B {\bf 79}, 276 (1974);
%%CITATION = NUPHA,B79,276;%%
A.~M.~Polyakov,
%``Particle Spectrum In Quantum Field Theory,''
Pisma Zh.\ Eksp.\ Teor.\ Fiz.\  {\bf 20}, 430 (1974)
[JETP Lett.\  {\bf 20}, 194 (1974)].
%%CITATION = JTPLA,20,194;%%




\bibitem{HSZ}
A.~Hanany, M.~J.~Strassler and A.~Zaffaroni,
%``Confinement and strings in M{QCD},''
Nucl.\ Phys.\ B {\bf 513}, 87 (1998)
[hep-th/9707244].
%%CITATION = HEP-TH 9707244;%%

\bibitem{VY}
A.~I.~Vainshtein and A.~Yung,
%``Type I superconductivity upon
%monopole condensation in Seiberg-Witten  theory,''
Nucl.\ Phys.\ B {\bf 614}, 3 (2001)
[hep-th/0012250].
%%CITATION = HEP-TH 0012250;%%

\bibitem{BarH}
K.~Bardakci and M.~B.~Halpern,
%``Spontaneous Breakdown And Hadronic Symmetries,''
Phys.\ Rev.\ D {\bf 6}, 696 (1972).
%%CITATION = PHRVA,D6,696;%%



\bibitem{ANO}
A.~Abrikosov, Sov.~Phys. JETP {\bf32} 1442  (1957)
[Reprinted in {\em Solitons and Particles}, Eds. C. Rebbi and G. Soliani
(World Scientific, Singapore, 1984), p. 356];\\
H.~Nielsen and P.~Olesen, Nucl.~Phys. {\bf B61} 45 (1973)
[Reprinted in {\em Solitons and Particles}, Eds. C. Rebbi and G. Soliani
(World Scientific, Singapore, 1984), p. 365].

\bibitem{asrptobe}
A. Ritz, M. Shifman, and A. Vainshtein, to appear.

\bibitem{SYmeta}
M.~Shifman and A.~Yung,
%``Metastable strings in Abelian Higgs models embedded in non-Abelian
%theories: Calculating the decay rate,''
Phys.\ Rev.\ D {\bf 66}, 045012 (2002), [hep-th/0205025].

\bibitem{VS}
H.~J.~de Vega and F.~A.~Schaposnik,
 %``Electrically Charged Vortices In Nonabelian
 %Gauge Theories With Chern-Simons
%Term,''
Phys.\ Rev.\ Lett.\  {\bf 56}, 2564 (1986);
%%CITATION = PRLTA,56,2564;%%
%``Vortices And Electrically Charged Vortices
%In Nonabelian Gauge Theories,''
Phys.\ Rev.\ D {\bf 34}, 3206 (1986).
%%CITATION = PHRVA,D34,3206;%%

\bibitem{HV}
J.~Heo and T.~Vachaspati,
%``Z(3) strings and their interactions,''
Phys.\ Rev.\ D {\bf 58}, 065011 (1998)
[hep-ph/9801455].
%%CITATION = HEP-PH 9801455;%%

\bibitem{Su}
P.~Suranyi,
%``Vortex solutions in SU(N) adjoint Higgs theories,''
Phys.\ Lett.\ B {\bf 481}, 136 (2000)
[hep-lat/9912023].
%%CITATION = HEP-LAT 9912023;%%

\bibitem{SS}
F.~A.~Schaposnik and P.~Suranyi,
%``New vortex solution in SU(3) gauge-Higgs theory,''
Phys.\ Rev.\ D {\bf 62}, 125002 (2000)
[hep-th/0005109].
%%CITATION = HEP-TH 0005109;%%

\bibitem{KB}
M.~A.~C.~Kneipp and P.~Brockill,
%``BPS string solutions in non-Abelian Yang-Mills theories,''
Phys.\ Rev.\ D {\bf 64}, 125012 (2001)
[hep-th/0104171].
%%CITATION = HEP-TH 0104171;%%

\bibitem{KoS}
K.~Konishi and L.~Spanu,
%``Non-Abelian vortex and confinement,''
Int.\ J.\ Mod.\ Phys.\ A {\bf 18}, 249 (2003)
[hep-th/0106175].
%%CITATION = HEP-TH 0106175;%%



%\bibitem{HS}
%Z.~Hlou\v{s}ek and D.~Spector,
%``Why topological charges imply extended supersymmetry,''
%Nucl.\ Phys.\ B {\bf 370}, 143 (1992);
%%CITATION = NUPHA,B370,143;%%
%J.~Edelstein, C.~Nu\~{n}ez and F.~Schaposnik,
%``Supersymmetry And Bogomolny Equations In
%The Abelian Higgs Model,''
%Phys.\ Lett.\ B {\bf 329}, 39 (1994)
%[hep-th/9311055].
%%CITATION = HEP-TH 9311055;%%

%\bibitem{DDT}
%S.~C.~Davis, A.~C.~Davis and M.~Trodden,
%``N = 1 supersymmetric cosmic strings,''
%Phys.\ Lett.\ B {\bf 405}, 257 (1997)
%[hep-ph/9702360].
%%CITATION = HEP-PH 9702360;%%

\bibitem{GS}
A.~Gorsky and M.~A.~Shifman,
%``More on the tensorial central charges
%in N = 1 supersymmetric gauge  theories (BPS wall junctions and strings),''
Phys.\ Rev.\ D {\bf 61}, 085001 (2000)
[hep-th/9909015].
%%CITATION = HEP-TH 9909015;%%









\bibitem{NSVZ}
V.~Novikov, M.~Shifman, A.~Vainshtein,  V.~Zakharov,
 %``Two-Dimensional Sigma Models: Modeling
 %Nonperturbative Effects Of Quantum
%Chromodynamics,''
Phys.\ Reports\  {\bf 116}, 103 (1984).
%%CITATION = PRPLC,116,103;%%






\bibitem{siaf}
A.~M.~Polyakov,
 %``Interaction Of Goldstone Particles In
 %Two-Dimensions. Applications To
%Ferromagnets And Massive Yang-Mills Fields,''
Phys.\ Lett.\ B {\bf 59}, 79 (1975).
%%CITATION = PHLTA,B59,79;%%

\bibitem{Losev}
A.~Losev and M.~Shifman,
 %``N = 2 sigma model with twisted mass
 %and superpotential: Central charges  and
%solitons,''
Phys.\ Rev.\ D {\bf 68}, 045006 (2003)
[hep-th/0304003].
%%CITATION = HEP-TH 0304003;%%

\bibitem{dwano}
G.~R.~Dvali and M.~A.~Shifman,
%``Domain walls in strongly coupled theories,''
Phys.\ Lett.\ B {\bf 396}, 64 (1997), (E)
\ B {\bf 407}, 452 (1997)
[hep-th/9612128];
%%CITATION = HEP-TH 9612128;%%
A.~Kovner, M.~A.~Shifman and A.~Smilga,
%``Domain walls in supersymmetric Yang-Mills theories,''
Phys.\ Rev.\ D {\bf 56}, 7978 (1997)
[hep-th/9706089];
%%CITATION = HEP-TH 9706089;%%
B.~Chibisov and M.~A.~Shifman,
%``BPS-saturated walls in supersymmetric theories,''
Phys.\ Rev.\ D {\bf 56}, 7990 (1997), (E)
\ D {\bf 58}, 109901 (1998)
[hep-th/9706141].
%%CITATION = HEP-TH 9706141;%%


\bibitem{WitIndex}
E.~Witten,
%``Constraints On Supersymmetry Breaking,''
Nucl.\ Phys.\ B {\bf 202}, 253 (1982).
%%CITATION = NUPHA,B202,253;%%

\bibitem{Alvarez}
L.~Alvarez-Gaum\'{e} and D.~Z.~Freedman,
%``Potentials For The Supersymmetric Nonlinear Sigma Model,''
Commun.\ Math.\ Phys.\  {\bf 91}, 87 (1983);
%%CITATION = CMPHA,91,87;%%
S.~J.~Gates,
%``Superspace Formulation Of New Nonlinear Sigma Models,''
Nucl.\ Phys.\ B {\bf 238}, 349 (1984);
%%CITATION = NUPHA,B238,349;%%
S.~J.~Gates, C.~M.~Hull and M.~Ro\v{c}ek,
%``Twisted Multiplets And New Supersymmetric
%Nonlinear Sigma Models,''
Nucl.\ Phys.\ B {\bf 248}, 157 (1984).
%%CITATION = NUPHA,B248,157;%%

\bibitem{HoVa}
K.~Hori and C.~Vafa,
{\em Mirror symmetry,}
hep-th/0002222, unpublished.
%%CITATION = HEP-TH 0002222;%%

\bibitem{CFIV}
S.~Cecotti, P.~Fendley, K.~A.~Intriligator and C.~Vafa,
%``A New supersymmetric index,''
Nucl.\ Phys.\ B {\bf 386}, 405 (1992)
[hep-th/9204102];
%%CITATION = HEP-TH 9204102;%%
P.~Fendley and K.~A.~Intriligator,
%``Scattering and thermodynamics of fractionally
%charged supersymmetric solitons,''
Nucl.\ Phys.\ B {\bf 372}, 533 (1992)
[hep-th/9111014];
%%CITATION = HEP-TH 9111014;%%
S.~Cecotti and C.~Vafa,
%``On classification of N=2 supersymmetric theories,''
Commun.\ Math.\ Phys.\  {\bf 158}, 569 (1993)
[hep-th/9211097].%%CITATION = HEP-TH 9211097;%%



\bibitem{Dorey}
N.~Dorey,
%``The BPS spectra of two-dimensional
%supersymmetric gauge theories
%with  twisted mass terms,''
JHEP {\bf 9811}, 005 (1998) [hep-th/9806056].
%%CITATION = HEP-TH 9806056;%%

\bibitem{B}
E.~B.~Bogomolny,
%``Stability Of Classical Solutions,''
Yad.\ Fiz.\  {\bf 24}, 861 (1976) [Sov.\ J.\ Nucl.\ Phys.\
{\bf 24}, 449 (1976),
reprinted in {\em Solitons and Particles}, Eds. C. Rebbi and G. Soliani
(World Scientific, Singapore, 1984), p. 389].
%%CITATION = SJNCA,24,449;%%



\bibitem{AV}
A.~Achucarro and T.~Vachaspati,
Phys.\ Rep.\  {\bf 327}, 347 (2000)
[hep-ph/9904229]

\bibitem{HaHo}
A.~Hanany and K.~Hori,
%``Branes and N = 2 theories in two dimensions'',
Nucl.\ Phys.\ B {\bf 513}, 119 (1998) [hep-th/9707192].
%%CITATION = HEP-TH 9707192;%%


\bibitem{AD}
P.~C.~Argyres and M.~R.~Douglas,
%``New phenomena in SU(3) supersymmetric gauge theory,''
Nucl.\ Phys.\ B {\bf 448}, 93 (1995)
[hep-th/9505062].
%%CITATION = HEP-TH 9505062;%%

\bibitem{BF}
A.~Bilal and F.~Ferrari,
 %``The BPS spectra and superconformal
 %points in massive N = 2  supersymmetric
%{QCD},''
Nucl.\ Phys.\ B {\bf 516}, 175 (1998)
[hep-th/9706145].
%%CITATION = HEP-TH 9706145;%%



\end{thebibliography}





\end{document}



