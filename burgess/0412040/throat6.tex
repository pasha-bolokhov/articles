\documentclass[12pt]{JHEP3}

\usepackage{epsfig}
\epsfclipon
\usepackage{multicol}
%\usepackage{epsfig,bbm,bm}
\usepackage{epsfig,bm}
\usepackage{amssymb,amsmath}

\newcommand{\mathbold}[1]{\mbox{\boldmath $#1$}}

\newcommand{\roughly}[1]{\mathrel{\raise.3ex\hbox{$#1$\kern-0.85em
\lower1ex\hbox{$\sim$}}}}

\newcommand{\lsim}{\roughly<}
\newcommand{\gsim}{\roughly>}
\newcommand{\sss}{\scriptscriptstyle}
%\usepackage{epsfig,subfigure}
%\documentstyle[12pt,amsfonts]{article}
%\input epsf.tex
\renewcommand{\topfraction}{0.8}
\renewcommand{\bottomfraction}{0.8}
\renewcommand{\baselinestretch}{1.3}
% shortcuts
\def\beq{\begin{equation}}
\def\eeq{\end{equation}}
\def\beqa{\begin{eqnarray}}
\def\eeqa{\end{eqnarray}}
\def\sfrac#1#2{\textstyle\frac#1#2}
\def\IK{\relax{\rm I\kern-.20em K}}
\def\IM{\relax{\rm I\kern-.15em M}}
\def\KKLMMT{{\IK L\IM T}}
\title{Warped Reheating in Brane-Antibrane Inflation}

\author{N.\ Barnaby,${}^1$ C.P.\ Burgess${}^{1,2,3}$ and J.M.\ Cline${}^1$\\
${}^1$ Physics Department, McGill University, 3600 University
Street\\
~~Montr{\'e}al, Qu{\'e}bec, Canada, H3A 2T8. \\
${}^2$ Department of Physics and Astronomy, McMaster University,\\
~~Hamilton, Ontario, Canada.\\
${}^3$ Perimeter Institute, Waterloo, Ontario, Canada. }


%in the sdocstyle case, use \abstact{    }

\abstract{We examine how reheating occurs after brane-antibrane
inflation in warped geometries, such as those which have recently
been considered for Type IIB string vacua. We adopt the standard
picture that the energy released by brane annihilation is
dominantly dumped into massive bulk (closed-string) modes which
eventually cascade down into massless particles, but argue that
the this need not mean that the result is mostly gravitons with
negligible visible radiation on the Standard Model brane. Rather,
in geometries having strongly warped throats we argue that this
energy can instead be predominantly dumped into massless modes
which are localized at the throat's tip, because of the
exponential growth of the massive Kaluza-Klein wave functions
there. The result is preferentially to reheat those branes which
lay in the most strongly warped throats. We argue that the
efficiency of this process removes a conceptual obstacle to the
construction of multi-throat models, wherein inflation occurs in a
different throat than the one in which the Standard Model brane
resides. Such multi-throat models are desirable because they may
resolve a difficulty in reconciling inflation with the
supersymmetry breaking scale on the Standard Model brane, and
because they may allow cosmic strings to be sufficiently
long-lived to be observable during the present epoch.}


%\keywords{Strings, Inflation, Cosmology} \preprint{}

%\preprint{McGill-04-xx\\  hep-th/0412040}


\begin{document}


%----------------------------------------------------------------------%
%  numbering equations with section number
%----------------------------------------------------------------------%
\makeatletter \@addtoreset{equation}{section} \makeatother
\renewcommand{\theequation}{\thesection.\arabic{equation}}
%----------------------------------------------------------------------%
%  title page
%----------------------------------------------------------------------%


%----------------------------------------------------------------------%
%  Resetting of counters
%----------------------------------------------------------------------%
\setcounter{page}{1} \pagestyle{plain}
\renewcommand{\thefootnote}{\arabic{footnote}}
\setcounter{footnote}{0}
%----------------------------------------------------------------------%
%  Paper begins
%----------------------------------------------------------------------%

%\end{document}


\section{Introduction}

There has been significant progress over the past years towards
the construction of {\it bona fide} string-theoretic models of
inflation. The main progress over early string-inspired
supergravity \cite{SugraInflation} and BPS-brane based
\cite{DvaliTye} models has come due to the recognition that
brane-antibrane \cite{BI1,BBbarInflation} and related
\cite{Angles,Others} systems can provide {\it calculable}
mechanisms for identifying potentially inflationary potentials.
Even better, they can suggest new observable signatures, such as
the natural generation of cosmic strings by the brane-antibrane
mechanism \cite{BI1,cosmicstrings}. The central problem to emerge
from these early studies was to understand how the many string
moduli get fixed, since such an understanding is a prerequisite
for a complete inflationary scenario.

Recent developments are based on current progress in
modulus stabilization within warped geometries with background
fluxes for Type IIB vacua \cite{GKP,Sethi,kklt}. Both
brane-antibrane inflation \cite{kklmmt} and modulus inflation
\cite{racetrackinflation} have been embedded into this context,
with an important role being played in each case by branes living
in strongly-warped `throat-like' regions within the extra
dimensions. These inflationary scenarios have generated
considerable activity \cite{OtherStringInflation,BIReviews}
because they open up the possibility of asking in a more focused
way how string theory might address the many issues which arise
when building inflationary models. For instance, one can more
fully compute the abundance and properties of any residual cosmic
strings which might survive into the present epoch
\cite{stringycosmicstrings}. Similarly, the possibility of having
quasi-realistic massless particle spectra in warped, fluxed Type
IIB vacua \cite{cmqu} opens up the possibility of locating where
the known elementary particles fit into the post-inflationary
world \cite{bcqs}, a prerequisite for any understanding of
reheating and the subsequent emergence of the Hot Big Bang.

Even at the present preliminary level of understanding, a
consistent phenomenological picture seems to require more
complicated models involving more than a single throat (in
addition to the orientifold images).\footnote{Two-throat models
are also considered for reasons different than those given here in
ref.~\cite{twothroat}.} This is mainly because for the
single-throat models the success of inflation and particle-physics
phenomenology place contradictory demands on the throat's warping.
They do so because the energy scale in the throat is typically
required to be of order $M_i \sim 10^{15}$ GeV to obtain
acceptably large temperature fluctuations in the CMB. But as was
found in ref.~\cite{bcqs}, this scale tends to give too large a
supersymmetry breaking scale for ordinary particles if the
Standard Model (SM) brane resides in the same throat. This problem
appears to be reasonably generic to the KKLT-type models discussed
to date, because these models tend to have supersymmetric anti-de
Sitter vacua until some sort of supersymmetry-breaking physics is
added to lift the vacuum energy to zero. The problem is that the
amount of supersymmetry-breaking required to zero the vacuum
energy also implies so large a gravitino mass that it threatens to
ruin the supersymmetric understanding of the low-energy
electroweak hierarchy problem.

No general no-go theorem exists, however, and there does appear to
be considerable room to try to address this issue through more
clever model-building. Ref.~\cite{RA} provides a first step in
this direction within the the framework of `racetrack' inflation
\cite{racetrackinflation}. Another possibility is a picture having
two (or more) throats, with inflation arising because of
brane-antibrane motion in one throat but with the Standard Model
situated in the other (more about this proposal below). By
separating the scales associated with the SM and inflationary
branes in this way, it may be possible to reconcile the
inflationary and supersymmetry-breaking scales with one another.

Besides possibly helping to resolve this problem of scales,
multi-throat models could also help ensure that string defects
formed at the end of inflation in the inflationary throat have a
chance of surviving into the present epoch and giving rise to new
observable effects \cite{stringycosmicstrings}. They are able to
do so because if the Standard Model were on a brane within the
same throat as the inflationary branes, these defects typically
break up and disappear by intersecting with the SM brane.

At first sight, however, any multi-throat scenario seems likely to
immediately founder on the rock of reheating.\footnote{See
ref.~\cite{braneheat} for a discussion of issues concerning
brane-related reheating within other contexts.} Given the absence
of direct couplings between the SM and inflationary branes, and
the energy barrier produced by the warping of the bulk separating
the two throats, one might expect the likely endpoint of
brane-antibrane annihilation to be dump energy only into
closed-string, bulk modes, such as gravitons, rather than visible
degrees of freedom on our brane. In such a universe the energy
which drove inflation could be converted almost entirely into
gravitons, leaving our observable universe out in the cold.

It is the purpose of the present work to argue that this picture
is too pessimistic, because strongly-warped geometries provide a
generic mechanism for channelling the post-inflationary energy
into massless modes localized on the throat having the strongest
warping. They can do so because the massive bulk Kaluza-Klein (KK)
modes produced by brane-antibrane annihilation prefer to decay
into massless particles which are localized on branes within
strongly-warped throats rather than to decay to massless bulk
modes. As such, they open a window for obtaining acceptable
reheating from brane-antibrane inflation, even if the inflationary
and SM branes are well separated on different throats within the
extra dimensions.

The remainder of the paper is organized as follows. In \S2 we
introduce a simple generalization of the Randall-Sundrum (RS)
model \cite{RS} containing two AdS$_5$ throats with different warp
factors, as a tractable model for the \KKLMMT\ inflationary
scenario \cite{kklmmt} with two throats.  Here we recall the form
of the KK graviton wave functions in the extra dimension. This is
followed in \S3 by an account of how the tachyonic fluid
describing the unstable brane-antibrane decays into excited
closed-string states, which quickly decay into KK gravitons. \S4
Gives an estimate of the reheating temperature on the SM brane
which results from the preferential decay of the KK gravitons into
SM particles. Our conclusions are given in \S5.

\section{Tale of Two Throats}

We wish to describe reheating in a situation where brane-antibrane
inflation occurs within an inflationary throat having an energy
scale of $M_i$, due to the warp factor $a_i = M_i/M_p$, where
$M_p$ is the 4D Planck mass. This throat is assumed to be
separated from other, more strongly warped, throats by a weakly
warped Giddings-Kachru-Polchinski (GKP) manifold \cite{GKP} whose
volume is only moderately larger than the string scale, so $M_s
\lsim M_p$. In the simplest situation there are only two throats
(plus their orientifold images), with the non-inflationary
(Standard Model) throat having warp factor $a_{sm} \ll a_i$.

There are two natural choices for the SM warp factor, depending on
whether or not the SM brane strongly breaks 4D supersymmetry. For
instance, if the SM resides on an anti-D3 brane then supersymmetry
is badly broken and the SM warp factor must describe the
electroweak hierarchy {\it \`a la} Randall and Sundrum \cite{RS}, with
$a_{sm} \sim M_W/M_p \sim 10^{-16}$. Alternatively, if the SM
resides on a D3 or D7 brane which preserves the bulk's $N=1$
supersymmetry in 4D, then SUSY breaking on the SM brane is
naturally suppressed by powers of $1/M_p$ because it is only
mediated by virtual effects involving other SUSY-breaking anti-D3
branes. In this case the electroweak hierarchy might instead be
described by an intermediate-scale scenario \cite{IntScale}, where
$a_{sm} \sim M_{int}/M_p \sim (M_W/M_p)^{1/2} \sim 10^{-8}$.

A potential problem arises with the low-energy field theory
approximation if $a_{sm} < a_i^2$, because in this case the string
scale in the SM throat, $M_{sm} \sim a_{sm} M_p$, is smaller than
the inflationary Hubble scale $H_i \sim M_i^2/M_p \sim a_i^2 M_p$
\cite{Rob}. In this case string physics is expected to become
important in the SM throat, and stringy corrections may
change the low-energy description. The intermediate
scale is somewhat more attractive from this point of view, since
for it the field-theory approximation may be justified if the
underlying GKP geometry's volume is somewhat larger than the
string scale. In this case one instead has $M_s \sim f \, M_p$,
for $f < 1$, and so $M_i \sim f a_i M_p$, $M_{sm} \sim f a_{sm}
M_p$ and $H_i \sim a_i^2 f^2 M_p$. This implies $H_i/M_{sm} \sim f
(a_i^2/a_{sm})$, and so small $f$ can ensure $H_i \ll M_{sm}$ even
if $a_i^2 \sim a_{sm}$.

To proceed we use the fact that within the GKP compactification the
geometry within the throat is well approximated by
%
\beq \label{RS}
    ds^2 = a^2(y)(dt^2 - dx^2) - dy^2 - y^2 \, d\Omega^2_5 \,,
\eeq
%
where $y$ represents the proper distance along the throat, $a(y) =
e^{-k|y|}$ is the throat's warp factor and $d\Omega^2_5$ is the
metric on the base space of the corresponding conifold singularity
of the underlying Calabi-Yau space \cite{throatmetric}. Of most
interest is the 5D metric built from the observable 4 dimensions
and $y$, which is well approximated by the metric of 5-dimensional
anti-de Sitter space.

A simple model of the two-throat situation then consists of
placing inflationary brane-antibranes in a throat at $y=-y_i$ and
putting the Standard Model brane at $y=+y_{sm}$, as is illustrated
in Fig.~\ref{fig1}. Our analysis of this geometry follows the
spirit of ref.~\cite{dkkls}. Since most of the interest is in the
throats, we simplify the description of the intervening bulk
geometry by replacing them with a Planck brane at $y=0$, with the
resulting discontinuity in the derivative of $a(y)$ chosen to
reproduce the smoother (but otherwise similar) change due to the
weakly-warped bulk. This approximation is illustrated in
Fig.~\ref{fig2}, with the smooth dashed curve representing the
warp factor in the real bulk geometry and the solid spiked curve
representing the result using an intervening Planck brane instead.

\DOUBLEFIGURE[ht]{manifold.eps, width=1.2\hsize} {warping.eps,
width=1.2  \hsize}{A Type IIB vacuum with a mildly warped
inflationary throat and a strongly warped Standard Model throat.
This diagram suppresses any image throats arising due to any
orientifolds which appear in the compactification.\label{fig1}}
{The warp factor as function of a bulk radial coordinate in a
simplified model of two asymmetric throats. As shown in the
figure, the part of the internal space outside of the throats can
be regarded as a regularization of a `Planck' brane of a
Randall-Sundrum geometry.\label{fig2}}

Of particular interest in what follows are the massive
Kaluza-Klein modes in the bulk, since these are arguably the most
abundantly-produced modes after brane-antibrane annihilation. For
instance, focussing on the 5 dimensions which resemble AdS space
in the throat, a representative set of metric fluctuations can be
parameterized as $h(x,y)$ in the line element,
%
\beq \label{RSh}
    ds^2 = a^2(y)(dt^2 - dx^2 + h_{\mu\nu} dx^\mu dx^\nu) - dy^2
    \,.
\eeq
%
In the static AdS background, the KK modes have spatial
wavefunctions of the form
%
\beq
    h(x,y) = \sum_n\phi_n(y) \; e^{ip\cdot x}
\eeq
%
with $p\cdot x = - E_n t + {\bf p} \cdot {\bf x}$, and $\phi_n(y)$
satisfying the equation of motion
%
\beq
    -{d\over dy}\left(e^{-4k|y|} \,{d\phi_n\over dy}\right)
    = m^2_n e^{-2k|y|}\phi_n \,.
\eeq
%
Here $m_n^2 = p \cdot p$ is the mode's 4D mass as viewed by
brane-bound observers.

Exact solutions for $\phi_n(y)$ are possible in the Planck-brane
approximation \cite{RS,dkkls,PlanckBraneModes}, and are linear
combinations of Bessel functions times an exponential
%
\beq \label{modefunctions}
    \phi_n(y) = N_n \, e^{2k|y|}
    \left[ J_2\left({m_n\over k}e^{k|y|}\right)
    + {b_n} \,
    Y_2\left({m_n\over k}e^{k|y|}\right) \right]
 \,
\eeq
%
where, for low lying KK modes ($m_n \ll k$) one has
%
\beq
    b_n \cong \frac{\pi m_n^2}{4 k^2}
\eeq
%
while for heavy KK modes ($m_n / k \cong 1$) one has
%
\beq
    b_n \cong -0.47 + 1.04 \left( \frac{m_n}{k} \right).
\eeq
%
$N_n$ is determined by the orthonormality condition, which ensures
that the kinetic terms of the KK modes are independent of
$a_{sm}$:
%
\beq
    \int_{-y_i}^{y_{sm}} dy\, e^{-2k|y|} \phi_n \phi_m =
    \delta_{nm}\,.
\eeq
%
These wavefunctions are graphed schematically in Fig.~\ref{fig3}.

For strongly-warped throats it is the exponential dependence which
is most important for the KK modes. Because of the exponential
arguments of the Bessel functions in eq.~(\ref{modefunctions}),
the presence of the Bessel functions modifies the large-$y$
behaviour slightly. Due to the asymptotic forms $J_2(z) \propto
z^{-1/2}$ for large $|z|$, and similarly for $Y_2(z)$, we
see that $\phi_n(y) \sim e^{3k|y|/2}$ for $m_n e^{k|y|} \gg k$. It
is only this behaviour which we follow from here on. Taking the
most warping to occur in the SM throat we find that $m_n$ is
approximately quantized in units of $M_{sm} \equiv a_{sm} M_p$,
which is either of order $M_W \sim 10^3$ GeV or $M_{int} \sim
10^{10}$ GeV depending on whether or not supersymmetry breaks on
the SM brane. Keeping only the exponentials we find that
orthonormality requires $N_n^{-2} \sim \int^{y_{sm}} dy \;
e^{-2ky} \, \left( e^{3k |y|/2} \right)^2 \sim (k \,
a_{sm})^{-1}$, and so
%
\beq
\label{lownwf}
    \phi_n(y) \sim  ({a_{sm}\, k})^{1/2}\, e^{3k|y|/2} \,,
\eeq
%
showing that these modes are strongly peaked deep within the
throat. This is intuitively easy to understand, since being
localized near the most highly-warped region allows them to
minimize their energy most effectively.

Thus, even the most energetic KK modes still have exponentially
larger wave functions on the TeV brane, with the more energetic
modes reaching the asymptotic region for smaller $y$. This is
illustrated in Figure \ref{fig5}, which shows $\ln|\phi_n(y)|$
versus $y$ in the representative case of a throat having warp
factor $a = e^{-10}$, for a series of KK states with masses going
as high as $M_p$ ($n=20,000$). As the figure shows, the wave
functions grow exponentially toward the TeV brane, with the onset
of the asymptotic exponential form setting in earlier for larger
mode number.\footnote{For the lowest-lying modes having the
smallest nonzero masses it can happen that the asymptotic form of
the Bessel functions is not yet reached even when $y = y_{sm}$, in
which case the exponential peaking is slightly stronger than
discussed above.} This behaviour is central to the estimates which
follow.

Among the KK modes it is the zero modes which are the exceptional
case because their wavefunction is constant, $\phi_0 \sim
\sqrt{k}$, and so they are not exponentially peaked inside the
throat. It is the strong exponential peaking of the lightest
massive KK modes relative to the massless modes which is central
to the reheating arguments which follow.

\DOUBLEFIGURE[ht]{wavefunc.eps, width=6cm}{wf3.eps,
width=6cm}{Wave functions of KK gravitons on the internal
space.\label{fig3}}{Unnormalized wave functions for highly excited
KK gravitons with KK numbers $n=1$, 100, 1000 and
20,000.\label{fig5}}
%\clearpage

\EPSFIGURE[ht]{orbifold.eps, width=6cm}{How to place two throats
on an $S_1$ with $Z_2$ orbifold symmetry.\label{fig4}}
%\clearpage

In our simplified model, the presence of two throats is not much
more complicated than the original RS model.  Mathematically it is
the same, except that RS identified the two sides
$y\leftrightarrow -y$ through orbifolding.  Instead we interpret
them as two separate throats with different depths defined by the
brane locations $-y_i$ and $+y_{sm}$.  One can imagine doubling
this entire system on $S_1$ and orbifolding as shown in
Fig.~\ref{fig4}, to define the boundary conditions on the metric
and its perturbations at the infrared branes. In this figure, the
orbifold identification acts horizontally so that the inflation
and SM branes are distinct fixed points.

\section{Brane-Antibrane Annihilation}

In brane-antibrane inflation the energy released during reheating
is provided by the tensions of the annihilating branes. Although
this annihilation process is not yet completely understood,
present understanding indicates that the energy released passes
through an intermediate stage involving very highly-excited string
states, before generically being transferred into massless
closed-string modes. The time frame for this process is expected
to be the local string scale.

For instance many of the features of brane-antibrane annihilation
are believed to be captured by the dynamics of the open-string
tachyon which emerges for small separations for those strings that
stretch between the annihilating branes.\footnote{See, however,
ref.~\cite{BraneRad} for a discussion of an alternative mechanism
for which the relevant highly-excited strings are open strings,
but for which the annihilation energy nonetheless eventually ends
up in massless closed string modes.} In flat space and at zero
string coupling ($g_s=0$), the annihilation instability has been
argued to be described by the following tachyon Lagrangian
\cite{Sen}
%
\beq \label{Sen}
    {\cal L}_T = -2\tau_0\, e^{-|T|^2/l_s^2} \sqrt{1-|\partial_\mu T|^2}
\eeq
%
where $T$ is the complex tachyon field, $\tau_0$ is the tension of
either of the branes, and $l_s$ is the string length scale. During
inflation, when the brane and anti-brane are well separated, $\dot
T = 0$ and the pressure of the system $p_i$ is simply the negative
of the tension of the two branes, $p_i=-\rho_i$, while afterward
$\dot T\to 1$ and $p_i\to 0$. In this description the pressureless
tachyonic fluid would dominate the energy density of the universe
and lead to no reheating whatsoever.

However, for nonvanishing $g_s$, the time evolution of the tachyon
fluid instead very quickly generates highly excited closed-string
states \cite{BraneDecay,Sen-review}. For D$p$ systems with $p>2$
the rate of closed string production in this process is formally
finite, whereas it diverges for $p\le 2$ (and so passes beyond the
domain of validity of the calculation). This divergence is
interpreted to mean that for branes with $p\le 2$ all of the
energy liberated from the initial brane tensions goes very
efficiently into closed string modes. For spatially homogeneous
branes with $p>2$ the conversion is less efficient and so can be
dominated by other, faster processes. In particular, it is
believed that these higher-dimensional branes will decay more
efficiently inhomogeneously, since they can then take advantage of
the more efficient channels which are available to the
lower-dimensional branes. For example a D3 brane could be regarded
as a collection of densely packed but smeared-out D0 branes, each
of which decays very efficiently into closed strings. Since the
decay time is of order the local string scale, $l_s = 1/M_s$, the
causally-connected regions in this kind of decay are only of order
$l_s$ in size, and so have a total energy of order the brane
tension $\tau_0 \sim M_s^4/g_s$.

These flat-space calculations also provide the distribution of
closed-string states as a function of their energy. The energy
density deposited by annihilating D3-branes into any given string
level is of order $M_s^4$, and so due to the exponentially large
density of excited string states the total energy density produced
is dominated by the most highly-excited states into which decays
are possible. Since the available energy density goes like $1/g_s$
the typical closed-string state produced turns out to have a mass
of order $M_s/g_s$, corresponding to string mode numbers of order
$N \sim 1/g_s$. On the other hand, the momentum transverse to the
decaying branes for these states turns out to be relatively small,
$p_{\sss T} \sim M_s/\sqrt{g_s}$ \cite{BraneDecay,Sen-review}, and
so the most abundantly produced closed-string states are
nonrelativistic.

How do these flat-space conclusions generalize to the warped Type
IIB geometries which arise in string inflationary models? If the
annihilating 3-branes are localized in the inflationary throat,
then the tension of the annihilating branes is of order $\tau_0
\sim (a_i M_s)^4/e^{\phi_i}$, where $\phi_i$ denotes the value
taken by the dilaton field at the throat's tip. The highly-excited
closed-string states that are produced in this way live in the
bulk, with the energy density produced being dominated by those
whose masses are of order $a_i M_s/e^{\phi_i}$. Once produced,
these closed-string bulk modes decay down to lower energies and,
as might be expected from phase space arguments, most of them
typically drop down to massless string states very quickly. An
important exception to this would arise for those states carrying
the most angular momentum at any given string mass level, since
these must cascade more slowly down to lower energies in order to
lose their angular momentum \cite{HiJCascade}. However these seem
unlikely to be produced in appreciable numbers by brane-antibrane
annihilation.

We are led in this way to expect that the annihilation energy is
distributed relatively quickly amongst massless string states, or
equivalently to KK modes of the higher-dimensional supergravity
which describes these states. Although the initial massive string
modes would be nonrelativistic, with $M \sim a_i M_s/e^{\phi_i}$
and $p_{\sss T} \sim a_i M_s/e^{\phi_i/2}$, the same need not be
true for the secondary string states produced by their decay,
whose masses are now of order the KK mass scales. Consequently
these states may be expected not to remain localized in the
inflationary throat, and so if the extra dimensions are not too
large compared with the string scale these modes would have time
to move to the vicinity of the SM throat before decaying further.
Once there, they would be free to fall into the potential wells
formed by the throats as their energy is lost by subsequent decays
into lower-energy levels.

This physical picture is supported by the exponential peaking of
the KK-mode wave-functions in the most deeply-warped throats. In
order to estimate the efficiency with which energy can be
transferred amongst KK modes, we can use the approximate behavior
of the wave functions given in the previous section to keep track
of powers of the throat's warp factor, $a_{sm}$. For instance,
consider the trilinear vertex among 3 KK states having mode
numbers $n_1$, $n_2$ and $n_3$ which is obtained by dimensionally
reducing the higher-dimensional Einstein-Hilbert action,
$\sqrt{g}R$. Keeping in mind that $\sqrt{g} g^{\mu\nu} \propto
a^2$ and that $\psi_n \propto a_{sm}^{1/2}/a^{3/2}$ for the
nonzero modes (eq.\ (\ref{lownwf})), we find that the trilinear
vertex involving $0 \le r \le 3$ massive KK modes (and $3-r$
massless KK modes) has the following representative estimate
%
\beqa
    {\cal L}_{\rm int} &\sim& \int^{y_{sm}}_{-y_i} dy \, \sqrt{g}
    \, g^{\mu\nu} g^{\alpha\beta} g^{\kappa\sigma} g^{\rho\delta}
    h_{\alpha\kappa} \partial_\mu h_{\sigma\rho} \partial_\nu
    h_{\beta\delta} \nonumber \\
    &\sim& \int^{y_{sm}}_{-y_i} dy\, e^{-2k|y|}\, \eta^{\mu\nu}
    \psi_{n_1}(x,y)\,
    \partial_\mu \psi_{n_2}(x,y)\, \partial_\nu
    \psi_{n_3}(x,y) \nonumber\\
    &\sim&   \psi_{n_1}(x)\, \psi_{n_2}(x)\, \psi_{n_3}(x)\,
     p_2\!\cdot\!p_3\; a_{sm}^{r/2} \,
     \int^{y_{sm}}_{-y_i} dy\, e^{-2k|y|}
     (e^{3k|y|/2})^r \nonumber\\
    &\sim& \psi_{n_1}(x)\,
    \psi_{n_2}(x)\,\psi_{n_3}(x)\, \left(
    \frac{p_2\!\cdot\!p_3}{k}
    \right) \; a_{sm}^{\eta} \,
\eeqa
%
where
%
\beq
    \eta = 2-r \qquad \hbox{if} \quad r \ge 2, \qquad \hbox{and}
    \qquad \eta = \frac{r}{2} \qquad \hbox{if} \quad r = 0,1  \,.
\eeq
%
Notice for this estimate that since derivatives in the
compactified directions are proportional to $g^{mn}$ rather than
$g^{\mu\nu}$, they suffer from additional suppression by powers of
$a = e^{-ky}$ within the throat. Here $m$, $n$ label the internal
directions perpendicular to the large $3+1$-dimensional Minkowski
space.

Thus a trilinear interaction amongst generic KK modes ($r=3$),
even those with very large $n$, is proportional to $1/(a_{sm} k)
\sim 1/M_{sm}$, and so is only suppressed by inverse powers of the
low scale. Similarly, $r=2$ processes involving two massive KK
modes $B$ and $B'$, and one massless bulk mode $ZM$ --- such as
the reaction $B \to B' + ZM$ --- are $\propto 1/k \sim 1/M_p$ and
so have the strength of 4D gravity inasmuch as they are Planck
suppressed. The same is also true of the $r=0$ couplings which
purely couple the zero modes amongst themselves.\footnote{The
appendix shows that this agrees with the size of the couplings
found in the effective 4D supergravity lagrangian which describes
the zero-mode and brane couplings.} Finally, those couplings
involving only a single low-lying massive mode and two zero modes
($r = 1$) --- such as for $B \to ZM + ZM'$
--- are proportional to $a_{sm}^{1/2}/k \sim (M_{sm}/M_p^3)^{1/2}$
and so are even weaker than Planck-suppressed.

Similar estimates may also be made for the couplings of the
generic and the massless KK modes to degrees of freedom on a brane
sitting deep within the most strongly-warped throat. Using the
expressions $\phi_0(y_{sm}) \sim 1$ and $\phi_n(y_{sm}) \sim
1/a_{sm}$ for massless and massive KK modes respectively, this
gives:
%
\beqa
    {\cal L}_i &=& M_p^{-1}\left({h_{\mu\nu}^{(0)}\phi_0(y_{sm})} +
    \sum_n{h_{\mu\nu}^{(n)}\phi_n(y_{sm})}\right)
    T^{\mu\nu}_{sm} \nonumber\\
     &\sim& \left({h_{\mu\nu}^{(0)}\over M_p} +
    \sum_n{h_{\mu\nu}^{(n)}\over M_{sm}}\right) T^{\mu\nu}_{sm}
    \,.
\eeqa
%
We see here the standard Planck-suppressed couplings of the
massless modes (such as the graviton) as compared with the
$O(1/M_{sm})$ couplings of the massive KK modes.

The picture which emerges is one for which the energy released by
brane-antibrane annihilation ends up distributed among the massive
KK modes of the massless string states. Because the wavefunctions
of these modes tend to pile up at the tip of the most warped (SM)
throat, their couplings amongst themselves --- and their couplings
with states localized on branes in this throat --- are set by the
low scale $M_{sm}$ rather than by $M_p$. Furthermore, because the
${\cal O}(1/M_{sm})$ couplings to the massless modes on the SM
branes are much stronger than the Planck-suppressed couplings to
the massless bulk modes, we see that the ultimate decay of these
massive KK modes is likely to be into brane states. Thus the final
production of {\it massless} KK zero modes is likely to be
practically nil.\footnote{Detailed calculations of these rates
would be worthwhile, in particular to verify the extent to which
resonant enhancement of various bulk modes along the lines found
in ref.~\cite{RobnJoe} can overwhelm the exponential growth we
track here.} Although we make the argument here for gravitons, the
same warp-counting applies equally well to the other fields
describing the massless closed-string sector.

In summary, we see that strong warping can provide a mechanism for
dumping most of the energy released by the decay of the unstable
brane-antibrane system into massless modes localized on branes
localized at the most strongly-warped throat, regardless of
whether the initial brane-antibrane annihilation is located in
this throat. It does so because the primary daughter states
produced by the decaying brane-antibrane system are expected to be
very energetic closed strings, which in turn rapidly decay into
massive KK modes of the massless string levels. The strong warping
then generically channels the decay energy into massless modes
which are localized within the most strongly-warped throats,
rather than into massless bulk modes.

\section{Warped Reheating}

{}From the previous section we see that a strongly-warped internal
geometry provides a natural mechanism for channelling the energy
released by brane-antibrane annihilation onto those branes which
sit deep within the most strongly-warped throats, rather than
dumping it into massless bulk states like the graviton.
Surprisingly, this is true even if the brane-antibrane
annihilation is localized within a region which is well separated
within the extra dimensions from the most-warped throat.

A naive estimate for the expected reheat temperature is obtained
by following standard arguments \cite{Reheat}. Neglecting
dimensionless factors like string couplings and $2 \pi$, the total
4D energy density released by the annihilation is of order $\rho_0
\sim \tau_0 \sim M_i^4$, and this energy is dumped into massless
string states with a rate set by the local string scale, $\Gamma_s
\sim M_i$. Estimating the rate for the resulting KK modes to decay
into massless brane states to be also of this order $\Gamma_{KK}
\sim M_i$, makes the total rate for producing massless states
$\Gamma_{tot} \sim M_i$.

During this decay process the initial energy density dilutes due
to the expansion of the large 4 dimensions, according to $\rho(a)
= \rho_0 (a_0/a)^n$, with the power $n$ depending on the equation
of state of the universe during this time. Since the massless
modes cannot come into thermal equilibrium until the Hubble scale,
$H = \dot{a}/a \sim \rho^{1/2}/M_p \sim H_0 (a_0/a)^{n/2}$, falls
below $\Gamma_{tot}$, it cannot occur earlier than $a = a_{eq}$,
where $a_{eq}$ is defined by the condition $H(a_{eq}) =
\Gamma_{tot}$ and so $(a_0/a_{eq})^{n/2} \sim \Gamma_{tot}/H_0
\sim \Gamma_{tot} M_p/\rho_0^{1/2}$.

Assuming that equilibrium happens immediately, once $H$ reaches
$\Gamma_{tot}$, the resulting equilibrium temperature is estimated
by writing $\rho(a_{eq}) \sim T_{RH}^4$, and so $T_{RH} \sim
\rho_0^{1/4} ( a_0/a_{eq})^{n/4} \sim (\Gamma_{tot} M_p)^{1/2}$.
Once order-unity factors are included this leads to
%
\beq
\label{reheat}
    T_{RH} \sim 0.1 \, (\Gamma_{tot} M_p)^{1/2} \sim 0.1 \,
    a_i^{1/2} \, M_p \sim 10^{-3} \, M_p \,,
\eeq
%
where the last estimate uses $a_i \sim 10^{-4}$. Notice that this is
independent of the equation of state parameter $n$, so long as the
universal energy dilution is well-described by a power law. If reliable,
it would also be high enough to adequately reheat the post-inflationary
universe to avoid potential problems to which a low reheat temperature
can give rise.  This can only be regarded as illustrative, because the
result is larger than both $M_i$ and $M_{sm}$: because it is larger than
the string scale in the throats it invalidates the 4D field-theoretic
calculation on which it is based. A more careful calculation must instead
be based on a higher-dimensional, string-theoretic estimate of the energy
loss, which goes beyond the scope of this article.

In conventional inflation models, such a high reheating temperature would
be in conflict with the gravitino bound (overproduction of gravitinos,
whose late decays disrupt big bang nucleosythesis).  It is interesting in
this regard that the KKLT scenario gives a very large gravitino mass,
around $m_{3/2} =  6\times 10^{10}$ GeV \cite{RA}, which is so large that
there is effectively no upper limit on the reheat temperature (see for
instance ref.\ \cite{KKM}).  The disadvantage of such a large gravitino
mass is that supersymmetry is broken at too high a scale to explain the
weak scale of the SM. If SUSY is this badly broken, one possibility for
explaining the weak hierarchy is that the large landscape of string vacua
provides a finely-tuned Higgs mass, as well as cosmological constant, as
has been suggested in ref.\ \cite{splitsusy}.  If this is the case, then
the degree of warping in the SM model brane would not be crucial for
determining the TeV scale, and the existence of an extra throat to
contain the SM model brane would be unnecessary.  However, given the
large number of 3-cycles in a typical Calabi-Yau manifold,  each of which
can carry nontrivial fluxes, the existence of many throats should be
quite generic, and it would not be surprising to find the SM brane in a
different throat from the inflationary one.


\section{Conclusions}

We have argued that for brane-antibrane inflation in
strongly-warped extra-dimensional vacua --- such as have been
considered in detail for Type IIB string models --- there is a
natural mechanism which channels the released energy into
reheating the Standard Model degrees of freedom. This is because a
sizeable fraction of the false vacuum energy of the
brane-antibrane system naturally ends up being deposited into
massless modes on branes which are localized inside the most
strongly-warped throats, rather than being dumped into massless
bulk-state modes graviton KK modes.

This process relies on what is known about brane-antibrane
annihilation in flat space, where it is believed that the
annihilation energy dominantly produces very massive closed-string
states, which then quickly themselves decay to produce massive KK
modes for massless string states. What is important for our
purposes is that the wave functions for all of the massive KK
modes of this type are typically exponentially enhanced at the
bottom of warped throats, while those for the massless KK bulk
modes are not. This enhancement arises because the energies of
these states are minimized if their probabilities are greatest in
the most highly warped regions. This peaking is crucial because it
acts to suppress the couplings of the massive KK modes to the
massless bulk states, while enhancing their couplings to brane
modes in the most warped throats.

The upshot is that most of the energy density of the
brane-antibrane system ends up as radiation on the SM brane,
regardless of whether or not this brane is physically separated
from the place where the brane-antibrane annihilation actually
occurs. Our arguments are generic, and do not require us to make
any unusual assumptions or restrictions on the parameters. From
this point of view, it is unavoidable to efficiently reheat the SM
brane after brane-antibrane inflation, so long as there are no
other hidden branes lying in even deeper throats than the SM. This
observation is all the more interesting given the attention which
multiple-throat inflationary models are now receiving, both due to
the better understanding which they permit for the relation
between the inflationary scale and those of low-energy particle
physics, and to the prospects they raise for producing long-lived
cosmic string networks with potentially observable consequences.

\section*{Acknowledgements}
It is a pleasure to thank Shamit Kachru, Renata Kallosh, Andrei
Linde, Juan Maldacena, Anupam Mazumdar, Liam McAllister, Rob
Myers, Joe Polchinski, Fernando Quevedo, Raul Rabad\'an, Horace
Stoica and Henry Tye for fruitful discussions. This research is
supported in part by funds from NSERC of Canada, FQRNT of Qu\'ebec
and McGill University.

\section{Appendix: The 4D View}

In this appendix we compute the low-energy couplings amongst the
bulk zero modes and brane modes in the effective 4D supergravity
obtained after modulus stabilization {\it \`a la} KKLT \cite{kklt}.
Besides checking the scaling of the kinetic terms obtained by
dimensionally reducing the Einstein-Hilbert action, this also
allows the study of the couplings in the scalar potential which
arise from modulus stabilization and so are more difficult to
analyze from a semiclassical, higher-dimensional point of view.

To this end imagine integrating out all of the extra-dimensional
physics to obtain the low-energy effective 4D supergravity for a
Type IIB GKP vacuum having only the mandatory volume modulus (and
its supersymmetric friends) plus various low-energy brane modes
(such as those describing the motion of various D3 branes). The
terms in this supergravity involving up to two derivatives are
completely described once the K\"ahler function, $K$,
superpotential, $W$, and gauge kinetic function, $f_{ab}$, are
specified.

Denoting the bulk-modulus supermultiplet by $T$ and the brane
multiplets by $\phi^I$, we use the K\"ahler potential
\cite{truncation,kklt,kklmmt,louis}
%
\begin{equation}
    K = -3 \log \left[ r \right] \,,
\end{equation}
%
where $r = T + T^* + k(\phi,\phi^*)$. For instance, if $\phi^I$
denotes the position of single brane, then $k$ is the K\"ahler
potential for the underlying 6D manifold. This implies the scalar
kinetic terms are governed by the following K\"ahler metric in
field space
%
\begin{equation}
    K_{TT^*} = \frac{3}{r^2} \,, \qquad
    K_{IT^*} = \frac{3 \, k_I}{r^2} \, \qquad \hbox{and} \qquad
    K_{IJ^*} = \frac{3}{r^2} \, \left[ k_I \, k_{J^*} - r\, k_{IJ^*}
    \right] \,,
\end{equation}
%
with inverse
%
\begin{equation}
    K^{T^*T} = \frac{r}{3} \, \left[ r - k^{L^*N} k_{L^*} k_N \right]
    \,, \quad
    K^{J^*T} = \frac{r\, k^{J^*L} k_L }{3} \quad \hbox{and} \quad
    K^{J^*I} = - \frac{r\, k^{J^*I}}{3}  \,.
\end{equation}

In the absence of modulus stabilization the superpotential of the
effective theory is a constant \cite{gvw}, $W = w_0$, and the
supergravity takes the usual no-scale form \cite{noscale}, with
vanishing scalar potential. If, however, there are low-energy
gauge multiplets associated with any of the D7 branes of the model
then their gauge kinetic function is $f_{ab} = T \, \delta_{ab}$.
For nonabelian multiplets of this type gaugino condensation
\cite{gc,ourgc} can generate a nontrivial superpotential, of the
form
%
\begin{equation}
    \qquad W = w_0 + A \, \exp \left[ - a \, T \right] \,,
\end{equation}
%
where $A$ and $a$ are calculable constants.

With these choices the K\"ahler derivatives of the superpotential
become
%
\begin{equation}
    D_T W = W_T - \frac{3 W}{r} \,, \qquad \hbox{and} \qquad
    D_I W = - \frac{3 k_I \, W}{r} \,,
\end{equation}
%
and so the supersymmetric scalar potential \cite{cremmeretal}
becomes
%
\begin{equation}
    V = \frac{1}{3r^2} \, \left[ \left(r - k^{I^*J} k_{I*} k_J
    \right) |W_T|^2 - 3 (W^* W_T + W W_T^*) \right] \,.
\end{equation}
%
Notice that use of these expression implicitly requires that we
work in the 4D Einstein frame, and so are using 4D Planck units
for which $M_p = {\cal O}(1)$.

If we specialize to the case of several branes, for which
$\{\phi^I \} = \{\phi^i_n \}$, with $i$ labelling the fields on a
given brane and $n = 1,\dots,N$ labelling which brane is involved,
then we typically have
%
\begin{equation}
    k(\phi^I,\phi^{I*}) = \sum_n k^{(n)}(\phi^i_n, \phi^{i*}_n)
    \,.
\end{equation}
%
In this case the K\"ahler metric built from $k$ is block diagonal,
with $k_{i_n j_m} = k^{(n)}_{ij} \, \delta_{mn}$, and so $k^{I^*J}
k_{I^*} k_J = \sum_n k_{(n)}^{i^* j} k^{(n)}_{i^*} k^{(n)}_{j}$
and so on.

We may now see how strongly the bulk KK zero modes, $g_{\mu\nu}$
and $T$, couple to one another and to the brane modes. Setting $k
= 0$ in the above shows that the couplings of $T$ and $g_{\mu\nu}$
to one another are order unity, and since our use of the standard
4D supergravity formalism requires us to be in the Einstein frame,
this implies these are all of 4D Planck strength (in agreement
with our higher-dimensional estimates).

Couplings to the branes are obtained by keeping $k$ nonzero, and
in the event that the branes are located in highly warped regions,
we must take $k^{(n)} = {\cal O}(a_n^2)$ with $a_n \ll 1$ denoting
the warp factor at the position of brane $n$.\footnote{For
instance, this power of $a_n$ reproduces the $a_n$-dependence of
the factor $\sqrt{g} g^{\mu\nu}$ obtained by dimensionally
reducing the higher-dimensional kinetic terms.} In this case the
combination $k_{(n)}^{i^* j} k^{(n)}_{i^*} k^{(n)}_{j}$ is also
${\cal O}(a_n^2)$.

Suppose we now expand the functions $k^{(n)}$ in powers of $\phi$
and keep only the leading powers:
%
\begin{equation}
    k^{(n)} \approx a_n^2 \sum_i \phi^{i*}_n \, \phi^i_n \,.
\end{equation}
%
Then, since the $\phi_n$ kinetic terms are ${\cal O}(a_n^2)$, we
see that the canonically-normalized fields are $\chi^i_n = a_n
\phi^i_n$. Once this is done the leading couplings to $T$ and
$g_{\mu\nu}$ are those which involve those parts of $k^{(n)}$ that
are quadratic in $\chi^i_n$, and since these are also order unity,
these couplings are also of Planck strength (again in agreement
with our earlier estimates).

Alternatively, consider now those couplings which only involve the
brane modes. Working to leading order in $a_n^2$, we see that a
term in $k^{(n)}$ of the form $(\chi^i_n)^k$ has a strength which
is of order $a_n^{2-k}$. For instance the case $k = 3$ generates
cubic couplings from the kinetic lagrangian of order $a_n^{-1}
\chi \partial \chi \partial \chi$, whose coefficient is of order
$(a_n M_p)^{-1} = M_{sm}^{-1}$. These are larger than Planck
suppressed, as expected.

\begin{thebibliography}{4}

\bibitem{SugraInflation}
D.V. Nanopoulos, K.A. Olive, M. Srednicki and K. Tamvakis, Phys.\
Lett.\ {\bf B123} (1983) 41;
%
A.B. Goncharov and A.D. Linde, Phys.\ Lett.\ {\bf B139} (1984) 27;
%
B. Gato, J. Leon, M. Quiros and M. Ramon-Medrano, Z.\ Phys.\ {\bf
C22} (1984) 345;
%
G. Gelmini, C. Kounnas and D.V. Nanopoulos, Nucl.\ Phys.\ {\bf
B250} (1985) 177;
%
L.G. Jensen and K.A. Olive, Nucl.\ Phys.\ {\bf B263} (1986) 731;
%
P.~Binetruy and M.~K.~Gaillard,
%``Candidates For The Inflaton Field In Superstring Models,''
Phys.\ Rev.\ {\bf D34} (1986) 3069;
%%CITATION = PHRVA,D34,3069;%%
%
G.L. Cardoso and B.A. Ovrut, Phys.\ Lett.\ {\bf B298} (1993)
292-298 [hep-th/9210114];
%
R.~Brustein and P.~J.~Steinhardt,
%``Challenges For Superstring Cosmology,''
Phys.\ Lett.\ {\bf B302}, 196 (1993) [hep-th/9212049];
%%CITATION = HEP-TH 9212049;%%
%
J.A. Adams, G.G. Ross and S. Sarkar, Phys.\ Lett.\ {\bf B391}
(1997) 271-280 [hep-ph/9608336];
%
E. Halyo, Phys.\ Lett.\ {\bf B387} (1996) 43-47 [hep-ph/9606423];
%
P.~Binetruy and G.~R.~Dvali,
%``D-term inflation,''
Phys.\ Lett.\ {\bf B388} (1996) 241 [hep-ph/9606342];
%
A.D. Linde and A. Riotto, Phys.\ Rev.\ {\bf D56} (1997) 1841-1844
[hep-ph/9703209].

\bibitem{DvaliTye}
G.~R.~Dvali and S.~H.~H.~Tye,
%``Brane inflation,''
Phys.\ Lett.\ {\bf B450} (1999) 72 [hep-ph/9812483].

\bibitem{BI1}
C.~P.~Burgess, M.~Majumdar, D.~Nolte, F.~Quevedo, G.~Rajesh and
R.~J.~Zhang,
%``The inflationary brane-antibrane universe,''
JHEP {\bf 0107} (2001) 047 [hep-th/0105204].
%%CITATION = HEP-TH 0105204;%%

\bibitem{BBbarInflation}
S.~H.~Alexander,
%``Inflation from D - anti-D brane annihilation,''
Phys.\ Rev.\ {\bf D65} (2002) 023507 [hep-th/0105032];
%
G.~R.~Dvali, Q.~Shafi and S.~Solganik, ``D-brane inflation,''
[hep-th/0105203].

\bibitem{Angles}
J.~Garcia-Bellido, R.~Rabadan and F.~Zamora, ``Inflationary
scenarios from branes at angles,'' JHEP {\bf 0201} (2002) 036
[hep-th/0112147];
%
N.~Jones, H.~Stoica and S.~H.~Tye, ``Brane interaction as the
origin of inflation,'' JHEP {\bf 0207} (2002) 051
[hep-th/0203163];
%
M.~Gomez-Reino and I.~Zavala, ``Recombination of intersecting
D-branes and cosmological inflation,'' JHEP {\bf 0209} (2002) 020
[hep-th/0207278].

\bibitem{Others}
A. Mazumdar, S. Panda and A. P\'erez-Lorenzana, ``Assisted
Inflation via Tachyon Condensation,'' Nucl.\ Phys.\ {\bf B614}
(2001) 101-116 [hep-ph/0107058];
%
C.~P.~Burgess, P.~Martineau, F.~Quevedo, G.~Rajesh and
R.~J.~Zhang, ``Brane antibrane inflation in orbifold and
orientifold models,'' JHEP {\bf 0203} (2002) 052 [hep-th/0111025];
%%CITATION = HEP-TH 0111025;%%
%
C.~Herdeiro, S.~Hirano and R.~Kallosh, ``String theory and hybrid
inflation / acceleration,'' JHEP {\bf 0112} (2001) 027
[hep-th/0110271];
%
K.~Dasgupta, C.~Herdeiro, S.~Hirano and R.~Kallosh, ``D3/D7
inflationary model and M-theory,'' Phys.\ Rev.\ D {\bf 65} (2002)
126002 [hep-th/0203019];
%
L.~Pilo, A.~Riotto and A.~Zaffaroni, ``Old inflation in string
theory,'' [hep-th/0401004];
%
J.~P.~Hsu, R.~Kallosh and S.~Prokushkin, ``On brane inflation with
volume stabilization,'' JCAP {\bf 0312} (2003) 009
[hep-th/0311077];
%%CITATION = HEP-TH 0311077;%%
%
%\bibitem{Koyama:2003yc}
F.~Koyama, Y.~Tachikawa and T.~Watari, ``Supergravity analysis of
hybrid inflation model from D3-D7 system'', [hep-th/0311191];
%%CITATION = HEP-TH 0311191;%%
%
%\bibitem{henry}
H.~Firouzjahi and S.~H.~H.~Tye, ``Closer towards inflation in
string theory,'' Phys.\ Lett.\ B {\bf 584} (2004) 147
[hep-th/0312020].
%%CITATION = HEP-TH 0312020;%%

\bibitem{cosmicstrings}
S.~Sarangi and S.~H.~H.~Tye, ``Cosmic string production towards
the end of brane inflation,'' Phys.\ Lett.\ B {\bf 536} (2002) 185
[hep-th/0204074];
%
G.~Dvali, R.~Kallosh and A.~Van Proeyen, ``D-term strings,'' JHEP
{\bf 0401} (2004) 035 [hep-th/0312005];
%%CITATION = HEP-TH 0312005;%%
%
G.~Dvali and A.~Vilenkin, ``Formation and evolution of cosmic
D-strings,'' JCAP {\bf 0403} (2004) 010 [hep-th/0312007].
%%CITATION = HEP-TH 0312007;%%
%
J. Urrestilla, A. Achucarro and A.C. Davis, Phys.\ Rev.\ Lett.\
{\bf 92} (2004) 251302 [hep-th/0402032].

\bibitem{GKP}
S.~B.~Giddings, S.~Kachru and J.~Polchinski, ``Hierarchies from
fluxes in string compactifications,'' Phys. Rev. {\bf D66}, 106006
(2002).

\bibitem{Sethi}
S.~Sethi, C.~Vafa and E.~Witten, ``Constraints on low-dimensional
string compactifications,'' Nucl.\ Phys.\ B {\bf 480} (1996) 213
[hep-th/9606122];
%
K.~Dasgupta, G.~Rajesh and S.~Sethi, ``M theory, orientifolds and
G-flux,'' JHEP {\bf 9908} (1999) 023 [hep-th/9908088].

\bibitem{kklt}
S. Kachru, R. Kallosh, A. Linde and S. P. Trivedi, ``de Sitter
Vacua in String Theory,'' [hep-th/0301240].

\bibitem{kklmmt}
S.~Kachru, R.~Kallosh, A.~Linde, J.~Maldacena, L.~McAllister and
S.~P.~Trivedi, ``Towards inflation in string theory,'' JCAP {\bf
0310} (2003) 013 [hep-th/0308055].

\bibitem{racetrackinflation}
J.J.~ Blanco-Pillado, C.P.~Burgess, J.M.~Cline, C.~Escoda,
M.~G\'omez-Reino, R.~Kallosh, A.~Linde, and F.~Quevedo,
``Racetrack Inflation,'' (hep-th/0406230).
%%CITATION = HEP-TH 0406230;%%

\bibitem{OtherStringInflation}
A.R. Frey, M. Lippert and B. Williams, ``The fall of stringy de
Sitter", Phys.\ Rev.\ {\bf D68} (2003) 046008 [hep-th/0305018];
%%CITATION = HEP-TH 0305018;%%
%
S.~Buchan, B.~Shlaer, H.~Stoica and S.~H.~H.~Tye, ``Inter-brane
interactions in compact spaces and brane inflation,''
[hep-th/0311207];
%
C.~Escoda, M.~Gomez-Reino and F.~Quevedo, ``Saltatory de Sitter
string vacua,''JHEP {\bf 0311} (2003) 065, [hep-th/0307160];
%
C.~P.~Burgess, R.~Kallosh and F.~Quevedo, ``de Sitter string vacua
from supersymmetric D-terms,'' JHEP {\bf 0310} (2003) 056
[hep-th/0309187]
%%CITATION = HEP-TH 0309187;%%
%
%
S.~Buchan, B.~Shlaer, H.~Stoica and S.~H.~H.~Tye, ``Inter-brane
interactions in compact spaces and brane inflation,''
[hep-th/0311207];
%
J.~P.~Hsu, R.~Kallosh and S.~Prokushkin, ``On brane inflation with
volume stabilization,'' JCAP {\bf 0312} (2003) 009
[hep-th/0311077];
%
A.~Buchel and R.~Roiban, ``Inflation in warped geometries,''
hep-th/0311154;
%%CITATION = HEP-TH 0311154;%%
%
H.~Firouzjahi and S.~H.~H.~Tye, ``Closer towards inflation in
string theory,'' [hep-th/0312020];
%%CITATION = HEP-TH 0312020;%%
%
A. Saltman and E. Silverstein, ``The Scaling of the No Scale
Potential and de Sitter Model Building,'' [hep-th/0402135];
%
E.~Halyo, ``D-brane inflation on conifolds,'' hep-th/0402155;
%
R.~Kallosh and S.~Prokushkin, ``SuperCosmology,''
[hep-th/0403060];
%%CITATION = HEP-TH 0403060;%%
%
M. Becker, G. Curio and A. Krause, Nucl.\ Phys.\ {\bf B693} (2004)
223-260 [hep-th/0403027];
%
M. Alishahiha, E. Silverstein and D. Tong, ``DBI in the sky,''
(hep-th/0404084);
%
J.~P.~Hsu and R.~Kallosh, ``Volume stabilization and the origin of
the inflaton shift symmetry in string theory,'' JHEP {\bf 0404}
(2004) 042 [hep-th/0402047];
%%CITATION = HEP-TH 0402047;%%
%
 O.~DeWolfe, S.~Kachru and H.~Verlinde,
``The giant inflaton,'' JHEP {\bf 0405} (2004) 017
[hep-th/0403123].
%%CITATION = HEP-TH 0403123;%%
%
N.~Iizuka and S.~P.~Trivedi, ``An inflationary model in string
theory,'' hep-th/0403203.
%%CITATION = HEP-TH 0403203;%%

\bibitem{BIReviews}
F.~Quevedo, Class.\ Quant.\ Grav.\  {\bf 19} (2002) 5721,
[hep-th/0210292];
%
A.~Linde, ``Prospects of inflation,'' [hep-th/0402051];
%%CITATION = HEP-TH 0402051;%%
%
C.P. Burgess, ``Inflationary String Theory?,'' Pramana (to appear)
[hep-th/0408037].

\bibitem{twothroat}
N.~Iizuka and S.~P.~Trivedi, ``An inflationary model in string
theory,'' hep-th/0403203.
%%CITATION = HEP-TH 0403203;%%
%
X. Chen, ``Multi-throat Brane Inflation,'' [hep-th/040084].

\bibitem{stringycosmicstrings}
E.~J.~Copeland, R.~C.~Myers and J.~Polchinski, ``Cosmic F- and
D-strings,'' JHEP {\bf 0406} (2004) 013 [hep-th/0312067];
%%CITATION = HEP-TH 0312067;%%
%
L.~Leblond and S.~H.~H.~Tye, ``Stability of D1-strings inside a
D3-brane,'' JHEP {\bf 0403} (2004) 055 [hep-th/0402072];
%%CITATION = HEP-TH 0402072;%%
%
K.~Dasgupta, J.~P.~Hsu, R.~Kallosh, A.~Linde and M.~Zagermann,
``D3/D7 brane inflation and semilocal strings,'' (hep-th/0405247);
%
L. Leblond and S.H. Tye, ``Stability of D1 Strings Inside a D3
Brane,'' JHEP 0403 (2004) 055 [hep-th/0402072].

\bibitem{cmqu}
J.~F.~G.~Cascales, M.~P.~Garcia del Moral, F.~Quevedo and
A.~M.~Uranga, ``Realistic D-brane models on warped throats:
Fluxes, hierarchies and moduli stabilization,'' hep-th/0312051.

\bibitem{bcqs}
C.~P.~Burgess, J.~M.~Cline, H.~Stoica and F.~Quevedo, ``Inflation
in realistic D-brane models,'' [hep-th/0403119].
%%CITATION = HEP-TH 0403119;%%

\bibitem{RA}
R. Kallosh and A. Linde,
%``LANDSCAPE, THE SCALE OF SUSY BREAKING, AND INFLATION,''
[hep-th/0411011].

\bibitem{braneheat}
Y. Himemoto and T. Tanaka, Phys.\ Rev.\ {\bf D67} (2003) 084014
[gr-qc/0212114];
%
R. Allahverdi, A. Mazumdar and A. Perez-Lorenzana, Phys.\ Lett.\
{\bf B516} (2001) 431-438 [hep-ph/0105125];
%
J.H. Brodie and D.A. Easson, JCAP 0312 (2003) 004
[hep-th/0301138];
%
Y. Takamizu and K. Maeda, ``Collision of Domain Walls and
Reheating of the Brane Universe,'' [hep-th/0406235];
%
E. Papantonopoulos and V. Zamarias, JHEP 0410 (2004) 051
[hep-th/0408227];
%
C.J. Chang and I.G. Moss, ``Brane Inflation With Dark Reheating,''
[hep-ph/0411021].

\bibitem{RS}
L.~Randall and R.~Sundrum,
``A large mass hierarchy from a small extra dimension,''
Phys.\ Rev. Lett. {\bf 83} ({1999}) {3370} [\hepph{9905221}];
%
L.~Randall and R.~Sundrum,
``An alternative to compactification,''
Phys. Rev. Lett. {\bf 83} ({1999}) {4690} [\hepth{9906064}].

\bibitem{IntScale}
K.~Benakli, {\it Phenomenology of low quantum gravity scale
models,} Phys.\ Rev.\ D {\bf 60}, 104002 (1999) [hep-ph/9809582];
%%CITATION = HEP-PH 9809582;%%
C.~P.~Burgess, L.~E.~Ibanez and F.~Quevedo, {\it Strings at the
intermediate scale or is the Fermi scale dual to the  Planck
scale?} Phys.\ Lett.\ B {\bf 447}, 257 (1999) [hep-ph/9810535].
%%CITATION = HEP-PH 9810535;%%

\bibitem{Rob}
A. Frey, R. Myers and A. Mazumdar, in preparation.

\bibitem{throatmetric}
O. De Wolfe and S.B. Giddings, ``Scales and Hierarchies in Warped
Compactifications and Brane Worlds,'' Phys.\ Rev.\ {\bf D67}
(2003) 066008 [hep-th/0208123].

\bibitem{dkkls}
S. Dimopoulos, S. Kachru, N. Kaloper, A. Lawrence and E.
Silverstein,
%``Small numbers from tunnelling between brane throats,''
[hep-th/0104239].

\bibitem{PlanckBraneModes}
W.~D.~Goldberger and M.~B.~Wise, ``Bulk fields in the
Randall-Sundrum compactification scenario,'' Phys.\ Rev.\ D {\bf
60}, 107505 (1999) [hep-ph/9907218];
%
%%CITATION = HEP-PH 9907218;%%
H.~Davoudiasl, J.~L.~Hewett and T.~G.~Rizzo, ``Phenomenology of
the Randall-Sundrum gauge hierarchy model,'' Phys.\ Rev.\ Lett.\
{\bf 84}, 2080 (2000) [hep-ph/9909255].
%%CITATION = HEP-PH 9909255;%%

\bibitem{BraneRad}
L. McAllister and I. Mitra, ``Relativistic D-Brane Scattering is
Extremely Inelastic,'' [hep-th/0408085].

\bibitem{Sen}
A. Sen, ``Tachyon Matter,'' JHEP 0207 (2002) 065 [hep-th/0203265].

\bibitem{BraneDecay}
A. Sen,
%``Rolling Tachyon,''
JHEP {\bf 0204} (2002) 048 [hep-th/0203211];
%
N. Lambert, H. Liu and J. Maldacena,
%``CLOSED STRINGS FROM DECAYING D-BRANES,''
[hep-th/0303139].

\bibitem{Sen-review}
A.~Sen, ``Tachyon dynamics in open string theory,''
hep-th/0410103.
%%CITATION = HEP-TH 0410103;%%

\bibitem{HiJCascade}
R. Iengo and J.G. Russo,
%``THE DECAY OF MASSIVE CLOSED SUPERSTRINGS WITH
%MAXIMUM ANGULAR MOMENTUM,''
JHEP {\bf 0211} (2002) 045 [hep-th/0210245];
%
R. Iengo and J.G. Russo,
%``SEMICLASSICAL DECAY OF STRINGS WITH MAXIMUM ANGULAR MOMENTUM,''
JHEP {\bf 0303} (2003) 030 [hep-th/0301109];
%
D. Chialva, R. Iengo and J.G. Russo,
%``DECAY OF LONG-LIVED MASSIVE CLOSED SUPERSTRING STATES:
%EXACT RESULTS,''
JHEP {\bf 0312} (2003) 014 [hep-th/0310283];
%
D. Chialva, R. Iengo and J.G. Russo,
%``SEARCH FOR THE MOST STABLE MASSIVE STATE IN
%SUPERSTRING THEORY,''
[hep-th/0410152].

\bibitem{RobnJoe}
J. Lykken, R. Myers and J. Wang, ``Gravity in a Box,'' JHEP {\bf
0009} (2000)009 [hep-th/0006191].

\bibitem{Reheat}
L.~F.~Abbott, E.~Farhi and M.~B.~Wise,
``Particle Production In The New Inflationary Cosmology,''
Phys.\ Lett.\ B {\bf 117}, 29 (1982);
%%CITATION = PHLTA,B117,29;%%
A.~D.~Linde,
``Particle Physics And Inflationary Cosmology,''
 Chur, Switzerland: Harwood (1990) 362 p. (Contemporary concepts in physics, 5).
%\href{http://www.slac.stanford.edu/spires/find/hep/www?irn=2352990}{SPIRES entry}

\bibitem{truncation}
E.~Witten, ``Dimensional Reduction Of Superstring Models,'' Phys.\
Lett.\ B {\bf 155} (1985) 151;
%
C.~P.~Burgess, A.~Font and F.~Quevedo, ``Low-Energy Effective
Action For The Superstring,'' Nucl.\ Phys.\ B {\bf 272} (1986)
661.

\bibitem{louis}
M. Gra\~a, T.W. Grimm, H. Jockers and J. Louis, Nucl.\ Phys.\ {\bf
B690} (2004) 21-61 [hep-th/0312232];
%
T.W. Grimm and J. Louis, Nucl.\ Phys.\ {\bf B699} (2004) 387-426
[hep-th/0403067].

\bibitem{gvw}
S. Gukov, C. Vafa and E. Witten, ``CFTs from Calabi-Yau
Fourfolds,'' Nucl. Phys. {\bf B584}, 69 (2000).

\bibitem{noscale}
E. Cremmer, S. Ferrara, C. Kounnas and D.V. Nanonpoulos,
``Naturally vanishing cosmological constant in $N=1$
supergravity,'' Phys. Lett. {\bf B133}, 61 (1983);
%
J. Ellis, A.B. Lahanas, D.V. Nanopoulos and K. Tamvakis,
``No-scale Supersymmetric Standard Model,'' Phys. Lett. {\bf
B134}, 429 (1984).

\bibitem{gc}
J.~P.~Derendinger, L.~E.~Ibanez and H.~P.~Nilles, ``On The
Low-Energy D = 4, N=1 Supergravity Theory Extracted From The D =
10, N=1 Superstring,'' Phys.\ Lett.\ B {\bf 155} (1985) 65;
%%CITATION = PHLTA,B155,65;%%
%
M.~Dine, R.~Rohm, N.~Seiberg and E.~Witten, ``Gluino Condensation
In Superstring Models,'' Phys.\ Lett.\ B {\bf 156} (1985) 55.
%%CITATION = PHLTA,B156,55;%%

\bibitem{ourgc}
C.P. Burgess, J.-P. Derendinger, F. Quevedo and M. Quir\'os,
``Gaugino Condensates and Chiral-Linear Duality: An
Effective-Lagrangian Analysis'', Phys.\ Lett.\ B {\bf 348} (1995)
428--442;
%%CITATION = PHLTA,B348,428;%%
``On Gaugino Condensation with Field-Dependent Gauge Couplings'',
Ann.\ Phys.\ {\bf 250} (1996) 193-233.

\bibitem{cremmeretal}
E.~Cremmer, B.~Julia, J.~Scherk, S.~Ferrara, L.~Girardello and
P.~van Nieuwenhuizen, ``Spontaneous Symmetry Breaking And Higgs
Effect In Supergravity Without Cosmological Constant,'' Nucl.\
Phys.\ B {\bf 147}, 105 (1979).
%%CITATION = NUPHA,B147,105;%%

\bibitem{KKM}
M.~Kawasaki, K.~Kohri and T.~Moroi,
``Hadronic decay of the gravitino in the early universe and its implications
to inflation,''
arXiv:hep-ph/0410287.
%%CITATION = HEP-PH 0410287;%%


\bibitem{splitsusy}
N.~Arkani-Hamed and S.~Dimopoulos,
``Supersymmetric unification without low energy supersymmetry and signatures
for fine-tuning at the LHC,''
arXiv:hep-th/0405159;

%%CITATION = HEP-TH 0405159;%%
G.~F.~Giudice and A.~Romanino,
``Split supersymmetry,''
Nucl.\ Phys.\ B {\bf 699}, 65 (2004)
[arXiv:hep-ph/0406088].
%%CITATION = HEP-PH 0406088;%%



\end{thebibliography}



\end{document}
